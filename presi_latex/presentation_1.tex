%%%%%%%%%%%%%%%%%%%%%%%%%%%%%%%%%%%%%%%%%
% Beamer Presentation
% LaTeX Template
% Version 1.0 (10/11/12)
%
% This template has been downloaded from:
% http://www.LaTeXTemplates.com
%
% License:
% CC BY-NC-SA 3.0 (http://creativecommons.org/licenses/by-nc-sa/3.0/)
%
%%%%%%%%%%%%%%%%%%%%%%%%%%%%%%%%%%%%%%%%%

%----------------------------------------------------------------------------------------
%	PACKAGES AND THEMES
%----------------------------------------------------------------------------------------

\documentclass{beamer}

\mode<presentation> {
\date{12. November 2015}
% The Beamer class comes with a number of default slide themes
% which change the colors and layouts of slides. Below this is a list
% of all the themes, uncomment each in turn to see what they look like.

%\usetheme{default}
%\usetheme{AnnArbor}
%\usetheme{Antibes}
%\usetheme{Bergen}
%\usetheme{Berkeley}
%\usetheme{Berlin}
%\usetheme{Boadilla}
%\usetheme{CambridgeUS}
%\usetheme{Copenhagen}
%\usetheme{Darmstadt}
%\usetheme{Dresden}
%\usetheme{Frankfurt}
%\usetheme{Goettingen}
\usetheme{Hannover}
%\usetheme{Ilmenau}
%\usetheme{JuanLesPins}
%\usetheme{Luebeck}
%\usetheme{Madrid}
%\usetheme{Malmoe}
%\usetheme{Marburg}
%\usetheme{Montpellier}
%\usetheme{PaloAlto}
%\usetheme{Pittsburgh}
%\usetheme{Rochester}
%\usetheme{Singapore}
%\usetheme{Szeged}
%\usetheme{Warsaw}

% As well as themes, the Beamer class has a number of color themes
% for any slide theme. Uncomment each of these in turn to see how it
% changes the colors of your current slide theme.

%\usecolortheme{albatross}
%\usecolortheme{beaver}
%\usecolortheme{beetle}
%\usecolortheme{crane}
%\usecolortheme{dolphin}
%\usecolortheme{dove}
%\usecolortheme{fly}
%\usecolortheme{lily}
%\usecolortheme{orchid}
%\usecolortheme{rose}
%\usecolortheme{seagull}
%\usecolortheme{seahorse}
%\usecolortheme{whale}
%\usecolortheme{wolverine}

%\setbeamertemplate{footline} % To remove the footer line in all slides uncomment this line
%\setbeamertemplate{footline}[page number] % To replace the footer line in all slides with a simple slide count uncomment this line

%\setbeamertemplate{navigation symbols}{} % To remove the navigation symbols from the bottom of all slides uncomment this line
\institute{Institut für Softwaretechnik und Programmiersprachen}
\subject{Thesis}
\keywords{DNA, Thesis}
\date{12. November 2015}
}

\usepackage{filecontents}
\usepackage{underscore}
\usepackage[T1]{fontenc}
\usepackage[utf8]{inputenc}                 % Eingabekodierung
\usepackage[ngerman]{babel}                 % Deutsch 
\usepackage{graphicx} % Allows including images
\usepackage{booktabs} % Allows the use of \toprule, \midrule and \bottomrule in tables


\usepackage{tikz}                   % TikZ pictures
\usetikzlibrary{shadows}
\usepackage{amsmath,amssymb,amsthm} % Math


\usepackage[ruled,vlined,linesnumbered]{algorithm2e} % Algorithms
\newcommand{\twodots}{\mathrel{{.}\,{.}}\nobreak}

%----------------------------------------------------------------------------------------
%	TITLE PAGE
%----------------------------------------------------------------------------------------

\title[]{Browser-basierte Cache-Angriffe auf die RSA-Schlüsselgenerierung} % The short title appears at the bottom of every slide, the full title is only on the title page

\author[M. Krebbel]{Moritz Krebbel} % Your name
\institute[ITS] % Your institution as it will appear on the bottom of every slide, may be shorthand to save space
{Universität zu Lübeck\\
Institut für IT-Sicherheit% Your institution for the title page
\medskip
\textit{} % Your email address
}
\date{30. Oktober 2018} % Date, can be changed to a custom date

\begin{document}

\begin{frame}
\titlepage % Print the title page as the first slide
%Ich möchte im folgenden die Ergebenisse meiner Masterarbeit mit dem Titel Browser-basierte Cache-Angriffe auf die RSA-Schlusselgenerierung vorstellen.
\end{frame}

\begin{frame}
\frametitle{Überblick} % Table of contents slide, comment this block out to remove it
\tableofcontents[hideallsubsections] % Throughout your presentation, if you choose to use \section{} and \subsection{} commands, these will automatically be printed on this slide as an overview of your presentation

%Zunächst einmal ein kleiner Überlick über die Inhalte.
%Zuerst möchte ich für das Thema motivieren und den Themenschwerpunkt der Arbeit näher vorstellen.
%Danach möchte ich zeigen wie Angriffe im Browser umgesetzt werden und welche Unterschiede zu nativen Angriffen bestehen.
%Insebsondere möchte ich dort auf die Initalisierungsphase des Angriffs eingehen und optmieren dieser beschreiben.
%Nachfolgend möchte ich die Probleme beschrieben, die beim portieren eines nativen Angriffs auftreten und mögliche Lösungen dafür diskutieren.
%Abschließend gibt es ein Fazit und einen Ausblick auf zukünftige Forschungemöglichkeiten in diesem Bereich.
\end{frame}

%bis hier max 1 min 30 sec wenn flüssig läuft

%----------------------------------------------------------------------------------------
%	PRESENTATION SLIDES
%----------------------------------------------------------------------------------------

%------------------------------------------------
\section{Motivation} % Sections can be created in order to organize your presentation into discrete blocks, all sections and subsections are automatically printed in the table of contents as an overview of the talk
%------------------------------------------------

%\subsection{Subsection Example} % A subsection can be created just before a set of slides with a common theme to further break down your presentation into chunks

\begin{frame}
\frametitle{Motivation}
\begin{itemize}
\item Seitenkanalangriffe auf die Mikroarchitektur sind mächtig
%weil keine lokale Nähe nötig, "sichere" Programme über die Hardware angreifbar, hebeln schutzmaßnahmen wie virtualisierung, getrennte Prozesse aus
\item Mehr potenzielle Ziele durch Angriff aus dem Browser
%native Angriffe nur bei Cloud-computing oder Mehrbenutzersystemen sinnvoll,
%Mit Browserangriffen alle Arten von Endgeräten angreifbar
\item Portierung nativer Angriffe zu browser-basierten
\item Im Folgenden: Angriff auf den geteilten L3-Cache von Intel-CPUs
%nicht von den Gegenmaßnahmen gegen Meltdown und Spectre betroffen
\item Speicherzugriffe des Opferprogramms durch unterschiedliche Zugriffszeiten bei Cache-Hit und Miss

\end{itemize}
\end{frame}

%Seitenkanalangriffe sind mächtig da sich damit kryptographische schlüssel aus geräten extrahieren lassen. so kann man etwa den stromverbrauch einer chipkarte messen und daraus dann den schlüssel ableiten. Eine einschränkung ist hier, dass man lokalen zugriff auf das geräte bestitzen muss. Seitenkanalangriffe auf die Mirkroarchtiektur eines prozessors sind interssant, da man hier keine lokale Nähde zum system besitzen muss. außerdem können diese angriffe schutzmaßnahme wie virtualisierung oder den getrennten adressbereich von prozessen aushebeln.

%meistens ist jedoch so, dass für die angriffe die ausführung von nativen code, sprich x86-maschinenbefehlen, vorrausgesetzt wird. das schränkt den angriffsmöglichkeiten jedoch massiv ein, da wenn ich sowieso beliebigen maschinencode auf einem rechner ausführen kann, dann sind cache-angriffe eher ein geringeres Problem. Außnahmen sind hier zum beispiel mehrbenutzersystem wie in unserer uni oder cloud dienste, wo die cpu zwischen meherern benutzern geteilt wird.

%spannend wäre es somit wenn man die einschränkung, dass man nativen code ausführen können muss wegfallen lassen könnte. und genau das wurde gezeigt indem cache-angriffe im browser mittels javascript und websassembly umgesetzt wurden. dadurch, dass die angriffe im browser laufen ist zahl der potentiellen ziele natürlich deutlich erhöht.

%bei heutigen intel prozessoren ist der l3-cache zwischen den kernen aufgeteilt. außerdem ist der cache inklusiv, das heißt daten die im l1 bzw. l2 cache liegen, liegen immer auch im l3-cache

%durch diese cache-architektur können speicherzugriffe von programmen geleakt werden, indem halt unterschiedliche zugriffszeiten bei cache-hits und misses auftreten.
%dies möchte ich auf der nächsten folie noch näher erläutern.

%bis hier 2m30s

%------------------------------------------------


\section{Angriffe im Browser}
\subsection{Prime-and-Probe}
\begin{frame}
\frametitle{Prime-and-Probe}
\SetKwProg{Fn}{Function}{}{}
\begin{algorithm}[H]
\DontPrintSemicolon
\caption{Pseudo-Code für Prime-and-Probe}
\label{alg:prime_and_pribe}

\Fn{$primeAndProbe(evictionSet)$}{
    \ForEach{address in evictionSet}{
        readMem(address)\;
    }
    wait()\;
	timestampBefore $\leftarrow$ getTimestamp()\;
	\ForEach{address in evictionSet}{
        readMem(address)\;
    }
	\Return getTimestamp() - timestampBefore $>$ threshold
}

\end{algorithm}
\begin{itemize}
\item Timer im Browser durch Counter-Thread
\end{itemize}
%erläutere wie zeitmessung in JS mit Timethread
\end{frame}

%Nun möchte ich kurz den Prime-and-Probe-Angriff vorstellen der in dieser Arbeit verwendet wird.
%Mit diesem kann ermittelt werden ob das Opferprogramm auf einen bestimmten Speicheradressen zugegriffen hat.
%Hierzu wird ein Eviction-Set benötigt, auf dessen Einträge in Zeile 2 und 3 lesend zugegriffen wird.
%Dies sorgt dafür, dass das Cache-Set ausschließlich mit Einträgen aus dem Eviction-Set belegt ist.
%Die wait()-Operation soll andeuten, dass das Opferprogramm in dieser Phase Berechnungen ausführt.
%In den Zeilen 5 bis 8 wird nun gemessen wie lange der Zugriffe auf alle Einträge des Eviction-Sets dauert.
%Wenn das Opferprogarmm in der wait-phase einträge aus dem Cache-set verdrängt hat, wird eine höhere Zugriffszeit über dem threshold-Wert gemessen.
%Andernfalls sind noch alle Einträge des Eviction-Sets im Cache und es wird eine niedrige Zugriffszeit gemessen.
%Diese Unterschiede befinden sich im Nanosekundenbereich, weswegen sehr genaue Timer benötigt werden.
%Im Browser kann zurzeit nicht direkt auf höchpräsize Timer zurückgegriffen werden, weshalb hier eine Counter-Thread verwendet wird.
%Dieser inkrementiert in einem zusätzlichen Thread in einer Endlosschleife eine Integer-Variable, die dann als Zeitstempel dient.
%Solche eine Inkrementierung benötigt nur wenige Taktzyklen, weshalb dieser Workaround ausreichend genau ist.

%------------------------------------------------
\subsection{Eviction-Set-Suche}
\begin{frame}
\frametitle{Eviction-Set-Suche}

\tiny
\begin{figure}[H]
\tikzset{every picture/.style={line width=0.75pt}} %set default line width to 0.75pt        
\begin{tikzpicture}[x=0.5pt,y=0.5pt,yscale=-1,xscale=1]
%uncomment if require: \path (0,147); %set diagram left start at 0, and has height of 147
%Shape: Rectangle [id:dp378127113165748] 
\draw   (139.4,25) -- (399,25) -- (399,50) -- (139.4,50) -- cycle ;
%Shape: Rectangle [id:dp46142039984044336] 
\draw   (139.8,64) -- (444,64) -- (444,89) -- (139.8,89) -- cycle ;
%Shape: Rectangle [id:dp8338046143479867] 
\draw   (399,25) -- (487,25) -- (487,50) -- (399,50) -- cycle ;
%Shape: Rectangle [id:dp940182630841411] 
\draw   (487,25) -- (545,25) -- (545,50) -- (487,50) -- cycle ;
%Shape: Rectangle [id:dp12367960229267094] 
\draw   (444,64) -- (545,64) -- (545,89) -- (444,89) -- cycle ;
%Shape: Rectangle [id:dp7768353739474505] 
\draw   (315,106) -- (545.33,106) -- (545.33,131) -- (315,131) -- cycle ;
%Straight Lines [id:da18312510628843603] 
\draw  [dash pattern={on 0.84pt off 2.51pt}]  (443.9,105.6) -- (443.9,129.6) ;
% Text Node
\draw (721,21) node   {$0$};
% Text Node
\draw (701,71) node   {$0$};
% Text Node
\draw (67,37) node  [align=left] {Physische Adresse};
% Text Node
\draw (71,76) node  [align=left] {Virtuelle Adresse};
% Text Node
\draw (75,120) node  [align=left] {Webassembly's\\ \ \ \ \ \ Adresse};
% Text Node
\draw (516,38) node [scale=0.7] [align=left] {Cache-Line\\ \ \ \ Offset};
% Text Node
\draw (443,38) node [scale=1] [align=left] {Cache-Set};
% Text Node
\draw (266,38) node [scale=1] [align=left] {Slice (durch Hashfunktion)};
% Text Node
\draw (134,16) node [scale=0.7] [align=left] {64};
% Text Node
\draw (393,16) node [scale=0.7] [align=left] {17};
% Text Node
\draw (481,16) node [scale=0.7] [align=left] {6};
% Text Node
\draw (548,16) node [scale=0.7] [align=left] {0};
% Text Node
\draw (298,76) node [scale=1] [align=left] {Page};
% Text Node
\draw (495,76) node [scale=1] [align=left] {Page Offset};
% Text Node
\draw (438,57) node [scale=0.7] [align=left] {12};
% Text Node
\draw (309,98) node [scale=0.7] [align=left] {32};
% Text Node
\draw (548,57) node [scale=0.7] [align=left] {0};
% Text Node
\draw (548,98) node [scale=0.7] [align=left] {0};
% Text Node
\draw (438,98) node [scale=0.7] [align=left] {12};
\draw (134,57) node [scale=0.7] [align=left] {64};
% Text Node
\draw (435,118) node [scale=1] [align=left] {Adressbits};
\end{tikzpicture}
\end{figure}

\normalsize
\begin{itemize}
\item Erzeuge Adresspool durch Allokation eines Buffers
\item Wähle zufällige Adresse aus dem Pool als Zeugen
\item Teilmenge des Pools ist Eviction-Set für den Zeugen
\item Verbesserter Adresspool mittels colliding-addresses
\end{itemize}

\end{frame}

%Im Folgenden möchte ich nun darauf eingehen wie man diese im Prime-and-Probe Angriff benötigten Eviction-Sets finden kann.
%Zuerst einmal basiert das Mapping einer Speicheradresse auf ein Cache-Set auf physischen Adressen.
%Diese folgende Zuordnung gilt für die in dieser Arbeit angegriffene L3-Cache der Intel Core-Architektur.
%In den ersten Zeile sieht man, wie Adressen Cache-Sets zugeordnet werden. Eine Cache-Line hat 64 Byte, d.h. Bit 0 bis 5 sind der Cache-Line Offset. 
%Die 11 Bit 6 bis 16 bestimmen nun in welchem Cache-Set die Adresse landet.
%Des Weiteren ist jedem physischen CPU-Kern ein Teil des L3-Caches auch Slice genannt zugeordnet,
%Und die Bits 17 bis 63 der Adresse bestimmen in welchem Slice die Adresse gemappt wird, wobei dies mit einer Hashfunktion passiert.
%Aus Programmsicht sind aber nur die Virtuellen Adressen sichtbar bei denen nur die letzten 12 Bits der Adressen mit der physischen übereinstimmen, die sogenannten Page Bits.
%Im Webkontext kommt noch zusätzlich eine Abstraktionsschicht durch die eigene Adressierung von Web-Assembly hinzu.
%Hier sind aber auch die letzten 12 Adressbits identisch mit der physischen Adresse.
%Mit diesen Voraussetzungen sollen jetzt Eviction-Sets gefunden werden.
%Hierfür benötigt man einen Adresspool, den man durch Allokation eines großen Buffers erhält.
%In dem Adresspool landen nun nur Adressen der letzten 12 Bit gleich z.b. alle Adressen deren letzten 12 Bits 0 sind.
%Ziel ist es eine Eviction-Set zu finden indem alle Einträge auf das gleich Cache-Set gemappt werden.
%Somit landen im Pool nur Adressen deren physischen Adressbits 6 bis 11 bereits gleich sind.
%Um jetzt ein Eviction-Set zu finden wird eine beliebige Adresse aus dem Pool als Zeugenadresse gewählt.
%Die soll nun im Folgenden aus dem Cache durch ein Eviction-Set verdrängt werden.
%Es wird nun so eine Menge gebaut der wiederholt eine Adresse aus dem Pool hinzugefügt wird.
%Irgendwann wird diese Menge ein Eviction-Set für den Zeugen.
%Diese Menge hat dann in der Regel mehrere hundert Einträge und muss abschließend noch verkleinert werden, aber so lassen sich in dieser Umgebung Eviction-Sets finden.
%Wir können dank der Pages garantieren, dass die letzten 12 Bits der Adresse übereinstimmen.
%Bersser wäre es natürlich wenn beispielweise Adressen hätten, wo wir wüssten, dass die letzten 18 Bit gleich sind.
%Wie in Zeile 1 zu sehen wäre dann nur noch die Slice unbekannt und die vorhin beschriebene Eviction-Set Suche könnte deutlich beschleunigt werden, da der Adresspool deutlich kleiner ist.


%------------------------------------------------
\subsection{Ergebnisse}
\begin{frame}
\begin{itemize}
\item Nutzer besucht Website für wenige Minuten
\item Reduzierung der Suchzeit (Median) durch optimierten Standardalgorithmus: 46 s $\rightarrow$ 35 s
\item Reduzierung der Suchzeit (Median) durch neuen Ansatz:\\ 35 s $\rightarrow$ 11 s
\item Standardalgorithmus hat Ausreißer von über 100 Sekunden
\item Durch Parallelisierung weitere Verbesserung möglich
%suche nach colliding-addresses und eviction-set suche kann nebenläufig geschehen, Standardalgorithmus nicht parallelsierbar
\end{itemize}
\frametitle{Beschleunigung der Initialisierung}

\end{frame}

%Ohne die Eviction-Sets kann keine Angriff stattfinden, d.h. die Suche sollte nicht zu lange dauern.
%Insbesondere da bei einem Webangriff davon auszugehen ist, das ein Nutzer eine Website mit einer bösartigen Werbeanzeige nicht länger als ein paar Minuten besucht.
%Deshalb ist es interssant die Eviction-Set-Suche möglichst kurz zu halten.
%Der Standardsuchalgorithmus, der nur auf Adressen zurückgreifen kann deren letzten 12 Bits gleich sind braucht in der optimierten Version etwas 35 sekunden um alle eviciton-Sets zu finden.
%Der Ansatz mit den colliding-addresses kann die Suchzeit auf 11sekunden reudzieren.
%Dies sind median-werte, da der Standardalgo Außreißer von über 200 Sekunden hat. In der Praxis würde man einfach nach 100 Sekunden abbrechen und die Suche erneut starten.
%der neue algoritmus hat weniger Probleme mit den Außreißern, dort hat der längste Lauf von 37 Sekunden.

%------------------------------------------------
\section{Portierung eines nativen Angriffs}
%------------------------------------------------
\subsection{RSA-Primzahl-generierung}

\begin{frame}
\frametitle{Portierung eines Angriffs auf OpenSSL}
\begin{itemize}
\item Nativer Angriff auf die Berechnung von $gcd(e,p-1)$ bzw. $gcd(e,q-1)$ in OpenSSL als Grundlage
%sonst kein modulares inverses
\item Primzahlrekonstruktion durch Überwachung von Shift- und Subtraktionsoperationen
%bestimme Anzahl von Shift-Operationen zwischen zwei Subtraktionsoperationen
\item Leakage auch in Mozilla NSS ausnutzbar
%als Browserangriff
\item Problem: Ausführung einer Prime-and-Probe Operation bei Aktivität etwa 1500 Taktzyklen
\end{itemize}
\begin{table}[H]
\caption{Taktzyklendauer der Shift- und Subtraktionsoperation}
\label{tbl:ShiftSubCycles}
\begin{tabular}{ccc}
\toprule
Bitlänge & Shift & Subtraktion  \\
\midrule
2048     & 340   & 477                                     \\
4096     & 577   & 718                                       \\
%8192     & 959  & 1223                                     \\
\bottomrule
\end{tabular}
\end{table}

\end{frame}

%Im folgenden möchte ich die Portierung eines nativen Angriffs anhand eines Angriffs auf OpenSSL zeigen.
%In diesem Angriff wird die Schlüsselerzeugung von OpenSSL angegriffen.
%Hierzu werden leakages in der Berechnung von gcd(e,p-1) und gcd(e,q-1) ausgenutzt.
%Diese Prüfungen sind nach der Generierung der Primzahlen p und q notwendig um sicherzustellen, dass das die RSA Entschlüsselung und Verschlüsselung richtig funktioniert.
%Es gibt eine optimerierte gcd-Fundktion die nur mit Shift- und Subtraktionsoperationen auskommt, d.h. wenn die Shift- und Subtraktionsfunktionen überwacht werden können die Primzahlen p und q rekonstruiert werden.
%Diese Leakage ist nicht nur in OpenSSL sondern auch in MozillaNSS vorhanden.
%Jedoch ist die Ausführung der Abfolge der Shift-und Subtraktionen zu schnell um mittels der Prime-and-Probe Variante im Browser überwacht zu werden.
%Denn eine Ausführung einer Prime-and-Probe-Operation bei AKtivität auf dem Cache-Set dauert etwa 1500 Taktzyklen.
%In der Tabelle sieht man, dass bei dem heute gängigen RSA-2048 340 Taktzyklen für die Shift und 477 Taktzyklen für die Subtraktion benötigt werden.

%------------------------------------------------
\subsection{Bremsen von Code}
\begin{frame}
\frametitle{Bremsen von Code im Browser}
\begin{itemize}
\item clflush-Instruktion im Browser nicht verfügbar
\item Prime-and-Probe Operation ohne Zeitmessung ausführen
\item Welcher Code-Teil soll gebremst werden?
\end{itemize}

\tiny
\begin{figure}[H]
\begin{scaletikzpicturetowidth}{\textwidth}
\begin{tikzpicture}[level distance=1.5cm,
  level 1/.style={sibling distance=2cm},
  level 2/.style={sibling distance=1.5cm},
  grow=right,sibling distance        = 3em,
    level distance          = 10em]
  \node {\textit{mp_gcd}}
    child {node {\textit{s_mp_div_2}}
      child {node {\textit{s_mp_div_2d}}
        child {node {\textit{s_mp_rshd}}}
        child {node {\textit{s_mp_clamp}}}
      }
    }
    child {node {\textit{mp_sub}}
    child {node {\textit{s_mp_sub_3arg}}
        child {node {\textit{s_mp_clamp}}}
      }
    };
\end{tikzpicture}
\end{scaletikzpicturetowidth}
\end{figure}

\normalsize
\begin{itemize}

\end{itemize}
\end{frame}

%Bei nativen Angriffen wird auch häufig der Code gebremst, wobei dabei in der Regel die clflush-Instruktion verwendet wird.
%Diese ist im Browser aber nicht verfügbar, sodass auf andere Mittel zurückgegriffen werden muss.
%Ziel ist den Code der Shift- bzw. Subtraktionsoperation aus dem Cache zu verdrängen. 
%Dazu eignen sich die Eviction-Sets die auch im Prime-and-Probe Angriff verwendet werden.
%Somit kann eine Prime-and-Probe Operation ohne Zeitmessung verwendet werden, um den Code zu bremsen.
%Da die Zahlen bei RSA-2048 nicht in Register passen, sind die FUnktion für die Shift- bzw. Subtraktion deutlich aufwendiger.
%Im der Grafik kann man sehen welche Unterfunktionen bei der Ausführung der SHift- bzw. Subtraktion aufgerufen werden.
%Interessasnt ist hierbei insbesonderer die s_mp_clamp Funktion, da diese von der Shift und der Subtraktion verwendet wird.
%die hier gezeigten funktionen wie z.b. die s_mp_clamp funktion sind wiederrum größer als 64 Byte oder liegen nicht direkt an einer Cache-Line Grenze, sodass die s_mp_clamp funktion etwa in zwei verschiedenen Cahce-lines liegt.

%------------------------------------------------
\subsection{Ergebnisse}
\begin{frame}
\frametitle{Ergebnisse}


\small
\begin{table}[h]
\label{tbl:PerformanceDegShift}
\begin{tabular}{cclccc}
\toprule
Funktion & Shift \\
\midrule
Referenz      & 340          \\
$*clamp_1$  & 541           \\
$*clamp_1,*rshd_4$   & 643     \\
$*clamp_1,*rshd_4,*div\_2d_3$ & 623    \\
\midrule
$*clamp_1$ \& $*clamp_1$ & 623     \\
$*clamp_1$ \& $*rshd_4$  & 959      \\
$*clamp_1$ \& $*rshd_4$,$*div\_2d_3$ & 880        \\
$*clamp_1$,$*rshd_4$ \& $*clamp_2$,$*div\_2d_3$ & 745   \\
$*clamp_1$ \& $*rshd_4$ \& $*div\_2d_1$   & 1017  \\
$*clamp_{1,2}, *rshd_4, *div\_2d_{1,3,4}$ (6 Threads) & 1109 \\
\bottomrule
\end{tabular}
\end{table}
%letzten beiden zeilen nicht wirklich besser da über 4 threads verwendet werden

\normalsize
\begin{itemize}
\item Opferprogramm, Timer, Messung jeweils einen Thread
\item Bremsung auf 4c/8t CPU mit nur einem physischen Kern
\end{itemize}

\end{frame}

%------------------------------------------------
\subsection{Bewertung}
\begin{frame}
\frametitle{Bewertung}

\begin{itemize}
\item Überwachung der Folge 4s2s1s5s2s1 in einem Thread
%Messung unterschiedlicher Cache-Lines zeigt keinen signifikanten Untreschied
\end{itemize}

\tiny
\begin{figure}[H]
\centering
\begin{scaletikzpicturetowidth}{\textwidth}
% Created by tikzDevice version 0.12 on 2018-10-06 03:06:08
% !TEX encoding = UTF-8 Unicode
\begin{tikzpicture}[x=0.6pt,y=0.4pt]
\definecolor{fillColor}{RGB}{255,255,255}
\path[use as bounding box,fill=fillColor,fill opacity=0.00] (0,0) rectangle (432.17,289.08);
\begin{scope}
\path[clip] (  0.00,  0.00) rectangle ( 82.92,289.08);
\definecolor{drawColor}{RGB}{255,255,255}
\definecolor{fillColor}{RGB}{255,255,255}

\path[draw=drawColor,line width= 0.6pt,line join=round,line cap=round,fill=fillColor] (  0.00,  0.00) rectangle ( 82.92,289.08);
\end{scope}
\begin{scope}
\path[clip] ( 82.92,  0.00) rectangle (170.23,289.08);
\definecolor{drawColor}{RGB}{255,255,255}
\definecolor{fillColor}{RGB}{255,255,255}

\path[draw=drawColor,line width= 0.6pt,line join=round,line cap=round,fill=fillColor] ( 82.92,  0.00) rectangle (170.23,289.08);
\end{scope}
\begin{scope}
\path[clip] (170.23,  0.00) rectangle (257.55,289.08);
\definecolor{drawColor}{RGB}{255,255,255}
\definecolor{fillColor}{RGB}{255,255,255}

\path[draw=drawColor,line width= 0.6pt,line join=round,line cap=round,fill=fillColor] (170.23,  0.00) rectangle (257.55,289.08);
\end{scope}
\begin{scope}
\path[clip] (257.55,  0.00) rectangle (344.86,289.08);
\definecolor{drawColor}{RGB}{255,255,255}
\definecolor{fillColor}{RGB}{255,255,255}

\path[draw=drawColor,line width= 0.6pt,line join=round,line cap=round,fill=fillColor] (257.55,  0.00) rectangle (344.86,289.08);
\end{scope}
\begin{scope}
\path[clip] (344.86,  0.00) rectangle (432.17,289.08);
\definecolor{drawColor}{RGB}{255,255,255}
\definecolor{fillColor}{RGB}{255,255,255}

\path[draw=drawColor,line width= 0.6pt,line join=round,line cap=round,fill=fillColor] (344.86,  0.00) rectangle (432.17,289.08);
\end{scope}
\begin{scope}
\path[clip] ( 14.85, 16.51) rectangle ( 77.42,269.49);
\definecolor{drawColor}{RGB}{255,255,255}

\path[draw=drawColor,line width= 0.3pt,line join=round] ( 14.85, 28.01) --
	( 77.42, 28.01);

\path[draw=drawColor,line width= 0.3pt,line join=round] ( 14.85, 39.51) --
	( 77.42, 39.51);

\path[draw=drawColor,line width= 0.3pt,line join=round] ( 14.85, 62.51) --
	( 77.42, 62.51);

\path[draw=drawColor,line width= 0.3pt,line join=round] ( 14.85, 85.50) --
	( 77.42, 85.50);

\path[draw=drawColor,line width= 0.3pt,line join=round] ( 14.85,108.50) --
	( 77.42,108.50);

\path[draw=drawColor,line width= 0.3pt,line join=round] ( 14.85,131.50) --
	( 77.42,131.50);

\path[draw=drawColor,line width= 0.3pt,line join=round] ( 14.85,154.50) --
	( 77.42,154.50);

\path[draw=drawColor,line width= 0.3pt,line join=round] ( 14.85,177.50) --
	( 77.42,177.50);

\path[draw=drawColor,line width= 0.3pt,line join=round] ( 14.85,200.49) --
	( 77.42,200.49);

\path[draw=drawColor,line width= 0.3pt,line join=round] ( 14.85,223.49) --
	( 77.42,223.49);

\path[draw=drawColor,line width= 0.3pt,line join=round] ( 14.85,246.49) --
	( 77.42,246.49);

\path[draw=drawColor,line width= 0.3pt,line join=round] ( 14.85,257.99) --
	( 77.42,257.99);

\path[draw=drawColor,line width= 0.6pt,line join=round] ( 14.85, 51.01) --
	( 77.42, 51.01);

\path[draw=drawColor,line width= 0.6pt,line join=round] ( 14.85, 74.01) --
	( 77.42, 74.01);

\path[draw=drawColor,line width= 0.6pt,line join=round] ( 14.85, 97.00) --
	( 77.42, 97.00);

\path[draw=drawColor,line width= 0.6pt,line join=round] ( 14.85,120.00) --
	( 77.42,120.00);

\path[draw=drawColor,line width= 0.6pt,line join=round] ( 14.85,143.00) --
	( 77.42,143.00);

\path[draw=drawColor,line width= 0.6pt,line join=round] ( 14.85,166.00) --
	( 77.42,166.00);

\path[draw=drawColor,line width= 0.6pt,line join=round] ( 14.85,189.00) --
	( 77.42,189.00);

\path[draw=drawColor,line width= 0.6pt,line join=round] ( 14.85,211.99) --
	( 77.42,211.99);

\path[draw=drawColor,line width= 0.6pt,line join=round] ( 14.85,234.99) --
	( 77.42,234.99);

\path[draw=drawColor,line width= 0.6pt,line join=round] ( 31.91, 16.51) --
	( 31.91,269.49);

\path[draw=drawColor,line width= 0.6pt,line join=round] ( 60.35, 16.51) --
	( 60.35,269.49);
\definecolor{fillColor}{RGB}{255,255,255}

\path[fill=fillColor] ( 17.69, 39.51) rectangle ( 46.13, 62.51);
\definecolor{fillColor}{gray}{0.59}

\path[fill=fillColor] ( 17.69, 62.51) rectangle ( 46.13, 85.50);
\definecolor{fillColor}{RGB}{211,211,211}

\path[fill=fillColor] ( 17.69, 85.50) rectangle ( 46.13,108.50);
\definecolor{fillColor}{RGB}{210,210,210}

\path[fill=fillColor] ( 17.69,108.50) rectangle ( 46.13,131.50);
\definecolor{fillColor}{RGB}{208,208,208}

\path[fill=fillColor] ( 17.69,131.50) rectangle ( 46.13,154.50);
\definecolor{fillColor}{RGB}{155,155,155}

\path[fill=fillColor] ( 17.69,154.50) rectangle ( 46.13,177.50);
\definecolor{fillColor}{gray}{0.52}

\path[fill=fillColor] ( 17.69,177.50) rectangle ( 46.13,200.49);
\definecolor{fillColor}{RGB}{37,37,37}

\path[fill=fillColor] ( 17.69,200.49) rectangle ( 46.13,223.49);
\definecolor{fillColor}{RGB}{119,119,119}

\path[fill=fillColor] ( 17.69,223.49) rectangle ( 46.13,246.49);
\definecolor{fillColor}{gray}{0.22}

\path[fill=fillColor] ( 46.13, 39.51) rectangle ( 74.57, 62.51);
\definecolor{fillColor}{gray}{0.60}

\path[fill=fillColor] ( 46.13, 62.51) rectangle ( 74.57, 85.50);
\definecolor{fillColor}{RGB}{58,58,58}

\path[fill=fillColor] ( 46.13, 85.50) rectangle ( 74.57,108.50);
\definecolor{fillColor}{gray}{0.66}

\path[fill=fillColor] ( 46.13,108.50) rectangle ( 74.57,131.50);
\definecolor{fillColor}{gray}{0.22}

\path[fill=fillColor] ( 46.13,131.50) rectangle ( 74.57,154.50);
\definecolor{fillColor}{RGB}{215,215,215}

\path[fill=fillColor] ( 46.13,154.50) rectangle ( 74.57,177.50);
\definecolor{fillColor}{RGB}{37,37,37}

\path[fill=fillColor] ( 46.13,177.50) rectangle ( 74.57,200.49);
\definecolor{fillColor}{RGB}{211,211,211}

\path[fill=fillColor] ( 46.13,200.49) rectangle ( 74.57,223.49);

\path[fill=fillColor] ( 46.13,223.49) rectangle ( 74.57,246.49);
\end{scope}
\begin{scope}
\path[clip] (102.16, 16.51) rectangle (164.73,269.49);
\definecolor{drawColor}{RGB}{255,255,255}

\path[draw=drawColor,line width= 0.3pt,line join=round] (102.16, 28.01) --
	(164.73, 28.01);

\path[draw=drawColor,line width= 0.3pt,line join=round] (102.16, 55.07) --
	(164.73, 55.07);

\path[draw=drawColor,line width= 0.3pt,line join=round] (102.16, 82.12) --
	(164.73, 82.12);

\path[draw=drawColor,line width= 0.3pt,line join=round] (102.16,109.18) --
	(164.73,109.18);

\path[draw=drawColor,line width= 0.3pt,line join=round] (102.16,136.24) --
	(164.73,136.24);

\path[draw=drawColor,line width= 0.3pt,line join=round] (102.16,163.29) --
	(164.73,163.29);

\path[draw=drawColor,line width= 0.3pt,line join=round] (102.16,190.35) --
	(164.73,190.35);

\path[draw=drawColor,line width= 0.3pt,line join=round] (102.16,217.41) --
	(164.73,217.41);

\path[draw=drawColor,line width= 0.3pt,line join=round] (102.16,244.46) --
	(164.73,244.46);

\path[draw=drawColor,line width= 0.3pt,line join=round] (102.16,257.99) --
	(164.73,257.99);

\path[draw=drawColor,line width= 0.6pt,line join=round] (102.16, 41.54) --
	(164.73, 41.54);

\path[draw=drawColor,line width= 0.6pt,line join=round] (102.16, 68.59) --
	(164.73, 68.59);

\path[draw=drawColor,line width= 0.6pt,line join=round] (102.16, 95.65) --
	(164.73, 95.65);

\path[draw=drawColor,line width= 0.6pt,line join=round] (102.16,122.71) --
	(164.73,122.71);

\path[draw=drawColor,line width= 0.6pt,line join=round] (102.16,149.76) --
	(164.73,149.76);

\path[draw=drawColor,line width= 0.6pt,line join=round] (102.16,176.82) --
	(164.73,176.82);

\path[draw=drawColor,line width= 0.6pt,line join=round] (102.16,203.88) --
	(164.73,203.88);

\path[draw=drawColor,line width= 0.6pt,line join=round] (102.16,230.93) --
	(164.73,230.93);

\path[draw=drawColor,line width= 0.6pt,line join=round] (119.23, 16.51) --
	(119.23,269.49);

\path[draw=drawColor,line width= 0.6pt,line join=round] (147.67, 16.51) --
	(147.67,269.49);
\definecolor{fillColor}{gray}{0.94}

\path[fill=fillColor] (105.01, 34.77) rectangle (133.45, 48.30);
\definecolor{fillColor}{RGB}{239,239,239}

\path[fill=fillColor] (105.01, 48.30) rectangle (133.45, 61.83);
\definecolor{fillColor}{RGB}{238,238,238}

\path[fill=fillColor] (105.01, 61.83) rectangle (133.45, 75.36);
\definecolor{fillColor}{RGB}{239,239,239}

\path[fill=fillColor] (105.01, 75.36) rectangle (133.45, 88.89);
\definecolor{fillColor}{RGB}{111,111,111}

\path[fill=fillColor] (105.01, 88.89) rectangle (133.45,102.42);
\definecolor{fillColor}{RGB}{62,62,62}

\path[fill=fillColor] (105.01,102.42) rectangle (133.45,115.94);
\definecolor{fillColor}{gray}{0.65}

\path[fill=fillColor] (105.01,115.94) rectangle (133.45,129.47);
\definecolor{fillColor}{RGB}{37,37,37}

\path[fill=fillColor] (105.01,129.47) rectangle (133.45,143.00);
\definecolor{fillColor}{RGB}{60,60,60}

\path[fill=fillColor] (105.01,143.00) rectangle (133.45,156.53);
\definecolor{fillColor}{RGB}{241,241,241}

\path[fill=fillColor] (105.01,156.53) rectangle (133.45,170.06);
\definecolor{fillColor}{RGB}{239,239,239}

\path[fill=fillColor] (105.01,170.06) rectangle (133.45,183.58);
\definecolor{fillColor}{gray}{0.94}

\path[fill=fillColor] (105.01,183.58) rectangle (133.45,197.11);
\definecolor{fillColor}{RGB}{165,165,165}

\path[fill=fillColor] (105.01,197.11) rectangle (133.45,210.64);
\definecolor{fillColor}{RGB}{62,62,62}

\path[fill=fillColor] (105.01,210.64) rectangle (133.45,224.17);

\path[fill=fillColor] (105.01,224.17) rectangle (133.45,237.70);
\definecolor{fillColor}{RGB}{109,109,109}

\path[fill=fillColor] (105.01,237.70) rectangle (133.45,251.23);
\definecolor{fillColor}{RGB}{213,213,213}

\path[fill=fillColor] (133.45, 34.77) rectangle (161.89, 48.30);
\definecolor{fillColor}{RGB}{197,197,197}

\path[fill=fillColor] (133.45, 48.30) rectangle (161.89, 61.83);
\definecolor{fillColor}{gray}{0.66}

\path[fill=fillColor] (133.45, 61.83) rectangle (161.89, 75.36);

\path[fill=fillColor] (133.45, 75.36) rectangle (161.89, 88.89);
\definecolor{fillColor}{gray}{0.89}

\path[fill=fillColor] (133.45, 88.89) rectangle (161.89,102.42);
\definecolor{fillColor}{RGB}{188,188,188}

\path[fill=fillColor] (133.45,102.42) rectangle (161.89,115.94);
\definecolor{fillColor}{RGB}{183,183,183}

\path[fill=fillColor] (133.45,115.94) rectangle (161.89,129.47);
\definecolor{fillColor}{gray}{0.86}

\path[fill=fillColor] (133.45,129.47) rectangle (161.89,143.00);
\definecolor{fillColor}{RGB}{37,37,37}

\path[fill=fillColor] (133.45,143.00) rectangle (161.89,156.53);
\definecolor{fillColor}{RGB}{230,230,230}

\path[fill=fillColor] (133.45,156.53) rectangle (161.89,170.06);
\definecolor{fillColor}{gray}{0.65}

\path[fill=fillColor] (133.45,170.06) rectangle (161.89,183.58);
\definecolor{fillColor}{RGB}{197,197,197}

\path[fill=fillColor] (133.45,183.58) rectangle (161.89,197.11);
\definecolor{fillColor}{RGB}{195,195,195}

\path[fill=fillColor] (133.45,197.11) rectangle (161.89,210.64);
\definecolor{fillColor}{gray}{0.77}

\path[fill=fillColor] (133.45,210.64) rectangle (161.89,224.17);
\definecolor{fillColor}{gray}{0.85}

\path[fill=fillColor] (133.45,224.17) rectangle (161.89,237.70);
\definecolor{fillColor}{RGB}{216,216,216}

\path[fill=fillColor] (133.45,237.70) rectangle (161.89,251.23);
\end{scope}
\begin{scope}
\path[clip] (189.48, 16.51) rectangle (252.05,269.49);
\definecolor{drawColor}{RGB}{255,255,255}

\path[draw=drawColor,line width= 0.3pt,line join=round] (189.48, 21.62) --
	(252.05, 21.62);

\path[draw=drawColor,line width= 0.3pt,line join=round] (189.48, 47.17) --
	(252.05, 47.17);

\path[draw=drawColor,line width= 0.3pt,line join=round] (189.48, 72.73) --
	(252.05, 72.73);

\path[draw=drawColor,line width= 0.3pt,line join=round] (189.48, 98.28) --
	(252.05, 98.28);

\path[draw=drawColor,line width= 0.3pt,line join=round] (189.48,123.83) --
	(252.05,123.83);

\path[draw=drawColor,line width= 0.3pt,line join=round] (189.48,149.39) --
	(252.05,149.39);

\path[draw=drawColor,line width= 0.3pt,line join=round] (189.48,174.94) --
	(252.05,174.94);

\path[draw=drawColor,line width= 0.3pt,line join=round] (189.48,200.49) --
	(252.05,200.49);

\path[draw=drawColor,line width= 0.3pt,line join=round] (189.48,226.05) --
	(252.05,226.05);

\path[draw=drawColor,line width= 0.3pt,line join=round] (189.48,251.60) --
	(252.05,251.60);

\path[draw=drawColor,line width= 0.3pt,line join=round] (189.48,264.38) --
	(252.05,264.38);

\path[draw=drawColor,line width= 0.6pt,line join=round] (189.48, 34.40) --
	(252.05, 34.40);

\path[draw=drawColor,line width= 0.6pt,line join=round] (189.48, 59.95) --
	(252.05, 59.95);

\path[draw=drawColor,line width= 0.6pt,line join=round] (189.48, 85.50) --
	(252.05, 85.50);

\path[draw=drawColor,line width= 0.6pt,line join=round] (189.48,111.06) --
	(252.05,111.06);

\path[draw=drawColor,line width= 0.6pt,line join=round] (189.48,136.61) --
	(252.05,136.61);

\path[draw=drawColor,line width= 0.6pt,line join=round] (189.48,162.16) --
	(252.05,162.16);

\path[draw=drawColor,line width= 0.6pt,line join=round] (189.48,187.72) --
	(252.05,187.72);

\path[draw=drawColor,line width= 0.6pt,line join=round] (189.48,213.27) --
	(252.05,213.27);

\path[draw=drawColor,line width= 0.6pt,line join=round] (189.48,238.82) --
	(252.05,238.82);

\path[draw=drawColor,line width= 0.6pt,line join=round] (206.54, 16.51) --
	(206.54,269.49);

\path[draw=drawColor,line width= 0.6pt,line join=round] (234.98, 16.51) --
	(234.98,269.49);
\definecolor{fillColor}{RGB}{255,255,255}

\path[fill=fillColor] (192.32, 31.20) rectangle (220.76, 37.59);

\path[fill=fillColor] (192.32, 37.59) rectangle (220.76, 43.98);

\path[fill=fillColor] (192.32, 43.98) rectangle (220.76, 50.37);

\path[fill=fillColor] (192.32, 50.37) rectangle (220.76, 56.76);

\path[fill=fillColor] (192.32, 56.76) rectangle (220.76, 63.15);

\path[fill=fillColor] (192.32, 63.15) rectangle (220.76, 69.53);

\path[fill=fillColor] (192.32, 69.53) rectangle (220.76, 75.92);

\path[fill=fillColor] (192.32, 75.92) rectangle (220.76, 82.31);
\definecolor{fillColor}{gray}{0.55}

\path[fill=fillColor] (192.32, 82.31) rectangle (220.76, 88.70);
\definecolor{fillColor}{RGB}{159,159,159}

\path[fill=fillColor] (192.32, 88.70) rectangle (220.76, 95.09);
\definecolor{fillColor}{RGB}{255,255,255}

\path[fill=fillColor] (192.32, 95.09) rectangle (220.76,101.48);

\path[fill=fillColor] (192.32,101.48) rectangle (220.76,107.86);

\path[fill=fillColor] (192.32,107.86) rectangle (220.76,114.25);
\definecolor{fillColor}{gray}{0.91}

\path[fill=fillColor] (192.32,114.25) rectangle (220.76,120.64);
\definecolor{fillColor}{RGB}{144,144,144}

\path[fill=fillColor] (192.32,120.64) rectangle (220.76,127.03);
\definecolor{fillColor}{RGB}{255,255,255}

\path[fill=fillColor] (192.32,127.03) rectangle (220.76,133.42);

\path[fill=fillColor] (192.32,133.42) rectangle (220.76,139.81);
\definecolor{fillColor}{gray}{0.40}

\path[fill=fillColor] (192.32,139.81) rectangle (220.76,146.19);
\definecolor{fillColor}{RGB}{108,108,108}

\path[fill=fillColor] (192.32,146.19) rectangle (220.76,152.58);
\definecolor{fillColor}{RGB}{255,255,255}

\path[fill=fillColor] (192.32,152.58) rectangle (220.76,158.97);

\path[fill=fillColor] (192.32,158.97) rectangle (220.76,165.36);

\path[fill=fillColor] (192.32,165.36) rectangle (220.76,171.75);

\path[fill=fillColor] (192.32,171.75) rectangle (220.76,178.14);

\path[fill=fillColor] (192.32,178.14) rectangle (220.76,184.52);

\path[fill=fillColor] (192.32,184.52) rectangle (220.76,190.91);

\path[fill=fillColor] (192.32,190.91) rectangle (220.76,197.30);

\path[fill=fillColor] (192.32,197.30) rectangle (220.76,203.69);
\definecolor{fillColor}{RGB}{180,180,180}

\path[fill=fillColor] (192.32,203.69) rectangle (220.76,210.08);
\definecolor{fillColor}{gray}{0.77}

\path[fill=fillColor] (192.32,210.08) rectangle (220.76,216.47);
\definecolor{fillColor}{RGB}{255,255,255}

\path[fill=fillColor] (192.32,216.47) rectangle (220.76,222.85);

\path[fill=fillColor] (192.32,222.85) rectangle (220.76,229.24);
\definecolor{fillColor}{RGB}{63,63,63}

\path[fill=fillColor] (192.32,229.24) rectangle (220.76,235.63);
\definecolor{fillColor}{RGB}{104,104,104}

\path[fill=fillColor] (192.32,235.63) rectangle (220.76,242.02);
\definecolor{fillColor}{RGB}{37,37,37}

\path[fill=fillColor] (192.32,242.02) rectangle (220.76,248.41);
\definecolor{fillColor}{RGB}{144,144,144}

\path[fill=fillColor] (192.32,248.41) rectangle (220.76,254.80);
\definecolor{fillColor}{RGB}{126,126,126}

\path[fill=fillColor] (220.76, 31.20) rectangle (249.20, 37.59);
\definecolor{fillColor}{RGB}{37,37,37}

\path[fill=fillColor] (220.76, 37.59) rectangle (249.20, 43.98);
\definecolor{fillColor}{RGB}{255,255,255}

\path[fill=fillColor] (220.76, 43.98) rectangle (249.20, 50.37);
\definecolor{fillColor}{RGB}{221,221,221}

\path[fill=fillColor] (220.76, 50.37) rectangle (249.20, 56.76);
\definecolor{fillColor}{RGB}{126,126,126}

\path[fill=fillColor] (220.76, 56.76) rectangle (249.20, 63.15);
\definecolor{fillColor}{RGB}{39,39,39}

\path[fill=fillColor] (220.76, 63.15) rectangle (249.20, 69.53);
\definecolor{fillColor}{RGB}{255,255,255}

\path[fill=fillColor] (220.76, 69.53) rectangle (249.20, 75.92);

\path[fill=fillColor] (220.76, 75.92) rectangle (249.20, 82.31);

\path[fill=fillColor] (220.76, 82.31) rectangle (249.20, 88.70);
\definecolor{fillColor}{gray}{0.27}

\path[fill=fillColor] (220.76, 88.70) rectangle (249.20, 95.09);
\definecolor{fillColor}{RGB}{239,239,239}

\path[fill=fillColor] (220.76, 95.09) rectangle (249.20,101.48);
\definecolor{fillColor}{RGB}{255,255,255}

\path[fill=fillColor] (220.76,101.48) rectangle (249.20,107.86);

\path[fill=fillColor] (220.76,107.86) rectangle (249.20,114.25);

\path[fill=fillColor] (220.76,114.25) rectangle (249.20,120.64);
\definecolor{fillColor}{RGB}{195,195,195}

\path[fill=fillColor] (220.76,120.64) rectangle (249.20,127.03);
\definecolor{fillColor}{RGB}{121,121,121}

\path[fill=fillColor] (220.76,127.03) rectangle (249.20,133.42);
\definecolor{fillColor}{RGB}{255,255,255}

\path[fill=fillColor] (220.76,133.42) rectangle (249.20,139.81);

\path[fill=fillColor] (220.76,139.81) rectangle (249.20,146.19);
\definecolor{fillColor}{RGB}{88,88,88}

\path[fill=fillColor] (220.76,146.19) rectangle (249.20,152.58);
\definecolor{fillColor}{RGB}{160,160,160}

\path[fill=fillColor] (220.76,152.58) rectangle (249.20,158.97);
\definecolor{fillColor}{RGB}{255,255,255}

\path[fill=fillColor] (220.76,158.97) rectangle (249.20,165.36);
\definecolor{fillColor}{RGB}{85,85,85}

\path[fill=fillColor] (220.76,165.36) rectangle (249.20,171.75);
\definecolor{fillColor}{RGB}{160,160,160}

\path[fill=fillColor] (220.76,171.75) rectangle (249.20,178.14);
\definecolor{fillColor}{RGB}{255,255,255}

\path[fill=fillColor] (220.76,178.14) rectangle (249.20,184.52);

\path[fill=fillColor] (220.76,184.52) rectangle (249.20,190.91);
\definecolor{fillColor}{RGB}{226,226,226}

\path[fill=fillColor] (220.76,190.91) rectangle (249.20,197.30);
\definecolor{fillColor}{gray}{0.51}

\path[fill=fillColor] (220.76,197.30) rectangle (249.20,203.69);
\definecolor{fillColor}{gray}{0.27}

\path[fill=fillColor] (220.76,203.69) rectangle (249.20,210.08);
\definecolor{fillColor}{gray}{0.74}

\path[fill=fillColor] (220.76,210.08) rectangle (249.20,216.47);
\definecolor{fillColor}{RGB}{202,202,202}

\path[fill=fillColor] (220.76,216.47) rectangle (249.20,222.85);
\definecolor{fillColor}{gray}{0.27}

\path[fill=fillColor] (220.76,222.85) rectangle (249.20,229.24);

\path[fill=fillColor] (220.76,229.24) rectangle (249.20,235.63);
\definecolor{fillColor}{RGB}{195,195,195}

\path[fill=fillColor] (220.76,235.63) rectangle (249.20,242.02);
\definecolor{fillColor}{gray}{0.29}

\path[fill=fillColor] (220.76,242.02) rectangle (249.20,248.41);
\definecolor{fillColor}{gray}{0.27}

\path[fill=fillColor] (220.76,248.41) rectangle (249.20,254.80);
\end{scope}
\begin{scope}
\path[clip] (276.79, 16.51) rectangle (339.36,269.49);
\definecolor{drawColor}{RGB}{255,255,255}

\path[draw=drawColor,line width= 0.3pt,line join=round] (276.79, 21.11) --
	(339.36, 21.11);

\path[draw=drawColor,line width= 0.3pt,line join=round] (276.79, 44.11) --
	(339.36, 44.11);

\path[draw=drawColor,line width= 0.3pt,line join=round] (276.79, 67.11) --
	(339.36, 67.11);

\path[draw=drawColor,line width= 0.3pt,line join=round] (276.79, 90.10) --
	(339.36, 90.10);

\path[draw=drawColor,line width= 0.3pt,line join=round] (276.79,113.10) --
	(339.36,113.10);

\path[draw=drawColor,line width= 0.3pt,line join=round] (276.79,136.10) --
	(339.36,136.10);

\path[draw=drawColor,line width= 0.3pt,line join=round] (276.79,159.10) --
	(339.36,159.10);

\path[draw=drawColor,line width= 0.3pt,line join=round] (276.79,182.10) --
	(339.36,182.10);

\path[draw=drawColor,line width= 0.3pt,line join=round] (276.79,205.09) --
	(339.36,205.09);

\path[draw=drawColor,line width= 0.3pt,line join=round] (276.79,228.09) --
	(339.36,228.09);

\path[draw=drawColor,line width= 0.3pt,line join=round] (276.79,251.09) --
	(339.36,251.09);

\path[draw=drawColor,line width= 0.3pt,line join=round] (276.79,262.59) --
	(339.36,262.59);

\path[draw=drawColor,line width= 0.6pt,line join=round] (276.79, 32.61) --
	(339.36, 32.61);

\path[draw=drawColor,line width= 0.6pt,line join=round] (276.79, 55.61) --
	(339.36, 55.61);

\path[draw=drawColor,line width= 0.6pt,line join=round] (276.79, 78.61) --
	(339.36, 78.61);

\path[draw=drawColor,line width= 0.6pt,line join=round] (276.79,101.60) --
	(339.36,101.60);

\path[draw=drawColor,line width= 0.6pt,line join=round] (276.79,124.60) --
	(339.36,124.60);

\path[draw=drawColor,line width= 0.6pt,line join=round] (276.79,147.60) --
	(339.36,147.60);

\path[draw=drawColor,line width= 0.6pt,line join=round] (276.79,170.60) --
	(339.36,170.60);

\path[draw=drawColor,line width= 0.6pt,line join=round] (276.79,193.60) --
	(339.36,193.60);

\path[draw=drawColor,line width= 0.6pt,line join=round] (276.79,216.59) --
	(339.36,216.59);

\path[draw=drawColor,line width= 0.6pt,line join=round] (276.79,239.59) --
	(339.36,239.59);

\path[draw=drawColor,line width= 0.6pt,line join=round] (293.86, 16.51) --
	(293.86,269.49);

\path[draw=drawColor,line width= 0.6pt,line join=round] (322.30, 16.51) --
	(322.30,269.49);
\definecolor{fillColor}{RGB}{255,255,255}

\path[fill=fillColor] (279.64, 30.31) rectangle (308.08, 34.91);

\path[fill=fillColor] (279.64, 34.91) rectangle (308.08, 39.51);

\path[fill=fillColor] (279.64, 39.51) rectangle (308.08, 44.11);

\path[fill=fillColor] (279.64, 44.11) rectangle (308.08, 48.71);

\path[fill=fillColor] (279.64, 48.71) rectangle (308.08, 53.31);

\path[fill=fillColor] (279.64, 53.31) rectangle (308.08, 57.91);

\path[fill=fillColor] (279.64, 57.91) rectangle (308.08, 62.51);

\path[fill=fillColor] (279.64, 62.51) rectangle (308.08, 67.11);

\path[fill=fillColor] (279.64, 67.11) rectangle (308.08, 71.71);

\path[fill=fillColor] (279.64, 71.71) rectangle (308.08, 76.31);
\definecolor{fillColor}{RGB}{100,100,100}

\path[fill=fillColor] (279.64, 76.31) rectangle (308.08, 80.91);
\definecolor{fillColor}{gray}{0.81}

\path[fill=fillColor] (279.64, 80.91) rectangle (308.08, 85.50);
\definecolor{fillColor}{RGB}{255,255,255}

\path[fill=fillColor] (279.64, 85.50) rectangle (308.08, 90.10);

\path[fill=fillColor] (279.64, 90.10) rectangle (308.08, 94.70);

\path[fill=fillColor] (279.64, 94.70) rectangle (308.08, 99.30);

\path[fill=fillColor] (279.64, 99.30) rectangle (308.08,103.90);

\path[fill=fillColor] (279.64,103.90) rectangle (308.08,108.50);

\path[fill=fillColor] (279.64,108.50) rectangle (308.08,113.10);
\definecolor{fillColor}{RGB}{142,142,142}

\path[fill=fillColor] (279.64,113.10) rectangle (308.08,117.70);
\definecolor{fillColor}{gray}{0.66}

\path[fill=fillColor] (279.64,117.70) rectangle (308.08,122.30);
\definecolor{fillColor}{RGB}{255,255,255}

\path[fill=fillColor] (279.64,122.30) rectangle (308.08,126.90);

\path[fill=fillColor] (279.64,126.90) rectangle (308.08,131.50);

\path[fill=fillColor] (279.64,131.50) rectangle (308.08,136.10);
\definecolor{fillColor}{RGB}{170,170,170}

\path[fill=fillColor] (279.64,136.10) rectangle (308.08,140.70);
\definecolor{fillColor}{gray}{0.69}

\path[fill=fillColor] (279.64,140.70) rectangle (308.08,145.30);
\definecolor{fillColor}{RGB}{255,255,255}

\path[fill=fillColor] (279.64,145.30) rectangle (308.08,149.90);

\path[fill=fillColor] (279.64,149.90) rectangle (308.08,154.50);

\path[fill=fillColor] (279.64,154.50) rectangle (308.08,159.10);

\path[fill=fillColor] (279.64,159.10) rectangle (308.08,163.70);

\path[fill=fillColor] (279.64,163.70) rectangle (308.08,168.30);

\path[fill=fillColor] (279.64,168.30) rectangle (308.08,172.90);

\path[fill=fillColor] (279.64,172.90) rectangle (308.08,177.50);

\path[fill=fillColor] (279.64,177.50) rectangle (308.08,182.10);

\path[fill=fillColor] (279.64,182.10) rectangle (308.08,186.70);

\path[fill=fillColor] (279.64,186.70) rectangle (308.08,191.30);

\path[fill=fillColor] (279.64,191.30) rectangle (308.08,195.90);
\definecolor{fillColor}{RGB}{37,37,37}

\path[fill=fillColor] (279.64,195.90) rectangle (308.08,200.49);
\definecolor{fillColor}{RGB}{113,113,113}

\path[fill=fillColor] (279.64,200.49) rectangle (308.08,205.09);
\definecolor{fillColor}{RGB}{255,255,255}

\path[fill=fillColor] (279.64,205.09) rectangle (308.08,209.69);

\path[fill=fillColor] (279.64,209.69) rectangle (308.08,214.29);

\path[fill=fillColor] (279.64,214.29) rectangle (308.08,218.89);

\path[fill=fillColor] (279.64,218.89) rectangle (308.08,223.49);

\path[fill=fillColor] (279.64,223.49) rectangle (308.08,228.09);
\definecolor{fillColor}{RGB}{105,105,105}

\path[fill=fillColor] (279.64,228.09) rectangle (308.08,232.69);
\definecolor{fillColor}{gray}{0.56}

\path[fill=fillColor] (279.64,232.69) rectangle (308.08,237.29);
\definecolor{fillColor}{RGB}{255,255,255}

\path[fill=fillColor] (279.64,237.29) rectangle (308.08,241.89);

\path[fill=fillColor] (279.64,241.89) rectangle (308.08,246.49);
\definecolor{fillColor}{RGB}{131,131,131}

\path[fill=fillColor] (279.64,246.49) rectangle (308.08,251.09);
\definecolor{fillColor}{RGB}{177,177,177}

\path[fill=fillColor] (279.64,251.09) rectangle (308.08,255.69);
\definecolor{fillColor}{RGB}{58,58,58}

\path[fill=fillColor] (308.08, 30.31) rectangle (336.52, 34.91);
\definecolor{fillColor}{RGB}{208,208,208}

\path[fill=fillColor] (308.08, 34.91) rectangle (336.52, 39.51);
\definecolor{fillColor}{RGB}{75,75,75}

\path[fill=fillColor] (308.08, 39.51) rectangle (336.52, 44.11);
\definecolor{fillColor}{RGB}{134,134,134}

\path[fill=fillColor] (308.08, 44.11) rectangle (336.52, 48.71);
\definecolor{fillColor}{RGB}{37,37,37}

\path[fill=fillColor] (308.08, 48.71) rectangle (336.52, 53.31);
\definecolor{fillColor}{gray}{0.35}

\path[fill=fillColor] (308.08, 53.31) rectangle (336.52, 57.91);
\definecolor{fillColor}{RGB}{255,255,255}

\path[fill=fillColor] (308.08, 57.91) rectangle (336.52, 62.51);

\path[fill=fillColor] (308.08, 62.51) rectangle (336.52, 67.11);

\path[fill=fillColor] (308.08, 67.11) rectangle (336.52, 71.71);

\path[fill=fillColor] (308.08, 71.71) rectangle (336.52, 76.31);

\path[fill=fillColor] (308.08, 76.31) rectangle (336.52, 80.91);

\path[fill=fillColor] (308.08, 80.91) rectangle (336.52, 85.50);
\definecolor{fillColor}{gray}{0.24}

\path[fill=fillColor] (308.08, 85.50) rectangle (336.52, 90.10);
\definecolor{fillColor}{RGB}{183,183,183}

\path[fill=fillColor] (308.08, 90.10) rectangle (336.52, 94.70);
\definecolor{fillColor}{RGB}{239,239,239}

\path[fill=fillColor] (308.08, 94.70) rectangle (336.52, 99.30);

\path[fill=fillColor] (308.08, 99.30) rectangle (336.52,103.90);
\definecolor{fillColor}{RGB}{255,255,255}

\path[fill=fillColor] (308.08,103.90) rectangle (336.52,108.50);

\path[fill=fillColor] (308.08,108.50) rectangle (336.52,113.10);

\path[fill=fillColor] (308.08,113.10) rectangle (336.52,117.70);

\path[fill=fillColor] (308.08,117.70) rectangle (336.52,122.30);
\definecolor{fillColor}{RGB}{192,192,192}

\path[fill=fillColor] (308.08,122.30) rectangle (336.52,126.90);
\definecolor{fillColor}{gray}{0.72}

\path[fill=fillColor] (308.08,126.90) rectangle (336.52,131.50);
\definecolor{fillColor}{RGB}{255,255,255}

\path[fill=fillColor] (308.08,131.50) rectangle (336.52,136.10);

\path[fill=fillColor] (308.08,136.10) rectangle (336.52,140.70);

\path[fill=fillColor] (308.08,140.70) rectangle (336.52,145.30);
\definecolor{fillColor}{RGB}{164,164,164}

\path[fill=fillColor] (308.08,145.30) rectangle (336.52,149.90);
\definecolor{fillColor}{gray}{0.50}

\path[fill=fillColor] (308.08,149.90) rectangle (336.52,154.50);
\definecolor{fillColor}{RGB}{255,255,255}

\path[fill=fillColor] (308.08,154.50) rectangle (336.52,159.10);
\definecolor{fillColor}{RGB}{98,98,98}

\path[fill=fillColor] (308.08,159.10) rectangle (336.52,163.70);
\definecolor{fillColor}{RGB}{126,126,126}

\path[fill=fillColor] (308.08,163.70) rectangle (336.52,168.30);
\definecolor{fillColor}{RGB}{100,100,100}

\path[fill=fillColor] (308.08,168.30) rectangle (336.52,172.90);
\definecolor{fillColor}{gray}{0.60}

\path[fill=fillColor] (308.08,172.90) rectangle (336.52,177.50);
\definecolor{fillColor}{gray}{0.20}

\path[fill=fillColor] (308.08,177.50) rectangle (336.52,182.10);
\definecolor{fillColor}{RGB}{142,142,142}

\path[fill=fillColor] (308.08,182.10) rectangle (336.52,186.70);
\definecolor{fillColor}{RGB}{103,103,103}

\path[fill=fillColor] (308.08,186.70) rectangle (336.52,191.30);
\definecolor{fillColor}{gray}{0.55}

\path[fill=fillColor] (308.08,191.30) rectangle (336.52,195.90);
\definecolor{fillColor}{gray}{0.92}

\path[fill=fillColor] (308.08,195.90) rectangle (336.52,200.49);
\definecolor{fillColor}{RGB}{182,182,182}

\path[fill=fillColor] (308.08,200.49) rectangle (336.52,205.09);
\definecolor{fillColor}{RGB}{255,255,255}

\path[fill=fillColor] (308.08,205.09) rectangle (336.52,209.69);

\path[fill=fillColor] (308.08,209.69) rectangle (336.52,214.29);
\definecolor{fillColor}{gray}{0.55}

\path[fill=fillColor] (308.08,214.29) rectangle (336.52,218.89);
\definecolor{fillColor}{RGB}{208,208,208}

\path[fill=fillColor] (308.08,218.89) rectangle (336.52,223.49);
\definecolor{fillColor}{RGB}{255,255,255}

\path[fill=fillColor] (308.08,223.49) rectangle (336.52,228.09);

\path[fill=fillColor] (308.08,228.09) rectangle (336.52,232.69);
\definecolor{fillColor}{gray}{0.49}

\path[fill=fillColor] (308.08,232.69) rectangle (336.52,237.29);
\definecolor{fillColor}{RGB}{226,226,226}

\path[fill=fillColor] (308.08,237.29) rectangle (336.52,241.89);
\definecolor{fillColor}{gray}{0.85}

\path[fill=fillColor] (308.08,241.89) rectangle (336.52,246.49);
\definecolor{fillColor}{RGB}{70,70,70}

\path[fill=fillColor] (308.08,246.49) rectangle (336.52,251.09);
\definecolor{fillColor}{RGB}{202,202,202}

\path[fill=fillColor] (308.08,251.09) rectangle (336.52,255.69);
\end{scope}
\begin{scope}
\path[clip] (364.11, 16.51) rectangle (426.67,269.49);
\definecolor{drawColor}{RGB}{255,255,255}

\path[draw=drawColor,line width= 0.3pt,line join=round] (364.11, 21.14) --
	(426.67, 21.14);

\path[draw=drawColor,line width= 0.3pt,line join=round] (364.11, 41.74) --
	(426.67, 41.74);

\path[draw=drawColor,line width= 0.3pt,line join=round] (364.11, 62.34) --
	(426.67, 62.34);

\path[draw=drawColor,line width= 0.3pt,line join=round] (364.11, 82.93) --
	(426.67, 82.93);

\path[draw=drawColor,line width= 0.3pt,line join=round] (364.11,103.53) --
	(426.67,103.53);

\path[draw=drawColor,line width= 0.3pt,line join=round] (364.11,124.12) --
	(426.67,124.12);

\path[draw=drawColor,line width= 0.3pt,line join=round] (364.11,144.72) --
	(426.67,144.72);

\path[draw=drawColor,line width= 0.3pt,line join=round] (364.11,165.31) --
	(426.67,165.31);

\path[draw=drawColor,line width= 0.3pt,line join=round] (364.11,185.91) --
	(426.67,185.91);

\path[draw=drawColor,line width= 0.3pt,line join=round] (364.11,206.50) --
	(426.67,206.50);

\path[draw=drawColor,line width= 0.3pt,line join=round] (364.11,227.10) --
	(426.67,227.10);

\path[draw=drawColor,line width= 0.3pt,line join=round] (364.11,247.69) --
	(426.67,247.69);

\path[draw=drawColor,line width= 0.3pt,line join=round] (364.11,257.99) --
	(426.67,257.99);

\path[draw=drawColor,line width= 0.6pt,line join=round] (364.11, 31.44) --
	(426.67, 31.44);

\path[draw=drawColor,line width= 0.6pt,line join=round] (364.11, 52.04) --
	(426.67, 52.04);

\path[draw=drawColor,line width= 0.6pt,line join=round] (364.11, 72.63) --
	(426.67, 72.63);

\path[draw=drawColor,line width= 0.6pt,line join=round] (364.11, 93.23) --
	(426.67, 93.23);

\path[draw=drawColor,line width= 0.6pt,line join=round] (364.11,113.82) --
	(426.67,113.82);

\path[draw=drawColor,line width= 0.6pt,line join=round] (364.11,134.42) --
	(426.67,134.42);

\path[draw=drawColor,line width= 0.6pt,line join=round] (364.11,155.01) --
	(426.67,155.01);

\path[draw=drawColor,line width= 0.6pt,line join=round] (364.11,175.61) --
	(426.67,175.61);

\path[draw=drawColor,line width= 0.6pt,line join=round] (364.11,196.20) --
	(426.67,196.20);

\path[draw=drawColor,line width= 0.6pt,line join=round] (364.11,216.80) --
	(426.67,216.80);

\path[draw=drawColor,line width= 0.6pt,line join=round] (364.11,237.39) --
	(426.67,237.39);

\path[draw=drawColor,line width= 0.6pt,line join=round] (381.17, 16.51) --
	(381.17,269.49);

\path[draw=drawColor,line width= 0.6pt,line join=round] (409.61, 16.51) --
	(409.61,269.49);
\definecolor{fillColor}{RGB}{255,255,255}

\path[fill=fillColor] (366.95, 29.73) rectangle (395.39, 33.16);

\path[fill=fillColor] (366.95, 33.16) rectangle (395.39, 36.59);

\path[fill=fillColor] (366.95, 36.59) rectangle (395.39, 40.02);

\path[fill=fillColor] (366.95, 40.02) rectangle (395.39, 43.46);

\path[fill=fillColor] (366.95, 43.46) rectangle (395.39, 46.89);

\path[fill=fillColor] (366.95, 46.89) rectangle (395.39, 50.32);

\path[fill=fillColor] (366.95, 50.32) rectangle (395.39, 53.75);

\path[fill=fillColor] (366.95, 53.75) rectangle (395.39, 57.19);

\path[fill=fillColor] (366.95, 57.19) rectangle (395.39, 60.62);

\path[fill=fillColor] (366.95, 60.62) rectangle (395.39, 64.05);

\path[fill=fillColor] (366.95, 64.05) rectangle (395.39, 67.48);

\path[fill=fillColor] (366.95, 67.48) rectangle (395.39, 70.92);

\path[fill=fillColor] (366.95, 70.92) rectangle (395.39, 74.35);

\path[fill=fillColor] (366.95, 74.35) rectangle (395.39, 77.78);

\path[fill=fillColor] (366.95, 77.78) rectangle (395.39, 81.21);
\definecolor{fillColor}{RGB}{78,78,78}

\path[fill=fillColor] (366.95, 81.21) rectangle (395.39, 84.65);
\definecolor{fillColor}{gray}{0.88}

\path[fill=fillColor] (366.95, 84.65) rectangle (395.39, 88.08);
\definecolor{fillColor}{RGB}{255,255,255}

\path[fill=fillColor] (366.95, 88.08) rectangle (395.39, 91.51);

\path[fill=fillColor] (366.95, 91.51) rectangle (395.39, 94.94);

\path[fill=fillColor] (366.95, 94.94) rectangle (395.39, 98.38);

\path[fill=fillColor] (366.95, 98.38) rectangle (395.39,101.81);

\path[fill=fillColor] (366.95,101.81) rectangle (395.39,105.24);

\path[fill=fillColor] (366.95,105.24) rectangle (395.39,108.67);

\path[fill=fillColor] (366.95,108.67) rectangle (395.39,112.11);

\path[fill=fillColor] (366.95,112.11) rectangle (395.39,115.54);

\path[fill=fillColor] (366.95,115.54) rectangle (395.39,118.97);
\definecolor{fillColor}{RGB}{215,215,215}

\path[fill=fillColor] (366.95,118.97) rectangle (395.39,122.40);
\definecolor{fillColor}{RGB}{37,37,37}

\path[fill=fillColor] (366.95,122.40) rectangle (395.39,125.84);
\definecolor{fillColor}{RGB}{255,255,255}

\path[fill=fillColor] (366.95,125.84) rectangle (395.39,129.27);

\path[fill=fillColor] (366.95,129.27) rectangle (395.39,132.70);

\path[fill=fillColor] (366.95,132.70) rectangle (395.39,136.13);

\path[fill=fillColor] (366.95,136.13) rectangle (395.39,139.57);
\definecolor{fillColor}{gray}{0.92}

\path[fill=fillColor] (366.95,139.57) rectangle (395.39,143.00);
\definecolor{fillColor}{RGB}{221,221,221}

\path[fill=fillColor] (366.95,143.00) rectangle (395.39,146.43);
\definecolor{fillColor}{RGB}{206,206,206}

\path[fill=fillColor] (366.95,146.43) rectangle (395.39,149.86);
\definecolor{fillColor}{RGB}{255,255,255}

\path[fill=fillColor] (366.95,149.86) rectangle (395.39,153.30);

\path[fill=fillColor] (366.95,153.30) rectangle (395.39,156.73);

\path[fill=fillColor] (366.95,156.73) rectangle (395.39,160.16);

\path[fill=fillColor] (366.95,160.16) rectangle (395.39,163.60);

\path[fill=fillColor] (366.95,163.60) rectangle (395.39,167.03);

\path[fill=fillColor] (366.95,167.03) rectangle (395.39,170.46);

\path[fill=fillColor] (366.95,170.46) rectangle (395.39,173.89);

\path[fill=fillColor] (366.95,173.89) rectangle (395.39,177.33);

\path[fill=fillColor] (366.95,177.33) rectangle (395.39,180.76);

\path[fill=fillColor] (366.95,180.76) rectangle (395.39,184.19);

\path[fill=fillColor] (366.95,184.19) rectangle (395.39,187.62);

\path[fill=fillColor] (366.95,187.62) rectangle (395.39,191.06);

\path[fill=fillColor] (366.95,191.06) rectangle (395.39,194.49);

\path[fill=fillColor] (366.95,194.49) rectangle (395.39,197.92);

\path[fill=fillColor] (366.95,197.92) rectangle (395.39,201.35);
\definecolor{fillColor}{gray}{0.87}

\path[fill=fillColor] (366.95,201.35) rectangle (395.39,204.79);
\definecolor{fillColor}{RGB}{216,216,216}

\path[fill=fillColor] (366.95,204.79) rectangle (395.39,208.22);
\definecolor{fillColor}{RGB}{255,255,255}

\path[fill=fillColor] (366.95,208.22) rectangle (395.39,211.65);

\path[fill=fillColor] (366.95,211.65) rectangle (395.39,215.08);

\path[fill=fillColor] (366.95,215.08) rectangle (395.39,218.52);

\path[fill=fillColor] (366.95,218.52) rectangle (395.39,221.95);

\path[fill=fillColor] (366.95,221.95) rectangle (395.39,225.38);

\path[fill=fillColor] (366.95,225.38) rectangle (395.39,228.81);
\definecolor{fillColor}{RGB}{220,220,220}

\path[fill=fillColor] (366.95,228.81) rectangle (395.39,232.25);
\definecolor{fillColor}{RGB}{128,128,128}

\path[fill=fillColor] (366.95,232.25) rectangle (395.39,235.68);
\definecolor{fillColor}{RGB}{255,255,255}

\path[fill=fillColor] (366.95,235.68) rectangle (395.39,239.11);

\path[fill=fillColor] (366.95,239.11) rectangle (395.39,242.54);

\path[fill=fillColor] (366.95,242.54) rectangle (395.39,245.98);

\path[fill=fillColor] (366.95,245.98) rectangle (395.39,249.41);
\definecolor{fillColor}{RGB}{197,197,197}

\path[fill=fillColor] (366.95,249.41) rectangle (395.39,252.84);
\definecolor{fillColor}{RGB}{205,205,205}

\path[fill=fillColor] (366.95,252.84) rectangle (395.39,256.27);
\definecolor{fillColor}{RGB}{203,203,203}

\path[fill=fillColor] (395.39, 29.73) rectangle (423.83, 33.16);
\definecolor{fillColor}{gray}{0.97}

\path[fill=fillColor] (395.39, 33.16) rectangle (423.83, 36.59);
\definecolor{fillColor}{RGB}{255,255,255}

\path[fill=fillColor] (395.39, 36.59) rectangle (423.83, 40.02);
\definecolor{fillColor}{RGB}{226,226,226}

\path[fill=fillColor] (395.39, 40.02) rectangle (423.83, 43.46);
\definecolor{fillColor}{RGB}{95,95,95}

\path[fill=fillColor] (395.39, 43.46) rectangle (423.83, 46.89);
\definecolor{fillColor}{RGB}{255,255,255}

\path[fill=fillColor] (395.39, 46.89) rectangle (423.83, 50.32);

\path[fill=fillColor] (395.39, 50.32) rectangle (423.83, 53.75);
\definecolor{fillColor}{RGB}{149,149,149}

\path[fill=fillColor] (395.39, 53.75) rectangle (423.83, 57.19);
\definecolor{fillColor}{RGB}{255,255,255}

\path[fill=fillColor] (395.39, 57.19) rectangle (423.83, 60.62);

\path[fill=fillColor] (395.39, 60.62) rectangle (423.83, 64.05);

\path[fill=fillColor] (395.39, 64.05) rectangle (423.83, 67.48);
\definecolor{fillColor}{RGB}{231,231,231}

\path[fill=fillColor] (395.39, 67.48) rectangle (423.83, 70.92);
\definecolor{fillColor}{gray}{0.75}

\path[fill=fillColor] (395.39, 70.92) rectangle (423.83, 74.35);
\definecolor{fillColor}{RGB}{255,255,255}

\path[fill=fillColor] (395.39, 74.35) rectangle (423.83, 77.78);

\path[fill=fillColor] (395.39, 77.78) rectangle (423.83, 81.21);

\path[fill=fillColor] (395.39, 81.21) rectangle (423.83, 84.65);

\path[fill=fillColor] (395.39, 84.65) rectangle (423.83, 88.08);

\path[fill=fillColor] (395.39, 88.08) rectangle (423.83, 91.51);
\definecolor{fillColor}{RGB}{231,231,231}

\path[fill=fillColor] (395.39, 91.51) rectangle (423.83, 94.94);

\path[fill=fillColor] (395.39, 94.94) rectangle (423.83, 98.38);
\definecolor{fillColor}{RGB}{255,255,255}

\path[fill=fillColor] (395.39, 98.38) rectangle (423.83,101.81);

\path[fill=fillColor] (395.39,101.81) rectangle (423.83,105.24);
\definecolor{fillColor}{gray}{0.63}

\path[fill=fillColor] (395.39,105.24) rectangle (423.83,108.67);
\definecolor{fillColor}{RGB}{230,230,230}

\path[fill=fillColor] (395.39,108.67) rectangle (423.83,112.11);
\definecolor{fillColor}{RGB}{255,255,255}

\path[fill=fillColor] (395.39,112.11) rectangle (423.83,115.54);

\path[fill=fillColor] (395.39,115.54) rectangle (423.83,118.97);
\definecolor{fillColor}{gray}{0.89}

\path[fill=fillColor] (395.39,118.97) rectangle (423.83,122.40);
\definecolor{fillColor}{gray}{0.64}

\path[fill=fillColor] (395.39,122.40) rectangle (423.83,125.84);
\definecolor{fillColor}{RGB}{230,230,230}

\path[fill=fillColor] (395.39,125.84) rectangle (423.83,129.27);
\definecolor{fillColor}{RGB}{211,211,211}

\path[fill=fillColor] (395.39,129.27) rectangle (423.83,132.70);
\definecolor{fillColor}{RGB}{255,255,255}

\path[fill=fillColor] (395.39,132.70) rectangle (423.83,136.13);

\path[fill=fillColor] (395.39,136.13) rectangle (423.83,139.57);

\path[fill=fillColor] (395.39,139.57) rectangle (423.83,143.00);

\path[fill=fillColor] (395.39,143.00) rectangle (423.83,146.43);

\path[fill=fillColor] (395.39,146.43) rectangle (423.83,149.86);
\definecolor{fillColor}{RGB}{210,210,210}

\path[fill=fillColor] (395.39,149.86) rectangle (423.83,153.30);
\definecolor{fillColor}{gray}{0.79}

\path[fill=fillColor] (395.39,153.30) rectangle (423.83,156.73);
\definecolor{fillColor}{RGB}{255,255,255}

\path[fill=fillColor] (395.39,156.73) rectangle (423.83,160.16);

\path[fill=fillColor] (395.39,160.16) rectangle (423.83,163.60);
\definecolor{fillColor}{RGB}{151,151,151}

\path[fill=fillColor] (395.39,163.60) rectangle (423.83,167.03);
\definecolor{fillColor}{RGB}{95,95,95}

\path[fill=fillColor] (395.39,167.03) rectangle (423.83,170.46);
\definecolor{fillColor}{RGB}{255,255,255}

\path[fill=fillColor] (395.39,170.46) rectangle (423.83,173.89);
\definecolor{fillColor}{gray}{0.30}

\path[fill=fillColor] (395.39,173.89) rectangle (423.83,177.33);
\definecolor{fillColor}{gray}{0.42}

\path[fill=fillColor] (395.39,177.33) rectangle (423.83,180.76);
\definecolor{fillColor}{RGB}{243,243,243}

\path[fill=fillColor] (395.39,180.76) rectangle (423.83,184.19);
\definecolor{fillColor}{RGB}{85,85,85}

\path[fill=fillColor] (395.39,184.19) rectangle (423.83,187.62);
\definecolor{fillColor}{RGB}{255,255,255}

\path[fill=fillColor] (395.39,187.62) rectangle (423.83,191.06);
\definecolor{fillColor}{RGB}{37,37,37}

\path[fill=fillColor] (395.39,191.06) rectangle (423.83,194.49);
\definecolor{fillColor}{RGB}{88,88,88}

\path[fill=fillColor] (395.39,194.49) rectangle (423.83,197.92);
\definecolor{fillColor}{gray}{0.96}

\path[fill=fillColor] (395.39,197.92) rectangle (423.83,201.35);
\definecolor{fillColor}{RGB}{49,49,49}

\path[fill=fillColor] (395.39,201.35) rectangle (423.83,204.79);
\definecolor{fillColor}{gray}{0.42}

\path[fill=fillColor] (395.39,204.79) rectangle (423.83,208.22);
\definecolor{fillColor}{RGB}{255,255,255}

\path[fill=fillColor] (395.39,208.22) rectangle (423.83,211.65);

\path[fill=fillColor] (395.39,211.65) rectangle (423.83,215.08);

\path[fill=fillColor] (395.39,215.08) rectangle (423.83,218.52);
\definecolor{fillColor}{RGB}{211,211,211}

\path[fill=fillColor] (395.39,218.52) rectangle (423.83,221.95);
\definecolor{fillColor}{RGB}{75,75,75}

\path[fill=fillColor] (395.39,221.95) rectangle (423.83,225.38);
\definecolor{fillColor}{RGB}{255,255,255}

\path[fill=fillColor] (395.39,225.38) rectangle (423.83,228.81);
\definecolor{fillColor}{gray}{0.42}

\path[fill=fillColor] (395.39,228.81) rectangle (423.83,232.25);
\definecolor{fillColor}{RGB}{132,132,132}

\path[fill=fillColor] (395.39,232.25) rectangle (423.83,235.68);
\definecolor{fillColor}{RGB}{255,255,255}

\path[fill=fillColor] (395.39,235.68) rectangle (423.83,239.11);

\path[fill=fillColor] (395.39,239.11) rectangle (423.83,242.54);

\path[fill=fillColor] (395.39,242.54) rectangle (423.83,245.98);
\definecolor{fillColor}{RGB}{231,231,231}

\path[fill=fillColor] (395.39,245.98) rectangle (423.83,249.41);
\definecolor{fillColor}{gray}{0.36}

\path[fill=fillColor] (395.39,249.41) rectangle (423.83,252.84);
\definecolor{fillColor}{gray}{0.66}

\path[fill=fillColor] (395.39,252.84) rectangle (423.83,256.27);
\end{scope}
\begin{scope}
\path[clip] (  0.00,  0.00) rectangle (432.17,289.08);
\definecolor{drawColor}{gray}{0.30}

\node[text=drawColor,anchor=base east,inner sep=0pt, outer sep=0pt, scale=  0.88] at (  9.90, 47.98) {1};

\node[text=drawColor,anchor=base east,inner sep=0pt, outer sep=0pt, scale=  0.88] at (  9.90, 70.98) {2};

\node[text=drawColor,anchor=base east,inner sep=0pt, outer sep=0pt, scale=  0.88] at (  9.90, 93.97) {3};

\node[text=drawColor,anchor=base east,inner sep=0pt, outer sep=0pt, scale=  0.88] at (  9.90,116.97) {4};

\node[text=drawColor,anchor=base east,inner sep=0pt, outer sep=0pt, scale=  0.88] at (  9.90,139.97) {5};

\node[text=drawColor,anchor=base east,inner sep=0pt, outer sep=0pt, scale=  0.88] at (  9.90,162.97) {6};

\node[text=drawColor,anchor=base east,inner sep=0pt, outer sep=0pt, scale=  0.88] at (  9.90,185.97) {7};

\node[text=drawColor,anchor=base east,inner sep=0pt, outer sep=0pt, scale=  0.88] at (  9.90,208.96) {8};

\node[text=drawColor,anchor=base east,inner sep=0pt, outer sep=0pt, scale=  0.88] at (  9.90,231.96) {9};
\end{scope}
\begin{scope}
\path[clip] (  0.00,  0.00) rectangle (432.17,289.08);
\definecolor{drawColor}{gray}{0.20}

\path[draw=drawColor,line width= 0.6pt,line join=round] ( 12.10, 51.01) --
	( 14.85, 51.01);

\path[draw=drawColor,line width= 0.6pt,line join=round] ( 12.10, 74.01) --
	( 14.85, 74.01);

\path[draw=drawColor,line width= 0.6pt,line join=round] ( 12.10, 97.00) --
	( 14.85, 97.00);

\path[draw=drawColor,line width= 0.6pt,line join=round] ( 12.10,120.00) --
	( 14.85,120.00);

\path[draw=drawColor,line width= 0.6pt,line join=round] ( 12.10,143.00) --
	( 14.85,143.00);

\path[draw=drawColor,line width= 0.6pt,line join=round] ( 12.10,166.00) --
	( 14.85,166.00);

\path[draw=drawColor,line width= 0.6pt,line join=round] ( 12.10,189.00) --
	( 14.85,189.00);

\path[draw=drawColor,line width= 0.6pt,line join=round] ( 12.10,211.99) --
	( 14.85,211.99);

\path[draw=drawColor,line width= 0.6pt,line join=round] ( 12.10,234.99) --
	( 14.85,234.99);
\end{scope}
\begin{scope}
\path[clip] (  0.00,  0.00) rectangle (432.17,289.08);
\definecolor{drawColor}{gray}{0.30}

\node[text=drawColor,anchor=base east,inner sep=0pt, outer sep=0pt, scale=  0.88] at ( 97.21, 38.51) {1};

\node[text=drawColor,anchor=base east,inner sep=0pt, outer sep=0pt, scale=  0.88] at ( 97.21, 65.56) {3};

\node[text=drawColor,anchor=base east,inner sep=0pt, outer sep=0pt, scale=  0.88] at ( 97.21, 92.62) {5};

\node[text=drawColor,anchor=base east,inner sep=0pt, outer sep=0pt, scale=  0.88] at ( 97.21,119.68) {7};

\node[text=drawColor,anchor=base east,inner sep=0pt, outer sep=0pt, scale=  0.88] at ( 97.21,146.73) {9};

\node[text=drawColor,anchor=base east,inner sep=0pt, outer sep=0pt, scale=  0.88] at ( 97.21,173.79) {11};

\node[text=drawColor,anchor=base east,inner sep=0pt, outer sep=0pt, scale=  0.88] at ( 97.21,200.85) {13};

\node[text=drawColor,anchor=base east,inner sep=0pt, outer sep=0pt, scale=  0.88] at ( 97.21,227.90) {15};
\end{scope}
\begin{scope}
\path[clip] (  0.00,  0.00) rectangle (432.17,289.08);
\definecolor{drawColor}{gray}{0.20}

\path[draw=drawColor,line width= 0.6pt,line join=round] ( 99.41, 41.54) --
	(102.16, 41.54);

\path[draw=drawColor,line width= 0.6pt,line join=round] ( 99.41, 68.59) --
	(102.16, 68.59);

\path[draw=drawColor,line width= 0.6pt,line join=round] ( 99.41, 95.65) --
	(102.16, 95.65);

\path[draw=drawColor,line width= 0.6pt,line join=round] ( 99.41,122.71) --
	(102.16,122.71);

\path[draw=drawColor,line width= 0.6pt,line join=round] ( 99.41,149.76) --
	(102.16,149.76);

\path[draw=drawColor,line width= 0.6pt,line join=round] ( 99.41,176.82) --
	(102.16,176.82);

\path[draw=drawColor,line width= 0.6pt,line join=round] ( 99.41,203.88) --
	(102.16,203.88);

\path[draw=drawColor,line width= 0.6pt,line join=round] ( 99.41,230.93) --
	(102.16,230.93);
\end{scope}
\begin{scope}
\path[clip] (  0.00,  0.00) rectangle (432.17,289.08);
\definecolor{drawColor}{gray}{0.30}

\node[text=drawColor,anchor=base east,inner sep=0pt, outer sep=0pt, scale=  0.88] at (184.53, 31.37) {1};

\node[text=drawColor,anchor=base east,inner sep=0pt, outer sep=0pt, scale=  0.88] at (184.53, 56.92) {5};

\node[text=drawColor,anchor=base east,inner sep=0pt, outer sep=0pt, scale=  0.88] at (184.53, 82.47) {9};

\node[text=drawColor,anchor=base east,inner sep=0pt, outer sep=0pt, scale=  0.88] at (184.53,108.03) {13};

\node[text=drawColor,anchor=base east,inner sep=0pt, outer sep=0pt, scale=  0.88] at (184.53,133.58) {17};

\node[text=drawColor,anchor=base east,inner sep=0pt, outer sep=0pt, scale=  0.88] at (184.53,159.13) {21};

\node[text=drawColor,anchor=base east,inner sep=0pt, outer sep=0pt, scale=  0.88] at (184.53,184.69) {25};

\node[text=drawColor,anchor=base east,inner sep=0pt, outer sep=0pt, scale=  0.88] at (184.53,210.24) {29};

\node[text=drawColor,anchor=base east,inner sep=0pt, outer sep=0pt, scale=  0.88] at (184.53,235.79) {33};
\end{scope}
\begin{scope}
\path[clip] (  0.00,  0.00) rectangle (432.17,289.08);
\definecolor{drawColor}{gray}{0.20}

\path[draw=drawColor,line width= 0.6pt,line join=round] (186.73, 34.40) --
	(189.48, 34.40);

\path[draw=drawColor,line width= 0.6pt,line join=round] (186.73, 59.95) --
	(189.48, 59.95);

\path[draw=drawColor,line width= 0.6pt,line join=round] (186.73, 85.50) --
	(189.48, 85.50);

\path[draw=drawColor,line width= 0.6pt,line join=round] (186.73,111.06) --
	(189.48,111.06);

\path[draw=drawColor,line width= 0.6pt,line join=round] (186.73,136.61) --
	(189.48,136.61);

\path[draw=drawColor,line width= 0.6pt,line join=round] (186.73,162.16) --
	(189.48,162.16);

\path[draw=drawColor,line width= 0.6pt,line join=round] (186.73,187.72) --
	(189.48,187.72);

\path[draw=drawColor,line width= 0.6pt,line join=round] (186.73,213.27) --
	(189.48,213.27);

\path[draw=drawColor,line width= 0.6pt,line join=round] (186.73,238.82) --
	(189.48,238.82);
\end{scope}
\begin{scope}
\path[clip] (  0.00,  0.00) rectangle (432.17,289.08);
\definecolor{drawColor}{gray}{0.30}

\node[text=drawColor,anchor=base east,inner sep=0pt, outer sep=0pt, scale=  0.88] at (271.84, 29.58) {1};

\node[text=drawColor,anchor=base east,inner sep=0pt, outer sep=0pt, scale=  0.88] at (271.84, 52.58) {6};

\node[text=drawColor,anchor=base east,inner sep=0pt, outer sep=0pt, scale=  0.88] at (271.84, 75.58) {11};

\node[text=drawColor,anchor=base east,inner sep=0pt, outer sep=0pt, scale=  0.88] at (271.84, 98.57) {16};

\node[text=drawColor,anchor=base east,inner sep=0pt, outer sep=0pt, scale=  0.88] at (271.84,121.57) {21};

\node[text=drawColor,anchor=base east,inner sep=0pt, outer sep=0pt, scale=  0.88] at (271.84,144.57) {26};

\node[text=drawColor,anchor=base east,inner sep=0pt, outer sep=0pt, scale=  0.88] at (271.84,167.57) {31};

\node[text=drawColor,anchor=base east,inner sep=0pt, outer sep=0pt, scale=  0.88] at (271.84,190.57) {36};

\node[text=drawColor,anchor=base east,inner sep=0pt, outer sep=0pt, scale=  0.88] at (271.84,213.56) {41};

\node[text=drawColor,anchor=base east,inner sep=0pt, outer sep=0pt, scale=  0.88] at (271.84,236.56) {46};
\end{scope}
\begin{scope}
\path[clip] (  0.00,  0.00) rectangle (432.17,289.08);
\definecolor{drawColor}{gray}{0.20}

\path[draw=drawColor,line width= 0.6pt,line join=round] (274.04, 32.61) --
	(276.79, 32.61);

\path[draw=drawColor,line width= 0.6pt,line join=round] (274.04, 55.61) --
	(276.79, 55.61);

\path[draw=drawColor,line width= 0.6pt,line join=round] (274.04, 78.61) --
	(276.79, 78.61);

\path[draw=drawColor,line width= 0.6pt,line join=round] (274.04,101.60) --
	(276.79,101.60);

\path[draw=drawColor,line width= 0.6pt,line join=round] (274.04,124.60) --
	(276.79,124.60);

\path[draw=drawColor,line width= 0.6pt,line join=round] (274.04,147.60) --
	(276.79,147.60);

\path[draw=drawColor,line width= 0.6pt,line join=round] (274.04,170.60) --
	(276.79,170.60);

\path[draw=drawColor,line width= 0.6pt,line join=round] (274.04,193.60) --
	(276.79,193.60);

\path[draw=drawColor,line width= 0.6pt,line join=round] (274.04,216.59) --
	(276.79,216.59);

\path[draw=drawColor,line width= 0.6pt,line join=round] (274.04,239.59) --
	(276.79,239.59);
\end{scope}
\begin{scope}
\path[clip] (  0.00,  0.00) rectangle (432.17,289.08);
\definecolor{drawColor}{gray}{0.30}

\node[text=drawColor,anchor=base east,inner sep=0pt, outer sep=0pt, scale=  0.88] at (359.16, 28.41) {1};

\node[text=drawColor,anchor=base east,inner sep=0pt, outer sep=0pt, scale=  0.88] at (359.16, 49.01) {7};

\node[text=drawColor,anchor=base east,inner sep=0pt, outer sep=0pt, scale=  0.88] at (359.16, 69.60) {13};

\node[text=drawColor,anchor=base east,inner sep=0pt, outer sep=0pt, scale=  0.88] at (359.16, 90.20) {19};

\node[text=drawColor,anchor=base east,inner sep=0pt, outer sep=0pt, scale=  0.88] at (359.16,110.79) {25};

\node[text=drawColor,anchor=base east,inner sep=0pt, outer sep=0pt, scale=  0.88] at (359.16,131.39) {31};

\node[text=drawColor,anchor=base east,inner sep=0pt, outer sep=0pt, scale=  0.88] at (359.16,151.98) {37};

\node[text=drawColor,anchor=base east,inner sep=0pt, outer sep=0pt, scale=  0.88] at (359.16,172.58) {43};

\node[text=drawColor,anchor=base east,inner sep=0pt, outer sep=0pt, scale=  0.88] at (359.16,193.17) {49};

\node[text=drawColor,anchor=base east,inner sep=0pt, outer sep=0pt, scale=  0.88] at (359.16,213.77) {55};

\node[text=drawColor,anchor=base east,inner sep=0pt, outer sep=0pt, scale=  0.88] at (359.16,234.36) {61};
\end{scope}
\begin{scope}
\path[clip] (  0.00,  0.00) rectangle (432.17,289.08);
\definecolor{drawColor}{gray}{0.20}

\path[draw=drawColor,line width= 0.6pt,line join=round] (361.36, 31.44) --
	(364.11, 31.44);

\path[draw=drawColor,line width= 0.6pt,line join=round] (361.36, 52.04) --
	(364.11, 52.04);

\path[draw=drawColor,line width= 0.6pt,line join=round] (361.36, 72.63) --
	(364.11, 72.63);

\path[draw=drawColor,line width= 0.6pt,line join=round] (361.36, 93.23) --
	(364.11, 93.23);

\path[draw=drawColor,line width= 0.6pt,line join=round] (361.36,113.82) --
	(364.11,113.82);

\path[draw=drawColor,line width= 0.6pt,line join=round] (361.36,134.42) --
	(364.11,134.42);

\path[draw=drawColor,line width= 0.6pt,line join=round] (361.36,155.01) --
	(364.11,155.01);

\path[draw=drawColor,line width= 0.6pt,line join=round] (361.36,175.61) --
	(364.11,175.61);

\path[draw=drawColor,line width= 0.6pt,line join=round] (361.36,196.20) --
	(364.11,196.20);

\path[draw=drawColor,line width= 0.6pt,line join=round] (361.36,216.80) --
	(364.11,216.80);

\path[draw=drawColor,line width= 0.6pt,line join=round] (361.36,237.39) --
	(364.11,237.39);
\end{scope}
\begin{scope}
\path[clip] (  0.00,  0.00) rectangle (432.17,289.08);
\definecolor{drawColor}{gray}{0.20}

\path[draw=drawColor,line width= 0.6pt,line join=round] ( 31.91, 13.76) --
	( 31.91, 16.51);

\path[draw=drawColor,line width= 0.6pt,line join=round] ( 60.35, 13.76) --
	( 60.35, 16.51);
\end{scope}
\begin{scope}
\path[clip] (  0.00,  0.00) rectangle (432.17,289.08);
\definecolor{drawColor}{gray}{0.30}

\node[text=drawColor,anchor=base,inner sep=0pt, outer sep=0pt, scale=  0.88] at ( 31.91,  5.50) {Sub};

\node[text=drawColor,anchor=base,inner sep=0pt, outer sep=0pt, scale=  0.88] at ( 60.35,  5.50) {Shift};
\end{scope}
\begin{scope}
\path[clip] (  0.00,  0.00) rectangle (432.17,289.08);
\definecolor{drawColor}{gray}{0.20}

\path[draw=drawColor,line width= 0.6pt,line join=round] (119.23, 13.76) --
	(119.23, 16.51);

\path[draw=drawColor,line width= 0.6pt,line join=round] (147.67, 13.76) --
	(147.67, 16.51);
\end{scope}
\begin{scope}
\path[clip] (  0.00,  0.00) rectangle (432.17,289.08);
\definecolor{drawColor}{gray}{0.30}

\node[text=drawColor,anchor=base,inner sep=0pt, outer sep=0pt, scale=  0.88] at (119.23,  5.50) {Sub};

\node[text=drawColor,anchor=base,inner sep=0pt, outer sep=0pt, scale=  0.88] at (147.67,  5.50) {Shift};
\end{scope}
\begin{scope}
\path[clip] (  0.00,  0.00) rectangle (432.17,289.08);
\definecolor{drawColor}{gray}{0.20}

\path[draw=drawColor,line width= 0.6pt,line join=round] (206.54, 13.76) --
	(206.54, 16.51);

\path[draw=drawColor,line width= 0.6pt,line join=round] (234.98, 13.76) --
	(234.98, 16.51);
\end{scope}
\begin{scope}
\path[clip] (  0.00,  0.00) rectangle (432.17,289.08);
\definecolor{drawColor}{gray}{0.30}

\node[text=drawColor,anchor=base,inner sep=0pt, outer sep=0pt, scale=  0.88] at (206.54,  5.50) {Sub};

\node[text=drawColor,anchor=base,inner sep=0pt, outer sep=0pt, scale=  0.88] at (234.98,  5.50) {Shift};
\end{scope}
\begin{scope}
\path[clip] (  0.00,  0.00) rectangle (432.17,289.08);
\definecolor{drawColor}{gray}{0.20}

\path[draw=drawColor,line width= 0.6pt,line join=round] (293.86, 13.76) --
	(293.86, 16.51);

\path[draw=drawColor,line width= 0.6pt,line join=round] (322.30, 13.76) --
	(322.30, 16.51);
\end{scope}
\begin{scope}
\path[clip] (  0.00,  0.00) rectangle (432.17,289.08);
\definecolor{drawColor}{gray}{0.30}

\node[text=drawColor,anchor=base,inner sep=0pt, outer sep=0pt, scale=  0.88] at (293.86,  5.50) {Sub};

\node[text=drawColor,anchor=base,inner sep=0pt, outer sep=0pt, scale=  0.88] at (322.30,  5.50) {Shift};
\end{scope}
\begin{scope}
\path[clip] (  0.00,  0.00) rectangle (432.17,289.08);
\definecolor{drawColor}{gray}{0.20}

\path[draw=drawColor,line width= 0.6pt,line join=round] (381.17, 13.76) --
	(381.17, 16.51);

\path[draw=drawColor,line width= 0.6pt,line join=round] (409.61, 13.76) --
	(409.61, 16.51);
\end{scope}
\begin{scope}
\path[clip] (  0.00,  0.00) rectangle (432.17,289.08);
\definecolor{drawColor}{gray}{0.30}

\node[text=drawColor,anchor=base,inner sep=0pt, outer sep=0pt, scale=  0.88] at (381.17,  5.50) {Sub};

\node[text=drawColor,anchor=base,inner sep=0pt, outer sep=0pt, scale=  0.88] at (409.61,  5.50) {Shift};
\end{scope}
\begin{scope}
\path[clip] (  0.00,  0.00) rectangle (432.17,289.08);
\definecolor{drawColor}{RGB}{0,0,0}

\node[text=drawColor,anchor=base,inner sep=0pt, outer sep=0pt, scale=  1.32] at ( 46.13,264.49) {0};
\end{scope}
\begin{scope}
\path[clip] (  0.00,  0.00) rectangle (432.17,289.08);
\definecolor{drawColor}{RGB}{0,0,0}

\node[text=drawColor,anchor=base,inner sep=0pt, outer sep=0pt, scale=  1.32] at (133.45,264.49) {1000};
\end{scope}
\begin{scope}
\path[clip] (  0.00,  0.00) rectangle (432.17,289.08);
\definecolor{drawColor}{RGB}{0,0,0}

\node[text=drawColor,anchor=base,inner sep=0pt, outer sep=0pt, scale=  1.32] at (220.76,264.49) {3000};
\end{scope}
\begin{scope}
\path[clip] (  0.00,  0.00) rectangle (432.17,289.08);
\definecolor{drawColor}{RGB}{0,0,0}

\node[text=drawColor,anchor=base,inner sep=0pt, outer sep=0pt, scale=  1.32] at (308.08,264.49) {4000};
\end{scope}
\begin{scope}
\path[clip] (  0.00,  0.00) rectangle (432.17,289.08);
\definecolor{drawColor}{RGB}{0,0,0}

\node[text=drawColor,anchor=base,inner sep=0pt, outer sep=0pt, scale=  1.32] at (395.39,264.49) {5000};
\end{scope}
\end{tikzpicture}
\end{scaletikzpicturetowidth}
%\caption{Überwachung der Folge 4s2s1s5s2s1}
\end{figure}

\normalsize
\begin{itemize}
\item Bremsung von mindestens 3000 Taktzyklen oder mehr Kerne zur Unterscheidung nötig

\end{itemize}

\end{frame}

%------------------------------------------------
\section{Fazit und Ausblick}
%------------------------------------------------

\begin{frame}
\frametitle{Fazit und Ausblick}
\begin{itemize}
\item Angriffs-Initialisierung unter 10 Sekunden möglich
%Paralleliserung von colliding address search und eviction-set search

\item Limitierte Möglichkeiten erschweren Portierung von Angriffen im Browser

\item Wiedereinführung hoch-präziser Timer ermöglicht Angriffe
%Chrome führt Time wegen Site Isolation wieder ein, problematisch da keine Fixes gegen L3-Cache-Angriffe

%\item Erweiterte Angriffsmöglichkeiten wenn Opfer- und Angriffsprogramm auf gleichem physischen Kern

\item Standard-Webtechnologien ermöglichen Angriffe auf viele Endgeräte
%

\item Cache-Angriffe werden noch Jahre Bestand haben
%Intel exclusive Cache frühestens mit Ice Lake 2019, aber AMD jetzt schon nicht mit dieser Methode angreifbar

\end{itemize}
\end{frame}

%------------------------------------------------
%\subsection{Weitere Ansätze}
\begin{frame}
\frametitle{Weitere Bremsansätze}


\begin{itemize}
\item Implementierung von Prime-and-Probe in Javascript bringt ~10\% Performancegewinn
\item Standard für Prime-and-Probe ist Pointer-Chasing
\item Aufteilung in zwei Ketten erlaubt 50\% mehr Messungen pro Zeitintervall
%\item Aufteilung auf vier Ketten nicht erfolgreich
%Warum geht das eigentlich nicht
\item Pointer-Array statt Pointer-Chasing bremst besser wenn viele Cache-Lines in einem Thread gebremst werden (1 Thread bremst Shift auf ca. 800 Taktzyklen)
\item Aufteilen des Pointer-Arrays ist nachteilig
\item Weglassen von Back-Probing bringt nur Vorteile bei Pointer-Array
\end{itemize}

\end{frame}

%------------------------------------------------
%\subsection{colliding-addresses}
\begin{frame}
\frametitle{Beschleunigung der Eviction-Set-Suche}
\begin{itemize}
%schnelle Initaliserung des Angriffs notwendig
\item Verfeinerung des Adresspools mittels Leakage
\end{itemize}

\SetKwProg{Fn}{Function}{}{}
\small
\begin{algorithm}[H]
\DontPrintSemicolon
\caption{Pseudo-Code für colliding-address-Suche}
\label{alg:storeForward}

%\Fn{$FindCollidingAddresses()$}{
 %   addressPool = malloc(PageSize * PageCount)\;
    
 %       \For{i from 60 to 0}{
 %           addressPool[(p-i) * PageSize] $\leftarrow$ 0\;
 %       }
 %       timestampBefore $\leftarrow$ getTimestamp()\;
 %       read address\;
 %       result[p] $\leftarrow$ getTimestamp() - timestampBefore\;
%    }
%}

\Fn{$FindCollidingAddresses()$}{
    buffer = malloc(Size)\;
    
    \For{p from 60 to Size-1}{
        \For{i from 60 to 0}{
            buffer[p-i] $\leftarrow$ 0\;
        }
        timestampBefore $\leftarrow$ getTimestamp()\;
        read x\;
        result[p] $\leftarrow$ getTimestamp() - timestampBefore\;
    }
}
\end{algorithm}
\end{frame}

%------------------------------------------------
%\subsection{Ein Thread}
\begin{frame}
\frametitle{Ein Thread}

%TODO clamp_1 erste Zeile sub-wert untersuchen

\small
\begin{table}[h]
\label{tbl:PerformanceDegShift}
\begin{tabular}{cclccc}
\toprule
Funktion & Shift & Sub \\
\midrule
Referenz      & 340   & 477        \\
$*clamp_1$  & 541   & 608           \\
$*rshd_2$  & 451   & 542    \\
$mp\_sub_7$            & 359   & 766    \\
$*clamp_{1,2}$ & 542   & 557     \\
$*clamp_1,*rshd_4$   & 643   & 549     \\
$*clamp_1,*rshd_4,*div\_2d_3$ & 623   & 560    \\

\bottomrule
\end{tabular}
\end{table}

\normalsize
\begin{itemize}
\item Unterschiedliche Cache-Lines bremsen unterschiedlich gut
\item Besser mehrere Cache-Lines abwechselnd bremsen
%tebelle einfügen mit bester bremsung bei einer cache-line und bester bei mehreren cache-lines
\end{itemize}
\end{frame}

%Ohne die Eviction-Sets kann keine Angriff stattfinden, d.h. die Suche sollte nicht zu lange dauern.
%Insbesondere da bei einem Webangriff davon auszugehen ist, das ein Nutzer eine Website mit einer bösartigen Werbeanzeige nicht länger als ein paar Minuten besucht.
%Deshalb ist es interssant die Eviction-Set-Suche möglichst kurz zu halten. Hierfür möchte ich noch einmal kurz auf die vorherige Folie zurückgehen.
%Wir können dank der Pages garantieren, dass die letzten 12 Bits der Adresse übereinstimmen.
%Bersser wäre es natürlich wenn beispielweise Adressen hätten, wo wir wüssten, dass die letzten 18 Bit gleich sind.
%Wie in Zeile 1 zu sehen wäre dann nur noch die Slice unbekannt und die vorhin beschriebene Eviction-Set Suche könnte deutlich beschleunigt werden, da der Adresspool deutlich kleiner ist.
%Und genau diese Möglichkeit bietet ein Verhalten von Intel-Prozessoren, was ich kurz beschreiben möchte.
%Eine colliding-address zu Adresse x soll nun eine Adresse sein deren 20 letzten Adressbits mit x übereinstimmen.
%Dazu wird in Zeile 2 erstmal ein Buffer allokiert.
%Im Folgenden wird nun eine pseudo Read-after-Write Abhängigkeit erzeugt.
%Wir wollen nun erfahren ob p eine colliding-address zu x ist.
%Dazu fluten wir die CPU zuerst mit Schreibbefehlen, die auf Adressen ungleich p. Das sind die ersten 60 Iterationen der For-Schleife in Zeile 4,5.
%Abschließend wird noch ein Schreibbefehl auf Adresse p ausgeüfhrt, das ist die letzte Iteration der For-Schleiße in Zeile 4,5
%Danach wird die Zugriffszeit auf x gemessen. Wenn nun die letzten 20 Bits von x und p gleich sind lässt sich eine erhöhte Zugriffszeit messen, da die CPU eine pseudo-Abhängigkeit zwischen dem Schreiben von p und dem Lesen von x erkennt.
%Somit kann ein Adresspool gebaut werden bei dem die letzten 20 Bit der Adressen gleich sind und somit die vorhin beschriebene Suche nach Eviction-Set deutlich beschleunigt werden, da bei zufälliger Wahl aus dem Adresspool schneller ein Eviciton-Set gefunden wird.

%------------------------------------------------

%\begin{frame}
%\frametitle{Figure}
%Uncomment the code on this slide to include your own image from the same directory as the template .TeX file.
%\begin{figure}
%\includegraphics[width=0.8\linewidth]{test}
%\end{figure}
%\end{frame}

%------------------------------------------------

%\begin{frame}[fragile] % Need to use the fragile option when verbatim is used in the slide
%\frametitle{Citation}
%An example of the \verb|\cite| command to cite within the presentation:\\~

%This statement requires citation \cite{p1}.
%\end{frame}

%------------------------------------------------

%\begin{frame}
%\frametitle{References}
%\footnotesize{
%\begin{thebibliography}{99} % Beamer does not support BibTeX so references must be inserted manually as below
%\bibitem[Smith, 2012]{p1} John Smith (2012)
%\newblock Title of the publication
%\newblock \emph{Journal Name} 12(3), 45 -- 678.
%\end{thebibliography}
%}
%\end{frame}

%----------------------------------------------------------------------------------------

\end{document}