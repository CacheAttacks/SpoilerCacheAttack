\chapter{Implementierung}
\label{chapter:preparation}

\SetKwProg{Fn}{Function}{}{}

Das folgende Kapitel beschreibt, mit Hilfe welcher Softwaretools der Cache-Angriff implementiert wird.

Um das allgemeinere und praxisnähere Angriffsmodell umzusetzen und den Angriff schon durch den Besuch einer Website zu starten, liegt der komplette Angriffscode in JavaScript und Webassembly vor. 
Frühere Implementierungen von Cache-Angriffen im Browser \cite{TheSpyInTheSandbox} hatten noch keine Möglichkeit, Webassembly zu verwenden, weshalb diese mit ihrem kompletten Angriffscode in JavaScript geschrieben waren. 
Webassembly ermöglicht hardwarenähere Programmierung und den Vorteil, dass der Code nicht wie in JavaScript während der Laufzeit optimiert werden muss. 
Des Weiteren steht mit dem Emscripten-Compiler ein Tool bereit, welches die Übersetzung von C-Code in Webassembly anbietet. Somit kann ein bestehender Angriffscode in C, in diesem Fall von Mastik, als Grundlage verwendet werden, und eine komplette fehleranfällige Neuimplementierung in JavaScript entfällt.


\section{Timer in JavaScript}

Der hier ausgeführte Cache-Angriff benötigt wie in \ref beschrieben präzise Timer, welche eine Auflösung von unter 30 ns bereitstellen sollten. Dennoch könnte die Suche nach \textit{Eviction-Sets} auch mit schlechteren Timerauflösungen bewerkstelligt werden, indem Operationen mehrfach ausgeführt werden und die Differenz der aufsummierten Zeiten zur Bewertung herangezogen wird.
Im \textit{Eviction-Set}-Algorithmus könnte etwa die Funktion $checkevict$ wie in Algorithmus \ref{alg:checkevict_low_resolution} angepasst werden, wobei der Parameter $repeatIterations$ abhängig von der Timerauflösung gewählt wird. 
Weiter besteht jedoch das Problem, schwache Aktivitäten im Cache-Set während des eigentlichen Angriffs aufzudecken, da im Worst-Case nur ein Eintrag aus dem beobachteten Cache-Set verdrängt wird und somit lediglich die Zugriffszeit zwischen einem Hit und einem Miss ausschlaggebend ist. 
In diesem Fall könnte die Dauer mehrerer Prime and Probe-Iterationen gesamtheitlich gemessen werden, und zwar in der Vermutung, dass auf die für die Verdrängung verantwortliche Adresse über die Zeit mehrfach zugegriffen wird. Die direkten Auswirkungen eines niedrig aufgelösten Timers sind also eine geringere zeitliche Auflösung von Cache-Aktivitäten oder die Nichtregestrierung von schwachen.

\begin{algorithm}[h]
\DontPrintSemicolon
\caption{Psuedo-Code für $checkevict$ im Fall von einer niedrig aufgelösten getTimestamp}
\label{alg:checkevict_low_resolution}

\Fn{$checkevict(possibleEvictionSet, witness)$}{
    timestampBefore <- getTimestamp()\;
    \For{i=1 to repeatIterations}{
        accessMemory(possibleEvictionSet)\;
        accessMemory(witness)\;
    }
	timestampAfter <- getTimestamp()\;
	\Return timestampAfter - timestampBefore > threshold;
}

\end{algorithm}

Der W3C hat die High-Resolution-Time-API spezifiziert, welche die Methode performance.now() beinhaltet, die einen aktuellen Timestamp zurückgibt. Im Firefox hatte die Methode in früheren Versionen eine hinreichend genaue Auflösung im Nanosekundenbereich, wobei in Reaktion auf die Sicherheitslücken Meltdown und Spectre die Auflösung schrittweise auf 2 ms im aktuellen Firefox 60 abgesenkt wurde. 
Auch in den Browsern Edge und Chrome wurden im Zuge der Veröffentlichung von Meltdown und Spectre die Auflösung von perfomance.now() verringert.
Allerdings wird auf den zurückgegebenen Timestamp zusätzlich ein Timerjitter addiert. 

So bieten Edge und Chrome zurzeit (Stand Juni 18) eine Auflösung von 20 \textmu s + 20 \textmu s Jitter respektive 100 \textmu s + 100 \textmu s Jitter.
Das Paper "Fantastic Timers and where to find them" \cite{FantasticTimers} beschreibt diverse andere Methoden, um mit Hilfe von JavaScript Timer zu generieren. 
Allerdings ist nur eine geeignete Methode dabei, da die Auflösung aller anderen mindestens im hohen einstelligen \textmu s Bereich liegt und somit der Parameter $repeatIterations$ auf Werte von etwa 1000 gesetzt werden müsste, um zuverlässig \textit{Eviction Sets} zu finden. 
Hierdurch würde die benötigte Ausführungszeit zum Finden der \textit{Eviction Sets} auf ein Maß ansteigen, welches dann nicht mehr zum angenommenen Angriffsmodell passen würde.

\newtext

Als einziges angemessenes Zeitmessungswerkzeug verbleibt der SharedArrayBuffer aus Javascript. 
Die Ausführung von Javascript passiert in einem Thread, wobei die Möglichkeit besteht sogenannte Webworker zu starten, welche Code aus einem Skript in einem eigenen Thread ausführen.
Diese haben von Grund auf aber einen zu Mainthread abgetrennten Speicherbereich, sodass Daten ursprünglich über Nachrichten zwischen dem Mainthread und dem Webworker ausgetauscht werden mussten. Hier setzt der SharedArrayBuffer an, welcher einen geteilten Speicherbereich zwischen Mainthread und Webworker definiert.

Gemäß Code-Listing \ref{alg_list:sharedArrayBufferWorkerMain} wird im Mainthread zuerst ein SharedArrayBuffer von 4 Bytes angelegt. Anschließend wird der als Zeitgeber fungierende Webworker gestartet und ihm eine Referenz auf den eben angelegten SharedArrayBuffer übersandt. 
Die Zählvariable soll hier eine Größe von 32 Bit haben, weshalb abschließend ein Uint32 Array definiert wird, welches auf den SharedArrayBuffer referenziert. 
Ein Zählvariable kann nun durch Lesen des ersten Eintrags des Arrays erhalten werden. 
Problematisch ist jedoch, dass auf die Zählvariable sowohl lesend vom Mainthread als auch schreibend vom Webworker zugegriffen wird. 
Dadurch können die im Mainthread gelesenen Werte veraltet sein, da der SharedArrayBuffer noch nicht zwischen beiden Threads synchronisiert wurde. 
Abhilfe schafft hier die von Javascript bereitgestellte Atomics-Library, welche es ermöglicht die Leseoperation atomar auszuführen.

Der Webworker iteriert nun in einem eigenen Thread die Zählvariable in einer Endlosschleife (siehe auch Pseudocode \ref{alg_list:sharedArrayBufferWorker}). 
Zuerst wird dazu dem Webworker via message die Referenz auf einen im Mainthread erstellten SharedArrayBuffer übergeben.
Anschließend wird im Webworker ein \textit{Uint32Array} angelegt welches mit dem übergebenen SharedArrayBuffer verknüpft ist. 
Zum Schluss geht der Webworker in die Endlosschleife über, in welcher die Zählvariable \textit{sharedArray[0]} durchgehend iteriert.

Das Iterieren einer Variable benötigt nur wenige Taktzyklen, weshalb der aktuelle Wert der Zählvariable als Zeitstempel interpretiert werden kann. 
Die Auflösung dieser Methode hängt also von der Geschwindigkeit der Iteration, sowie der Speichersynchronisation des SharedArrayBuffers zwischen Mainthread und Webworker ab.

In Versuchen mit verschiedenen Webbrowsern und Hardwarekonfigurationen zeigte sich, dass die Auflösung mindestens im einstelligen Nanosekundenbreich liegt und somit ausreichend genau um den Unterschied zwischen einem Cache-Miss und Hit festzustellen.

Diese Methode geht allerdings mit dem Nachteil einher, dass der Webworker-Thread in der Messphase einen CPU-Kern komplett auslastet. Das heißt im Angriffsszenario rechnen das Opferprogramm, der Javascript-Mainthread und der Webworker gleichzeitig, sodass mindestens 3 CPU Kerne benötigt werden. 
Sofern sich der Webworker einen physischen Kern mit einen anderen aktiven Prozess teilt, können die gemessen Zeiten ein stärkeren Volatilität durch die erhöhte Iterationsdauer unterliegen.
In der Konsequenz reduziert sich die Auflösung des Zeitgebers, wobei ein ausreichende Genauigkeit dennoch gegeben ist, da beide Prozesse in etwa die gleiche Rechenzeit zugesprochen bekommen.

Aufgrund der Option den SharedArrayBuffer als Timer zweckzuentfremden, wurde dieser im Zuge der Veröffentlichung von Meltdown und Spectre in allen gängigen Webbrowser deaktiviert \cite{FirefoxSharedArrayBuffer}. Jedoch planen die Hersteller das Feature in Zukunft wieder zu aktivieren, sobald die Gefahr von Angriffen wie Meltdown und Spectre reduziert ist. 
Google ist der erste Hersteller der in seinem Chrome-Browser mit Version 68 SharedArrayBuffer wieder aktiviert hat \cite{ChromeSharedArrayBufferAgain}. 
Als logische Konsequenz wurde angekündigt in Zukunft wieder hochauflösende Timer bereitzustellen, da ein SharedArrayBuffer genau diese Möglichkeit schon jetzt bereitstellt \cite{ChromeHighResolutionTimerAgain}.

Aus diesen Gründen wird im Folgenden davon ausgegangen, dass das Opfer SharedArrayBuffer in seinem Webbrowser aktiviert hat.

\label{alg_list:sharedArrayBufferWorkerMain}[h]
\begin{lstlisting}[caption=main.js: Code welcher den counterWorker für Zeitmessungen verwendet]
var sharedArrayBuffer = new SharedArrayBuffer(4);
var counterWorker = new Webworker('counterWebworker.js');
counterWorker.postMessage(sharedArrayBuffer);
var sharedArrayBufferUin32Array = new Uint32Array(sharedArrayBuffer);

function measureTime(func){
    var t1 = Atomics.load(sharedArrayBufferUin32Array[0]);
    func();
    var t2 = Atomics.load(sharedArrayBufferUin32Array[0]);
    return t2 - t1;
}
\end{lstlisting}

\label{alg_list:sharedArrayBufferWorker}[h]
\begin{lstlisting}[caption=counterWebworker.js: Webworker welcher die Zählvariable in einer Endlosschleife iteriert]
self.addEventListener('message', (m) => {
  // Create an Int32Array on top of the shared memory array 
  const sharedArray = new Uint32Array(m.data);
  while{true}{
    sharedArray[0]++;
  }
});
\end{lstlisting}

%moved from section grundlagen
\section{Eviction-Set Algorithmus in der Javascript-Umgebung}
%\section{Cache-Angriff in JavaScript und Webassembly}
Der wichtigste Teil für ein Prime and Probe Angriff ist die Fähigkeit zuverlässig Eviction-Sets zu finden. Wie im Grundlagenkapitel beschrieben führt die CPU das Cache-Mapping anhand der physisches Adressen durch. Webassembly emuliert eine 32 Bit Umgebung, welche die internen Adressen in virtuelle Adressen des Hostprozesses, hier der Browser, übersetzt. 
Webassembly-Code verwendet nur Adressen der emulierten Umgebung und hat keinerlei Zugriff auf das Mapping zu den virtuellen Adressen. 

Somit sind die physischen Adressen in Webassembly durch gleich 2 Abstraktionsschichten geschützt. 
Jedoch lässt sich die Eigenschaft ausnutzen, dass im Betriebssystem 4-KiB alignte Pages existieren. 
Das heißt die letzten 12 Bit der virtuellen und physischen Adresse sind identisch. 
Des Weiteren alloziert Webassembly 4KiB-Größe Blöcke, sodass die 12 letzten Bits der Webassembly-Adresse mit der virtuellen und physischen übereinstimmen.

Um aus dieser Eigenschaft Kapital zu schlagen, wird ein Array mindestens entsprechend der Größe des L3-Caches in Webassembly angelegt. Im Folgenden soll wieder der Intel i7-4770 mit 8 MiB großem L3-Cache als Basis dienen. In diesem Array sind nun $x$ Blöcke der Größe jeweils 4 KiB dessen letzten 12 Adressbits mit der physischen Adresse übereinstimmen. Im sogenannten Adresspool seien nun die Adressen des Arrays bei denen die letzten 12 Bit gleich sind, also insgesamt $x$ Stück.

Der i7-4770 besitzt 8192 Cache-Sets die auf 4 Slices aufgeteilt sind, wobei für das Mapping der 2048 Cache-Sets innerhalb eines Slices nur die untersten 18 Bit der physischen Adresse relevant sind. Dabei bestimmen die Bits 6 bis 17 eindeutig das Cache-Set und Bits 0 bis 5 das Offset innerhalb der Cache-Line.

In welchem der 4 Slices die Daten landen, wird anhand der Adressbits 18 bis 63 bestimmt.
Durch die Kenntnis der untersten 12 Bits der physischen Adresse, sind gleichzeitig 6 Bit (6 bis 11) bekannt, welche für die Zuordnung zu den Cache-Sets verantwortlich sind.

Angenommen im Pool sind ausschließlich Adressen, bei denen die 12 untersten Bit jedes der 4 KiB großen Blöcke auf 0 gesetzt sind, dann kann erwartet werden, dass im Mittel jede 128. 
Adresse auf das gleiche Cache-Set gemappt wird. Es gibt 8192 Möglichkeiten eine Adresse einem Cache-Sets zuzuordnen, also 13 Bit an Unsicherheit.
Durch die Kenntnis der untersten 12 Bit der Adresse, sind davon 6 Bit bekannt, welche für eine eindeutige Zuordnung sorgen. 
Es bleiben noch 7 Bit an Unsicherheit, die durch Kenntnis der restlichen Adressbits beseitigt werden könnten.

\subsection{Wahl der Adresspoolgröße}

Um die Anzahl der Blöcke beziehungsweise die Arraygröße in Webassembly sinnvoll zu bestimmen, kann zuerst die vereinfachte Annahme getroffen werden, dass die physischen Adressbits 12 bis 63 der 4 KiB Blöcke zufällig sind. 
Die unbekannte Cache-Mapping Funktion nimmt nun die zufälligen Bits 12 bis 63 und die auf 0 gesetzten Bits 0 bis 11 entgegen und gibt eines von 128 möglichen Cache-Sets zurück.
Ziel ist es mit einer Poolgröße $x$ und einer hohen Wahrscheinlichkeit für jedes Cache-Sets 16 Zuordnungen beziehungsweise ein Eviction Set zu finden.
Gesucht sei die Wahrscheinlichkeit bei einer Poolgröße $x$ mindestens 16 Zuordnungen zu einem fixen Cache-Set $cs$ bei 128 Möglichkeiten zu finden.
Hierfür eignet sich das Urnenmodell mit Zurücklegen ohne Beachtung der Reihenfolge, wobei die $x$ Adressen im Pool die Ziehungen und die Cache-Sets die Kugeln repräsentieren. Sei $P(count(cs)>=16)$ die Wahrscheinlichkeit dafür, dass in der gezogenen Folge mindestens 16 mal das Cache-Set $cs$ auftaucht, unter der Bedingung, dass die Poolgröße $\#add = x$.
Leichter ist es hier die Gegenwahrscheinlichkeit zu berechnen die mit
\begin{align*}
P(count(cs)<16|\#add = x) &=
\left( \sum\limits_{i=0}^{15}P(count(cs)=i|\#add = x) \right) \\&=
\left( \sum\limits_{i=0}^{15} {x \choose i} \frac{127}{128}^{x-i} \cdot \frac{1}{128}^i  \right)
\end{align*}
beschrieben ist.

Die Wahrscheinlichkeit des $P(count(cs)>=16)$ ist etwa bedeutend im ersten Szenario:
Angenommen die Angreiferin hat es auf eine bestimmte Adresse abgesehen und möchte ein korrelierendes Eviction-Set finden. 
Wie wahrscheinlich ist ein erfolgreicher Angriff beziehungsweise wie hoch ist die Wahrscheinlichkeit ein korrelierendes Eviction-Set zu finden? 
Das Diagramm \ref{fig:prob_specific_es} gibt die Erfolgswahrscheinlichkeiten für verschiedene $x$-Werte an.
\label{fig:prob_specific_es}
\begin{center}
\begin{scaletikzpicturetowidth}{\textwidth}
% Created by tikzDevice version 0.12 on 2018-09-02 02:49:51
% !TEX encoding = UTF-8 Unicode
\begin{tikzpicture}[scale=\tikzscale,y=21pt]
\definecolor{fillColor}{RGB}{255,255,255}
\path[use as bounding box,fill=fillColor,fill opacity=0.00] (0,0) rectangle (505.89,505.89);
\begin{scope}
\path[clip] (  0.00,  0.00) rectangle (505.89,505.89);
\definecolor{drawColor}{RGB}{255,255,255}
\definecolor{fillColor}{RGB}{255,255,255}
\path[draw=drawColor,line width= 0.6pt,line join=round,line cap=round,fill=fillColor] (  0.00,  0.00) rectangle (505.89,505.89);
\end{scope}
\begin{scope}
\path[clip] ( 39.22, 32.09) rectangle (500.39,500.39);
\definecolor{fillColor}{gray}{0.92}
\path[fill=fillColor] ( 39.22, 32.09) rectangle (500.39,500.39);
\definecolor{drawColor}{RGB}{255,255,255}
\path[draw=drawColor,line width= 0.3pt,line join=round] ( 39.22, 95.95) --
	(500.39, 95.95);
\path[draw=drawColor,line width= 0.3pt,line join=round] ( 39.22,181.09) --
	(500.39,181.09);
\path[draw=drawColor,line width= 0.3pt,line join=round] ( 39.22,266.24) --
	(500.39,266.24);
\path[draw=drawColor,line width= 0.3pt,line join=round] ( 39.22,351.38) --
	(500.39,351.38);
\path[draw=drawColor,line width= 0.3pt,line join=round] ( 39.22,436.53) --
	(500.39,436.53);
\path[draw=drawColor,line width= 0.3pt,line join=round] (102.11, 32.09) --
	(102.11,500.39);
\path[draw=drawColor,line width= 0.3pt,line join=round] (185.96, 32.09) --
	(185.96,500.39);
\path[draw=drawColor,line width= 0.3pt,line join=round] (269.81, 32.09) --
	(269.81,500.39);
\path[draw=drawColor,line width= 0.3pt,line join=round] (353.65, 32.09) --
	(353.65,500.39);
\path[draw=drawColor,line width= 0.3pt,line join=round] (437.50, 32.09) --
	(437.50,500.39);
\path[draw=drawColor,line width= 0.6pt,line join=round] ( 39.22, 53.37) --
	(500.39, 53.37);
\path[draw=drawColor,line width= 0.6pt,line join=round] ( 39.22,138.52) --
	(500.39,138.52);
\path[draw=drawColor,line width= 0.6pt,line join=round] ( 39.22,223.66) --
	(500.39,223.66);
\path[draw=drawColor,line width= 0.6pt,line join=round] ( 39.22,308.81) --
	(500.39,308.81);
\path[draw=drawColor,line width= 0.6pt,line join=round] ( 39.22,393.96) --
	(500.39,393.96);
\path[draw=drawColor,line width= 0.6pt,line join=round] ( 39.22,479.10) --
	(500.39,479.10);
\path[draw=drawColor,line width= 0.6pt,line join=round] ( 60.18, 32.09) --
	( 60.18,500.39);
\path[draw=drawColor,line width= 0.6pt,line join=round] (144.03, 32.09) --
	(144.03,500.39);
\path[draw=drawColor,line width= 0.6pt,line join=round] (227.88, 32.09) --
	(227.88,500.39);
\path[draw=drawColor,line width= 0.6pt,line join=round] (311.73, 32.09) --
	(311.73,500.39);
\path[draw=drawColor,line width= 0.6pt,line join=round] (395.58, 32.09) --
	(395.58,500.39);
\path[draw=drawColor,line width= 0.6pt,line join=round] (479.43, 32.09) --
	(479.43,500.39);
\definecolor{drawColor}{RGB}{0,0,0}
\definecolor{fillColor}{RGB}{0,0,0}
\path[draw=drawColor,line width= 0.4pt,line join=round,line cap=round,fill=fillColor] ( 70.25, 82.02) circle (  1.96);
\path[draw=drawColor,line width= 0.4pt,line join=round,line cap=round,fill=fillColor] ( 70.46, 82.68) circle (  1.96);
\path[draw=drawColor,line width= 0.4pt,line join=round,line cap=round,fill=fillColor] ( 70.67, 83.34) circle (  1.96);
\path[draw=drawColor,line width= 0.4pt,line join=round,line cap=round,fill=fillColor] ( 70.88, 84.01) circle (  1.96);
\path[draw=drawColor,line width= 0.4pt,line join=round,line cap=round,fill=fillColor] ( 71.08, 84.67) circle (  1.96);
\path[draw=drawColor,line width= 0.4pt,line join=round,line cap=round,fill=fillColor] ( 71.29, 85.33) circle (  1.96);
\path[draw=drawColor,line width= 0.4pt,line join=round,line cap=round,fill=fillColor] ( 71.50, 85.99) circle (  1.96);
\path[draw=drawColor,line width= 0.4pt,line join=round,line cap=round,fill=fillColor] ( 71.71, 86.65) circle (  1.96);
\path[draw=drawColor,line width= 0.4pt,line join=round,line cap=round,fill=fillColor] ( 71.92, 87.31) circle (  1.96);
\path[draw=drawColor,line width= 0.4pt,line join=round,line cap=round,fill=fillColor] ( 72.13, 87.97) circle (  1.96);
\path[draw=drawColor,line width= 0.4pt,line join=round,line cap=round,fill=fillColor] ( 72.34, 88.63) circle (  1.96);
\path[draw=drawColor,line width= 0.4pt,line join=round,line cap=round,fill=fillColor] ( 72.55, 89.29) circle (  1.96);
\path[draw=drawColor,line width= 0.4pt,line join=round,line cap=round,fill=fillColor] ( 72.76, 89.94) circle (  1.96);
\path[draw=drawColor,line width= 0.4pt,line join=round,line cap=round,fill=fillColor] ( 72.97, 90.60) circle (  1.96);
\path[draw=drawColor,line width= 0.4pt,line join=round,line cap=round,fill=fillColor] ( 73.18, 91.26) circle (  1.96);
\path[draw=drawColor,line width= 0.4pt,line join=round,line cap=round,fill=fillColor] ( 73.39, 91.92) circle (  1.96);
\path[draw=drawColor,line width= 0.4pt,line join=round,line cap=round,fill=fillColor] ( 73.60, 92.58) circle (  1.96);
\path[draw=drawColor,line width= 0.4pt,line join=round,line cap=round,fill=fillColor] ( 73.81, 93.23) circle (  1.96);
\path[draw=drawColor,line width= 0.4pt,line join=round,line cap=round,fill=fillColor] ( 74.02, 93.89) circle (  1.96);
\path[draw=drawColor,line width= 0.4pt,line join=round,line cap=round,fill=fillColor] ( 74.23, 94.54) circle (  1.96);
\path[draw=drawColor,line width= 0.4pt,line join=round,line cap=round,fill=fillColor] ( 74.44, 95.20) circle (  1.96);
\path[draw=drawColor,line width= 0.4pt,line join=round,line cap=round,fill=fillColor] ( 74.65, 95.86) circle (  1.96);
\path[draw=drawColor,line width= 0.4pt,line join=round,line cap=round,fill=fillColor] ( 74.86, 96.51) circle (  1.96);
\path[draw=drawColor,line width= 0.4pt,line join=round,line cap=round,fill=fillColor] ( 75.07, 97.17) circle (  1.96);
\path[draw=drawColor,line width= 0.4pt,line join=round,line cap=round,fill=fillColor] ( 75.28, 97.82) circle (  1.96);
\path[draw=drawColor,line width= 0.4pt,line join=round,line cap=round,fill=fillColor] ( 75.49, 98.47) circle (  1.96);
\path[draw=drawColor,line width= 0.4pt,line join=round,line cap=round,fill=fillColor] ( 75.70, 99.13) circle (  1.96);
\path[draw=drawColor,line width= 0.4pt,line join=round,line cap=round,fill=fillColor] ( 75.91, 99.78) circle (  1.96);
\path[draw=drawColor,line width= 0.4pt,line join=round,line cap=round,fill=fillColor] ( 76.12,100.43) circle (  1.96);
\path[draw=drawColor,line width= 0.4pt,line join=round,line cap=round,fill=fillColor] ( 76.33,101.08) circle (  1.96);
\path[draw=drawColor,line width= 0.4pt,line join=round,line cap=round,fill=fillColor] ( 76.53,101.74) circle (  1.96);
\path[draw=drawColor,line width= 0.4pt,line join=round,line cap=round,fill=fillColor] ( 76.74,102.39) circle (  1.96);
\path[draw=drawColor,line width= 0.4pt,line join=round,line cap=round,fill=fillColor] ( 76.95,103.04) circle (  1.96);
\path[draw=drawColor,line width= 0.4pt,line join=round,line cap=round,fill=fillColor] ( 77.16,103.69) circle (  1.96);
\path[draw=drawColor,line width= 0.4pt,line join=round,line cap=round,fill=fillColor] ( 77.37,104.34) circle (  1.96);
\path[draw=drawColor,line width= 0.4pt,line join=round,line cap=round,fill=fillColor] ( 77.58,104.99) circle (  1.96);
\path[draw=drawColor,line width= 0.4pt,line join=round,line cap=round,fill=fillColor] ( 77.79,105.64) circle (  1.96);
\path[draw=drawColor,line width= 0.4pt,line join=round,line cap=round,fill=fillColor] ( 78.00,106.29) circle (  1.96);
\path[draw=drawColor,line width= 0.4pt,line join=round,line cap=round,fill=fillColor] ( 78.21,106.94) circle (  1.96);
\path[draw=drawColor,line width= 0.4pt,line join=round,line cap=round,fill=fillColor] ( 78.42,107.59) circle (  1.96);
\path[draw=drawColor,line width= 0.4pt,line join=round,line cap=round,fill=fillColor] ( 78.63,108.23) circle (  1.96);
\path[draw=drawColor,line width= 0.4pt,line join=round,line cap=round,fill=fillColor] ( 78.84,108.88) circle (  1.96);
\path[draw=drawColor,line width= 0.4pt,line join=round,line cap=round,fill=fillColor] ( 79.05,109.53) circle (  1.96);
\path[draw=drawColor,line width= 0.4pt,line join=round,line cap=round,fill=fillColor] ( 79.26,110.17) circle (  1.96);
\path[draw=drawColor,line width= 0.4pt,line join=round,line cap=round,fill=fillColor] ( 79.47,110.82) circle (  1.96);
\path[draw=drawColor,line width= 0.4pt,line join=round,line cap=round,fill=fillColor] ( 79.68,111.47) circle (  1.96);
\path[draw=drawColor,line width= 0.4pt,line join=round,line cap=round,fill=fillColor] ( 79.89,112.11) circle (  1.96);
\path[draw=drawColor,line width= 0.4pt,line join=round,line cap=round,fill=fillColor] ( 80.10,112.76) circle (  1.96);
\path[draw=drawColor,line width= 0.4pt,line join=round,line cap=round,fill=fillColor] ( 80.31,113.40) circle (  1.96);
\path[draw=drawColor,line width= 0.4pt,line join=round,line cap=round,fill=fillColor] ( 80.52,114.05) circle (  1.96);
\path[draw=drawColor,line width= 0.4pt,line join=round,line cap=round,fill=fillColor] ( 80.73,114.69) circle (  1.96);
\path[draw=drawColor,line width= 0.4pt,line join=round,line cap=round,fill=fillColor] ( 80.94,115.33) circle (  1.96);
\path[draw=drawColor,line width= 0.4pt,line join=round,line cap=round,fill=fillColor] ( 81.15,115.97) circle (  1.96);
\path[draw=drawColor,line width= 0.4pt,line join=round,line cap=round,fill=fillColor] ( 81.36,116.62) circle (  1.96);
\path[draw=drawColor,line width= 0.4pt,line join=round,line cap=round,fill=fillColor] ( 81.57,117.26) circle (  1.96);
\path[draw=drawColor,line width= 0.4pt,line join=round,line cap=round,fill=fillColor] ( 81.78,117.90) circle (  1.96);
\path[draw=drawColor,line width= 0.4pt,line join=round,line cap=round,fill=fillColor] ( 81.99,118.54) circle (  1.96);
\path[draw=drawColor,line width= 0.4pt,line join=round,line cap=round,fill=fillColor] ( 82.19,119.18) circle (  1.96);
\path[draw=drawColor,line width= 0.4pt,line join=round,line cap=round,fill=fillColor] ( 82.40,119.82) circle (  1.96);
\path[draw=drawColor,line width= 0.4pt,line join=round,line cap=round,fill=fillColor] ( 82.61,120.46) circle (  1.96);
\path[draw=drawColor,line width= 0.4pt,line join=round,line cap=round,fill=fillColor] ( 82.82,121.10) circle (  1.96);
\path[draw=drawColor,line width= 0.4pt,line join=round,line cap=round,fill=fillColor] ( 83.03,121.74) circle (  1.96);
\path[draw=drawColor,line width= 0.4pt,line join=round,line cap=round,fill=fillColor] ( 83.24,122.38) circle (  1.96);
\path[draw=drawColor,line width= 0.4pt,line join=round,line cap=round,fill=fillColor] ( 83.45,123.02) circle (  1.96);
\path[draw=drawColor,line width= 0.4pt,line join=round,line cap=round,fill=fillColor] ( 83.66,123.65) circle (  1.96);
\path[draw=drawColor,line width= 0.4pt,line join=round,line cap=round,fill=fillColor] ( 83.87,124.29) circle (  1.96);
\path[draw=drawColor,line width= 0.4pt,line join=round,line cap=round,fill=fillColor] ( 84.08,124.93) circle (  1.96);
\path[draw=drawColor,line width= 0.4pt,line join=round,line cap=round,fill=fillColor] ( 84.29,125.56) circle (  1.96);
\path[draw=drawColor,line width= 0.4pt,line join=round,line cap=round,fill=fillColor] ( 84.50,126.20) circle (  1.96);
\path[draw=drawColor,line width= 0.4pt,line join=round,line cap=round,fill=fillColor] ( 84.71,126.84) circle (  1.96);
\path[draw=drawColor,line width= 0.4pt,line join=round,line cap=round,fill=fillColor] ( 84.92,127.47) circle (  1.96);
\path[draw=drawColor,line width= 0.4pt,line join=round,line cap=round,fill=fillColor] ( 85.13,128.10) circle (  1.96);
\path[draw=drawColor,line width= 0.4pt,line join=round,line cap=round,fill=fillColor] ( 85.34,128.74) circle (  1.96);
\path[draw=drawColor,line width= 0.4pt,line join=round,line cap=round,fill=fillColor] ( 85.55,129.37) circle (  1.96);
\path[draw=drawColor,line width= 0.4pt,line join=round,line cap=round,fill=fillColor] ( 85.76,130.00) circle (  1.96);
\path[draw=drawColor,line width= 0.4pt,line join=round,line cap=round,fill=fillColor] ( 85.97,130.64) circle (  1.96);
\path[draw=drawColor,line width= 0.4pt,line join=round,line cap=round,fill=fillColor] ( 86.18,131.27) circle (  1.96);
\path[draw=drawColor,line width= 0.4pt,line join=round,line cap=round,fill=fillColor] ( 86.39,131.90) circle (  1.96);
\path[draw=drawColor,line width= 0.4pt,line join=round,line cap=round,fill=fillColor] ( 86.60,132.53) circle (  1.96);
\path[draw=drawColor,line width= 0.4pt,line join=round,line cap=round,fill=fillColor] ( 86.81,133.16) circle (  1.96);
\path[draw=drawColor,line width= 0.4pt,line join=round,line cap=round,fill=fillColor] ( 87.02,133.79) circle (  1.96);
\path[draw=drawColor,line width= 0.4pt,line join=round,line cap=round,fill=fillColor] ( 87.23,134.42) circle (  1.96);
\path[draw=drawColor,line width= 0.4pt,line join=round,line cap=round,fill=fillColor] ( 87.44,135.05) circle (  1.96);
\path[draw=drawColor,line width= 0.4pt,line join=round,line cap=round,fill=fillColor] ( 87.64,135.68) circle (  1.96);
\path[draw=drawColor,line width= 0.4pt,line join=round,line cap=round,fill=fillColor] ( 87.85,136.31) circle (  1.96);
\path[draw=drawColor,line width= 0.4pt,line join=round,line cap=round,fill=fillColor] ( 88.06,136.93) circle (  1.96);
\path[draw=drawColor,line width= 0.4pt,line join=round,line cap=round,fill=fillColor] ( 88.27,137.56) circle (  1.96);
\path[draw=drawColor,line width= 0.4pt,line join=round,line cap=round,fill=fillColor] ( 88.48,138.19) circle (  1.96);
\path[draw=drawColor,line width= 0.4pt,line join=round,line cap=round,fill=fillColor] ( 88.69,138.81) circle (  1.96);
\path[draw=drawColor,line width= 0.4pt,line join=round,line cap=round,fill=fillColor] ( 88.90,139.44) circle (  1.96);
\path[draw=drawColor,line width= 0.4pt,line join=round,line cap=round,fill=fillColor] ( 89.11,140.06) circle (  1.96);
\path[draw=drawColor,line width= 0.4pt,line join=round,line cap=round,fill=fillColor] ( 89.32,140.69) circle (  1.96);
\path[draw=drawColor,line width= 0.4pt,line join=round,line cap=round,fill=fillColor] ( 89.53,141.31) circle (  1.96);
\path[draw=drawColor,line width= 0.4pt,line join=round,line cap=round,fill=fillColor] ( 89.74,141.94) circle (  1.96);
\path[draw=drawColor,line width= 0.4pt,line join=round,line cap=round,fill=fillColor] ( 89.95,142.56) circle (  1.96);
\path[draw=drawColor,line width= 0.4pt,line join=round,line cap=round,fill=fillColor] ( 90.16,143.18) circle (  1.96);
\path[draw=drawColor,line width= 0.4pt,line join=round,line cap=round,fill=fillColor] ( 90.37,143.80) circle (  1.96);
\path[draw=drawColor,line width= 0.4pt,line join=round,line cap=round,fill=fillColor] ( 90.58,144.43) circle (  1.96);
\path[draw=drawColor,line width= 0.4pt,line join=round,line cap=round,fill=fillColor] ( 90.79,145.05) circle (  1.96);
\path[draw=drawColor,line width= 0.4pt,line join=round,line cap=round,fill=fillColor] ( 91.00,145.67) circle (  1.96);
\path[draw=drawColor,line width= 0.4pt,line join=round,line cap=round,fill=fillColor] ( 91.21,146.29) circle (  1.96);
\path[draw=drawColor,line width= 0.4pt,line join=round,line cap=round,fill=fillColor] ( 91.42,146.91) circle (  1.96);
\path[draw=drawColor,line width= 0.4pt,line join=round,line cap=round,fill=fillColor] ( 91.63,147.53) circle (  1.96);
\path[draw=drawColor,line width= 0.4pt,line join=round,line cap=round,fill=fillColor] ( 91.84,148.14) circle (  1.96);
\path[draw=drawColor,line width= 0.4pt,line join=round,line cap=round,fill=fillColor] ( 92.05,148.76) circle (  1.96);
\path[draw=drawColor,line width= 0.4pt,line join=round,line cap=round,fill=fillColor] ( 92.26,149.38) circle (  1.96);
\path[draw=drawColor,line width= 0.4pt,line join=round,line cap=round,fill=fillColor] ( 92.47,150.00) circle (  1.96);
\path[draw=drawColor,line width= 0.4pt,line join=round,line cap=round,fill=fillColor] ( 92.68,150.61) circle (  1.96);
\path[draw=drawColor,line width= 0.4pt,line join=round,line cap=round,fill=fillColor] ( 92.89,151.23) circle (  1.96);
\path[draw=drawColor,line width= 0.4pt,line join=round,line cap=round,fill=fillColor] ( 93.09,151.84) circle (  1.96);
\path[draw=drawColor,line width= 0.4pt,line join=round,line cap=round,fill=fillColor] ( 93.30,152.46) circle (  1.96);
\path[draw=drawColor,line width= 0.4pt,line join=round,line cap=round,fill=fillColor] ( 93.51,153.07) circle (  1.96);
\path[draw=drawColor,line width= 0.4pt,line join=round,line cap=round,fill=fillColor] ( 93.72,153.69) circle (  1.96);
\path[draw=drawColor,line width= 0.4pt,line join=round,line cap=round,fill=fillColor] ( 93.93,154.30) circle (  1.96);
\path[draw=drawColor,line width= 0.4pt,line join=round,line cap=round,fill=fillColor] ( 94.14,154.91) circle (  1.96);
\path[draw=drawColor,line width= 0.4pt,line join=round,line cap=round,fill=fillColor] ( 94.35,155.52) circle (  1.96);
\path[draw=drawColor,line width= 0.4pt,line join=round,line cap=round,fill=fillColor] ( 94.56,156.14) circle (  1.96);
\path[draw=drawColor,line width= 0.4pt,line join=round,line cap=round,fill=fillColor] ( 94.77,156.75) circle (  1.96);
\path[draw=drawColor,line width= 0.4pt,line join=round,line cap=round,fill=fillColor] ( 94.98,157.36) circle (  1.96);
\path[draw=drawColor,line width= 0.4pt,line join=round,line cap=round,fill=fillColor] ( 95.19,157.97) circle (  1.96);
\path[draw=drawColor,line width= 0.4pt,line join=round,line cap=round,fill=fillColor] ( 95.40,158.58) circle (  1.96);
\path[draw=drawColor,line width= 0.4pt,line join=round,line cap=round,fill=fillColor] ( 95.61,159.18) circle (  1.96);
\path[draw=drawColor,line width= 0.4pt,line join=round,line cap=round,fill=fillColor] ( 95.82,159.79) circle (  1.96);
\path[draw=drawColor,line width= 0.4pt,line join=round,line cap=round,fill=fillColor] ( 96.03,160.40) circle (  1.96);
\path[draw=drawColor,line width= 0.4pt,line join=round,line cap=round,fill=fillColor] ( 96.24,161.01) circle (  1.96);
\path[draw=drawColor,line width= 0.4pt,line join=round,line cap=round,fill=fillColor] ( 96.45,161.61) circle (  1.96);
\path[draw=drawColor,line width= 0.4pt,line join=round,line cap=round,fill=fillColor] ( 96.66,162.22) circle (  1.96);
\path[draw=drawColor,line width= 0.4pt,line join=round,line cap=round,fill=fillColor] ( 96.87,162.83) circle (  1.96);
\path[draw=drawColor,line width= 0.4pt,line join=round,line cap=round,fill=fillColor] ( 97.08,163.43) circle (  1.96);
\path[draw=drawColor,line width= 0.4pt,line join=round,line cap=round,fill=fillColor] ( 97.29,164.03) circle (  1.96);
\path[draw=drawColor,line width= 0.4pt,line join=round,line cap=round,fill=fillColor] ( 97.50,164.64) circle (  1.96);
\path[draw=drawColor,line width= 0.4pt,line join=round,line cap=round,fill=fillColor] ( 97.71,165.24) circle (  1.96);
\path[draw=drawColor,line width= 0.4pt,line join=round,line cap=round,fill=fillColor] ( 97.92,165.84) circle (  1.96);
\path[draw=drawColor,line width= 0.4pt,line join=round,line cap=round,fill=fillColor] ( 98.13,166.45) circle (  1.96);
\path[draw=drawColor,line width= 0.4pt,line join=round,line cap=round,fill=fillColor] ( 98.34,167.05) circle (  1.96);
\path[draw=drawColor,line width= 0.4pt,line join=round,line cap=round,fill=fillColor] ( 98.55,167.65) circle (  1.96);
\path[draw=drawColor,line width= 0.4pt,line join=round,line cap=round,fill=fillColor] ( 98.75,168.25) circle (  1.96);
\path[draw=drawColor,line width= 0.4pt,line join=round,line cap=round,fill=fillColor] ( 98.96,168.85) circle (  1.96);
\path[draw=drawColor,line width= 0.4pt,line join=round,line cap=round,fill=fillColor] ( 99.17,169.45) circle (  1.96);
\path[draw=drawColor,line width= 0.4pt,line join=round,line cap=round,fill=fillColor] ( 99.38,170.05) circle (  1.96);
\path[draw=drawColor,line width= 0.4pt,line join=round,line cap=round,fill=fillColor] ( 99.59,170.65) circle (  1.96);
\path[draw=drawColor,line width= 0.4pt,line join=round,line cap=round,fill=fillColor] ( 99.80,171.24) circle (  1.96);
\path[draw=drawColor,line width= 0.4pt,line join=round,line cap=round,fill=fillColor] (100.01,171.84) circle (  1.96);
\path[draw=drawColor,line width= 0.4pt,line join=round,line cap=round,fill=fillColor] (100.22,172.44) circle (  1.96);
\path[draw=drawColor,line width= 0.4pt,line join=round,line cap=round,fill=fillColor] (100.43,173.03) circle (  1.96);
\path[draw=drawColor,line width= 0.4pt,line join=round,line cap=round,fill=fillColor] (100.64,173.63) circle (  1.96);
\path[draw=drawColor,line width= 0.4pt,line join=round,line cap=round,fill=fillColor] (100.85,174.22) circle (  1.96);
\path[draw=drawColor,line width= 0.4pt,line join=round,line cap=round,fill=fillColor] (101.06,174.82) circle (  1.96);
\path[draw=drawColor,line width= 0.4pt,line join=round,line cap=round,fill=fillColor] (101.27,175.41) circle (  1.96);
\path[draw=drawColor,line width= 0.4pt,line join=round,line cap=round,fill=fillColor] (101.48,176.00) circle (  1.96);
\path[draw=drawColor,line width= 0.4pt,line join=round,line cap=round,fill=fillColor] (101.69,176.59) circle (  1.96);
\path[draw=drawColor,line width= 0.4pt,line join=round,line cap=round,fill=fillColor] (101.90,177.19) circle (  1.96);
\path[draw=drawColor,line width= 0.4pt,line join=round,line cap=round,fill=fillColor] (102.11,177.78) circle (  1.96);
\path[draw=drawColor,line width= 0.4pt,line join=round,line cap=round,fill=fillColor] (102.32,178.37) circle (  1.96);
\path[draw=drawColor,line width= 0.4pt,line join=round,line cap=round,fill=fillColor] (102.53,178.96) circle (  1.96);
\path[draw=drawColor,line width= 0.4pt,line join=round,line cap=round,fill=fillColor] (102.74,179.55) circle (  1.96);
\path[draw=drawColor,line width= 0.4pt,line join=round,line cap=round,fill=fillColor] (102.95,180.14) circle (  1.96);
\path[draw=drawColor,line width= 0.4pt,line join=round,line cap=round,fill=fillColor] (103.16,180.72) circle (  1.96);
\path[draw=drawColor,line width= 0.4pt,line join=round,line cap=round,fill=fillColor] (103.37,181.31) circle (  1.96);
\path[draw=drawColor,line width= 0.4pt,line join=round,line cap=round,fill=fillColor] (103.58,181.90) circle (  1.96);
\path[draw=drawColor,line width= 0.4pt,line join=round,line cap=round,fill=fillColor] (103.79,182.48) circle (  1.96);
\path[draw=drawColor,line width= 0.4pt,line join=round,line cap=round,fill=fillColor] (104.00,183.07) circle (  1.96);
\path[draw=drawColor,line width= 0.4pt,line join=round,line cap=round,fill=fillColor] (104.20,183.66) circle (  1.96);
\path[draw=drawColor,line width= 0.4pt,line join=round,line cap=round,fill=fillColor] (104.41,184.24) circle (  1.96);
\path[draw=drawColor,line width= 0.4pt,line join=round,line cap=round,fill=fillColor] (104.62,184.82) circle (  1.96);
\path[draw=drawColor,line width= 0.4pt,line join=round,line cap=round,fill=fillColor] (104.83,185.41) circle (  1.96);
\path[draw=drawColor,line width= 0.4pt,line join=round,line cap=round,fill=fillColor] (105.04,185.99) circle (  1.96);
\path[draw=drawColor,line width= 0.4pt,line join=round,line cap=round,fill=fillColor] (105.25,186.57) circle (  1.96);
\path[draw=drawColor,line width= 0.4pt,line join=round,line cap=round,fill=fillColor] (105.46,187.15) circle (  1.96);
\path[draw=drawColor,line width= 0.4pt,line join=round,line cap=round,fill=fillColor] (105.67,187.73) circle (  1.96);
\path[draw=drawColor,line width= 0.4pt,line join=round,line cap=round,fill=fillColor] (105.88,188.31) circle (  1.96);
\path[draw=drawColor,line width= 0.4pt,line join=round,line cap=round,fill=fillColor] (106.09,188.89) circle (  1.96);
\path[draw=drawColor,line width= 0.4pt,line join=round,line cap=round,fill=fillColor] (106.30,189.47) circle (  1.96);
\path[draw=drawColor,line width= 0.4pt,line join=round,line cap=round,fill=fillColor] (106.51,190.05) circle (  1.96);
\path[draw=drawColor,line width= 0.4pt,line join=round,line cap=round,fill=fillColor] (106.72,190.63) circle (  1.96);
\path[draw=drawColor,line width= 0.4pt,line join=round,line cap=round,fill=fillColor] (106.93,191.21) circle (  1.96);
\path[draw=drawColor,line width= 0.4pt,line join=round,line cap=round,fill=fillColor] (107.14,191.78) circle (  1.96);
\path[draw=drawColor,line width= 0.4pt,line join=round,line cap=round,fill=fillColor] (107.35,192.36) circle (  1.96);
\path[draw=drawColor,line width= 0.4pt,line join=round,line cap=round,fill=fillColor] (107.56,192.93) circle (  1.96);
\path[draw=drawColor,line width= 0.4pt,line join=round,line cap=round,fill=fillColor] (107.77,193.51) circle (  1.96);
\path[draw=drawColor,line width= 0.4pt,line join=round,line cap=round,fill=fillColor] (107.98,194.08) circle (  1.96);
\path[draw=drawColor,line width= 0.4pt,line join=round,line cap=round,fill=fillColor] (108.19,194.66) circle (  1.96);
\path[draw=drawColor,line width= 0.4pt,line join=round,line cap=round,fill=fillColor] (108.40,195.23) circle (  1.96);
\path[draw=drawColor,line width= 0.4pt,line join=round,line cap=round,fill=fillColor] (108.61,195.80) circle (  1.96);
\path[draw=drawColor,line width= 0.4pt,line join=round,line cap=round,fill=fillColor] (108.82,196.37) circle (  1.96);
\path[draw=drawColor,line width= 0.4pt,line join=round,line cap=round,fill=fillColor] (109.03,196.94) circle (  1.96);
\path[draw=drawColor,line width= 0.4pt,line join=round,line cap=round,fill=fillColor] (109.24,197.51) circle (  1.96);
\path[draw=drawColor,line width= 0.4pt,line join=round,line cap=round,fill=fillColor] (109.45,198.08) circle (  1.96);
\path[draw=drawColor,line width= 0.4pt,line join=round,line cap=round,fill=fillColor] (109.66,198.65) circle (  1.96);
\path[draw=drawColor,line width= 0.4pt,line join=round,line cap=round,fill=fillColor] (109.86,199.22) circle (  1.96);
\path[draw=drawColor,line width= 0.4pt,line join=round,line cap=round,fill=fillColor] (110.07,199.79) circle (  1.96);
\path[draw=drawColor,line width= 0.4pt,line join=round,line cap=round,fill=fillColor] (110.28,200.36) circle (  1.96);
\path[draw=drawColor,line width= 0.4pt,line join=round,line cap=round,fill=fillColor] (110.49,200.92) circle (  1.96);
\path[draw=drawColor,line width= 0.4pt,line join=round,line cap=round,fill=fillColor] (110.70,201.49) circle (  1.96);
\path[draw=drawColor,line width= 0.4pt,line join=round,line cap=round,fill=fillColor] (110.91,202.05) circle (  1.96);
\path[draw=drawColor,line width= 0.4pt,line join=round,line cap=round,fill=fillColor] (111.12,202.62) circle (  1.96);
\path[draw=drawColor,line width= 0.4pt,line join=round,line cap=round,fill=fillColor] (111.33,203.18) circle (  1.96);
\path[draw=drawColor,line width= 0.4pt,line join=round,line cap=round,fill=fillColor] (111.54,203.75) circle (  1.96);
\path[draw=drawColor,line width= 0.4pt,line join=round,line cap=round,fill=fillColor] (111.75,204.31) circle (  1.96);
\path[draw=drawColor,line width= 0.4pt,line join=round,line cap=round,fill=fillColor] (111.96,204.87) circle (  1.96);
\path[draw=drawColor,line width= 0.4pt,line join=round,line cap=round,fill=fillColor] (112.17,205.43) circle (  1.96);
\path[draw=drawColor,line width= 0.4pt,line join=round,line cap=round,fill=fillColor] (112.38,205.99) circle (  1.96);
\path[draw=drawColor,line width= 0.4pt,line join=round,line cap=round,fill=fillColor] (112.59,206.55) circle (  1.96);
\path[draw=drawColor,line width= 0.4pt,line join=round,line cap=round,fill=fillColor] (112.80,207.11) circle (  1.96);
\path[draw=drawColor,line width= 0.4pt,line join=round,line cap=round,fill=fillColor] (113.01,207.67) circle (  1.96);
\path[draw=drawColor,line width= 0.4pt,line join=round,line cap=round,fill=fillColor] (113.22,208.23) circle (  1.96);
\path[draw=drawColor,line width= 0.4pt,line join=round,line cap=round,fill=fillColor] (113.43,208.79) circle (  1.96);
\path[draw=drawColor,line width= 0.4pt,line join=round,line cap=round,fill=fillColor] (113.64,209.34) circle (  1.96);
\path[draw=drawColor,line width= 0.4pt,line join=round,line cap=round,fill=fillColor] (113.85,209.90) circle (  1.96);
\path[draw=drawColor,line width= 0.4pt,line join=round,line cap=round,fill=fillColor] (114.06,210.46) circle (  1.96);
\path[draw=drawColor,line width= 0.4pt,line join=round,line cap=round,fill=fillColor] (114.27,211.01) circle (  1.96);
\path[draw=drawColor,line width= 0.4pt,line join=round,line cap=round,fill=fillColor] (114.48,211.56) circle (  1.96);
\path[draw=drawColor,line width= 0.4pt,line join=round,line cap=round,fill=fillColor] (114.69,212.12) circle (  1.96);
\path[draw=drawColor,line width= 0.4pt,line join=round,line cap=round,fill=fillColor] (114.90,212.67) circle (  1.96);
\path[draw=drawColor,line width= 0.4pt,line join=round,line cap=round,fill=fillColor] (115.11,213.22) circle (  1.96);
\path[draw=drawColor,line width= 0.4pt,line join=round,line cap=round,fill=fillColor] (115.31,213.78) circle (  1.96);
\path[draw=drawColor,line width= 0.4pt,line join=round,line cap=round,fill=fillColor] (115.52,214.33) circle (  1.96);
\path[draw=drawColor,line width= 0.4pt,line join=round,line cap=round,fill=fillColor] (115.73,214.88) circle (  1.96);
\path[draw=drawColor,line width= 0.4pt,line join=round,line cap=round,fill=fillColor] (115.94,215.43) circle (  1.96);
\path[draw=drawColor,line width= 0.4pt,line join=round,line cap=round,fill=fillColor] (116.15,215.98) circle (  1.96);
\path[draw=drawColor,line width= 0.4pt,line join=round,line cap=round,fill=fillColor] (116.36,216.52) circle (  1.96);
\path[draw=drawColor,line width= 0.4pt,line join=round,line cap=round,fill=fillColor] (116.57,217.07) circle (  1.96);
\path[draw=drawColor,line width= 0.4pt,line join=round,line cap=round,fill=fillColor] (116.78,217.62) circle (  1.96);
\path[draw=drawColor,line width= 0.4pt,line join=round,line cap=round,fill=fillColor] (116.99,218.16) circle (  1.96);
\path[draw=drawColor,line width= 0.4pt,line join=round,line cap=round,fill=fillColor] (117.20,218.71) circle (  1.96);
\path[draw=drawColor,line width= 0.4pt,line join=round,line cap=round,fill=fillColor] (117.41,219.26) circle (  1.96);
\path[draw=drawColor,line width= 0.4pt,line join=round,line cap=round,fill=fillColor] (117.62,219.80) circle (  1.96);
\path[draw=drawColor,line width= 0.4pt,line join=round,line cap=round,fill=fillColor] (117.83,220.34) circle (  1.96);
\path[draw=drawColor,line width= 0.4pt,line join=round,line cap=round,fill=fillColor] (118.04,220.89) circle (  1.96);
\path[draw=drawColor,line width= 0.4pt,line join=round,line cap=round,fill=fillColor] (118.25,221.43) circle (  1.96);
\path[draw=drawColor,line width= 0.4pt,line join=round,line cap=round,fill=fillColor] (118.46,221.97) circle (  1.96);
\path[draw=drawColor,line width= 0.4pt,line join=round,line cap=round,fill=fillColor] (118.67,222.51) circle (  1.96);
\path[draw=drawColor,line width= 0.4pt,line join=round,line cap=round,fill=fillColor] (118.88,223.05) circle (  1.96);
\path[draw=drawColor,line width= 0.4pt,line join=round,line cap=round,fill=fillColor] (119.09,223.59) circle (  1.96);
\path[draw=drawColor,line width= 0.4pt,line join=round,line cap=round,fill=fillColor] (119.30,224.13) circle (  1.96);
\path[draw=drawColor,line width= 0.4pt,line join=round,line cap=round,fill=fillColor] (119.51,224.67) circle (  1.96);
\path[draw=drawColor,line width= 0.4pt,line join=round,line cap=round,fill=fillColor] (119.72,225.21) circle (  1.96);
\path[draw=drawColor,line width= 0.4pt,line join=round,line cap=round,fill=fillColor] (119.93,225.74) circle (  1.96);
\path[draw=drawColor,line width= 0.4pt,line join=round,line cap=round,fill=fillColor] (120.14,226.28) circle (  1.96);
\path[draw=drawColor,line width= 0.4pt,line join=round,line cap=round,fill=fillColor] (120.35,226.82) circle (  1.96);
\path[draw=drawColor,line width= 0.4pt,line join=round,line cap=round,fill=fillColor] (120.56,227.35) circle (  1.96);
\path[draw=drawColor,line width= 0.4pt,line join=round,line cap=round,fill=fillColor] (120.77,227.88) circle (  1.96);
\path[draw=drawColor,line width= 0.4pt,line join=round,line cap=round,fill=fillColor] (120.97,228.42) circle (  1.96);
\path[draw=drawColor,line width= 0.4pt,line join=round,line cap=round,fill=fillColor] (121.18,228.95) circle (  1.96);
\path[draw=drawColor,line width= 0.4pt,line join=round,line cap=round,fill=fillColor] (121.39,229.48) circle (  1.96);
\path[draw=drawColor,line width= 0.4pt,line join=round,line cap=round,fill=fillColor] (121.60,230.02) circle (  1.96);
\path[draw=drawColor,line width= 0.4pt,line join=round,line cap=round,fill=fillColor] (121.81,230.55) circle (  1.96);
\path[draw=drawColor,line width= 0.4pt,line join=round,line cap=round,fill=fillColor] (122.02,231.08) circle (  1.96);
\path[draw=drawColor,line width= 0.4pt,line join=round,line cap=round,fill=fillColor] (122.23,231.61) circle (  1.96);
\path[draw=drawColor,line width= 0.4pt,line join=round,line cap=round,fill=fillColor] (122.44,232.13) circle (  1.96);
\path[draw=drawColor,line width= 0.4pt,line join=round,line cap=round,fill=fillColor] (122.65,232.66) circle (  1.96);
\path[draw=drawColor,line width= 0.4pt,line join=round,line cap=round,fill=fillColor] (122.86,233.19) circle (  1.96);
\path[draw=drawColor,line width= 0.4pt,line join=round,line cap=round,fill=fillColor] (123.07,233.72) circle (  1.96);
\path[draw=drawColor,line width= 0.4pt,line join=round,line cap=round,fill=fillColor] (123.28,234.24) circle (  1.96);
\path[draw=drawColor,line width= 0.4pt,line join=round,line cap=round,fill=fillColor] (123.49,234.77) circle (  1.96);
\path[draw=drawColor,line width= 0.4pt,line join=round,line cap=round,fill=fillColor] (123.70,235.29) circle (  1.96);
\path[draw=drawColor,line width= 0.4pt,line join=round,line cap=round,fill=fillColor] (123.91,235.82) circle (  1.96);
\path[draw=drawColor,line width= 0.4pt,line join=round,line cap=round,fill=fillColor] (124.12,236.34) circle (  1.96);
\path[draw=drawColor,line width= 0.4pt,line join=round,line cap=round,fill=fillColor] (124.33,236.86) circle (  1.96);
\path[draw=drawColor,line width= 0.4pt,line join=round,line cap=round,fill=fillColor] (124.54,237.39) circle (  1.96);
\path[draw=drawColor,line width= 0.4pt,line join=round,line cap=round,fill=fillColor] (124.75,237.91) circle (  1.96);
\path[draw=drawColor,line width= 0.4pt,line join=round,line cap=round,fill=fillColor] (124.96,238.43) circle (  1.96);
\path[draw=drawColor,line width= 0.4pt,line join=round,line cap=round,fill=fillColor] (125.17,238.95) circle (  1.96);
\path[draw=drawColor,line width= 0.4pt,line join=round,line cap=round,fill=fillColor] (125.38,239.47) circle (  1.96);
\path[draw=drawColor,line width= 0.4pt,line join=round,line cap=round,fill=fillColor] (125.59,239.98) circle (  1.96);
\path[draw=drawColor,line width= 0.4pt,line join=round,line cap=round,fill=fillColor] (125.80,240.50) circle (  1.96);
\path[draw=drawColor,line width= 0.4pt,line join=round,line cap=round,fill=fillColor] (126.01,241.02) circle (  1.96);
\path[draw=drawColor,line width= 0.4pt,line join=round,line cap=round,fill=fillColor] (126.22,241.54) circle (  1.96);
\path[draw=drawColor,line width= 0.4pt,line join=round,line cap=round,fill=fillColor] (126.42,242.05) circle (  1.96);
\path[draw=drawColor,line width= 0.4pt,line join=round,line cap=round,fill=fillColor] (126.63,242.57) circle (  1.96);
\path[draw=drawColor,line width= 0.4pt,line join=round,line cap=round,fill=fillColor] (126.84,243.08) circle (  1.96);
\path[draw=drawColor,line width= 0.4pt,line join=round,line cap=round,fill=fillColor] (127.05,243.60) circle (  1.96);
\path[draw=drawColor,line width= 0.4pt,line join=round,line cap=round,fill=fillColor] (127.26,244.11) circle (  1.96);
\path[draw=drawColor,line width= 0.4pt,line join=round,line cap=round,fill=fillColor] (127.47,244.62) circle (  1.96);
\path[draw=drawColor,line width= 0.4pt,line join=round,line cap=round,fill=fillColor] (127.68,245.13) circle (  1.96);
\path[draw=drawColor,line width= 0.4pt,line join=round,line cap=round,fill=fillColor] (127.89,245.64) circle (  1.96);
\path[draw=drawColor,line width= 0.4pt,line join=round,line cap=round,fill=fillColor] (128.10,246.15) circle (  1.96);
\path[draw=drawColor,line width= 0.4pt,line join=round,line cap=round,fill=fillColor] (128.31,246.66) circle (  1.96);
\path[draw=drawColor,line width= 0.4pt,line join=round,line cap=round,fill=fillColor] (128.52,247.17) circle (  1.96);
\path[draw=drawColor,line width= 0.4pt,line join=round,line cap=round,fill=fillColor] (128.73,247.68) circle (  1.96);
\path[draw=drawColor,line width= 0.4pt,line join=round,line cap=round,fill=fillColor] (128.94,248.19) circle (  1.96);
\path[draw=drawColor,line width= 0.4pt,line join=round,line cap=round,fill=fillColor] (129.15,248.69) circle (  1.96);
\path[draw=drawColor,line width= 0.4pt,line join=round,line cap=round,fill=fillColor] (129.36,249.20) circle (  1.96);
\path[draw=drawColor,line width= 0.4pt,line join=round,line cap=round,fill=fillColor] (129.57,249.71) circle (  1.96);
\path[draw=drawColor,line width= 0.4pt,line join=round,line cap=round,fill=fillColor] (129.78,250.21) circle (  1.96);
\path[draw=drawColor,line width= 0.4pt,line join=round,line cap=round,fill=fillColor] (129.99,250.71) circle (  1.96);
\path[draw=drawColor,line width= 0.4pt,line join=round,line cap=round,fill=fillColor] (130.20,251.22) circle (  1.96);
\path[draw=drawColor,line width= 0.4pt,line join=round,line cap=round,fill=fillColor] (130.41,251.72) circle (  1.96);
\path[draw=drawColor,line width= 0.4pt,line join=round,line cap=round,fill=fillColor] (130.62,252.22) circle (  1.96);
\path[draw=drawColor,line width= 0.4pt,line join=round,line cap=round,fill=fillColor] (130.83,252.72) circle (  1.96);
\path[draw=drawColor,line width= 0.4pt,line join=round,line cap=round,fill=fillColor] (131.04,253.22) circle (  1.96);
\path[draw=drawColor,line width= 0.4pt,line join=round,line cap=round,fill=fillColor] (131.25,253.72) circle (  1.96);
\path[draw=drawColor,line width= 0.4pt,line join=round,line cap=round,fill=fillColor] (131.46,254.22) circle (  1.96);
\path[draw=drawColor,line width= 0.4pt,line join=round,line cap=round,fill=fillColor] (131.67,254.72) circle (  1.96);
\path[draw=drawColor,line width= 0.4pt,line join=round,line cap=round,fill=fillColor] (131.87,255.22) circle (  1.96);
\path[draw=drawColor,line width= 0.4pt,line join=round,line cap=round,fill=fillColor] (132.08,255.71) circle (  1.96);
\path[draw=drawColor,line width= 0.4pt,line join=round,line cap=round,fill=fillColor] (132.29,256.21) circle (  1.96);
\path[draw=drawColor,line width= 0.4pt,line join=round,line cap=round,fill=fillColor] (132.50,256.71) circle (  1.96);
\path[draw=drawColor,line width= 0.4pt,line join=round,line cap=round,fill=fillColor] (132.71,257.20) circle (  1.96);
\path[draw=drawColor,line width= 0.4pt,line join=round,line cap=round,fill=fillColor] (132.92,257.69) circle (  1.96);
\path[draw=drawColor,line width= 0.4pt,line join=round,line cap=round,fill=fillColor] (133.13,258.19) circle (  1.96);
\path[draw=drawColor,line width= 0.4pt,line join=round,line cap=round,fill=fillColor] (133.34,258.68) circle (  1.96);
\path[draw=drawColor,line width= 0.4pt,line join=round,line cap=round,fill=fillColor] (133.55,259.17) circle (  1.96);
\path[draw=drawColor,line width= 0.4pt,line join=round,line cap=round,fill=fillColor] (133.76,259.66) circle (  1.96);
\path[draw=drawColor,line width= 0.4pt,line join=round,line cap=round,fill=fillColor] (133.97,260.15) circle (  1.96);
\path[draw=drawColor,line width= 0.4pt,line join=round,line cap=round,fill=fillColor] (134.18,260.64) circle (  1.96);
\path[draw=drawColor,line width= 0.4pt,line join=round,line cap=round,fill=fillColor] (134.39,261.13) circle (  1.96);
\path[draw=drawColor,line width= 0.4pt,line join=round,line cap=round,fill=fillColor] (134.60,261.62) circle (  1.96);
\path[draw=drawColor,line width= 0.4pt,line join=round,line cap=round,fill=fillColor] (134.81,262.11) circle (  1.96);
\path[draw=drawColor,line width= 0.4pt,line join=round,line cap=round,fill=fillColor] (135.02,262.60) circle (  1.96);
\path[draw=drawColor,line width= 0.4pt,line join=round,line cap=round,fill=fillColor] (135.23,263.08) circle (  1.96);
\path[draw=drawColor,line width= 0.4pt,line join=round,line cap=round,fill=fillColor] (135.44,263.57) circle (  1.96);
\path[draw=drawColor,line width= 0.4pt,line join=round,line cap=round,fill=fillColor] (135.65,264.05) circle (  1.96);
\path[draw=drawColor,line width= 0.4pt,line join=round,line cap=round,fill=fillColor] (135.86,264.54) circle (  1.96);
\path[draw=drawColor,line width= 0.4pt,line join=round,line cap=round,fill=fillColor] (136.07,265.02) circle (  1.96);
\path[draw=drawColor,line width= 0.4pt,line join=round,line cap=round,fill=fillColor] (136.28,265.50) circle (  1.96);
\path[draw=drawColor,line width= 0.4pt,line join=round,line cap=round,fill=fillColor] (136.49,265.99) circle (  1.96);
\path[draw=drawColor,line width= 0.4pt,line join=round,line cap=round,fill=fillColor] (136.70,266.47) circle (  1.96);
\path[draw=drawColor,line width= 0.4pt,line join=round,line cap=round,fill=fillColor] (136.91,266.95) circle (  1.96);
\path[draw=drawColor,line width= 0.4pt,line join=round,line cap=round,fill=fillColor] (137.12,267.43) circle (  1.96);
\path[draw=drawColor,line width= 0.4pt,line join=round,line cap=round,fill=fillColor] (137.33,267.91) circle (  1.96);
\path[draw=drawColor,line width= 0.4pt,line join=round,line cap=round,fill=fillColor] (137.53,268.38) circle (  1.96);
\path[draw=drawColor,line width= 0.4pt,line join=round,line cap=round,fill=fillColor] (137.74,268.86) circle (  1.96);
\path[draw=drawColor,line width= 0.4pt,line join=round,line cap=round,fill=fillColor] (137.95,269.34) circle (  1.96);
\path[draw=drawColor,line width= 0.4pt,line join=round,line cap=round,fill=fillColor] (138.16,269.82) circle (  1.96);
\path[draw=drawColor,line width= 0.4pt,line join=round,line cap=round,fill=fillColor] (138.37,270.29) circle (  1.96);
\path[draw=drawColor,line width= 0.4pt,line join=round,line cap=round,fill=fillColor] (138.58,270.77) circle (  1.96);
\path[draw=drawColor,line width= 0.4pt,line join=round,line cap=round,fill=fillColor] (138.79,271.24) circle (  1.96);
\path[draw=drawColor,line width= 0.4pt,line join=round,line cap=round,fill=fillColor] (139.00,271.71) circle (  1.96);
\path[draw=drawColor,line width= 0.4pt,line join=round,line cap=round,fill=fillColor] (139.21,272.19) circle (  1.96);
\path[draw=drawColor,line width= 0.4pt,line join=round,line cap=round,fill=fillColor] (139.42,272.66) circle (  1.96);
\path[draw=drawColor,line width= 0.4pt,line join=round,line cap=round,fill=fillColor] (139.63,273.13) circle (  1.96);
\path[draw=drawColor,line width= 0.4pt,line join=round,line cap=round,fill=fillColor] (139.84,273.60) circle (  1.96);
\path[draw=drawColor,line width= 0.4pt,line join=round,line cap=round,fill=fillColor] (140.05,274.07) circle (  1.96);
\path[draw=drawColor,line width= 0.4pt,line join=round,line cap=round,fill=fillColor] (140.26,274.54) circle (  1.96);
\path[draw=drawColor,line width= 0.4pt,line join=round,line cap=round,fill=fillColor] (140.47,275.01) circle (  1.96);
\path[draw=drawColor,line width= 0.4pt,line join=round,line cap=round,fill=fillColor] (140.68,275.48) circle (  1.96);
\path[draw=drawColor,line width= 0.4pt,line join=round,line cap=round,fill=fillColor] (140.89,275.94) circle (  1.96);
\path[draw=drawColor,line width= 0.4pt,line join=round,line cap=round,fill=fillColor] (141.10,276.41) circle (  1.96);
\path[draw=drawColor,line width= 0.4pt,line join=round,line cap=round,fill=fillColor] (141.31,276.87) circle (  1.96);
\path[draw=drawColor,line width= 0.4pt,line join=round,line cap=round,fill=fillColor] (141.52,277.34) circle (  1.96);
\path[draw=drawColor,line width= 0.4pt,line join=round,line cap=round,fill=fillColor] (141.73,277.80) circle (  1.96);
\path[draw=drawColor,line width= 0.4pt,line join=round,line cap=round,fill=fillColor] (141.94,278.27) circle (  1.96);
\path[draw=drawColor,line width= 0.4pt,line join=round,line cap=round,fill=fillColor] (142.15,278.73) circle (  1.96);
\path[draw=drawColor,line width= 0.4pt,line join=round,line cap=round,fill=fillColor] (142.36,279.19) circle (  1.96);
\path[draw=drawColor,line width= 0.4pt,line join=round,line cap=round,fill=fillColor] (142.57,279.65) circle (  1.96);
\path[draw=drawColor,line width= 0.4pt,line join=round,line cap=round,fill=fillColor] (142.78,280.11) circle (  1.96);
\path[draw=drawColor,line width= 0.4pt,line join=round,line cap=round,fill=fillColor] (142.98,280.57) circle (  1.96);
\path[draw=drawColor,line width= 0.4pt,line join=round,line cap=round,fill=fillColor] (143.19,281.03) circle (  1.96);
\path[draw=drawColor,line width= 0.4pt,line join=round,line cap=round,fill=fillColor] (143.40,281.49) circle (  1.96);
\path[draw=drawColor,line width= 0.4pt,line join=round,line cap=round,fill=fillColor] (143.61,281.95) circle (  1.96);
\path[draw=drawColor,line width= 0.4pt,line join=round,line cap=round,fill=fillColor] (143.82,282.41) circle (  1.96);
\path[draw=drawColor,line width= 0.4pt,line join=round,line cap=round,fill=fillColor] (144.03,282.86) circle (  1.96);
\path[draw=drawColor,line width= 0.4pt,line join=round,line cap=round,fill=fillColor] (144.24,283.32) circle (  1.96);
\path[draw=drawColor,line width= 0.4pt,line join=round,line cap=round,fill=fillColor] (144.45,283.77) circle (  1.96);
\path[draw=drawColor,line width= 0.4pt,line join=round,line cap=round,fill=fillColor] (144.66,284.23) circle (  1.96);
\path[draw=drawColor,line width= 0.4pt,line join=round,line cap=round,fill=fillColor] (144.87,284.68) circle (  1.96);
\path[draw=drawColor,line width= 0.4pt,line join=round,line cap=round,fill=fillColor] (145.08,285.13) circle (  1.96);
\path[draw=drawColor,line width= 0.4pt,line join=round,line cap=round,fill=fillColor] (145.29,285.58) circle (  1.96);
\path[draw=drawColor,line width= 0.4pt,line join=round,line cap=round,fill=fillColor] (145.50,286.04) circle (  1.96);
\path[draw=drawColor,line width= 0.4pt,line join=round,line cap=round,fill=fillColor] (145.71,286.49) circle (  1.96);
\path[draw=drawColor,line width= 0.4pt,line join=round,line cap=round,fill=fillColor] (145.92,286.94) circle (  1.96);
\path[draw=drawColor,line width= 0.4pt,line join=round,line cap=round,fill=fillColor] (146.13,287.38) circle (  1.96);
\path[draw=drawColor,line width= 0.4pt,line join=round,line cap=round,fill=fillColor] (146.34,287.83) circle (  1.96);
\path[draw=drawColor,line width= 0.4pt,line join=round,line cap=round,fill=fillColor] (146.55,288.28) circle (  1.96);
\path[draw=drawColor,line width= 0.4pt,line join=round,line cap=round,fill=fillColor] (146.76,288.73) circle (  1.96);
\path[draw=drawColor,line width= 0.4pt,line join=round,line cap=round,fill=fillColor] (146.97,289.17) circle (  1.96);
\path[draw=drawColor,line width= 0.4pt,line join=round,line cap=round,fill=fillColor] (147.18,289.62) circle (  1.96);
\path[draw=drawColor,line width= 0.4pt,line join=round,line cap=round,fill=fillColor] (147.39,290.06) circle (  1.96);
\path[draw=drawColor,line width= 0.4pt,line join=round,line cap=round,fill=fillColor] (147.60,290.51) circle (  1.96);
\path[draw=drawColor,line width= 0.4pt,line join=round,line cap=round,fill=fillColor] (147.81,290.95) circle (  1.96);
\path[draw=drawColor,line width= 0.4pt,line join=round,line cap=round,fill=fillColor] (148.02,291.39) circle (  1.96);
\path[draw=drawColor,line width= 0.4pt,line join=round,line cap=round,fill=fillColor] (148.23,291.84) circle (  1.96);
\path[draw=drawColor,line width= 0.4pt,line join=round,line cap=round,fill=fillColor] (148.44,292.28) circle (  1.96);
\path[draw=drawColor,line width= 0.4pt,line join=round,line cap=round,fill=fillColor] (148.64,292.72) circle (  1.96);
\path[draw=drawColor,line width= 0.4pt,line join=round,line cap=round,fill=fillColor] (148.85,293.16) circle (  1.96);
\path[draw=drawColor,line width= 0.4pt,line join=round,line cap=round,fill=fillColor] (149.06,293.60) circle (  1.96);
\path[draw=drawColor,line width= 0.4pt,line join=round,line cap=round,fill=fillColor] (149.27,294.03) circle (  1.96);
\path[draw=drawColor,line width= 0.4pt,line join=round,line cap=round,fill=fillColor] (149.48,294.47) circle (  1.96);
\path[draw=drawColor,line width= 0.4pt,line join=round,line cap=round,fill=fillColor] (149.69,294.91) circle (  1.96);
\path[draw=drawColor,line width= 0.4pt,line join=round,line cap=round,fill=fillColor] (149.90,295.35) circle (  1.96);
\path[draw=drawColor,line width= 0.4pt,line join=round,line cap=round,fill=fillColor] (150.11,295.78) circle (  1.96);
\path[draw=drawColor,line width= 0.4pt,line join=round,line cap=round,fill=fillColor] (150.32,296.22) circle (  1.96);
\path[draw=drawColor,line width= 0.4pt,line join=round,line cap=round,fill=fillColor] (150.53,296.65) circle (  1.96);
\path[draw=drawColor,line width= 0.4pt,line join=round,line cap=round,fill=fillColor] (150.74,297.08) circle (  1.96);
\path[draw=drawColor,line width= 0.4pt,line join=round,line cap=round,fill=fillColor] (150.95,297.52) circle (  1.96);
\path[draw=drawColor,line width= 0.4pt,line join=round,line cap=round,fill=fillColor] (151.16,297.95) circle (  1.96);
\path[draw=drawColor,line width= 0.4pt,line join=round,line cap=round,fill=fillColor] (151.37,298.38) circle (  1.96);
\path[draw=drawColor,line width= 0.4pt,line join=round,line cap=round,fill=fillColor] (151.58,298.81) circle (  1.96);
\path[draw=drawColor,line width= 0.4pt,line join=round,line cap=round,fill=fillColor] (151.79,299.24) circle (  1.96);
\path[draw=drawColor,line width= 0.4pt,line join=round,line cap=round,fill=fillColor] (152.00,299.67) circle (  1.96);
\path[draw=drawColor,line width= 0.4pt,line join=round,line cap=round,fill=fillColor] (152.21,300.10) circle (  1.96);
\path[draw=drawColor,line width= 0.4pt,line join=round,line cap=round,fill=fillColor] (152.42,300.52) circle (  1.96);
\path[draw=drawColor,line width= 0.4pt,line join=round,line cap=round,fill=fillColor] (152.63,300.95) circle (  1.96);
\path[draw=drawColor,line width= 0.4pt,line join=round,line cap=round,fill=fillColor] (152.84,301.38) circle (  1.96);
\path[draw=drawColor,line width= 0.4pt,line join=round,line cap=round,fill=fillColor] (153.05,301.80) circle (  1.96);
\path[draw=drawColor,line width= 0.4pt,line join=round,line cap=round,fill=fillColor] (153.26,302.23) circle (  1.96);
\path[draw=drawColor,line width= 0.4pt,line join=round,line cap=round,fill=fillColor] (153.47,302.65) circle (  1.96);
\path[draw=drawColor,line width= 0.4pt,line join=round,line cap=round,fill=fillColor] (153.68,303.07) circle (  1.96);
\path[draw=drawColor,line width= 0.4pt,line join=round,line cap=round,fill=fillColor] (153.89,303.50) circle (  1.96);
\path[draw=drawColor,line width= 0.4pt,line join=round,line cap=round,fill=fillColor] (154.09,303.92) circle (  1.96);
\path[draw=drawColor,line width= 0.4pt,line join=round,line cap=round,fill=fillColor] (154.30,304.34) circle (  1.96);
\path[draw=drawColor,line width= 0.4pt,line join=round,line cap=round,fill=fillColor] (154.51,304.76) circle (  1.96);
\path[draw=drawColor,line width= 0.4pt,line join=round,line cap=round,fill=fillColor] (154.72,305.18) circle (  1.96);
\path[draw=drawColor,line width= 0.4pt,line join=round,line cap=round,fill=fillColor] (154.93,305.60) circle (  1.96);
\path[draw=drawColor,line width= 0.4pt,line join=round,line cap=round,fill=fillColor] (155.14,306.02) circle (  1.96);
\path[draw=drawColor,line width= 0.4pt,line join=round,line cap=round,fill=fillColor] (155.35,306.43) circle (  1.96);
\path[draw=drawColor,line width= 0.4pt,line join=round,line cap=round,fill=fillColor] (155.56,306.85) circle (  1.96);
\path[draw=drawColor,line width= 0.4pt,line join=round,line cap=round,fill=fillColor] (155.77,307.27) circle (  1.96);
\path[draw=drawColor,line width= 0.4pt,line join=round,line cap=round,fill=fillColor] (155.98,307.68) circle (  1.96);
\path[draw=drawColor,line width= 0.4pt,line join=round,line cap=round,fill=fillColor] (156.19,308.10) circle (  1.96);
\path[draw=drawColor,line width= 0.4pt,line join=round,line cap=round,fill=fillColor] (156.40,308.51) circle (  1.96);
\path[draw=drawColor,line width= 0.4pt,line join=round,line cap=round,fill=fillColor] (156.61,308.93) circle (  1.96);
\path[draw=drawColor,line width= 0.4pt,line join=round,line cap=round,fill=fillColor] (156.82,309.34) circle (  1.96);
\path[draw=drawColor,line width= 0.4pt,line join=round,line cap=round,fill=fillColor] (157.03,309.75) circle (  1.96);
\path[draw=drawColor,line width= 0.4pt,line join=round,line cap=round,fill=fillColor] (157.24,310.16) circle (  1.96);
\path[draw=drawColor,line width= 0.4pt,line join=round,line cap=round,fill=fillColor] (157.45,310.57) circle (  1.96);
\path[draw=drawColor,line width= 0.4pt,line join=round,line cap=round,fill=fillColor] (157.66,310.98) circle (  1.96);
\path[draw=drawColor,line width= 0.4pt,line join=round,line cap=round,fill=fillColor] (157.87,311.39) circle (  1.96);
\path[draw=drawColor,line width= 0.4pt,line join=round,line cap=round,fill=fillColor] (158.08,311.80) circle (  1.96);
\path[draw=drawColor,line width= 0.4pt,line join=round,line cap=round,fill=fillColor] (158.29,312.21) circle (  1.96);
\path[draw=drawColor,line width= 0.4pt,line join=round,line cap=round,fill=fillColor] (158.50,312.61) circle (  1.96);
\path[draw=drawColor,line width= 0.4pt,line join=round,line cap=round,fill=fillColor] (158.71,313.02) circle (  1.96);
\path[draw=drawColor,line width= 0.4pt,line join=round,line cap=round,fill=fillColor] (158.92,313.42) circle (  1.96);
\path[draw=drawColor,line width= 0.4pt,line join=round,line cap=round,fill=fillColor] (159.13,313.83) circle (  1.96);
\path[draw=drawColor,line width= 0.4pt,line join=round,line cap=round,fill=fillColor] (159.34,314.23) circle (  1.96);
\path[draw=drawColor,line width= 0.4pt,line join=round,line cap=round,fill=fillColor] (159.55,314.64) circle (  1.96);
\path[draw=drawColor,line width= 0.4pt,line join=round,line cap=round,fill=fillColor] (159.75,315.04) circle (  1.96);
\path[draw=drawColor,line width= 0.4pt,line join=round,line cap=round,fill=fillColor] (159.96,315.44) circle (  1.96);
\path[draw=drawColor,line width= 0.4pt,line join=round,line cap=round,fill=fillColor] (160.17,315.84) circle (  1.96);
\path[draw=drawColor,line width= 0.4pt,line join=round,line cap=round,fill=fillColor] (160.38,316.24) circle (  1.96);
\path[draw=drawColor,line width= 0.4pt,line join=round,line cap=round,fill=fillColor] (160.59,316.64) circle (  1.96);
\path[draw=drawColor,line width= 0.4pt,line join=round,line cap=round,fill=fillColor] (160.80,317.04) circle (  1.96);
\path[draw=drawColor,line width= 0.4pt,line join=round,line cap=round,fill=fillColor] (161.01,317.44) circle (  1.96);
\path[draw=drawColor,line width= 0.4pt,line join=round,line cap=round,fill=fillColor] (161.22,317.84) circle (  1.96);
\path[draw=drawColor,line width= 0.4pt,line join=round,line cap=round,fill=fillColor] (161.43,318.23) circle (  1.96);
\path[draw=drawColor,line width= 0.4pt,line join=round,line cap=round,fill=fillColor] (161.64,318.63) circle (  1.96);
\path[draw=drawColor,line width= 0.4pt,line join=round,line cap=round,fill=fillColor] (161.85,319.03) circle (  1.96);
\path[draw=drawColor,line width= 0.4pt,line join=round,line cap=round,fill=fillColor] (162.06,319.42) circle (  1.96);
\path[draw=drawColor,line width= 0.4pt,line join=round,line cap=round,fill=fillColor] (162.27,319.81) circle (  1.96);
\path[draw=drawColor,line width= 0.4pt,line join=round,line cap=round,fill=fillColor] (162.48,320.21) circle (  1.96);
\path[draw=drawColor,line width= 0.4pt,line join=round,line cap=round,fill=fillColor] (162.69,320.60) circle (  1.96);
\path[draw=drawColor,line width= 0.4pt,line join=round,line cap=round,fill=fillColor] (162.90,320.99) circle (  1.96);
\path[draw=drawColor,line width= 0.4pt,line join=round,line cap=round,fill=fillColor] (163.11,321.38) circle (  1.96);
\path[draw=drawColor,line width= 0.4pt,line join=round,line cap=round,fill=fillColor] (163.32,321.77) circle (  1.96);
\path[draw=drawColor,line width= 0.4pt,line join=round,line cap=round,fill=fillColor] (163.53,322.16) circle (  1.96);
\path[draw=drawColor,line width= 0.4pt,line join=round,line cap=round,fill=fillColor] (163.74,322.55) circle (  1.96);
\path[draw=drawColor,line width= 0.4pt,line join=round,line cap=round,fill=fillColor] (163.95,322.94) circle (  1.96);
\path[draw=drawColor,line width= 0.4pt,line join=round,line cap=round,fill=fillColor] (164.16,323.33) circle (  1.96);
\path[draw=drawColor,line width= 0.4pt,line join=round,line cap=round,fill=fillColor] (164.37,323.72) circle (  1.96);
\path[draw=drawColor,line width= 0.4pt,line join=round,line cap=round,fill=fillColor] (164.58,324.10) circle (  1.96);
\path[draw=drawColor,line width= 0.4pt,line join=round,line cap=round,fill=fillColor] (164.79,324.49) circle (  1.96);
\path[draw=drawColor,line width= 0.4pt,line join=round,line cap=round,fill=fillColor] (165.00,324.87) circle (  1.96);
\path[draw=drawColor,line width= 0.4pt,line join=round,line cap=round,fill=fillColor] (165.20,325.26) circle (  1.96);
\path[draw=drawColor,line width= 0.4pt,line join=round,line cap=round,fill=fillColor] (165.41,325.64) circle (  1.96);
\path[draw=drawColor,line width= 0.4pt,line join=round,line cap=round,fill=fillColor] (165.62,326.02) circle (  1.96);
\path[draw=drawColor,line width= 0.4pt,line join=round,line cap=round,fill=fillColor] (165.83,326.41) circle (  1.96);
\path[draw=drawColor,line width= 0.4pt,line join=round,line cap=round,fill=fillColor] (166.04,326.79) circle (  1.96);
\path[draw=drawColor,line width= 0.4pt,line join=round,line cap=round,fill=fillColor] (166.25,327.17) circle (  1.96);
\path[draw=drawColor,line width= 0.4pt,line join=round,line cap=round,fill=fillColor] (166.46,327.55) circle (  1.96);
\path[draw=drawColor,line width= 0.4pt,line join=round,line cap=round,fill=fillColor] (166.67,327.93) circle (  1.96);
\path[draw=drawColor,line width= 0.4pt,line join=round,line cap=round,fill=fillColor] (166.88,328.31) circle (  1.96);
\path[draw=drawColor,line width= 0.4pt,line join=round,line cap=round,fill=fillColor] (167.09,328.68) circle (  1.96);
\path[draw=drawColor,line width= 0.4pt,line join=round,line cap=round,fill=fillColor] (167.30,329.06) circle (  1.96);
\path[draw=drawColor,line width= 0.4pt,line join=round,line cap=round,fill=fillColor] (167.51,329.44) circle (  1.96);
\path[draw=drawColor,line width= 0.4pt,line join=round,line cap=round,fill=fillColor] (167.72,329.81) circle (  1.96);
\path[draw=drawColor,line width= 0.4pt,line join=round,line cap=round,fill=fillColor] (167.93,330.19) circle (  1.96);
\path[draw=drawColor,line width= 0.4pt,line join=round,line cap=round,fill=fillColor] (168.14,330.56) circle (  1.96);
\path[draw=drawColor,line width= 0.4pt,line join=round,line cap=round,fill=fillColor] (168.35,330.94) circle (  1.96);
\path[draw=drawColor,line width= 0.4pt,line join=round,line cap=round,fill=fillColor] (168.56,331.31) circle (  1.96);
\path[draw=drawColor,line width= 0.4pt,line join=round,line cap=round,fill=fillColor] (168.77,331.68) circle (  1.96);
\path[draw=drawColor,line width= 0.4pt,line join=round,line cap=round,fill=fillColor] (168.98,332.05) circle (  1.96);
\path[draw=drawColor,line width= 0.4pt,line join=round,line cap=round,fill=fillColor] (169.19,332.43) circle (  1.96);
\path[draw=drawColor,line width= 0.4pt,line join=round,line cap=round,fill=fillColor] (169.40,332.80) circle (  1.96);
\path[draw=drawColor,line width= 0.4pt,line join=round,line cap=round,fill=fillColor] (169.61,333.17) circle (  1.96);
\path[draw=drawColor,line width= 0.4pt,line join=round,line cap=round,fill=fillColor] (169.82,333.53) circle (  1.96);
\path[draw=drawColor,line width= 0.4pt,line join=round,line cap=round,fill=fillColor] (170.03,333.90) circle (  1.96);
\path[draw=drawColor,line width= 0.4pt,line join=round,line cap=round,fill=fillColor] (170.24,334.27) circle (  1.96);
\path[draw=drawColor,line width= 0.4pt,line join=round,line cap=round,fill=fillColor] (170.45,334.64) circle (  1.96);
\path[draw=drawColor,line width= 0.4pt,line join=round,line cap=round,fill=fillColor] (170.66,335.00) circle (  1.96);
\path[draw=drawColor,line width= 0.4pt,line join=round,line cap=round,fill=fillColor] (170.86,335.37) circle (  1.96);
\path[draw=drawColor,line width= 0.4pt,line join=round,line cap=round,fill=fillColor] (171.07,335.73) circle (  1.96);
\path[draw=drawColor,line width= 0.4pt,line join=round,line cap=round,fill=fillColor] (171.28,336.10) circle (  1.96);
\path[draw=drawColor,line width= 0.4pt,line join=round,line cap=round,fill=fillColor] (171.49,336.46) circle (  1.96);
\path[draw=drawColor,line width= 0.4pt,line join=round,line cap=round,fill=fillColor] (171.70,336.82) circle (  1.96);
\path[draw=drawColor,line width= 0.4pt,line join=round,line cap=round,fill=fillColor] (171.91,337.19) circle (  1.96);
\path[draw=drawColor,line width= 0.4pt,line join=round,line cap=round,fill=fillColor] (172.12,337.55) circle (  1.96);
\path[draw=drawColor,line width= 0.4pt,line join=round,line cap=round,fill=fillColor] (172.33,337.91) circle (  1.96);
\path[draw=drawColor,line width= 0.4pt,line join=round,line cap=round,fill=fillColor] (172.54,338.27) circle (  1.96);
\path[draw=drawColor,line width= 0.4pt,line join=round,line cap=round,fill=fillColor] (172.75,338.63) circle (  1.96);
\path[draw=drawColor,line width= 0.4pt,line join=round,line cap=round,fill=fillColor] (172.96,338.98) circle (  1.96);
\path[draw=drawColor,line width= 0.4pt,line join=round,line cap=round,fill=fillColor] (173.17,339.34) circle (  1.96);
\path[draw=drawColor,line width= 0.4pt,line join=round,line cap=round,fill=fillColor] (173.38,339.70) circle (  1.96);
\path[draw=drawColor,line width= 0.4pt,line join=round,line cap=round,fill=fillColor] (173.59,340.06) circle (  1.96);
\path[draw=drawColor,line width= 0.4pt,line join=round,line cap=round,fill=fillColor] (173.80,340.41) circle (  1.96);
\path[draw=drawColor,line width= 0.4pt,line join=round,line cap=round,fill=fillColor] (174.01,340.77) circle (  1.96);
\path[draw=drawColor,line width= 0.4pt,line join=round,line cap=round,fill=fillColor] (174.22,341.12) circle (  1.96);
\path[draw=drawColor,line width= 0.4pt,line join=round,line cap=round,fill=fillColor] (174.43,341.48) circle (  1.96);
\path[draw=drawColor,line width= 0.4pt,line join=round,line cap=round,fill=fillColor] (174.64,341.83) circle (  1.96);
\path[draw=drawColor,line width= 0.4pt,line join=round,line cap=round,fill=fillColor] (174.85,342.18) circle (  1.96);
\path[draw=drawColor,line width= 0.4pt,line join=round,line cap=round,fill=fillColor] (175.06,342.53) circle (  1.96);
\path[draw=drawColor,line width= 0.4pt,line join=round,line cap=round,fill=fillColor] (175.27,342.88) circle (  1.96);
\path[draw=drawColor,line width= 0.4pt,line join=round,line cap=round,fill=fillColor] (175.48,343.23) circle (  1.96);
\path[draw=drawColor,line width= 0.4pt,line join=round,line cap=round,fill=fillColor] (175.69,343.58) circle (  1.96);
\path[draw=drawColor,line width= 0.4pt,line join=round,line cap=round,fill=fillColor] (175.90,343.93) circle (  1.96);
\path[draw=drawColor,line width= 0.4pt,line join=round,line cap=round,fill=fillColor] (176.11,344.28) circle (  1.96);
\path[draw=drawColor,line width= 0.4pt,line join=round,line cap=round,fill=fillColor] (176.31,344.63) circle (  1.96);
\path[draw=drawColor,line width= 0.4pt,line join=round,line cap=round,fill=fillColor] (176.52,344.98) circle (  1.96);
\path[draw=drawColor,line width= 0.4pt,line join=round,line cap=round,fill=fillColor] (176.73,345.32) circle (  1.96);
\path[draw=drawColor,line width= 0.4pt,line join=round,line cap=round,fill=fillColor] (176.94,345.67) circle (  1.96);
\path[draw=drawColor,line width= 0.4pt,line join=round,line cap=round,fill=fillColor] (177.15,346.01) circle (  1.96);
\path[draw=drawColor,line width= 0.4pt,line join=round,line cap=round,fill=fillColor] (177.36,346.36) circle (  1.96);
\path[draw=drawColor,line width= 0.4pt,line join=round,line cap=round,fill=fillColor] (177.57,346.70) circle (  1.96);
\path[draw=drawColor,line width= 0.4pt,line join=round,line cap=round,fill=fillColor] (177.78,347.04) circle (  1.96);
\path[draw=drawColor,line width= 0.4pt,line join=round,line cap=round,fill=fillColor] (177.99,347.38) circle (  1.96);
\path[draw=drawColor,line width= 0.4pt,line join=round,line cap=round,fill=fillColor] (178.20,347.73) circle (  1.96);
\path[draw=drawColor,line width= 0.4pt,line join=round,line cap=round,fill=fillColor] (178.41,348.07) circle (  1.96);
\path[draw=drawColor,line width= 0.4pt,line join=round,line cap=round,fill=fillColor] (178.62,348.41) circle (  1.96);
\path[draw=drawColor,line width= 0.4pt,line join=round,line cap=round,fill=fillColor] (178.83,348.75) circle (  1.96);
\path[draw=drawColor,line width= 0.4pt,line join=round,line cap=round,fill=fillColor] (179.04,349.09) circle (  1.96);
\path[draw=drawColor,line width= 0.4pt,line join=round,line cap=round,fill=fillColor] (179.25,349.42) circle (  1.96);
\path[draw=drawColor,line width= 0.4pt,line join=round,line cap=round,fill=fillColor] (179.46,349.76) circle (  1.96);
\path[draw=drawColor,line width= 0.4pt,line join=round,line cap=round,fill=fillColor] (179.67,350.10) circle (  1.96);
\path[draw=drawColor,line width= 0.4pt,line join=round,line cap=round,fill=fillColor] (179.88,350.43) circle (  1.96);
\path[draw=drawColor,line width= 0.4pt,line join=round,line cap=round,fill=fillColor] (180.09,350.77) circle (  1.96);
\path[draw=drawColor,line width= 0.4pt,line join=round,line cap=round,fill=fillColor] (180.30,351.10) circle (  1.96);
\path[draw=drawColor,line width= 0.4pt,line join=round,line cap=round,fill=fillColor] (180.51,351.44) circle (  1.96);
\path[draw=drawColor,line width= 0.4pt,line join=round,line cap=round,fill=fillColor] (180.72,351.77) circle (  1.96);
\path[draw=drawColor,line width= 0.4pt,line join=round,line cap=round,fill=fillColor] (180.93,352.11) circle (  1.96);
\path[draw=drawColor,line width= 0.4pt,line join=round,line cap=round,fill=fillColor] (181.14,352.44) circle (  1.96);
\path[draw=drawColor,line width= 0.4pt,line join=round,line cap=round,fill=fillColor] (181.35,352.77) circle (  1.96);
\path[draw=drawColor,line width= 0.4pt,line join=round,line cap=round,fill=fillColor] (181.56,353.10) circle (  1.96);
\path[draw=drawColor,line width= 0.4pt,line join=round,line cap=round,fill=fillColor] (181.76,353.43) circle (  1.96);
\path[draw=drawColor,line width= 0.4pt,line join=round,line cap=round,fill=fillColor] (181.97,353.76) circle (  1.96);
\path[draw=drawColor,line width= 0.4pt,line join=round,line cap=round,fill=fillColor] (182.18,354.09) circle (  1.96);
\path[draw=drawColor,line width= 0.4pt,line join=round,line cap=round,fill=fillColor] (182.39,354.42) circle (  1.96);
\path[draw=drawColor,line width= 0.4pt,line join=round,line cap=round,fill=fillColor] (182.60,354.74) circle (  1.96);
\path[draw=drawColor,line width= 0.4pt,line join=round,line cap=round,fill=fillColor] (182.81,355.07) circle (  1.96);
\path[draw=drawColor,line width= 0.4pt,line join=round,line cap=round,fill=fillColor] (183.02,355.40) circle (  1.96);
\path[draw=drawColor,line width= 0.4pt,line join=round,line cap=round,fill=fillColor] (183.23,355.72) circle (  1.96);
\path[draw=drawColor,line width= 0.4pt,line join=round,line cap=round,fill=fillColor] (183.44,356.05) circle (  1.96);
\path[draw=drawColor,line width= 0.4pt,line join=round,line cap=round,fill=fillColor] (183.65,356.37) circle (  1.96);
\path[draw=drawColor,line width= 0.4pt,line join=round,line cap=round,fill=fillColor] (183.86,356.70) circle (  1.96);
\path[draw=drawColor,line width= 0.4pt,line join=round,line cap=round,fill=fillColor] (184.07,357.02) circle (  1.96);
\path[draw=drawColor,line width= 0.4pt,line join=round,line cap=round,fill=fillColor] (184.28,357.34) circle (  1.96);
\path[draw=drawColor,line width= 0.4pt,line join=round,line cap=round,fill=fillColor] (184.49,357.66) circle (  1.96);
\path[draw=drawColor,line width= 0.4pt,line join=round,line cap=round,fill=fillColor] (184.70,357.98) circle (  1.96);
\path[draw=drawColor,line width= 0.4pt,line join=round,line cap=round,fill=fillColor] (184.91,358.30) circle (  1.96);
\path[draw=drawColor,line width= 0.4pt,line join=round,line cap=round,fill=fillColor] (185.12,358.62) circle (  1.96);
\path[draw=drawColor,line width= 0.4pt,line join=round,line cap=round,fill=fillColor] (185.33,358.94) circle (  1.96);
\path[draw=drawColor,line width= 0.4pt,line join=round,line cap=round,fill=fillColor] (185.54,359.26) circle (  1.96);
\path[draw=drawColor,line width= 0.4pt,line join=round,line cap=round,fill=fillColor] (185.75,359.58) circle (  1.96);
\path[draw=drawColor,line width= 0.4pt,line join=round,line cap=round,fill=fillColor] (185.96,359.90) circle (  1.96);
\path[draw=drawColor,line width= 0.4pt,line join=round,line cap=round,fill=fillColor] (186.17,360.21) circle (  1.96);
\path[draw=drawColor,line width= 0.4pt,line join=round,line cap=round,fill=fillColor] (186.38,360.53) circle (  1.96);
\path[draw=drawColor,line width= 0.4pt,line join=round,line cap=round,fill=fillColor] (186.59,360.84) circle (  1.96);
\path[draw=drawColor,line width= 0.4pt,line join=round,line cap=round,fill=fillColor] (186.80,361.16) circle (  1.96);
\path[draw=drawColor,line width= 0.4pt,line join=round,line cap=round,fill=fillColor] (187.01,361.47) circle (  1.96);
\path[draw=drawColor,line width= 0.4pt,line join=round,line cap=round,fill=fillColor] (187.22,361.78) circle (  1.96);
\path[draw=drawColor,line width= 0.4pt,line join=round,line cap=round,fill=fillColor] (187.42,362.10) circle (  1.96);
\path[draw=drawColor,line width= 0.4pt,line join=round,line cap=round,fill=fillColor] (187.63,362.41) circle (  1.96);
\path[draw=drawColor,line width= 0.4pt,line join=round,line cap=round,fill=fillColor] (187.84,362.72) circle (  1.96);
\path[draw=drawColor,line width= 0.4pt,line join=round,line cap=round,fill=fillColor] (188.05,363.03) circle (  1.96);
\path[draw=drawColor,line width= 0.4pt,line join=round,line cap=round,fill=fillColor] (188.26,363.34) circle (  1.96);
\path[draw=drawColor,line width= 0.4pt,line join=round,line cap=round,fill=fillColor] (188.47,363.65) circle (  1.96);
\path[draw=drawColor,line width= 0.4pt,line join=round,line cap=round,fill=fillColor] (188.68,363.96) circle (  1.96);
\path[draw=drawColor,line width= 0.4pt,line join=round,line cap=round,fill=fillColor] (188.89,364.27) circle (  1.96);
\path[draw=drawColor,line width= 0.4pt,line join=round,line cap=round,fill=fillColor] (189.10,364.57) circle (  1.96);
\path[draw=drawColor,line width= 0.4pt,line join=round,line cap=round,fill=fillColor] (189.31,364.88) circle (  1.96);
\path[draw=drawColor,line width= 0.4pt,line join=round,line cap=round,fill=fillColor] (189.52,365.19) circle (  1.96);
\path[draw=drawColor,line width= 0.4pt,line join=round,line cap=round,fill=fillColor] (189.73,365.49) circle (  1.96);
\path[draw=drawColor,line width= 0.4pt,line join=round,line cap=round,fill=fillColor] (189.94,365.80) circle (  1.96);
\path[draw=drawColor,line width= 0.4pt,line join=round,line cap=round,fill=fillColor] (190.15,366.10) circle (  1.96);
\path[draw=drawColor,line width= 0.4pt,line join=round,line cap=round,fill=fillColor] (190.36,366.40) circle (  1.96);
\path[draw=drawColor,line width= 0.4pt,line join=round,line cap=round,fill=fillColor] (190.57,366.71) circle (  1.96);
\path[draw=drawColor,line width= 0.4pt,line join=round,line cap=round,fill=fillColor] (190.78,367.01) circle (  1.96);
\path[draw=drawColor,line width= 0.4pt,line join=round,line cap=round,fill=fillColor] (190.99,367.31) circle (  1.96);
\path[draw=drawColor,line width= 0.4pt,line join=round,line cap=round,fill=fillColor] (191.20,367.61) circle (  1.96);
\path[draw=drawColor,line width= 0.4pt,line join=round,line cap=round,fill=fillColor] (191.41,367.91) circle (  1.96);
\path[draw=drawColor,line width= 0.4pt,line join=round,line cap=round,fill=fillColor] (191.62,368.21) circle (  1.96);
\path[draw=drawColor,line width= 0.4pt,line join=round,line cap=round,fill=fillColor] (191.83,368.51) circle (  1.96);
\path[draw=drawColor,line width= 0.4pt,line join=round,line cap=round,fill=fillColor] (192.04,368.81) circle (  1.96);
\path[draw=drawColor,line width= 0.4pt,line join=round,line cap=round,fill=fillColor] (192.25,369.11) circle (  1.96);
\path[draw=drawColor,line width= 0.4pt,line join=round,line cap=round,fill=fillColor] (192.46,369.40) circle (  1.96);
\path[draw=drawColor,line width= 0.4pt,line join=round,line cap=round,fill=fillColor] (192.67,369.70) circle (  1.96);
\path[draw=drawColor,line width= 0.4pt,line join=round,line cap=round,fill=fillColor] (192.87,369.99) circle (  1.96);
\path[draw=drawColor,line width= 0.4pt,line join=round,line cap=round,fill=fillColor] (193.08,370.29) circle (  1.96);
\path[draw=drawColor,line width= 0.4pt,line join=round,line cap=round,fill=fillColor] (193.29,370.58) circle (  1.96);
\path[draw=drawColor,line width= 0.4pt,line join=round,line cap=round,fill=fillColor] (193.50,370.88) circle (  1.96);
\path[draw=drawColor,line width= 0.4pt,line join=round,line cap=round,fill=fillColor] (193.71,371.17) circle (  1.96);
\path[draw=drawColor,line width= 0.4pt,line join=round,line cap=round,fill=fillColor] (193.92,371.47) circle (  1.96);
\path[draw=drawColor,line width= 0.4pt,line join=round,line cap=round,fill=fillColor] (194.13,371.76) circle (  1.96);
\path[draw=drawColor,line width= 0.4pt,line join=round,line cap=round,fill=fillColor] (194.34,372.05) circle (  1.96);
\path[draw=drawColor,line width= 0.4pt,line join=round,line cap=round,fill=fillColor] (194.55,372.34) circle (  1.96);
\path[draw=drawColor,line width= 0.4pt,line join=round,line cap=round,fill=fillColor] (194.76,372.63) circle (  1.96);
\path[draw=drawColor,line width= 0.4pt,line join=round,line cap=round,fill=fillColor] (194.97,372.92) circle (  1.96);
\path[draw=drawColor,line width= 0.4pt,line join=round,line cap=round,fill=fillColor] (195.18,373.21) circle (  1.96);
\path[draw=drawColor,line width= 0.4pt,line join=round,line cap=round,fill=fillColor] (195.39,373.50) circle (  1.96);
\path[draw=drawColor,line width= 0.4pt,line join=round,line cap=round,fill=fillColor] (195.60,373.78) circle (  1.96);
\path[draw=drawColor,line width= 0.4pt,line join=round,line cap=round,fill=fillColor] (195.81,374.07) circle (  1.96);
\path[draw=drawColor,line width= 0.4pt,line join=round,line cap=round,fill=fillColor] (196.02,374.36) circle (  1.96);
\path[draw=drawColor,line width= 0.4pt,line join=round,line cap=round,fill=fillColor] (196.23,374.64) circle (  1.96);
\path[draw=drawColor,line width= 0.4pt,line join=round,line cap=round,fill=fillColor] (196.44,374.93) circle (  1.96);
\path[draw=drawColor,line width= 0.4pt,line join=round,line cap=round,fill=fillColor] (196.65,375.21) circle (  1.96);
\path[draw=drawColor,line width= 0.4pt,line join=round,line cap=round,fill=fillColor] (196.86,375.50) circle (  1.96);
\path[draw=drawColor,line width= 0.4pt,line join=round,line cap=round,fill=fillColor] (197.07,375.78) circle (  1.96);
\path[draw=drawColor,line width= 0.4pt,line join=round,line cap=round,fill=fillColor] (197.28,376.06) circle (  1.96);
\path[draw=drawColor,line width= 0.4pt,line join=round,line cap=round,fill=fillColor] (197.49,376.35) circle (  1.96);
\path[draw=drawColor,line width= 0.4pt,line join=round,line cap=round,fill=fillColor] (197.70,376.63) circle (  1.96);
\path[draw=drawColor,line width= 0.4pt,line join=round,line cap=round,fill=fillColor] (197.91,376.91) circle (  1.96);
\path[draw=drawColor,line width= 0.4pt,line join=round,line cap=round,fill=fillColor] (198.12,377.19) circle (  1.96);
\path[draw=drawColor,line width= 0.4pt,line join=round,line cap=round,fill=fillColor] (198.33,377.47) circle (  1.96);
\path[draw=drawColor,line width= 0.4pt,line join=round,line cap=round,fill=fillColor] (198.53,377.75) circle (  1.96);
\path[draw=drawColor,line width= 0.4pt,line join=round,line cap=round,fill=fillColor] (198.74,378.03) circle (  1.96);
\path[draw=drawColor,line width= 0.4pt,line join=round,line cap=round,fill=fillColor] (198.95,378.30) circle (  1.96);
\path[draw=drawColor,line width= 0.4pt,line join=round,line cap=round,fill=fillColor] (199.16,378.58) circle (  1.96);
\path[draw=drawColor,line width= 0.4pt,line join=round,line cap=round,fill=fillColor] (199.37,378.86) circle (  1.96);
\path[draw=drawColor,line width= 0.4pt,line join=round,line cap=round,fill=fillColor] (199.58,379.13) circle (  1.96);
\path[draw=drawColor,line width= 0.4pt,line join=round,line cap=round,fill=fillColor] (199.79,379.41) circle (  1.96);
\path[draw=drawColor,line width= 0.4pt,line join=round,line cap=round,fill=fillColor] (200.00,379.68) circle (  1.96);
\path[draw=drawColor,line width= 0.4pt,line join=round,line cap=round,fill=fillColor] (200.21,379.96) circle (  1.96);
\path[draw=drawColor,line width= 0.4pt,line join=round,line cap=round,fill=fillColor] (200.42,380.23) circle (  1.96);
\path[draw=drawColor,line width= 0.4pt,line join=round,line cap=round,fill=fillColor] (200.63,380.51) circle (  1.96);
\path[draw=drawColor,line width= 0.4pt,line join=round,line cap=round,fill=fillColor] (200.84,380.78) circle (  1.96);
\path[draw=drawColor,line width= 0.4pt,line join=round,line cap=round,fill=fillColor] (201.05,381.05) circle (  1.96);
\path[draw=drawColor,line width= 0.4pt,line join=round,line cap=round,fill=fillColor] (201.26,381.32) circle (  1.96);
\path[draw=drawColor,line width= 0.4pt,line join=round,line cap=round,fill=fillColor] (201.47,381.59) circle (  1.96);
\path[draw=drawColor,line width= 0.4pt,line join=round,line cap=round,fill=fillColor] (201.68,381.86) circle (  1.96);
\path[draw=drawColor,line width= 0.4pt,line join=round,line cap=round,fill=fillColor] (201.89,382.13) circle (  1.96);
\path[draw=drawColor,line width= 0.4pt,line join=round,line cap=round,fill=fillColor] (202.10,382.40) circle (  1.96);
\path[draw=drawColor,line width= 0.4pt,line join=round,line cap=round,fill=fillColor] (202.31,382.67) circle (  1.96);
\path[draw=drawColor,line width= 0.4pt,line join=round,line cap=round,fill=fillColor] (202.52,382.94) circle (  1.96);
\path[draw=drawColor,line width= 0.4pt,line join=round,line cap=round,fill=fillColor] (202.73,383.20) circle (  1.96);
\path[draw=drawColor,line width= 0.4pt,line join=round,line cap=round,fill=fillColor] (202.94,383.47) circle (  1.96);
\path[draw=drawColor,line width= 0.4pt,line join=round,line cap=round,fill=fillColor] (203.15,383.73) circle (  1.96);
\path[draw=drawColor,line width= 0.4pt,line join=round,line cap=round,fill=fillColor] (203.36,384.00) circle (  1.96);
\path[draw=drawColor,line width= 0.4pt,line join=round,line cap=round,fill=fillColor] (203.57,384.27) circle (  1.96);
\path[draw=drawColor,line width= 0.4pt,line join=round,line cap=round,fill=fillColor] (203.78,384.53) circle (  1.96);
\path[draw=drawColor,line width= 0.4pt,line join=round,line cap=round,fill=fillColor] (203.98,384.79) circle (  1.96);
\path[draw=drawColor,line width= 0.4pt,line join=round,line cap=round,fill=fillColor] (204.19,385.06) circle (  1.96);
\path[draw=drawColor,line width= 0.4pt,line join=round,line cap=round,fill=fillColor] (204.40,385.32) circle (  1.96);
\path[draw=drawColor,line width= 0.4pt,line join=round,line cap=round,fill=fillColor] (204.61,385.58) circle (  1.96);
\path[draw=drawColor,line width= 0.4pt,line join=round,line cap=round,fill=fillColor] (204.82,385.84) circle (  1.96);
\path[draw=drawColor,line width= 0.4pt,line join=round,line cap=round,fill=fillColor] (205.03,386.10) circle (  1.96);
\path[draw=drawColor,line width= 0.4pt,line join=round,line cap=round,fill=fillColor] (205.24,386.36) circle (  1.96);
\path[draw=drawColor,line width= 0.4pt,line join=round,line cap=round,fill=fillColor] (205.45,386.62) circle (  1.96);
\path[draw=drawColor,line width= 0.4pt,line join=round,line cap=round,fill=fillColor] (205.66,386.88) circle (  1.96);
\path[draw=drawColor,line width= 0.4pt,line join=round,line cap=round,fill=fillColor] (205.87,387.14) circle (  1.96);
\path[draw=drawColor,line width= 0.4pt,line join=round,line cap=round,fill=fillColor] (206.08,387.40) circle (  1.96);
\path[draw=drawColor,line width= 0.4pt,line join=round,line cap=round,fill=fillColor] (206.29,387.65) circle (  1.96);
\path[draw=drawColor,line width= 0.4pt,line join=round,line cap=round,fill=fillColor] (206.50,387.91) circle (  1.96);
\path[draw=drawColor,line width= 0.4pt,line join=round,line cap=round,fill=fillColor] (206.71,388.16) circle (  1.96);
\path[draw=drawColor,line width= 0.4pt,line join=round,line cap=round,fill=fillColor] (206.92,388.42) circle (  1.96);
\path[draw=drawColor,line width= 0.4pt,line join=round,line cap=round,fill=fillColor] (207.13,388.67) circle (  1.96);
\path[draw=drawColor,line width= 0.4pt,line join=round,line cap=round,fill=fillColor] (207.34,388.93) circle (  1.96);
\path[draw=drawColor,line width= 0.4pt,line join=round,line cap=round,fill=fillColor] (207.55,389.18) circle (  1.96);
\path[draw=drawColor,line width= 0.4pt,line join=round,line cap=round,fill=fillColor] (207.76,389.44) circle (  1.96);
\path[draw=drawColor,line width= 0.4pt,line join=round,line cap=round,fill=fillColor] (207.97,389.69) circle (  1.96);
\path[draw=drawColor,line width= 0.4pt,line join=round,line cap=round,fill=fillColor] (208.18,389.94) circle (  1.96);
\path[draw=drawColor,line width= 0.4pt,line join=round,line cap=round,fill=fillColor] (208.39,390.19) circle (  1.96);
\path[draw=drawColor,line width= 0.4pt,line join=round,line cap=round,fill=fillColor] (208.60,390.44) circle (  1.96);
\path[draw=drawColor,line width= 0.4pt,line join=round,line cap=round,fill=fillColor] (208.81,390.69) circle (  1.96);
\path[draw=drawColor,line width= 0.4pt,line join=round,line cap=round,fill=fillColor] (209.02,390.94) circle (  1.96);
\path[draw=drawColor,line width= 0.4pt,line join=round,line cap=round,fill=fillColor] (209.23,391.19) circle (  1.96);
\path[draw=drawColor,line width= 0.4pt,line join=round,line cap=round,fill=fillColor] (209.44,391.44) circle (  1.96);
\path[draw=drawColor,line width= 0.4pt,line join=round,line cap=round,fill=fillColor] (209.64,391.69) circle (  1.96);
\path[draw=drawColor,line width= 0.4pt,line join=round,line cap=round,fill=fillColor] (209.85,391.93) circle (  1.96);
\path[draw=drawColor,line width= 0.4pt,line join=round,line cap=round,fill=fillColor] (210.06,392.18) circle (  1.96);
\path[draw=drawColor,line width= 0.4pt,line join=round,line cap=round,fill=fillColor] (210.27,392.43) circle (  1.96);
\path[draw=drawColor,line width= 0.4pt,line join=round,line cap=round,fill=fillColor] (210.48,392.67) circle (  1.96);
\path[draw=drawColor,line width= 0.4pt,line join=round,line cap=round,fill=fillColor] (210.69,392.92) circle (  1.96);
\path[draw=drawColor,line width= 0.4pt,line join=round,line cap=round,fill=fillColor] (210.90,393.16) circle (  1.96);
\path[draw=drawColor,line width= 0.4pt,line join=round,line cap=round,fill=fillColor] (211.11,393.41) circle (  1.96);
\path[draw=drawColor,line width= 0.4pt,line join=round,line cap=round,fill=fillColor] (211.32,393.65) circle (  1.96);
\path[draw=drawColor,line width= 0.4pt,line join=round,line cap=round,fill=fillColor] (211.53,393.89) circle (  1.96);
\path[draw=drawColor,line width= 0.4pt,line join=round,line cap=round,fill=fillColor] (211.74,394.13) circle (  1.96);
\path[draw=drawColor,line width= 0.4pt,line join=round,line cap=round,fill=fillColor] (211.95,394.38) circle (  1.96);
\path[draw=drawColor,line width= 0.4pt,line join=round,line cap=round,fill=fillColor] (212.16,394.62) circle (  1.96);
\path[draw=drawColor,line width= 0.4pt,line join=round,line cap=round,fill=fillColor] (212.37,394.86) circle (  1.96);
\path[draw=drawColor,line width= 0.4pt,line join=round,line cap=round,fill=fillColor] (212.58,395.10) circle (  1.96);
\path[draw=drawColor,line width= 0.4pt,line join=round,line cap=round,fill=fillColor] (212.79,395.34) circle (  1.96);
\path[draw=drawColor,line width= 0.4pt,line join=round,line cap=round,fill=fillColor] (213.00,395.58) circle (  1.96);
\path[draw=drawColor,line width= 0.4pt,line join=round,line cap=round,fill=fillColor] (213.21,395.81) circle (  1.96);
\path[draw=drawColor,line width= 0.4pt,line join=round,line cap=round,fill=fillColor] (213.42,396.05) circle (  1.96);
\path[draw=drawColor,line width= 0.4pt,line join=round,line cap=round,fill=fillColor] (213.63,396.29) circle (  1.96);
\path[draw=drawColor,line width= 0.4pt,line join=round,line cap=round,fill=fillColor] (213.84,396.52) circle (  1.96);
\path[draw=drawColor,line width= 0.4pt,line join=round,line cap=round,fill=fillColor] (214.05,396.76) circle (  1.96);
\path[draw=drawColor,line width= 0.4pt,line join=round,line cap=round,fill=fillColor] (214.26,397.00) circle (  1.96);
\path[draw=drawColor,line width= 0.4pt,line join=round,line cap=round,fill=fillColor] (214.47,397.23) circle (  1.96);
\path[draw=drawColor,line width= 0.4pt,line join=round,line cap=round,fill=fillColor] (214.68,397.47) circle (  1.96);
\path[draw=drawColor,line width= 0.4pt,line join=round,line cap=round,fill=fillColor] (214.89,397.70) circle (  1.96);
\path[draw=drawColor,line width= 0.4pt,line join=round,line cap=round,fill=fillColor] (215.09,397.93) circle (  1.96);
\path[draw=drawColor,line width= 0.4pt,line join=round,line cap=round,fill=fillColor] (215.30,398.17) circle (  1.96);
\path[draw=drawColor,line width= 0.4pt,line join=round,line cap=round,fill=fillColor] (215.51,398.40) circle (  1.96);
\path[draw=drawColor,line width= 0.4pt,line join=round,line cap=round,fill=fillColor] (215.72,398.63) circle (  1.96);
\path[draw=drawColor,line width= 0.4pt,line join=round,line cap=round,fill=fillColor] (215.93,398.86) circle (  1.96);
\path[draw=drawColor,line width= 0.4pt,line join=round,line cap=round,fill=fillColor] (216.14,399.09) circle (  1.96);
\path[draw=drawColor,line width= 0.4pt,line join=round,line cap=round,fill=fillColor] (216.35,399.32) circle (  1.96);
\path[draw=drawColor,line width= 0.4pt,line join=round,line cap=round,fill=fillColor] (216.56,399.55) circle (  1.96);
\path[draw=drawColor,line width= 0.4pt,line join=round,line cap=round,fill=fillColor] (216.77,399.78) circle (  1.96);
\path[draw=drawColor,line width= 0.4pt,line join=round,line cap=round,fill=fillColor] (216.98,400.01) circle (  1.96);
\path[draw=drawColor,line width= 0.4pt,line join=round,line cap=round,fill=fillColor] (217.19,400.24) circle (  1.96);
\path[draw=drawColor,line width= 0.4pt,line join=round,line cap=round,fill=fillColor] (217.40,400.46) circle (  1.96);
\path[draw=drawColor,line width= 0.4pt,line join=round,line cap=round,fill=fillColor] (217.61,400.69) circle (  1.96);
\path[draw=drawColor,line width= 0.4pt,line join=round,line cap=round,fill=fillColor] (217.82,400.92) circle (  1.96);
\path[draw=drawColor,line width= 0.4pt,line join=round,line cap=round,fill=fillColor] (218.03,401.14) circle (  1.96);
\path[draw=drawColor,line width= 0.4pt,line join=round,line cap=round,fill=fillColor] (218.24,401.37) circle (  1.96);
\path[draw=drawColor,line width= 0.4pt,line join=round,line cap=round,fill=fillColor] (218.45,401.59) circle (  1.96);
\path[draw=drawColor,line width= 0.4pt,line join=round,line cap=round,fill=fillColor] (218.66,401.82) circle (  1.96);
\path[draw=drawColor,line width= 0.4pt,line join=round,line cap=round,fill=fillColor] (218.87,402.04) circle (  1.96);
\path[draw=drawColor,line width= 0.4pt,line join=round,line cap=round,fill=fillColor] (219.08,402.26) circle (  1.96);
\path[draw=drawColor,line width= 0.4pt,line join=round,line cap=round,fill=fillColor] (219.29,402.48) circle (  1.96);
\path[draw=drawColor,line width= 0.4pt,line join=round,line cap=round,fill=fillColor] (219.50,402.71) circle (  1.96);
\path[draw=drawColor,line width= 0.4pt,line join=round,line cap=round,fill=fillColor] (219.71,402.93) circle (  1.96);
\path[draw=drawColor,line width= 0.4pt,line join=round,line cap=round,fill=fillColor] (219.92,403.15) circle (  1.96);
\path[draw=drawColor,line width= 0.4pt,line join=round,line cap=round,fill=fillColor] (220.13,403.37) circle (  1.96);
\path[draw=drawColor,line width= 0.4pt,line join=round,line cap=round,fill=fillColor] (220.34,403.59) circle (  1.96);
\path[draw=drawColor,line width= 0.4pt,line join=round,line cap=round,fill=fillColor] (220.54,403.81) circle (  1.96);
\path[draw=drawColor,line width= 0.4pt,line join=round,line cap=round,fill=fillColor] (220.75,404.03) circle (  1.96);
\path[draw=drawColor,line width= 0.4pt,line join=round,line cap=round,fill=fillColor] (220.96,404.25) circle (  1.96);
\path[draw=drawColor,line width= 0.4pt,line join=round,line cap=round,fill=fillColor] (221.17,404.46) circle (  1.96);
\path[draw=drawColor,line width= 0.4pt,line join=round,line cap=round,fill=fillColor] (221.38,404.68) circle (  1.96);
\path[draw=drawColor,line width= 0.4pt,line join=round,line cap=round,fill=fillColor] (221.59,404.90) circle (  1.96);
\path[draw=drawColor,line width= 0.4pt,line join=round,line cap=round,fill=fillColor] (221.80,405.11) circle (  1.96);
\path[draw=drawColor,line width= 0.4pt,line join=round,line cap=round,fill=fillColor] (222.01,405.33) circle (  1.96);
\path[draw=drawColor,line width= 0.4pt,line join=round,line cap=round,fill=fillColor] (222.22,405.54) circle (  1.96);
\path[draw=drawColor,line width= 0.4pt,line join=round,line cap=round,fill=fillColor] (222.43,405.76) circle (  1.96);
\path[draw=drawColor,line width= 0.4pt,line join=round,line cap=round,fill=fillColor] (222.64,405.97) circle (  1.96);
\path[draw=drawColor,line width= 0.4pt,line join=round,line cap=round,fill=fillColor] (222.85,406.19) circle (  1.96);
\path[draw=drawColor,line width= 0.4pt,line join=round,line cap=round,fill=fillColor] (223.06,406.40) circle (  1.96);
\path[draw=drawColor,line width= 0.4pt,line join=round,line cap=round,fill=fillColor] (223.27,406.61) circle (  1.96);
\path[draw=drawColor,line width= 0.4pt,line join=round,line cap=round,fill=fillColor] (223.48,406.82) circle (  1.96);
\path[draw=drawColor,line width= 0.4pt,line join=round,line cap=round,fill=fillColor] (223.69,407.03) circle (  1.96);
\path[draw=drawColor,line width= 0.4pt,line join=round,line cap=round,fill=fillColor] (223.90,407.25) circle (  1.96);
\path[draw=drawColor,line width= 0.4pt,line join=round,line cap=round,fill=fillColor] (224.11,407.46) circle (  1.96);
\path[draw=drawColor,line width= 0.4pt,line join=round,line cap=round,fill=fillColor] (224.32,407.67) circle (  1.96);
\path[draw=drawColor,line width= 0.4pt,line join=round,line cap=round,fill=fillColor] (224.53,407.88) circle (  1.96);
\path[draw=drawColor,line width= 0.4pt,line join=round,line cap=round,fill=fillColor] (224.74,408.08) circle (  1.96);
\path[draw=drawColor,line width= 0.4pt,line join=round,line cap=round,fill=fillColor] (224.95,408.29) circle (  1.96);
\path[draw=drawColor,line width= 0.4pt,line join=round,line cap=round,fill=fillColor] (225.16,408.50) circle (  1.96);
\path[draw=drawColor,line width= 0.4pt,line join=round,line cap=round,fill=fillColor] (225.37,408.71) circle (  1.96);
\path[draw=drawColor,line width= 0.4pt,line join=round,line cap=round,fill=fillColor] (225.58,408.92) circle (  1.96);
\path[draw=drawColor,line width= 0.4pt,line join=round,line cap=round,fill=fillColor] (225.79,409.12) circle (  1.96);
\path[draw=drawColor,line width= 0.4pt,line join=round,line cap=round,fill=fillColor] (226.00,409.33) circle (  1.96);
\path[draw=drawColor,line width= 0.4pt,line join=round,line cap=round,fill=fillColor] (226.20,409.53) circle (  1.96);
\path[draw=drawColor,line width= 0.4pt,line join=round,line cap=round,fill=fillColor] (226.41,409.74) circle (  1.96);
\path[draw=drawColor,line width= 0.4pt,line join=round,line cap=round,fill=fillColor] (226.62,409.94) circle (  1.96);
\path[draw=drawColor,line width= 0.4pt,line join=round,line cap=round,fill=fillColor] (226.83,410.15) circle (  1.96);
\path[draw=drawColor,line width= 0.4pt,line join=round,line cap=round,fill=fillColor] (227.04,410.35) circle (  1.96);
\path[draw=drawColor,line width= 0.4pt,line join=round,line cap=round,fill=fillColor] (227.25,410.55) circle (  1.96);
\path[draw=drawColor,line width= 0.4pt,line join=round,line cap=round,fill=fillColor] (227.46,410.76) circle (  1.96);
\path[draw=drawColor,line width= 0.4pt,line join=round,line cap=round,fill=fillColor] (227.67,410.96) circle (  1.96);
\path[draw=drawColor,line width= 0.4pt,line join=round,line cap=round,fill=fillColor] (227.88,411.16) circle (  1.96);
\path[draw=drawColor,line width= 0.4pt,line join=round,line cap=round,fill=fillColor] (228.09,411.36) circle (  1.96);
\path[draw=drawColor,line width= 0.4pt,line join=round,line cap=round,fill=fillColor] (228.30,411.56) circle (  1.96);
\path[draw=drawColor,line width= 0.4pt,line join=round,line cap=round,fill=fillColor] (228.51,411.76) circle (  1.96);
\path[draw=drawColor,line width= 0.4pt,line join=round,line cap=round,fill=fillColor] (228.72,411.96) circle (  1.96);
\path[draw=drawColor,line width= 0.4pt,line join=round,line cap=round,fill=fillColor] (228.93,412.16) circle (  1.96);
\path[draw=drawColor,line width= 0.4pt,line join=round,line cap=round,fill=fillColor] (229.14,412.36) circle (  1.96);
\path[draw=drawColor,line width= 0.4pt,line join=round,line cap=round,fill=fillColor] (229.35,412.55) circle (  1.96);
\path[draw=drawColor,line width= 0.4pt,line join=round,line cap=round,fill=fillColor] (229.56,412.75) circle (  1.96);
\path[draw=drawColor,line width= 0.4pt,line join=round,line cap=round,fill=fillColor] (229.77,412.95) circle (  1.96);
\path[draw=drawColor,line width= 0.4pt,line join=round,line cap=round,fill=fillColor] (229.98,413.14) circle (  1.96);
\path[draw=drawColor,line width= 0.4pt,line join=round,line cap=round,fill=fillColor] (230.19,413.34) circle (  1.96);
\path[draw=drawColor,line width= 0.4pt,line join=round,line cap=round,fill=fillColor] (230.40,413.54) circle (  1.96);
\path[draw=drawColor,line width= 0.4pt,line join=round,line cap=round,fill=fillColor] (230.61,413.73) circle (  1.96);
\path[draw=drawColor,line width= 0.4pt,line join=round,line cap=round,fill=fillColor] (230.82,413.93) circle (  1.96);
\path[draw=drawColor,line width= 0.4pt,line join=round,line cap=round,fill=fillColor] (231.03,414.12) circle (  1.96);
\path[draw=drawColor,line width= 0.4pt,line join=round,line cap=round,fill=fillColor] (231.24,414.31) circle (  1.96);
\path[draw=drawColor,line width= 0.4pt,line join=round,line cap=round,fill=fillColor] (231.45,414.51) circle (  1.96);
\path[draw=drawColor,line width= 0.4pt,line join=round,line cap=round,fill=fillColor] (231.65,414.70) circle (  1.96);
\path[draw=drawColor,line width= 0.4pt,line join=round,line cap=round,fill=fillColor] (231.86,414.89) circle (  1.96);
\path[draw=drawColor,line width= 0.4pt,line join=round,line cap=round,fill=fillColor] (232.07,415.08) circle (  1.96);
\path[draw=drawColor,line width= 0.4pt,line join=round,line cap=round,fill=fillColor] (232.28,415.27) circle (  1.96);
\path[draw=drawColor,line width= 0.4pt,line join=round,line cap=round,fill=fillColor] (232.49,415.47) circle (  1.96);
\path[draw=drawColor,line width= 0.4pt,line join=round,line cap=round,fill=fillColor] (232.70,415.66) circle (  1.96);
\path[draw=drawColor,line width= 0.4pt,line join=round,line cap=round,fill=fillColor] (232.91,415.84) circle (  1.96);
\path[draw=drawColor,line width= 0.4pt,line join=round,line cap=round,fill=fillColor] (233.12,416.03) circle (  1.96);
\path[draw=drawColor,line width= 0.4pt,line join=round,line cap=round,fill=fillColor] (233.33,416.22) circle (  1.96);
\path[draw=drawColor,line width= 0.4pt,line join=round,line cap=round,fill=fillColor] (233.54,416.41) circle (  1.96);
\path[draw=drawColor,line width= 0.4pt,line join=round,line cap=round,fill=fillColor] (233.75,416.60) circle (  1.96);
\path[draw=drawColor,line width= 0.4pt,line join=round,line cap=round,fill=fillColor] (233.96,416.79) circle (  1.96);
\path[draw=drawColor,line width= 0.4pt,line join=round,line cap=round,fill=fillColor] (234.17,416.97) circle (  1.96);
\path[draw=drawColor,line width= 0.4pt,line join=round,line cap=round,fill=fillColor] (234.38,417.16) circle (  1.96);
\path[draw=drawColor,line width= 0.4pt,line join=round,line cap=round,fill=fillColor] (234.59,417.35) circle (  1.96);
\path[draw=drawColor,line width= 0.4pt,line join=round,line cap=round,fill=fillColor] (234.80,417.53) circle (  1.96);
\path[draw=drawColor,line width= 0.4pt,line join=round,line cap=round,fill=fillColor] (235.01,417.72) circle (  1.96);
\path[draw=drawColor,line width= 0.4pt,line join=round,line cap=round,fill=fillColor] (235.22,417.90) circle (  1.96);
\path[draw=drawColor,line width= 0.4pt,line join=round,line cap=round,fill=fillColor] (235.43,418.08) circle (  1.96);
\path[draw=drawColor,line width= 0.4pt,line join=round,line cap=round,fill=fillColor] (235.64,418.27) circle (  1.96);
\path[draw=drawColor,line width= 0.4pt,line join=round,line cap=round,fill=fillColor] (235.85,418.45) circle (  1.96);
\path[draw=drawColor,line width= 0.4pt,line join=round,line cap=round,fill=fillColor] (236.06,418.63) circle (  1.96);
\path[draw=drawColor,line width= 0.4pt,line join=round,line cap=round,fill=fillColor] (236.27,418.82) circle (  1.96);
\path[draw=drawColor,line width= 0.4pt,line join=round,line cap=round,fill=fillColor] (236.48,419.00) circle (  1.96);
\path[draw=drawColor,line width= 0.4pt,line join=round,line cap=round,fill=fillColor] (236.69,419.18) circle (  1.96);
\path[draw=drawColor,line width= 0.4pt,line join=round,line cap=round,fill=fillColor] (236.90,419.36) circle (  1.96);
\path[draw=drawColor,line width= 0.4pt,line join=round,line cap=round,fill=fillColor] (237.11,419.54) circle (  1.96);
\path[draw=drawColor,line width= 0.4pt,line join=round,line cap=round,fill=fillColor] (237.31,419.72) circle (  1.96);
\path[draw=drawColor,line width= 0.4pt,line join=round,line cap=round,fill=fillColor] (237.52,419.90) circle (  1.96);
\path[draw=drawColor,line width= 0.4pt,line join=round,line cap=round,fill=fillColor] (237.73,420.08) circle (  1.96);
\path[draw=drawColor,line width= 0.4pt,line join=round,line cap=round,fill=fillColor] (237.94,420.26) circle (  1.96);
\path[draw=drawColor,line width= 0.4pt,line join=round,line cap=round,fill=fillColor] (238.15,420.43) circle (  1.96);
\path[draw=drawColor,line width= 0.4pt,line join=round,line cap=round,fill=fillColor] (238.36,420.61) circle (  1.96);
\path[draw=drawColor,line width= 0.4pt,line join=round,line cap=round,fill=fillColor] (238.57,420.79) circle (  1.96);
\path[draw=drawColor,line width= 0.4pt,line join=round,line cap=round,fill=fillColor] (238.78,420.97) circle (  1.96);
\path[draw=drawColor,line width= 0.4pt,line join=round,line cap=round,fill=fillColor] (238.99,421.14) circle (  1.96);
\path[draw=drawColor,line width= 0.4pt,line join=round,line cap=round,fill=fillColor] (239.20,421.32) circle (  1.96);
\path[draw=drawColor,line width= 0.4pt,line join=round,line cap=round,fill=fillColor] (239.41,421.49) circle (  1.96);
\path[draw=drawColor,line width= 0.4pt,line join=round,line cap=round,fill=fillColor] (239.62,421.67) circle (  1.96);
\path[draw=drawColor,line width= 0.4pt,line join=round,line cap=round,fill=fillColor] (239.83,421.84) circle (  1.96);
\path[draw=drawColor,line width= 0.4pt,line join=round,line cap=round,fill=fillColor] (240.04,422.02) circle (  1.96);
\path[draw=drawColor,line width= 0.4pt,line join=round,line cap=round,fill=fillColor] (240.25,422.19) circle (  1.96);
\path[draw=drawColor,line width= 0.4pt,line join=round,line cap=round,fill=fillColor] (240.46,422.36) circle (  1.96);
\path[draw=drawColor,line width= 0.4pt,line join=round,line cap=round,fill=fillColor] (240.67,422.53) circle (  1.96);
\path[draw=drawColor,line width= 0.4pt,line join=round,line cap=round,fill=fillColor] (240.88,422.71) circle (  1.96);
\path[draw=drawColor,line width= 0.4pt,line join=round,line cap=round,fill=fillColor] (241.09,422.88) circle (  1.96);
\path[draw=drawColor,line width= 0.4pt,line join=round,line cap=round,fill=fillColor] (241.30,423.05) circle (  1.96);
\path[draw=drawColor,line width= 0.4pt,line join=round,line cap=round,fill=fillColor] (241.51,423.22) circle (  1.96);
\path[draw=drawColor,line width= 0.4pt,line join=round,line cap=round,fill=fillColor] (241.72,423.39) circle (  1.96);
\path[draw=drawColor,line width= 0.4pt,line join=round,line cap=round,fill=fillColor] (241.93,423.56) circle (  1.96);
\path[draw=drawColor,line width= 0.4pt,line join=round,line cap=round,fill=fillColor] (242.14,423.73) circle (  1.96);
\path[draw=drawColor,line width= 0.4pt,line join=round,line cap=round,fill=fillColor] (242.35,423.90) circle (  1.96);
\path[draw=drawColor,line width= 0.4pt,line join=round,line cap=round,fill=fillColor] (242.56,424.07) circle (  1.96);
\path[draw=drawColor,line width= 0.4pt,line join=round,line cap=round,fill=fillColor] (242.76,424.24) circle (  1.96);
\path[draw=drawColor,line width= 0.4pt,line join=round,line cap=round,fill=fillColor] (242.97,424.40) circle (  1.96);
\path[draw=drawColor,line width= 0.4pt,line join=round,line cap=round,fill=fillColor] (243.18,424.57) circle (  1.96);
\path[draw=drawColor,line width= 0.4pt,line join=round,line cap=round,fill=fillColor] (243.39,424.74) circle (  1.96);
\path[draw=drawColor,line width= 0.4pt,line join=round,line cap=round,fill=fillColor] (243.60,424.91) circle (  1.96);
\path[draw=drawColor,line width= 0.4pt,line join=round,line cap=round,fill=fillColor] (243.81,425.07) circle (  1.96);
\path[draw=drawColor,line width= 0.4pt,line join=round,line cap=round,fill=fillColor] (244.02,425.24) circle (  1.96);
\path[draw=drawColor,line width= 0.4pt,line join=round,line cap=round,fill=fillColor] (244.23,425.40) circle (  1.96);
\path[draw=drawColor,line width= 0.4pt,line join=round,line cap=round,fill=fillColor] (244.44,425.57) circle (  1.96);
\path[draw=drawColor,line width= 0.4pt,line join=round,line cap=round,fill=fillColor] (244.65,425.73) circle (  1.96);
\path[draw=drawColor,line width= 0.4pt,line join=round,line cap=round,fill=fillColor] (244.86,425.89) circle (  1.96);
\path[draw=drawColor,line width= 0.4pt,line join=round,line cap=round,fill=fillColor] (245.07,426.06) circle (  1.96);
\path[draw=drawColor,line width= 0.4pt,line join=round,line cap=round,fill=fillColor] (245.28,426.22) circle (  1.96);
\path[draw=drawColor,line width= 0.4pt,line join=round,line cap=round,fill=fillColor] (245.49,426.38) circle (  1.96);
\path[draw=drawColor,line width= 0.4pt,line join=round,line cap=round,fill=fillColor] (245.70,426.55) circle (  1.96);
\path[draw=drawColor,line width= 0.4pt,line join=round,line cap=round,fill=fillColor] (245.91,426.71) circle (  1.96);
\path[draw=drawColor,line width= 0.4pt,line join=round,line cap=round,fill=fillColor] (246.12,426.87) circle (  1.96);
\path[draw=drawColor,line width= 0.4pt,line join=round,line cap=round,fill=fillColor] (246.33,427.03) circle (  1.96);
\path[draw=drawColor,line width= 0.4pt,line join=round,line cap=round,fill=fillColor] (246.54,427.19) circle (  1.96);
\path[draw=drawColor,line width= 0.4pt,line join=round,line cap=round,fill=fillColor] (246.75,427.35) circle (  1.96);
\path[draw=drawColor,line width= 0.4pt,line join=round,line cap=round,fill=fillColor] (246.96,427.51) circle (  1.96);
\path[draw=drawColor,line width= 0.4pt,line join=round,line cap=round,fill=fillColor] (247.17,427.67) circle (  1.96);
\path[draw=drawColor,line width= 0.4pt,line join=round,line cap=round,fill=fillColor] (247.38,427.83) circle (  1.96);
\path[draw=drawColor,line width= 0.4pt,line join=round,line cap=round,fill=fillColor] (247.59,427.99) circle (  1.96);
\path[draw=drawColor,line width= 0.4pt,line join=round,line cap=round,fill=fillColor] (247.80,428.14) circle (  1.96);
\path[draw=drawColor,line width= 0.4pt,line join=round,line cap=round,fill=fillColor] (248.01,428.30) circle (  1.96);
\path[draw=drawColor,line width= 0.4pt,line join=round,line cap=round,fill=fillColor] (248.22,428.46) circle (  1.96);
\path[draw=drawColor,line width= 0.4pt,line join=round,line cap=round,fill=fillColor] (248.42,428.62) circle (  1.96);
\path[draw=drawColor,line width= 0.4pt,line join=round,line cap=round,fill=fillColor] (248.63,428.77) circle (  1.96);
\path[draw=drawColor,line width= 0.4pt,line join=round,line cap=round,fill=fillColor] (248.84,428.93) circle (  1.96);
\path[draw=drawColor,line width= 0.4pt,line join=round,line cap=round,fill=fillColor] (249.05,429.08) circle (  1.96);
\path[draw=drawColor,line width= 0.4pt,line join=round,line cap=round,fill=fillColor] (249.26,429.24) circle (  1.96);
\path[draw=drawColor,line width= 0.4pt,line join=round,line cap=round,fill=fillColor] (249.47,429.39) circle (  1.96);
\path[draw=drawColor,line width= 0.4pt,line join=round,line cap=round,fill=fillColor] (249.68,429.55) circle (  1.96);
\path[draw=drawColor,line width= 0.4pt,line join=round,line cap=round,fill=fillColor] (249.89,429.70) circle (  1.96);
\path[draw=drawColor,line width= 0.4pt,line join=round,line cap=round,fill=fillColor] (250.10,429.85) circle (  1.96);
\path[draw=drawColor,line width= 0.4pt,line join=round,line cap=round,fill=fillColor] (250.31,430.01) circle (  1.96);
\path[draw=drawColor,line width= 0.4pt,line join=round,line cap=round,fill=fillColor] (250.52,430.16) circle (  1.96);
\path[draw=drawColor,line width= 0.4pt,line join=round,line cap=round,fill=fillColor] (250.73,430.31) circle (  1.96);
\path[draw=drawColor,line width= 0.4pt,line join=round,line cap=round,fill=fillColor] (250.94,430.46) circle (  1.96);
\path[draw=drawColor,line width= 0.4pt,line join=round,line cap=round,fill=fillColor] (251.15,430.61) circle (  1.96);
\path[draw=drawColor,line width= 0.4pt,line join=round,line cap=round,fill=fillColor] (251.36,430.77) circle (  1.96);
\path[draw=drawColor,line width= 0.4pt,line join=round,line cap=round,fill=fillColor] (251.57,430.92) circle (  1.96);
\path[draw=drawColor,line width= 0.4pt,line join=round,line cap=round,fill=fillColor] (251.78,431.07) circle (  1.96);
\path[draw=drawColor,line width= 0.4pt,line join=round,line cap=round,fill=fillColor] (251.99,431.22) circle (  1.96);
\path[draw=drawColor,line width= 0.4pt,line join=round,line cap=round,fill=fillColor] (252.20,431.36) circle (  1.96);
\path[draw=drawColor,line width= 0.4pt,line join=round,line cap=round,fill=fillColor] (252.41,431.51) circle (  1.96);
\path[draw=drawColor,line width= 0.4pt,line join=round,line cap=round,fill=fillColor] (252.62,431.66) circle (  1.96);
\path[draw=drawColor,line width= 0.4pt,line join=round,line cap=round,fill=fillColor] (252.83,431.81) circle (  1.96);
\path[draw=drawColor,line width= 0.4pt,line join=round,line cap=round,fill=fillColor] (253.04,431.96) circle (  1.96);
\path[draw=drawColor,line width= 0.4pt,line join=round,line cap=round,fill=fillColor] (253.25,432.11) circle (  1.96);
\path[draw=drawColor,line width= 0.4pt,line join=round,line cap=round,fill=fillColor] (253.46,432.25) circle (  1.96);
\path[draw=drawColor,line width= 0.4pt,line join=round,line cap=round,fill=fillColor] (253.67,432.40) circle (  1.96);
\path[draw=drawColor,line width= 0.4pt,line join=round,line cap=round,fill=fillColor] (253.87,432.54) circle (  1.96);
\path[draw=drawColor,line width= 0.4pt,line join=round,line cap=round,fill=fillColor] (254.08,432.69) circle (  1.96);
\path[draw=drawColor,line width= 0.4pt,line join=round,line cap=round,fill=fillColor] (254.29,432.84) circle (  1.96);
\path[draw=drawColor,line width= 0.4pt,line join=round,line cap=round,fill=fillColor] (254.50,432.98) circle (  1.96);
\path[draw=drawColor,line width= 0.4pt,line join=round,line cap=round,fill=fillColor] (254.71,433.13) circle (  1.96);
\path[draw=drawColor,line width= 0.4pt,line join=round,line cap=round,fill=fillColor] (254.92,433.27) circle (  1.96);
\path[draw=drawColor,line width= 0.4pt,line join=round,line cap=round,fill=fillColor] (255.13,433.41) circle (  1.96);
\path[draw=drawColor,line width= 0.4pt,line join=round,line cap=round,fill=fillColor] (255.34,433.56) circle (  1.96);
\path[draw=drawColor,line width= 0.4pt,line join=round,line cap=round,fill=fillColor] (255.55,433.70) circle (  1.96);
\path[draw=drawColor,line width= 0.4pt,line join=round,line cap=round,fill=fillColor] (255.76,433.84) circle (  1.96);
\path[draw=drawColor,line width= 0.4pt,line join=round,line cap=round,fill=fillColor] (255.97,433.98) circle (  1.96);
\path[draw=drawColor,line width= 0.4pt,line join=round,line cap=round,fill=fillColor] (256.18,434.13) circle (  1.96);
\path[draw=drawColor,line width= 0.4pt,line join=round,line cap=round,fill=fillColor] (256.39,434.27) circle (  1.96);
\path[draw=drawColor,line width= 0.4pt,line join=round,line cap=round,fill=fillColor] (256.60,434.41) circle (  1.96);
\path[draw=drawColor,line width= 0.4pt,line join=round,line cap=round,fill=fillColor] (256.81,434.55) circle (  1.96);
\path[draw=drawColor,line width= 0.4pt,line join=round,line cap=round,fill=fillColor] (257.02,434.69) circle (  1.96);
\path[draw=drawColor,line width= 0.4pt,line join=round,line cap=round,fill=fillColor] (257.23,434.83) circle (  1.96);
\path[draw=drawColor,line width= 0.4pt,line join=round,line cap=round,fill=fillColor] (257.44,434.97) circle (  1.96);
\path[draw=drawColor,line width= 0.4pt,line join=round,line cap=round,fill=fillColor] (257.65,435.11) circle (  1.96);
\path[draw=drawColor,line width= 0.4pt,line join=round,line cap=round,fill=fillColor] (257.86,435.25) circle (  1.96);
\path[draw=drawColor,line width= 0.4pt,line join=round,line cap=round,fill=fillColor] (258.07,435.39) circle (  1.96);
\path[draw=drawColor,line width= 0.4pt,line join=round,line cap=round,fill=fillColor] (258.28,435.52) circle (  1.96);
\path[draw=drawColor,line width= 0.4pt,line join=round,line cap=round,fill=fillColor] (258.49,435.66) circle (  1.96);
\path[draw=drawColor,line width= 0.4pt,line join=round,line cap=round,fill=fillColor] (258.70,435.80) circle (  1.96);
\path[draw=drawColor,line width= 0.4pt,line join=round,line cap=round,fill=fillColor] (258.91,435.94) circle (  1.96);
\path[draw=drawColor,line width= 0.4pt,line join=round,line cap=round,fill=fillColor] (259.12,436.07) circle (  1.96);
\path[draw=drawColor,line width= 0.4pt,line join=round,line cap=round,fill=fillColor] (259.33,436.21) circle (  1.96);
\path[draw=drawColor,line width= 0.4pt,line join=round,line cap=round,fill=fillColor] (259.53,436.34) circle (  1.96);
\path[draw=drawColor,line width= 0.4pt,line join=round,line cap=round,fill=fillColor] (259.74,436.48) circle (  1.96);
\path[draw=drawColor,line width= 0.4pt,line join=round,line cap=round,fill=fillColor] (259.95,436.61) circle (  1.96);
\path[draw=drawColor,line width= 0.4pt,line join=round,line cap=round,fill=fillColor] (260.16,436.75) circle (  1.96);
\path[draw=drawColor,line width= 0.4pt,line join=round,line cap=round,fill=fillColor] (260.37,436.88) circle (  1.96);
\path[draw=drawColor,line width= 0.4pt,line join=round,line cap=round,fill=fillColor] (260.58,437.02) circle (  1.96);
\path[draw=drawColor,line width= 0.4pt,line join=round,line cap=round,fill=fillColor] (260.79,437.15) circle (  1.96);
\path[draw=drawColor,line width= 0.4pt,line join=round,line cap=round,fill=fillColor] (261.00,437.28) circle (  1.96);
\path[draw=drawColor,line width= 0.4pt,line join=round,line cap=round,fill=fillColor] (261.21,437.42) circle (  1.96);
\path[draw=drawColor,line width= 0.4pt,line join=round,line cap=round,fill=fillColor] (261.42,437.55) circle (  1.96);
\path[draw=drawColor,line width= 0.4pt,line join=round,line cap=round,fill=fillColor] (261.63,437.68) circle (  1.96);
\path[draw=drawColor,line width= 0.4pt,line join=round,line cap=round,fill=fillColor] (261.84,437.81) circle (  1.96);
\path[draw=drawColor,line width= 0.4pt,line join=round,line cap=round,fill=fillColor] (262.05,437.94) circle (  1.96);
\path[draw=drawColor,line width= 0.4pt,line join=round,line cap=round,fill=fillColor] (262.26,438.08) circle (  1.96);
\path[draw=drawColor,line width= 0.4pt,line join=round,line cap=round,fill=fillColor] (262.47,438.21) circle (  1.96);
\path[draw=drawColor,line width= 0.4pt,line join=round,line cap=round,fill=fillColor] (262.68,438.34) circle (  1.96);
\path[draw=drawColor,line width= 0.4pt,line join=round,line cap=round,fill=fillColor] (262.89,438.47) circle (  1.96);
\path[draw=drawColor,line width= 0.4pt,line join=round,line cap=round,fill=fillColor] (263.10,438.60) circle (  1.96);
\path[draw=drawColor,line width= 0.4pt,line join=round,line cap=round,fill=fillColor] (263.31,438.72) circle (  1.96);
\path[draw=drawColor,line width= 0.4pt,line join=round,line cap=round,fill=fillColor] (263.52,438.85) circle (  1.96);
\path[draw=drawColor,line width= 0.4pt,line join=round,line cap=round,fill=fillColor] (263.73,438.98) circle (  1.96);
\path[draw=drawColor,line width= 0.4pt,line join=round,line cap=round,fill=fillColor] (263.94,439.11) circle (  1.96);
\path[draw=drawColor,line width= 0.4pt,line join=round,line cap=round,fill=fillColor] (264.15,439.24) circle (  1.96);
\path[draw=drawColor,line width= 0.4pt,line join=round,line cap=round,fill=fillColor] (264.36,439.37) circle (  1.96);
\path[draw=drawColor,line width= 0.4pt,line join=round,line cap=round,fill=fillColor] (264.57,439.49) circle (  1.96);
\path[draw=drawColor,line width= 0.4pt,line join=round,line cap=round,fill=fillColor] (264.78,439.62) circle (  1.96);
\path[draw=drawColor,line width= 0.4pt,line join=round,line cap=round,fill=fillColor] (264.98,439.75) circle (  1.96);
\path[draw=drawColor,line width= 0.4pt,line join=round,line cap=round,fill=fillColor] (265.19,439.87) circle (  1.96);
\path[draw=drawColor,line width= 0.4pt,line join=round,line cap=round,fill=fillColor] (265.40,440.00) circle (  1.96);
\path[draw=drawColor,line width= 0.4pt,line join=round,line cap=round,fill=fillColor] (265.61,440.12) circle (  1.96);
\path[draw=drawColor,line width= 0.4pt,line join=round,line cap=round,fill=fillColor] (265.82,440.25) circle (  1.96);
\path[draw=drawColor,line width= 0.4pt,line join=round,line cap=round,fill=fillColor] (266.03,440.37) circle (  1.96);
\path[draw=drawColor,line width= 0.4pt,line join=round,line cap=round,fill=fillColor] (266.24,440.50) circle (  1.96);
\path[draw=drawColor,line width= 0.4pt,line join=round,line cap=round,fill=fillColor] (266.45,440.62) circle (  1.96);
\path[draw=drawColor,line width= 0.4pt,line join=round,line cap=round,fill=fillColor] (266.66,440.74) circle (  1.96);
\path[draw=drawColor,line width= 0.4pt,line join=round,line cap=round,fill=fillColor] (266.87,440.87) circle (  1.96);
\path[draw=drawColor,line width= 0.4pt,line join=round,line cap=round,fill=fillColor] (267.08,440.99) circle (  1.96);
\path[draw=drawColor,line width= 0.4pt,line join=round,line cap=round,fill=fillColor] (267.29,441.11) circle (  1.96);
\path[draw=drawColor,line width= 0.4pt,line join=round,line cap=round,fill=fillColor] (267.50,441.23) circle (  1.96);
\path[draw=drawColor,line width= 0.4pt,line join=round,line cap=round,fill=fillColor] (267.71,441.36) circle (  1.96);
\path[draw=drawColor,line width= 0.4pt,line join=round,line cap=round,fill=fillColor] (267.92,441.48) circle (  1.96);
\path[draw=drawColor,line width= 0.4pt,line join=round,line cap=round,fill=fillColor] (268.13,441.60) circle (  1.96);
\path[draw=drawColor,line width= 0.4pt,line join=round,line cap=round,fill=fillColor] (268.34,441.72) circle (  1.96);
\path[draw=drawColor,line width= 0.4pt,line join=round,line cap=round,fill=fillColor] (268.55,441.84) circle (  1.96);
\path[draw=drawColor,line width= 0.4pt,line join=round,line cap=round,fill=fillColor] (268.76,441.96) circle (  1.96);
\path[draw=drawColor,line width= 0.4pt,line join=round,line cap=round,fill=fillColor] (268.97,442.08) circle (  1.96);
\path[draw=drawColor,line width= 0.4pt,line join=round,line cap=round,fill=fillColor] (269.18,442.20) circle (  1.96);
\path[draw=drawColor,line width= 0.4pt,line join=round,line cap=round,fill=fillColor] (269.39,442.32) circle (  1.96);
\path[draw=drawColor,line width= 0.4pt,line join=round,line cap=round,fill=fillColor] (269.60,442.44) circle (  1.96);
\path[draw=drawColor,line width= 0.4pt,line join=round,line cap=round,fill=fillColor] (269.81,442.56) circle (  1.96);
\path[draw=drawColor,line width= 0.4pt,line join=round,line cap=round,fill=fillColor] (270.02,442.67) circle (  1.96);
\path[draw=drawColor,line width= 0.4pt,line join=round,line cap=round,fill=fillColor] (270.23,442.79) circle (  1.96);
\path[draw=drawColor,line width= 0.4pt,line join=round,line cap=round,fill=fillColor] (270.43,442.91) circle (  1.96);
\path[draw=drawColor,line width= 0.4pt,line join=round,line cap=round,fill=fillColor] (270.64,443.03) circle (  1.96);
\path[draw=drawColor,line width= 0.4pt,line join=round,line cap=round,fill=fillColor] (270.85,443.14) circle (  1.96);
\path[draw=drawColor,line width= 0.4pt,line join=round,line cap=round,fill=fillColor] (271.06,443.26) circle (  1.96);
\path[draw=drawColor,line width= 0.4pt,line join=round,line cap=round,fill=fillColor] (271.27,443.38) circle (  1.96);
\path[draw=drawColor,line width= 0.4pt,line join=round,line cap=round,fill=fillColor] (271.48,443.49) circle (  1.96);
\path[draw=drawColor,line width= 0.4pt,line join=round,line cap=round,fill=fillColor] (271.69,443.61) circle (  1.96);
\path[draw=drawColor,line width= 0.4pt,line join=round,line cap=round,fill=fillColor] (271.90,443.72) circle (  1.96);
\path[draw=drawColor,line width= 0.4pt,line join=round,line cap=round,fill=fillColor] (272.11,443.84) circle (  1.96);
\path[draw=drawColor,line width= 0.4pt,line join=round,line cap=round,fill=fillColor] (272.32,443.95) circle (  1.96);
\path[draw=drawColor,line width= 0.4pt,line join=round,line cap=round,fill=fillColor] (272.53,444.06) circle (  1.96);
\path[draw=drawColor,line width= 0.4pt,line join=round,line cap=round,fill=fillColor] (272.74,444.18) circle (  1.96);
\path[draw=drawColor,line width= 0.4pt,line join=round,line cap=round,fill=fillColor] (272.95,444.29) circle (  1.96);
\path[draw=drawColor,line width= 0.4pt,line join=round,line cap=round,fill=fillColor] (273.16,444.41) circle (  1.96);
\path[draw=drawColor,line width= 0.4pt,line join=round,line cap=round,fill=fillColor] (273.37,444.52) circle (  1.96);
\path[draw=drawColor,line width= 0.4pt,line join=round,line cap=round,fill=fillColor] (273.58,444.63) circle (  1.96);
\path[draw=drawColor,line width= 0.4pt,line join=round,line cap=round,fill=fillColor] (273.79,444.74) circle (  1.96);
\path[draw=drawColor,line width= 0.4pt,line join=round,line cap=round,fill=fillColor] (274.00,444.86) circle (  1.96);
\path[draw=drawColor,line width= 0.4pt,line join=round,line cap=round,fill=fillColor] (274.21,444.97) circle (  1.96);
\path[draw=drawColor,line width= 0.4pt,line join=round,line cap=round,fill=fillColor] (274.42,445.08) circle (  1.96);
\path[draw=drawColor,line width= 0.4pt,line join=round,line cap=round,fill=fillColor] (274.63,445.19) circle (  1.96);
\path[draw=drawColor,line width= 0.4pt,line join=round,line cap=round,fill=fillColor] (274.84,445.30) circle (  1.96);
\path[draw=drawColor,line width= 0.4pt,line join=round,line cap=round,fill=fillColor] (275.05,445.41) circle (  1.96);
\path[draw=drawColor,line width= 0.4pt,line join=round,line cap=round,fill=fillColor] (275.26,445.52) circle (  1.96);
\path[draw=drawColor,line width= 0.4pt,line join=round,line cap=round,fill=fillColor] (275.47,445.63) circle (  1.96);
\path[draw=drawColor,line width= 0.4pt,line join=round,line cap=round,fill=fillColor] (275.68,445.74) circle (  1.96);
\path[draw=drawColor,line width= 0.4pt,line join=round,line cap=round,fill=fillColor] (275.89,445.85) circle (  1.96);
\path[draw=drawColor,line width= 0.4pt,line join=round,line cap=round,fill=fillColor] (276.09,445.96) circle (  1.96);
\path[draw=drawColor,line width= 0.4pt,line join=round,line cap=round,fill=fillColor] (276.30,446.07) circle (  1.96);
\path[draw=drawColor,line width= 0.4pt,line join=round,line cap=round,fill=fillColor] (276.51,446.17) circle (  1.96);
\path[draw=drawColor,line width= 0.4pt,line join=round,line cap=round,fill=fillColor] (276.72,446.28) circle (  1.96);
\path[draw=drawColor,line width= 0.4pt,line join=round,line cap=round,fill=fillColor] (276.93,446.39) circle (  1.96);
\path[draw=drawColor,line width= 0.4pt,line join=round,line cap=round,fill=fillColor] (277.14,446.50) circle (  1.96);
\path[draw=drawColor,line width= 0.4pt,line join=round,line cap=round,fill=fillColor] (277.35,446.60) circle (  1.96);
\path[draw=drawColor,line width= 0.4pt,line join=round,line cap=round,fill=fillColor] (277.56,446.71) circle (  1.96);
\path[draw=drawColor,line width= 0.4pt,line join=round,line cap=round,fill=fillColor] (277.77,446.82) circle (  1.96);
\path[draw=drawColor,line width= 0.4pt,line join=round,line cap=round,fill=fillColor] (277.98,446.92) circle (  1.96);
\path[draw=drawColor,line width= 0.4pt,line join=round,line cap=round,fill=fillColor] (278.19,447.03) circle (  1.96);
\path[draw=drawColor,line width= 0.4pt,line join=round,line cap=round,fill=fillColor] (278.40,447.13) circle (  1.96);
\path[draw=drawColor,line width= 0.4pt,line join=round,line cap=round,fill=fillColor] (278.61,447.24) circle (  1.96);
\path[draw=drawColor,line width= 0.4pt,line join=round,line cap=round,fill=fillColor] (278.82,447.34) circle (  1.96);
\path[draw=drawColor,line width= 0.4pt,line join=round,line cap=round,fill=fillColor] (279.03,447.45) circle (  1.96);
\path[draw=drawColor,line width= 0.4pt,line join=round,line cap=round,fill=fillColor] (279.24,447.55) circle (  1.96);
\path[draw=drawColor,line width= 0.4pt,line join=round,line cap=round,fill=fillColor] (279.45,447.65) circle (  1.96);
\path[draw=drawColor,line width= 0.4pt,line join=round,line cap=round,fill=fillColor] (279.66,447.76) circle (  1.96);
\path[draw=drawColor,line width= 0.4pt,line join=round,line cap=round,fill=fillColor] (279.87,447.86) circle (  1.96);
\path[draw=drawColor,line width= 0.4pt,line join=round,line cap=round,fill=fillColor] (280.08,447.96) circle (  1.96);
\path[draw=drawColor,line width= 0.4pt,line join=round,line cap=round,fill=fillColor] (280.29,448.07) circle (  1.96);
\path[draw=drawColor,line width= 0.4pt,line join=round,line cap=round,fill=fillColor] (280.50,448.17) circle (  1.96);
\path[draw=drawColor,line width= 0.4pt,line join=round,line cap=round,fill=fillColor] (280.71,448.27) circle (  1.96);
\path[draw=drawColor,line width= 0.4pt,line join=round,line cap=round,fill=fillColor] (280.92,448.37) circle (  1.96);
\path[draw=drawColor,line width= 0.4pt,line join=round,line cap=round,fill=fillColor] (281.13,448.47) circle (  1.96);
\path[draw=drawColor,line width= 0.4pt,line join=round,line cap=round,fill=fillColor] (281.34,448.57) circle (  1.96);
\path[draw=drawColor,line width= 0.4pt,line join=round,line cap=round,fill=fillColor] (281.54,448.68) circle (  1.96);
\path[draw=drawColor,line width= 0.4pt,line join=round,line cap=round,fill=fillColor] (281.75,448.78) circle (  1.96);
\path[draw=drawColor,line width= 0.4pt,line join=round,line cap=round,fill=fillColor] (281.96,448.88) circle (  1.96);
\path[draw=drawColor,line width= 0.4pt,line join=round,line cap=round,fill=fillColor] (282.17,448.98) circle (  1.96);
\path[draw=drawColor,line width= 0.4pt,line join=round,line cap=round,fill=fillColor] (282.38,449.08) circle (  1.96);
\path[draw=drawColor,line width= 0.4pt,line join=round,line cap=round,fill=fillColor] (282.59,449.18) circle (  1.96);
\path[draw=drawColor,line width= 0.4pt,line join=round,line cap=round,fill=fillColor] (282.80,449.27) circle (  1.96);
\path[draw=drawColor,line width= 0.4pt,line join=round,line cap=round,fill=fillColor] (283.01,449.37) circle (  1.96);
\path[draw=drawColor,line width= 0.4pt,line join=round,line cap=round,fill=fillColor] (283.22,449.47) circle (  1.96);
\path[draw=drawColor,line width= 0.4pt,line join=round,line cap=round,fill=fillColor] (283.43,449.57) circle (  1.96);
\path[draw=drawColor,line width= 0.4pt,line join=round,line cap=round,fill=fillColor] (283.64,449.67) circle (  1.96);
\path[draw=drawColor,line width= 0.4pt,line join=round,line cap=round,fill=fillColor] (283.85,449.77) circle (  1.96);
\path[draw=drawColor,line width= 0.4pt,line join=round,line cap=round,fill=fillColor] (284.06,449.86) circle (  1.96);
\path[draw=drawColor,line width= 0.4pt,line join=round,line cap=round,fill=fillColor] (284.27,449.96) circle (  1.96);
\path[draw=drawColor,line width= 0.4pt,line join=round,line cap=round,fill=fillColor] (284.48,450.06) circle (  1.96);
\path[draw=drawColor,line width= 0.4pt,line join=round,line cap=round,fill=fillColor] (284.69,450.15) circle (  1.96);
\path[draw=drawColor,line width= 0.4pt,line join=round,line cap=round,fill=fillColor] (284.90,450.25) circle (  1.96);
\path[draw=drawColor,line width= 0.4pt,line join=round,line cap=round,fill=fillColor] (285.11,450.34) circle (  1.96);
\path[draw=drawColor,line width= 0.4pt,line join=round,line cap=round,fill=fillColor] (285.32,450.44) circle (  1.96);
\path[draw=drawColor,line width= 0.4pt,line join=round,line cap=round,fill=fillColor] (285.53,450.54) circle (  1.96);
\path[draw=drawColor,line width= 0.4pt,line join=round,line cap=round,fill=fillColor] (285.74,450.63) circle (  1.96);
\path[draw=drawColor,line width= 0.4pt,line join=round,line cap=round,fill=fillColor] (285.95,450.73) circle (  1.96);
\path[draw=drawColor,line width= 0.4pt,line join=round,line cap=round,fill=fillColor] (286.16,450.82) circle (  1.96);
\path[draw=drawColor,line width= 0.4pt,line join=round,line cap=round,fill=fillColor] (286.37,450.91) circle (  1.96);
\path[draw=drawColor,line width= 0.4pt,line join=round,line cap=round,fill=fillColor] (286.58,451.01) circle (  1.96);
\path[draw=drawColor,line width= 0.4pt,line join=round,line cap=round,fill=fillColor] (286.79,451.10) circle (  1.96);
\path[draw=drawColor,line width= 0.4pt,line join=round,line cap=round,fill=fillColor] (287.00,451.20) circle (  1.96);
\path[draw=drawColor,line width= 0.4pt,line join=round,line cap=round,fill=fillColor] (287.20,451.29) circle (  1.96);
\path[draw=drawColor,line width= 0.4pt,line join=round,line cap=round,fill=fillColor] (287.41,451.38) circle (  1.96);
\path[draw=drawColor,line width= 0.4pt,line join=round,line cap=round,fill=fillColor] (287.62,451.47) circle (  1.96);
\path[draw=drawColor,line width= 0.4pt,line join=round,line cap=round,fill=fillColor] (287.83,451.57) circle (  1.96);
\path[draw=drawColor,line width= 0.4pt,line join=round,line cap=round,fill=fillColor] (288.04,451.66) circle (  1.96);
\path[draw=drawColor,line width= 0.4pt,line join=round,line cap=round,fill=fillColor] (288.25,451.75) circle (  1.96);
\path[draw=drawColor,line width= 0.4pt,line join=round,line cap=round,fill=fillColor] (288.46,451.84) circle (  1.96);
\path[draw=drawColor,line width= 0.4pt,line join=round,line cap=round,fill=fillColor] (288.67,451.93) circle (  1.96);
\path[draw=drawColor,line width= 0.4pt,line join=round,line cap=round,fill=fillColor] (288.88,452.02) circle (  1.96);
\path[draw=drawColor,line width= 0.4pt,line join=round,line cap=round,fill=fillColor] (289.09,452.11) circle (  1.96);
\path[draw=drawColor,line width= 0.4pt,line join=round,line cap=round,fill=fillColor] (289.30,452.20) circle (  1.96);
\path[draw=drawColor,line width= 0.4pt,line join=round,line cap=round,fill=fillColor] (289.51,452.29) circle (  1.96);
\path[draw=drawColor,line width= 0.4pt,line join=round,line cap=round,fill=fillColor] (289.72,452.38) circle (  1.96);
\path[draw=drawColor,line width= 0.4pt,line join=round,line cap=round,fill=fillColor] (289.93,452.47) circle (  1.96);
\path[draw=drawColor,line width= 0.4pt,line join=round,line cap=round,fill=fillColor] (290.14,452.56) circle (  1.96);
\path[draw=drawColor,line width= 0.4pt,line join=round,line cap=round,fill=fillColor] (290.35,452.65) circle (  1.96);
\path[draw=drawColor,line width= 0.4pt,line join=round,line cap=round,fill=fillColor] (290.56,452.74) circle (  1.96);
\path[draw=drawColor,line width= 0.4pt,line join=round,line cap=round,fill=fillColor] (290.77,452.83) circle (  1.96);
\path[draw=drawColor,line width= 0.4pt,line join=round,line cap=round,fill=fillColor] (290.98,452.92) circle (  1.96);
\path[draw=drawColor,line width= 0.4pt,line join=round,line cap=round,fill=fillColor] (291.19,453.01) circle (  1.96);
\path[draw=drawColor,line width= 0.4pt,line join=round,line cap=round,fill=fillColor] (291.40,453.09) circle (  1.96);
\path[draw=drawColor,line width= 0.4pt,line join=round,line cap=round,fill=fillColor] (291.61,453.18) circle (  1.96);
\path[draw=drawColor,line width= 0.4pt,line join=round,line cap=round,fill=fillColor] (291.82,453.27) circle (  1.96);
\path[draw=drawColor,line width= 0.4pt,line join=round,line cap=round,fill=fillColor] (292.03,453.36) circle (  1.96);
\path[draw=drawColor,line width= 0.4pt,line join=round,line cap=round,fill=fillColor] (292.24,453.44) circle (  1.96);
\path[draw=drawColor,line width= 0.4pt,line join=round,line cap=round,fill=fillColor] (292.45,453.53) circle (  1.96);
\path[draw=drawColor,line width= 0.4pt,line join=round,line cap=round,fill=fillColor] (292.65,453.61) circle (  1.96);
\path[draw=drawColor,line width= 0.4pt,line join=round,line cap=round,fill=fillColor] (292.86,453.70) circle (  1.96);
\path[draw=drawColor,line width= 0.4pt,line join=round,line cap=round,fill=fillColor] (293.07,453.79) circle (  1.96);
\path[draw=drawColor,line width= 0.4pt,line join=round,line cap=round,fill=fillColor] (293.28,453.87) circle (  1.96);
\path[draw=drawColor,line width= 0.4pt,line join=round,line cap=round,fill=fillColor] (293.49,453.96) circle (  1.96);
\path[draw=drawColor,line width= 0.4pt,line join=round,line cap=round,fill=fillColor] (293.70,454.04) circle (  1.96);
\path[draw=drawColor,line width= 0.4pt,line join=round,line cap=round,fill=fillColor] (293.91,454.13) circle (  1.96);
\path[draw=drawColor,line width= 0.4pt,line join=round,line cap=round,fill=fillColor] (294.12,454.21) circle (  1.96);
\path[draw=drawColor,line width= 0.4pt,line join=round,line cap=round,fill=fillColor] (294.33,454.29) circle (  1.96);
\path[draw=drawColor,line width= 0.4pt,line join=round,line cap=round,fill=fillColor] (294.54,454.38) circle (  1.96);
\path[draw=drawColor,line width= 0.4pt,line join=round,line cap=round,fill=fillColor] (294.75,454.46) circle (  1.96);
\path[draw=drawColor,line width= 0.4pt,line join=round,line cap=round,fill=fillColor] (294.96,454.55) circle (  1.96);
\path[draw=drawColor,line width= 0.4pt,line join=round,line cap=round,fill=fillColor] (295.17,454.63) circle (  1.96);
\path[draw=drawColor,line width= 0.4pt,line join=round,line cap=round,fill=fillColor] (295.38,454.71) circle (  1.96);
\path[draw=drawColor,line width= 0.4pt,line join=round,line cap=round,fill=fillColor] (295.59,454.79) circle (  1.96);
\path[draw=drawColor,line width= 0.4pt,line join=round,line cap=round,fill=fillColor] (295.80,454.88) circle (  1.96);
\path[draw=drawColor,line width= 0.4pt,line join=round,line cap=round,fill=fillColor] (296.01,454.96) circle (  1.96);
\path[draw=drawColor,line width= 0.4pt,line join=round,line cap=round,fill=fillColor] (296.22,455.04) circle (  1.96);
\path[draw=drawColor,line width= 0.4pt,line join=round,line cap=round,fill=fillColor] (296.43,455.12) circle (  1.96);
\path[draw=drawColor,line width= 0.4pt,line join=round,line cap=round,fill=fillColor] (296.64,455.20) circle (  1.96);
\path[draw=drawColor,line width= 0.4pt,line join=round,line cap=round,fill=fillColor] (296.85,455.28) circle (  1.96);
\path[draw=drawColor,line width= 0.4pt,line join=round,line cap=round,fill=fillColor] (297.06,455.36) circle (  1.96);
\path[draw=drawColor,line width= 0.4pt,line join=round,line cap=round,fill=fillColor] (297.27,455.45) circle (  1.96);
\path[draw=drawColor,line width= 0.4pt,line join=round,line cap=round,fill=fillColor] (297.48,455.53) circle (  1.96);
\path[draw=drawColor,line width= 0.4pt,line join=round,line cap=round,fill=fillColor] (297.69,455.61) circle (  1.96);
\path[draw=drawColor,line width= 0.4pt,line join=round,line cap=round,fill=fillColor] (297.90,455.69) circle (  1.96);
\path[draw=drawColor,line width= 0.4pt,line join=round,line cap=round,fill=fillColor] (298.11,455.77) circle (  1.96);
\path[draw=drawColor,line width= 0.4pt,line join=round,line cap=round,fill=fillColor] (298.31,455.84) circle (  1.96);
\path[draw=drawColor,line width= 0.4pt,line join=round,line cap=round,fill=fillColor] (298.52,455.92) circle (  1.96);
\path[draw=drawColor,line width= 0.4pt,line join=round,line cap=round,fill=fillColor] (298.73,456.00) circle (  1.96);
\path[draw=drawColor,line width= 0.4pt,line join=round,line cap=round,fill=fillColor] (298.94,456.08) circle (  1.96);
\path[draw=drawColor,line width= 0.4pt,line join=round,line cap=round,fill=fillColor] (299.15,456.16) circle (  1.96);
\path[draw=drawColor,line width= 0.4pt,line join=round,line cap=round,fill=fillColor] (299.36,456.24) circle (  1.96);
\path[draw=drawColor,line width= 0.4pt,line join=round,line cap=round,fill=fillColor] (299.57,456.32) circle (  1.96);
\path[draw=drawColor,line width= 0.4pt,line join=round,line cap=round,fill=fillColor] (299.78,456.39) circle (  1.96);
\path[draw=drawColor,line width= 0.4pt,line join=round,line cap=round,fill=fillColor] (299.99,456.47) circle (  1.96);
\path[draw=drawColor,line width= 0.4pt,line join=round,line cap=round,fill=fillColor] (300.20,456.55) circle (  1.96);
\path[draw=drawColor,line width= 0.4pt,line join=round,line cap=round,fill=fillColor] (300.41,456.63) circle (  1.96);
\path[draw=drawColor,line width= 0.4pt,line join=round,line cap=round,fill=fillColor] (300.62,456.70) circle (  1.96);
\path[draw=drawColor,line width= 0.4pt,line join=round,line cap=round,fill=fillColor] (300.83,456.78) circle (  1.96);
\path[draw=drawColor,line width= 0.4pt,line join=round,line cap=round,fill=fillColor] (301.04,456.86) circle (  1.96);
\path[draw=drawColor,line width= 0.4pt,line join=round,line cap=round,fill=fillColor] (301.25,456.93) circle (  1.96);
\path[draw=drawColor,line width= 0.4pt,line join=round,line cap=round,fill=fillColor] (301.46,457.01) circle (  1.96);
\path[draw=drawColor,line width= 0.4pt,line join=round,line cap=round,fill=fillColor] (301.67,457.08) circle (  1.96);
\path[draw=drawColor,line width= 0.4pt,line join=round,line cap=round,fill=fillColor] (301.88,457.16) circle (  1.96);
\path[draw=drawColor,line width= 0.4pt,line join=round,line cap=round,fill=fillColor] (302.09,457.23) circle (  1.96);
\path[draw=drawColor,line width= 0.4pt,line join=round,line cap=round,fill=fillColor] (302.30,457.31) circle (  1.96);
\path[draw=drawColor,line width= 0.4pt,line join=round,line cap=round,fill=fillColor] (302.51,457.38) circle (  1.96);
\path[draw=drawColor,line width= 0.4pt,line join=round,line cap=round,fill=fillColor] (302.72,457.46) circle (  1.96);
\path[draw=drawColor,line width= 0.4pt,line join=round,line cap=round,fill=fillColor] (302.93,457.53) circle (  1.96);
\path[draw=drawColor,line width= 0.4pt,line join=round,line cap=round,fill=fillColor] (303.14,457.61) circle (  1.96);
\path[draw=drawColor,line width= 0.4pt,line join=round,line cap=round,fill=fillColor] (303.35,457.68) circle (  1.96);
\path[draw=drawColor,line width= 0.4pt,line join=round,line cap=round,fill=fillColor] (303.56,457.75) circle (  1.96);
\path[draw=drawColor,line width= 0.4pt,line join=round,line cap=round,fill=fillColor] (303.76,457.83) circle (  1.96);
\path[draw=drawColor,line width= 0.4pt,line join=round,line cap=round,fill=fillColor] (303.97,457.90) circle (  1.96);
\path[draw=drawColor,line width= 0.4pt,line join=round,line cap=round,fill=fillColor] (304.18,457.97) circle (  1.96);
\path[draw=drawColor,line width= 0.4pt,line join=round,line cap=round,fill=fillColor] (304.39,458.05) circle (  1.96);
\path[draw=drawColor,line width= 0.4pt,line join=round,line cap=round,fill=fillColor] (304.60,458.12) circle (  1.96);
\path[draw=drawColor,line width= 0.4pt,line join=round,line cap=round,fill=fillColor] (304.81,458.19) circle (  1.96);
\path[draw=drawColor,line width= 0.4pt,line join=round,line cap=round,fill=fillColor] (305.02,458.26) circle (  1.96);
\path[draw=drawColor,line width= 0.4pt,line join=round,line cap=round,fill=fillColor] (305.23,458.33) circle (  1.96);
\path[draw=drawColor,line width= 0.4pt,line join=round,line cap=round,fill=fillColor] (305.44,458.40) circle (  1.96);
\path[draw=drawColor,line width= 0.4pt,line join=round,line cap=round,fill=fillColor] (305.65,458.48) circle (  1.96);
\path[draw=drawColor,line width= 0.4pt,line join=round,line cap=round,fill=fillColor] (305.86,458.55) circle (  1.96);
\path[draw=drawColor,line width= 0.4pt,line join=round,line cap=round,fill=fillColor] (306.07,458.62) circle (  1.96);
\path[draw=drawColor,line width= 0.4pt,line join=round,line cap=round,fill=fillColor] (306.28,458.69) circle (  1.96);
\path[draw=drawColor,line width= 0.4pt,line join=round,line cap=round,fill=fillColor] (306.49,458.76) circle (  1.96);
\path[draw=drawColor,line width= 0.4pt,line join=round,line cap=round,fill=fillColor] (306.70,458.83) circle (  1.96);
\path[draw=drawColor,line width= 0.4pt,line join=round,line cap=round,fill=fillColor] (306.91,458.90) circle (  1.96);
\path[draw=drawColor,line width= 0.4pt,line join=round,line cap=round,fill=fillColor] (307.12,458.97) circle (  1.96);
\path[draw=drawColor,line width= 0.4pt,line join=round,line cap=round,fill=fillColor] (307.33,459.04) circle (  1.96);
\path[draw=drawColor,line width= 0.4pt,line join=round,line cap=round,fill=fillColor] (307.54,459.11) circle (  1.96);
\path[draw=drawColor,line width= 0.4pt,line join=round,line cap=round,fill=fillColor] (307.75,459.18) circle (  1.96);
\path[draw=drawColor,line width= 0.4pt,line join=round,line cap=round,fill=fillColor] (307.96,459.25) circle (  1.96);
\path[draw=drawColor,line width= 0.4pt,line join=round,line cap=round,fill=fillColor] (308.17,459.31) circle (  1.96);
\path[draw=drawColor,line width= 0.4pt,line join=round,line cap=round,fill=fillColor] (308.38,459.38) circle (  1.96);
\path[draw=drawColor,line width= 0.4pt,line join=round,line cap=round,fill=fillColor] (308.59,459.45) circle (  1.96);
\path[draw=drawColor,line width= 0.4pt,line join=round,line cap=round,fill=fillColor] (308.80,459.52) circle (  1.96);
\path[draw=drawColor,line width= 0.4pt,line join=round,line cap=round,fill=fillColor] (309.01,459.59) circle (  1.96);
\path[draw=drawColor,line width= 0.4pt,line join=round,line cap=round,fill=fillColor] (309.21,459.66) circle (  1.96);
\path[draw=drawColor,line width= 0.4pt,line join=round,line cap=round,fill=fillColor] (309.42,459.72) circle (  1.96);
\path[draw=drawColor,line width= 0.4pt,line join=round,line cap=round,fill=fillColor] (309.63,459.79) circle (  1.96);
\path[draw=drawColor,line width= 0.4pt,line join=round,line cap=round,fill=fillColor] (309.84,459.86) circle (  1.96);
\path[draw=drawColor,line width= 0.4pt,line join=round,line cap=round,fill=fillColor] (310.05,459.92) circle (  1.96);
\path[draw=drawColor,line width= 0.4pt,line join=round,line cap=round,fill=fillColor] (310.26,459.99) circle (  1.96);
\path[draw=drawColor,line width= 0.4pt,line join=round,line cap=round,fill=fillColor] (310.47,460.06) circle (  1.96);
\path[draw=drawColor,line width= 0.4pt,line join=round,line cap=round,fill=fillColor] (310.68,460.12) circle (  1.96);
\path[draw=drawColor,line width= 0.4pt,line join=round,line cap=round,fill=fillColor] (310.89,460.19) circle (  1.96);
\path[draw=drawColor,line width= 0.4pt,line join=round,line cap=round,fill=fillColor] (311.10,460.25) circle (  1.96);
\path[draw=drawColor,line width= 0.4pt,line join=round,line cap=round,fill=fillColor] (311.31,460.32) circle (  1.96);
\path[draw=drawColor,line width= 0.4pt,line join=round,line cap=round,fill=fillColor] (311.52,460.38) circle (  1.96);
\path[draw=drawColor,line width= 0.4pt,line join=round,line cap=round,fill=fillColor] (311.73,460.45) circle (  1.96);
\path[draw=drawColor,line width= 0.4pt,line join=round,line cap=round,fill=fillColor] (311.94,460.52) circle (  1.96);
\path[draw=drawColor,line width= 0.4pt,line join=round,line cap=round,fill=fillColor] (312.15,460.58) circle (  1.96);
\path[draw=drawColor,line width= 0.4pt,line join=round,line cap=round,fill=fillColor] (312.36,460.64) circle (  1.96);
\path[draw=drawColor,line width= 0.4pt,line join=round,line cap=round,fill=fillColor] (312.57,460.71) circle (  1.96);
\path[draw=drawColor,line width= 0.4pt,line join=round,line cap=round,fill=fillColor] (312.78,460.77) circle (  1.96);
\path[draw=drawColor,line width= 0.4pt,line join=round,line cap=round,fill=fillColor] (312.99,460.84) circle (  1.96);
\path[draw=drawColor,line width= 0.4pt,line join=round,line cap=round,fill=fillColor] (313.20,460.90) circle (  1.96);
\path[draw=drawColor,line width= 0.4pt,line join=round,line cap=round,fill=fillColor] (313.41,460.96) circle (  1.96);
\path[draw=drawColor,line width= 0.4pt,line join=round,line cap=round,fill=fillColor] (313.62,461.03) circle (  1.96);
\path[draw=drawColor,line width= 0.4pt,line join=round,line cap=round,fill=fillColor] (313.83,461.09) circle (  1.96);
\path[draw=drawColor,line width= 0.4pt,line join=round,line cap=round,fill=fillColor] (314.04,461.15) circle (  1.96);
\path[draw=drawColor,line width= 0.4pt,line join=round,line cap=round,fill=fillColor] (314.25,461.22) circle (  1.96);
\path[draw=drawColor,line width= 0.4pt,line join=round,line cap=round,fill=fillColor] (314.46,461.28) circle (  1.96);
\path[draw=drawColor,line width= 0.4pt,line join=round,line cap=round,fill=fillColor] (314.67,461.34) circle (  1.96);
\path[draw=drawColor,line width= 0.4pt,line join=round,line cap=round,fill=fillColor] (314.87,461.40) circle (  1.96);
\path[draw=drawColor,line width= 0.4pt,line join=round,line cap=round,fill=fillColor] (315.08,461.46) circle (  1.96);
\path[draw=drawColor,line width= 0.4pt,line join=round,line cap=round,fill=fillColor] (315.29,461.53) circle (  1.96);
\path[draw=drawColor,line width= 0.4pt,line join=round,line cap=round,fill=fillColor] (315.50,461.59) circle (  1.96);
\path[draw=drawColor,line width= 0.4pt,line join=round,line cap=round,fill=fillColor] (315.71,461.65) circle (  1.96);
\path[draw=drawColor,line width= 0.4pt,line join=round,line cap=round,fill=fillColor] (315.92,461.71) circle (  1.96);
\path[draw=drawColor,line width= 0.4pt,line join=round,line cap=round,fill=fillColor] (316.13,461.77) circle (  1.96);
\path[draw=drawColor,line width= 0.4pt,line join=round,line cap=round,fill=fillColor] (316.34,461.83) circle (  1.96);
\path[draw=drawColor,line width= 0.4pt,line join=round,line cap=round,fill=fillColor] (316.55,461.89) circle (  1.96);
\path[draw=drawColor,line width= 0.4pt,line join=round,line cap=round,fill=fillColor] (316.76,461.95) circle (  1.96);
\path[draw=drawColor,line width= 0.4pt,line join=round,line cap=round,fill=fillColor] (316.97,462.01) circle (  1.96);
\path[draw=drawColor,line width= 0.4pt,line join=round,line cap=round,fill=fillColor] (317.18,462.07) circle (  1.96);
\path[draw=drawColor,line width= 0.4pt,line join=round,line cap=round,fill=fillColor] (317.39,462.13) circle (  1.96);
\path[draw=drawColor,line width= 0.4pt,line join=round,line cap=round,fill=fillColor] (317.60,462.19) circle (  1.96);
\path[draw=drawColor,line width= 0.4pt,line join=round,line cap=round,fill=fillColor] (317.81,462.25) circle (  1.96);
\path[draw=drawColor,line width= 0.4pt,line join=round,line cap=round,fill=fillColor] (318.02,462.31) circle (  1.96);
\path[draw=drawColor,line width= 0.4pt,line join=round,line cap=round,fill=fillColor] (318.23,462.37) circle (  1.96);
\path[draw=drawColor,line width= 0.4pt,line join=round,line cap=round,fill=fillColor] (318.44,462.43) circle (  1.96);
\path[draw=drawColor,line width= 0.4pt,line join=round,line cap=round,fill=fillColor] (318.65,462.49) circle (  1.96);
\path[draw=drawColor,line width= 0.4pt,line join=round,line cap=round,fill=fillColor] (318.86,462.55) circle (  1.96);
\path[draw=drawColor,line width= 0.4pt,line join=round,line cap=round,fill=fillColor] (319.07,462.60) circle (  1.96);
\path[draw=drawColor,line width= 0.4pt,line join=round,line cap=round,fill=fillColor] (319.28,462.66) circle (  1.96);
\path[draw=drawColor,line width= 0.4pt,line join=round,line cap=round,fill=fillColor] (319.49,462.72) circle (  1.96);
\path[draw=drawColor,line width= 0.4pt,line join=round,line cap=round,fill=fillColor] (319.70,462.78) circle (  1.96);
\path[draw=drawColor,line width= 0.4pt,line join=round,line cap=round,fill=fillColor] (319.91,462.84) circle (  1.96);
\path[draw=drawColor,line width= 0.4pt,line join=round,line cap=round,fill=fillColor] (320.12,462.89) circle (  1.96);
\path[draw=drawColor,line width= 0.4pt,line join=round,line cap=round,fill=fillColor] (320.32,462.95) circle (  1.96);
\path[draw=drawColor,line width= 0.4pt,line join=round,line cap=round,fill=fillColor] (320.53,463.01) circle (  1.96);
\path[draw=drawColor,line width= 0.4pt,line join=round,line cap=round,fill=fillColor] (320.74,463.06) circle (  1.96);
\path[draw=drawColor,line width= 0.4pt,line join=round,line cap=round,fill=fillColor] (320.95,463.12) circle (  1.96);
\path[draw=drawColor,line width= 0.4pt,line join=round,line cap=round,fill=fillColor] (321.16,463.18) circle (  1.96);
\path[draw=drawColor,line width= 0.4pt,line join=round,line cap=round,fill=fillColor] (321.37,463.23) circle (  1.96);
\path[draw=drawColor,line width= 0.4pt,line join=round,line cap=round,fill=fillColor] (321.58,463.29) circle (  1.96);
\path[draw=drawColor,line width= 0.4pt,line join=round,line cap=round,fill=fillColor] (321.79,463.35) circle (  1.96);
\path[draw=drawColor,line width= 0.4pt,line join=round,line cap=round,fill=fillColor] (322.00,463.40) circle (  1.96);
\path[draw=drawColor,line width= 0.4pt,line join=round,line cap=round,fill=fillColor] (322.21,463.46) circle (  1.96);
\path[draw=drawColor,line width= 0.4pt,line join=round,line cap=round,fill=fillColor] (322.42,463.51) circle (  1.96);
\path[draw=drawColor,line width= 0.4pt,line join=round,line cap=round,fill=fillColor] (322.63,463.57) circle (  1.96);
\path[draw=drawColor,line width= 0.4pt,line join=round,line cap=round,fill=fillColor] (322.84,463.62) circle (  1.96);
\path[draw=drawColor,line width= 0.4pt,line join=round,line cap=round,fill=fillColor] (323.05,463.68) circle (  1.96);
\path[draw=drawColor,line width= 0.4pt,line join=round,line cap=round,fill=fillColor] (323.26,463.73) circle (  1.96);
\path[draw=drawColor,line width= 0.4pt,line join=round,line cap=round,fill=fillColor] (323.47,463.79) circle (  1.96);
\path[draw=drawColor,line width= 0.4pt,line join=round,line cap=round,fill=fillColor] (323.68,463.84) circle (  1.96);
\path[draw=drawColor,line width= 0.4pt,line join=round,line cap=round,fill=fillColor] (323.89,463.89) circle (  1.96);
\path[draw=drawColor,line width= 0.4pt,line join=round,line cap=round,fill=fillColor] (324.10,463.95) circle (  1.96);
\path[draw=drawColor,line width= 0.4pt,line join=round,line cap=round,fill=fillColor] (324.31,464.00) circle (  1.96);
\path[draw=drawColor,line width= 0.4pt,line join=round,line cap=round,fill=fillColor] (324.52,464.06) circle (  1.96);
\path[draw=drawColor,line width= 0.4pt,line join=round,line cap=round,fill=fillColor] (324.73,464.11) circle (  1.96);
\path[draw=drawColor,line width= 0.4pt,line join=round,line cap=round,fill=fillColor] (324.94,464.16) circle (  1.96);
\path[draw=drawColor,line width= 0.4pt,line join=round,line cap=round,fill=fillColor] (325.15,464.22) circle (  1.96);
\path[draw=drawColor,line width= 0.4pt,line join=round,line cap=round,fill=fillColor] (325.36,464.27) circle (  1.96);
\path[draw=drawColor,line width= 0.4pt,line join=round,line cap=round,fill=fillColor] (325.57,464.32) circle (  1.96);
\path[draw=drawColor,line width= 0.4pt,line join=round,line cap=round,fill=fillColor] (325.78,464.37) circle (  1.96);
\path[draw=drawColor,line width= 0.4pt,line join=round,line cap=round,fill=fillColor] (325.98,464.43) circle (  1.96);
\path[draw=drawColor,line width= 0.4pt,line join=round,line cap=round,fill=fillColor] (326.19,464.48) circle (  1.96);
\path[draw=drawColor,line width= 0.4pt,line join=round,line cap=round,fill=fillColor] (326.40,464.53) circle (  1.96);
\path[draw=drawColor,line width= 0.4pt,line join=round,line cap=round,fill=fillColor] (326.61,464.58) circle (  1.96);
\path[draw=drawColor,line width= 0.4pt,line join=round,line cap=round,fill=fillColor] (326.82,464.63) circle (  1.96);
\path[draw=drawColor,line width= 0.4pt,line join=round,line cap=round,fill=fillColor] (327.03,464.69) circle (  1.96);
\path[draw=drawColor,line width= 0.4pt,line join=round,line cap=round,fill=fillColor] (327.24,464.74) circle (  1.96);
\path[draw=drawColor,line width= 0.4pt,line join=round,line cap=round,fill=fillColor] (327.45,464.79) circle (  1.96);
\path[draw=drawColor,line width= 0.4pt,line join=round,line cap=round,fill=fillColor] (327.66,464.84) circle (  1.96);
\path[draw=drawColor,line width= 0.4pt,line join=round,line cap=round,fill=fillColor] (327.87,464.89) circle (  1.96);
\path[draw=drawColor,line width= 0.4pt,line join=round,line cap=round,fill=fillColor] (328.08,464.94) circle (  1.96);
\path[draw=drawColor,line width= 0.4pt,line join=round,line cap=round,fill=fillColor] (328.29,464.99) circle (  1.96);
\path[draw=drawColor,line width= 0.4pt,line join=round,line cap=round,fill=fillColor] (328.50,465.04) circle (  1.96);
\path[draw=drawColor,line width= 0.4pt,line join=round,line cap=round,fill=fillColor] (328.71,465.09) circle (  1.96);
\path[draw=drawColor,line width= 0.4pt,line join=round,line cap=round,fill=fillColor] (328.92,465.14) circle (  1.96);
\path[draw=drawColor,line width= 0.4pt,line join=round,line cap=round,fill=fillColor] (329.13,465.19) circle (  1.96);
\path[draw=drawColor,line width= 0.4pt,line join=round,line cap=round,fill=fillColor] (329.34,465.24) circle (  1.96);
\path[draw=drawColor,line width= 0.4pt,line join=round,line cap=round,fill=fillColor] (329.55,465.29) circle (  1.96);
\path[draw=drawColor,line width= 0.4pt,line join=round,line cap=round,fill=fillColor] (329.76,465.34) circle (  1.96);
\path[draw=drawColor,line width= 0.4pt,line join=round,line cap=round,fill=fillColor] (329.97,465.39) circle (  1.96);
\path[draw=drawColor,line width= 0.4pt,line join=round,line cap=round,fill=fillColor] (330.18,465.44) circle (  1.96);
\path[draw=drawColor,line width= 0.4pt,line join=round,line cap=round,fill=fillColor] (330.39,465.49) circle (  1.96);
\path[draw=drawColor,line width= 0.4pt,line join=round,line cap=round,fill=fillColor] (330.60,465.54) circle (  1.96);
\path[draw=drawColor,line width= 0.4pt,line join=round,line cap=round,fill=fillColor] (330.81,465.59) circle (  1.96);
\path[draw=drawColor,line width= 0.4pt,line join=round,line cap=round,fill=fillColor] (331.02,465.64) circle (  1.96);
\path[draw=drawColor,line width= 0.4pt,line join=round,line cap=round,fill=fillColor] (331.23,465.68) circle (  1.96);
\path[draw=drawColor,line width= 0.4pt,line join=round,line cap=round,fill=fillColor] (331.43,465.73) circle (  1.96);
\path[draw=drawColor,line width= 0.4pt,line join=round,line cap=round,fill=fillColor] (331.64,465.78) circle (  1.96);
\path[draw=drawColor,line width= 0.4pt,line join=round,line cap=round,fill=fillColor] (331.85,465.83) circle (  1.96);
\path[draw=drawColor,line width= 0.4pt,line join=round,line cap=round,fill=fillColor] (332.06,465.87) circle (  1.96);
\path[draw=drawColor,line width= 0.4pt,line join=round,line cap=round,fill=fillColor] (332.27,465.92) circle (  1.96);
\path[draw=drawColor,line width= 0.4pt,line join=round,line cap=round,fill=fillColor] (332.48,465.97) circle (  1.96);
\path[draw=drawColor,line width= 0.4pt,line join=round,line cap=round,fill=fillColor] (332.69,466.02) circle (  1.96);
\path[draw=drawColor,line width= 0.4pt,line join=round,line cap=round,fill=fillColor] (332.90,466.06) circle (  1.96);
\path[draw=drawColor,line width= 0.4pt,line join=round,line cap=round,fill=fillColor] (333.11,466.11) circle (  1.96);
\path[draw=drawColor,line width= 0.4pt,line join=round,line cap=round,fill=fillColor] (333.32,466.16) circle (  1.96);
\path[draw=drawColor,line width= 0.4pt,line join=round,line cap=round,fill=fillColor] (333.53,466.20) circle (  1.96);
\path[draw=drawColor,line width= 0.4pt,line join=round,line cap=round,fill=fillColor] (333.74,466.25) circle (  1.96);
\path[draw=drawColor,line width= 0.4pt,line join=round,line cap=round,fill=fillColor] (333.95,466.30) circle (  1.96);
\path[draw=drawColor,line width= 0.4pt,line join=round,line cap=round,fill=fillColor] (334.16,466.34) circle (  1.96);
\path[draw=drawColor,line width= 0.4pt,line join=round,line cap=round,fill=fillColor] (334.37,466.39) circle (  1.96);
\path[draw=drawColor,line width= 0.4pt,line join=round,line cap=round,fill=fillColor] (334.58,466.43) circle (  1.96);
\path[draw=drawColor,line width= 0.4pt,line join=round,line cap=round,fill=fillColor] (334.79,466.48) circle (  1.96);
\path[draw=drawColor,line width= 0.4pt,line join=round,line cap=round,fill=fillColor] (335.00,466.53) circle (  1.96);
\path[draw=drawColor,line width= 0.4pt,line join=round,line cap=round,fill=fillColor] (335.21,466.57) circle (  1.96);
\path[draw=drawColor,line width= 0.4pt,line join=round,line cap=round,fill=fillColor] (335.42,466.62) circle (  1.96);
\path[draw=drawColor,line width= 0.4pt,line join=round,line cap=round,fill=fillColor] (335.63,466.66) circle (  1.96);
\path[draw=drawColor,line width= 0.4pt,line join=round,line cap=round,fill=fillColor] (335.84,466.71) circle (  1.96);
\path[draw=drawColor,line width= 0.4pt,line join=round,line cap=round,fill=fillColor] (336.05,466.75) circle (  1.96);
\path[draw=drawColor,line width= 0.4pt,line join=round,line cap=round,fill=fillColor] (336.26,466.80) circle (  1.96);
\path[draw=drawColor,line width= 0.4pt,line join=round,line cap=round,fill=fillColor] (336.47,466.84) circle (  1.96);
\path[draw=drawColor,line width= 0.4pt,line join=round,line cap=round,fill=fillColor] (336.68,466.88) circle (  1.96);
\path[draw=drawColor,line width= 0.4pt,line join=round,line cap=round,fill=fillColor] (336.89,466.93) circle (  1.96);
\path[draw=drawColor,line width= 0.4pt,line join=round,line cap=round,fill=fillColor] (337.09,466.97) circle (  1.96);
\path[draw=drawColor,line width= 0.4pt,line join=round,line cap=round,fill=fillColor] (337.30,467.02) circle (  1.96);
\path[draw=drawColor,line width= 0.4pt,line join=round,line cap=round,fill=fillColor] (337.51,467.06) circle (  1.96);
\path[draw=drawColor,line width= 0.4pt,line join=round,line cap=round,fill=fillColor] (337.72,467.10) circle (  1.96);
\path[draw=drawColor,line width= 0.4pt,line join=round,line cap=round,fill=fillColor] (337.93,467.15) circle (  1.96);
\path[draw=drawColor,line width= 0.4pt,line join=round,line cap=round,fill=fillColor] (338.14,467.19) circle (  1.96);
\path[draw=drawColor,line width= 0.4pt,line join=round,line cap=round,fill=fillColor] (338.35,467.23) circle (  1.96);
\path[draw=drawColor,line width= 0.4pt,line join=round,line cap=round,fill=fillColor] (338.56,467.28) circle (  1.96);
\path[draw=drawColor,line width= 0.4pt,line join=round,line cap=round,fill=fillColor] (338.77,467.32) circle (  1.96);
\path[draw=drawColor,line width= 0.4pt,line join=round,line cap=round,fill=fillColor] (338.98,467.36) circle (  1.96);
\path[draw=drawColor,line width= 0.4pt,line join=round,line cap=round,fill=fillColor] (339.19,467.41) circle (  1.96);
\path[draw=drawColor,line width= 0.4pt,line join=round,line cap=round,fill=fillColor] (339.40,467.45) circle (  1.96);
\path[draw=drawColor,line width= 0.4pt,line join=round,line cap=round,fill=fillColor] (339.61,467.49) circle (  1.96);
\path[draw=drawColor,line width= 0.4pt,line join=round,line cap=round,fill=fillColor] (339.82,467.53) circle (  1.96);
\path[draw=drawColor,line width= 0.4pt,line join=round,line cap=round,fill=fillColor] (340.03,467.57) circle (  1.96);
\path[draw=drawColor,line width= 0.4pt,line join=round,line cap=round,fill=fillColor] (340.24,467.62) circle (  1.96);
\path[draw=drawColor,line width= 0.4pt,line join=round,line cap=round,fill=fillColor] (340.45,467.66) circle (  1.96);
\path[draw=drawColor,line width= 0.4pt,line join=round,line cap=round,fill=fillColor] (340.66,467.70) circle (  1.96);
\path[draw=drawColor,line width= 0.4pt,line join=round,line cap=round,fill=fillColor] (340.87,467.74) circle (  1.96);
\path[draw=drawColor,line width= 0.4pt,line join=round,line cap=round,fill=fillColor] (341.08,467.78) circle (  1.96);
\path[draw=drawColor,line width= 0.4pt,line join=round,line cap=round,fill=fillColor] (341.29,467.82) circle (  1.96);
\path[draw=drawColor,line width= 0.4pt,line join=round,line cap=round,fill=fillColor] (341.50,467.87) circle (  1.96);
\path[draw=drawColor,line width= 0.4pt,line join=round,line cap=round,fill=fillColor] (341.71,467.91) circle (  1.96);
\path[draw=drawColor,line width= 0.4pt,line join=round,line cap=round,fill=fillColor] (341.92,467.95) circle (  1.96);
\path[draw=drawColor,line width= 0.4pt,line join=round,line cap=round,fill=fillColor] (342.13,467.99) circle (  1.96);
\path[draw=drawColor,line width= 0.4pt,line join=round,line cap=round,fill=fillColor] (342.34,468.03) circle (  1.96);
\path[draw=drawColor,line width= 0.4pt,line join=round,line cap=round,fill=fillColor] (342.54,468.07) circle (  1.96);
\path[draw=drawColor,line width= 0.4pt,line join=round,line cap=round,fill=fillColor] (342.75,468.11) circle (  1.96);
\path[draw=drawColor,line width= 0.4pt,line join=round,line cap=round,fill=fillColor] (342.96,468.15) circle (  1.96);
\path[draw=drawColor,line width= 0.4pt,line join=round,line cap=round,fill=fillColor] (343.17,468.19) circle (  1.96);
\path[draw=drawColor,line width= 0.4pt,line join=round,line cap=round,fill=fillColor] (343.38,468.23) circle (  1.96);
\path[draw=drawColor,line width= 0.4pt,line join=round,line cap=round,fill=fillColor] (343.59,468.27) circle (  1.96);
\path[draw=drawColor,line width= 0.4pt,line join=round,line cap=round,fill=fillColor] (343.80,468.31) circle (  1.96);
\path[draw=drawColor,line width= 0.4pt,line join=round,line cap=round,fill=fillColor] (344.01,468.35) circle (  1.96);
\path[draw=drawColor,line width= 0.4pt,line join=round,line cap=round,fill=fillColor] (344.22,468.39) circle (  1.96);
\path[draw=drawColor,line width= 0.4pt,line join=round,line cap=round,fill=fillColor] (344.43,468.43) circle (  1.96);
\path[draw=drawColor,line width= 0.4pt,line join=round,line cap=round,fill=fillColor] (344.64,468.47) circle (  1.96);
\path[draw=drawColor,line width= 0.4pt,line join=round,line cap=round,fill=fillColor] (344.85,468.50) circle (  1.96);
\path[draw=drawColor,line width= 0.4pt,line join=round,line cap=round,fill=fillColor] (345.06,468.54) circle (  1.96);
\path[draw=drawColor,line width= 0.4pt,line join=round,line cap=round,fill=fillColor] (345.27,468.58) circle (  1.96);
\path[draw=drawColor,line width= 0.4pt,line join=round,line cap=round,fill=fillColor] (345.48,468.62) circle (  1.96);
\path[draw=drawColor,line width= 0.4pt,line join=round,line cap=round,fill=fillColor] (345.69,468.66) circle (  1.96);
\path[draw=drawColor,line width= 0.4pt,line join=round,line cap=round,fill=fillColor] (345.90,468.70) circle (  1.96);
\path[draw=drawColor,line width= 0.4pt,line join=round,line cap=round,fill=fillColor] (346.11,468.74) circle (  1.96);
\path[draw=drawColor,line width= 0.4pt,line join=round,line cap=round,fill=fillColor] (346.32,468.77) circle (  1.96);
\path[draw=drawColor,line width= 0.4pt,line join=round,line cap=round,fill=fillColor] (346.53,468.81) circle (  1.96);
\path[draw=drawColor,line width= 0.4pt,line join=round,line cap=round,fill=fillColor] (346.74,468.85) circle (  1.96);
\path[draw=drawColor,line width= 0.4pt,line join=round,line cap=round,fill=fillColor] (346.95,468.89) circle (  1.96);
\path[draw=drawColor,line width= 0.4pt,line join=round,line cap=round,fill=fillColor] (347.16,468.92) circle (  1.96);
\path[draw=drawColor,line width= 0.4pt,line join=round,line cap=round,fill=fillColor] (347.37,468.96) circle (  1.96);
\path[draw=drawColor,line width= 0.4pt,line join=round,line cap=round,fill=fillColor] (347.58,469.00) circle (  1.96);
\path[draw=drawColor,line width= 0.4pt,line join=round,line cap=round,fill=fillColor] (347.79,469.04) circle (  1.96);
\path[draw=drawColor,line width= 0.4pt,line join=round,line cap=round,fill=fillColor] (348.00,469.07) circle (  1.96);
\path[draw=drawColor,line width= 0.4pt,line join=round,line cap=round,fill=fillColor] (348.20,469.11) circle (  1.96);
\path[draw=drawColor,line width= 0.4pt,line join=round,line cap=round,fill=fillColor] (348.41,469.15) circle (  1.96);
\path[draw=drawColor,line width= 0.4pt,line join=round,line cap=round,fill=fillColor] (348.62,469.18) circle (  1.96);
\path[draw=drawColor,line width= 0.4pt,line join=round,line cap=round,fill=fillColor] (348.83,469.22) circle (  1.96);
\path[draw=drawColor,line width= 0.4pt,line join=round,line cap=round,fill=fillColor] (349.04,469.26) circle (  1.96);
\path[draw=drawColor,line width= 0.4pt,line join=round,line cap=round,fill=fillColor] (349.25,469.29) circle (  1.96);
\path[draw=drawColor,line width= 0.4pt,line join=round,line cap=round,fill=fillColor] (349.46,469.33) circle (  1.96);
\path[draw=drawColor,line width= 0.4pt,line join=round,line cap=round,fill=fillColor] (349.67,469.36) circle (  1.96);
\path[draw=drawColor,line width= 0.4pt,line join=round,line cap=round,fill=fillColor] (349.88,469.40) circle (  1.96);
\path[draw=drawColor,line width= 0.4pt,line join=round,line cap=round,fill=fillColor] (350.09,469.44) circle (  1.96);
\path[draw=drawColor,line width= 0.4pt,line join=round,line cap=round,fill=fillColor] (350.30,469.47) circle (  1.96);
\path[draw=drawColor,line width= 0.4pt,line join=round,line cap=round,fill=fillColor] (350.51,469.51) circle (  1.96);
\path[draw=drawColor,line width= 0.4pt,line join=round,line cap=round,fill=fillColor] (350.72,469.54) circle (  1.96);
\path[draw=drawColor,line width= 0.4pt,line join=round,line cap=round,fill=fillColor] (350.93,469.58) circle (  1.96);
\path[draw=drawColor,line width= 0.4pt,line join=round,line cap=round,fill=fillColor] (351.14,469.61) circle (  1.96);
\path[draw=drawColor,line width= 0.4pt,line join=round,line cap=round,fill=fillColor] (351.35,469.65) circle (  1.96);
\path[draw=drawColor,line width= 0.4pt,line join=round,line cap=round,fill=fillColor] (351.56,469.68) circle (  1.96);
\path[draw=drawColor,line width= 0.4pt,line join=round,line cap=round,fill=fillColor] (351.77,469.72) circle (  1.96);
\path[draw=drawColor,line width= 0.4pt,line join=round,line cap=round,fill=fillColor] (351.98,469.75) circle (  1.96);
\path[draw=drawColor,line width= 0.4pt,line join=round,line cap=round,fill=fillColor] (352.19,469.79) circle (  1.96);
\path[draw=drawColor,line width= 0.4pt,line join=round,line cap=round,fill=fillColor] (352.40,469.82) circle (  1.96);
\path[draw=drawColor,line width= 0.4pt,line join=round,line cap=round,fill=fillColor] (352.61,469.86) circle (  1.96);
\path[draw=drawColor,line width= 0.4pt,line join=round,line cap=round,fill=fillColor] (352.82,469.89) circle (  1.96);
\path[draw=drawColor,line width= 0.4pt,line join=round,line cap=round,fill=fillColor] (353.03,469.92) circle (  1.96);
\path[draw=drawColor,line width= 0.4pt,line join=round,line cap=round,fill=fillColor] (353.24,469.96) circle (  1.96);
\path[draw=drawColor,line width= 0.4pt,line join=round,line cap=round,fill=fillColor] (353.45,469.99) circle (  1.96);
\path[draw=drawColor,line width= 0.4pt,line join=round,line cap=round,fill=fillColor] (353.65,470.03) circle (  1.96);
\path[draw=drawColor,line width= 0.4pt,line join=round,line cap=round,fill=fillColor] (353.86,470.06) circle (  1.96);
\path[draw=drawColor,line width= 0.4pt,line join=round,line cap=round,fill=fillColor] (354.07,470.09) circle (  1.96);
\path[draw=drawColor,line width= 0.4pt,line join=round,line cap=round,fill=fillColor] (354.28,470.13) circle (  1.96);
\path[draw=drawColor,line width= 0.4pt,line join=round,line cap=round,fill=fillColor] (354.49,470.16) circle (  1.96);
\path[draw=drawColor,line width= 0.4pt,line join=round,line cap=round,fill=fillColor] (354.70,470.19) circle (  1.96);
\path[draw=drawColor,line width= 0.4pt,line join=round,line cap=round,fill=fillColor] (354.91,470.23) circle (  1.96);
\path[draw=drawColor,line width= 0.4pt,line join=round,line cap=round,fill=fillColor] (355.12,470.26) circle (  1.96);
\path[draw=drawColor,line width= 0.4pt,line join=round,line cap=round,fill=fillColor] (355.33,470.29) circle (  1.96);
\path[draw=drawColor,line width= 0.4pt,line join=round,line cap=round,fill=fillColor] (355.54,470.32) circle (  1.96);
\path[draw=drawColor,line width= 0.4pt,line join=round,line cap=round,fill=fillColor] (355.75,470.36) circle (  1.96);
\path[draw=drawColor,line width= 0.4pt,line join=round,line cap=round,fill=fillColor] (355.96,470.39) circle (  1.96);
\path[draw=drawColor,line width= 0.4pt,line join=round,line cap=round,fill=fillColor] (356.17,470.42) circle (  1.96);
\path[draw=drawColor,line width= 0.4pt,line join=round,line cap=round,fill=fillColor] (356.38,470.45) circle (  1.96);
\path[draw=drawColor,line width= 0.4pt,line join=round,line cap=round,fill=fillColor] (356.59,470.49) circle (  1.96);
\path[draw=drawColor,line width= 0.4pt,line join=round,line cap=round,fill=fillColor] (356.80,470.52) circle (  1.96);
\path[draw=drawColor,line width= 0.4pt,line join=round,line cap=round,fill=fillColor] (357.01,470.55) circle (  1.96);
\path[draw=drawColor,line width= 0.4pt,line join=round,line cap=round,fill=fillColor] (357.22,470.58) circle (  1.96);
\path[draw=drawColor,line width= 0.4pt,line join=round,line cap=round,fill=fillColor] (357.43,470.61) circle (  1.96);
\path[draw=drawColor,line width= 0.4pt,line join=round,line cap=round,fill=fillColor] (357.64,470.65) circle (  1.96);
\path[draw=drawColor,line width= 0.4pt,line join=round,line cap=round,fill=fillColor] (357.85,470.68) circle (  1.96);
\path[draw=drawColor,line width= 0.4pt,line join=round,line cap=round,fill=fillColor] (358.06,470.71) circle (  1.96);
\path[draw=drawColor,line width= 0.4pt,line join=round,line cap=round,fill=fillColor] (358.27,470.74) circle (  1.96);
\path[draw=drawColor,line width= 0.4pt,line join=round,line cap=round,fill=fillColor] (358.48,470.77) circle (  1.96);
\path[draw=drawColor,line width= 0.4pt,line join=round,line cap=round,fill=fillColor] (358.69,470.80) circle (  1.96);
\path[draw=drawColor,line width= 0.4pt,line join=round,line cap=round,fill=fillColor] (358.90,470.83) circle (  1.96);
\path[draw=drawColor,line width= 0.4pt,line join=round,line cap=round,fill=fillColor] (359.10,470.86) circle (  1.96);
\path[draw=drawColor,line width= 0.4pt,line join=round,line cap=round,fill=fillColor] (359.31,470.89) circle (  1.96);
\path[draw=drawColor,line width= 0.4pt,line join=round,line cap=round,fill=fillColor] (359.52,470.93) circle (  1.96);
\path[draw=drawColor,line width= 0.4pt,line join=round,line cap=round,fill=fillColor] (359.73,470.96) circle (  1.96);
\path[draw=drawColor,line width= 0.4pt,line join=round,line cap=round,fill=fillColor] (359.94,470.99) circle (  1.96);
\path[draw=drawColor,line width= 0.4pt,line join=round,line cap=round,fill=fillColor] (360.15,471.02) circle (  1.96);
\path[draw=drawColor,line width= 0.4pt,line join=round,line cap=round,fill=fillColor] (360.36,471.05) circle (  1.96);
\path[draw=drawColor,line width= 0.4pt,line join=round,line cap=round,fill=fillColor] (360.57,471.08) circle (  1.96);
\path[draw=drawColor,line width= 0.4pt,line join=round,line cap=round,fill=fillColor] (360.78,471.11) circle (  1.96);
\path[draw=drawColor,line width= 0.4pt,line join=round,line cap=round,fill=fillColor] (360.99,471.14) circle (  1.96);
\path[draw=drawColor,line width= 0.4pt,line join=round,line cap=round,fill=fillColor] (361.20,471.17) circle (  1.96);
\path[draw=drawColor,line width= 0.4pt,line join=round,line cap=round,fill=fillColor] (361.41,471.20) circle (  1.96);
\path[draw=drawColor,line width= 0.4pt,line join=round,line cap=round,fill=fillColor] (361.62,471.23) circle (  1.96);
\path[draw=drawColor,line width= 0.4pt,line join=round,line cap=round,fill=fillColor] (361.83,471.26) circle (  1.96);
\path[draw=drawColor,line width= 0.4pt,line join=round,line cap=round,fill=fillColor] (362.04,471.29) circle (  1.96);
\path[draw=drawColor,line width= 0.4pt,line join=round,line cap=round,fill=fillColor] (362.25,471.31) circle (  1.96);
\path[draw=drawColor,line width= 0.4pt,line join=round,line cap=round,fill=fillColor] (362.46,471.34) circle (  1.96);
\path[draw=drawColor,line width= 0.4pt,line join=round,line cap=round,fill=fillColor] (362.67,471.37) circle (  1.96);
\path[draw=drawColor,line width= 0.4pt,line join=round,line cap=round,fill=fillColor] (362.88,471.40) circle (  1.96);
\path[draw=drawColor,line width= 0.4pt,line join=round,line cap=round,fill=fillColor] (363.09,471.43) circle (  1.96);
\path[draw=drawColor,line width= 0.4pt,line join=round,line cap=round,fill=fillColor] (363.30,471.46) circle (  1.96);
\path[draw=drawColor,line width= 0.4pt,line join=round,line cap=round,fill=fillColor] (363.51,471.49) circle (  1.96);
\path[draw=drawColor,line width= 0.4pt,line join=round,line cap=round,fill=fillColor] (363.72,471.52) circle (  1.96);
\path[draw=drawColor,line width= 0.4pt,line join=round,line cap=round,fill=fillColor] (363.93,471.55) circle (  1.96);
\path[draw=drawColor,line width= 0.4pt,line join=round,line cap=round,fill=fillColor] (364.14,471.57) circle (  1.96);
\path[draw=drawColor,line width= 0.4pt,line join=round,line cap=round,fill=fillColor] (364.35,471.60) circle (  1.96);
\path[draw=drawColor,line width= 0.4pt,line join=round,line cap=round,fill=fillColor] (364.56,471.63) circle (  1.96);
\path[draw=drawColor,line width= 0.4pt,line join=round,line cap=round,fill=fillColor] (364.76,471.66) circle (  1.96);
\path[draw=drawColor,line width= 0.4pt,line join=round,line cap=round,fill=fillColor] (364.97,471.69) circle (  1.96);
\path[draw=drawColor,line width= 0.4pt,line join=round,line cap=round,fill=fillColor] (365.18,471.71) circle (  1.96);
\path[draw=drawColor,line width= 0.4pt,line join=round,line cap=round,fill=fillColor] (365.39,471.74) circle (  1.96);
\path[draw=drawColor,line width= 0.4pt,line join=round,line cap=round,fill=fillColor] (365.60,471.77) circle (  1.96);
\path[draw=drawColor,line width= 0.4pt,line join=round,line cap=round,fill=fillColor] (365.81,471.80) circle (  1.96);
\path[draw=drawColor,line width= 0.4pt,line join=round,line cap=round,fill=fillColor] (366.02,471.83) circle (  1.96);
\path[draw=drawColor,line width= 0.4pt,line join=round,line cap=round,fill=fillColor] (366.23,471.85) circle (  1.96);
\path[draw=drawColor,line width= 0.4pt,line join=round,line cap=round,fill=fillColor] (366.44,471.88) circle (  1.96);
\path[draw=drawColor,line width= 0.4pt,line join=round,line cap=round,fill=fillColor] (366.65,471.91) circle (  1.96);
\path[draw=drawColor,line width= 0.4pt,line join=round,line cap=round,fill=fillColor] (366.86,471.93) circle (  1.96);
\path[draw=drawColor,line width= 0.4pt,line join=round,line cap=round,fill=fillColor] (367.07,471.96) circle (  1.96);
\path[draw=drawColor,line width= 0.4pt,line join=round,line cap=round,fill=fillColor] (367.28,471.99) circle (  1.96);
\path[draw=drawColor,line width= 0.4pt,line join=round,line cap=round,fill=fillColor] (367.49,472.02) circle (  1.96);
\path[draw=drawColor,line width= 0.4pt,line join=round,line cap=round,fill=fillColor] (367.70,472.04) circle (  1.96);
\path[draw=drawColor,line width= 0.4pt,line join=round,line cap=round,fill=fillColor] (367.91,472.07) circle (  1.96);
\path[draw=drawColor,line width= 0.4pt,line join=round,line cap=round,fill=fillColor] (368.12,472.10) circle (  1.96);
\path[draw=drawColor,line width= 0.4pt,line join=round,line cap=round,fill=fillColor] (368.33,472.12) circle (  1.96);
\path[draw=drawColor,line width= 0.4pt,line join=round,line cap=round,fill=fillColor] (368.54,472.15) circle (  1.96);
\path[draw=drawColor,line width= 0.4pt,line join=round,line cap=round,fill=fillColor] (368.75,472.17) circle (  1.96);
\path[draw=drawColor,line width= 0.4pt,line join=round,line cap=round,fill=fillColor] (368.96,472.20) circle (  1.96);
\path[draw=drawColor,line width= 0.4pt,line join=round,line cap=round,fill=fillColor] (369.17,472.23) circle (  1.96);
\path[draw=drawColor,line width= 0.4pt,line join=round,line cap=round,fill=fillColor] (369.38,472.25) circle (  1.96);
\path[draw=drawColor,line width= 0.4pt,line join=round,line cap=round,fill=fillColor] (369.59,472.28) circle (  1.96);
\path[draw=drawColor,line width= 0.4pt,line join=round,line cap=round,fill=fillColor] (369.80,472.30) circle (  1.96);
\path[draw=drawColor,line width= 0.4pt,line join=round,line cap=round,fill=fillColor] (370.01,472.33) circle (  1.96);
\path[draw=drawColor,line width= 0.4pt,line join=round,line cap=round,fill=fillColor] (370.21,472.36) circle (  1.96);
\path[draw=drawColor,line width= 0.4pt,line join=round,line cap=round,fill=fillColor] (370.42,472.38) circle (  1.96);
\path[draw=drawColor,line width= 0.4pt,line join=round,line cap=round,fill=fillColor] (370.63,472.41) circle (  1.96);
\path[draw=drawColor,line width= 0.4pt,line join=round,line cap=round,fill=fillColor] (370.84,472.43) circle (  1.96);
\path[draw=drawColor,line width= 0.4pt,line join=round,line cap=round,fill=fillColor] (371.05,472.46) circle (  1.96);
\path[draw=drawColor,line width= 0.4pt,line join=round,line cap=round,fill=fillColor] (371.26,472.48) circle (  1.96);
\path[draw=drawColor,line width= 0.4pt,line join=round,line cap=round,fill=fillColor] (371.47,472.51) circle (  1.96);
\path[draw=drawColor,line width= 0.4pt,line join=round,line cap=round,fill=fillColor] (371.68,472.53) circle (  1.96);
\path[draw=drawColor,line width= 0.4pt,line join=round,line cap=round,fill=fillColor] (371.89,472.56) circle (  1.96);
\path[draw=drawColor,line width= 0.4pt,line join=round,line cap=round,fill=fillColor] (372.10,472.58) circle (  1.96);
\path[draw=drawColor,line width= 0.4pt,line join=round,line cap=round,fill=fillColor] (372.31,472.61) circle (  1.96);
\path[draw=drawColor,line width= 0.4pt,line join=round,line cap=round,fill=fillColor] (372.52,472.63) circle (  1.96);
\path[draw=drawColor,line width= 0.4pt,line join=round,line cap=round,fill=fillColor] (372.73,472.66) circle (  1.96);
\path[draw=drawColor,line width= 0.4pt,line join=round,line cap=round,fill=fillColor] (372.94,472.68) circle (  1.96);
\path[draw=drawColor,line width= 0.4pt,line join=round,line cap=round,fill=fillColor] (373.15,472.71) circle (  1.96);
\path[draw=drawColor,line width= 0.4pt,line join=round,line cap=round,fill=fillColor] (373.36,472.73) circle (  1.96);
\path[draw=drawColor,line width= 0.4pt,line join=round,line cap=round,fill=fillColor] (373.57,472.75) circle (  1.96);
\path[draw=drawColor,line width= 0.4pt,line join=round,line cap=round,fill=fillColor] (373.78,472.78) circle (  1.96);
\path[draw=drawColor,line width= 0.4pt,line join=round,line cap=round,fill=fillColor] (373.99,472.80) circle (  1.96);
\path[draw=drawColor,line width= 0.4pt,line join=round,line cap=round,fill=fillColor] (374.20,472.83) circle (  1.96);
\path[draw=drawColor,line width= 0.4pt,line join=round,line cap=round,fill=fillColor] (374.41,472.85) circle (  1.96);
\path[draw=drawColor,line width= 0.4pt,line join=round,line cap=round,fill=fillColor] (374.62,472.87) circle (  1.96);
\path[draw=drawColor,line width= 0.4pt,line join=round,line cap=round,fill=fillColor] (374.83,472.90) circle (  1.96);
\path[draw=drawColor,line width= 0.4pt,line join=round,line cap=round,fill=fillColor] (375.04,472.92) circle (  1.96);
\path[draw=drawColor,line width= 0.4pt,line join=round,line cap=round,fill=fillColor] (375.25,472.94) circle (  1.96);
\path[draw=drawColor,line width= 0.4pt,line join=round,line cap=round,fill=fillColor] (375.46,472.97) circle (  1.96);
\path[draw=drawColor,line width= 0.4pt,line join=round,line cap=round,fill=fillColor] (375.67,472.99) circle (  1.96);
\path[draw=drawColor,line width= 0.4pt,line join=round,line cap=round,fill=fillColor] (375.87,473.02) circle (  1.96);
\path[draw=drawColor,line width= 0.4pt,line join=round,line cap=round,fill=fillColor] (376.08,473.04) circle (  1.96);
\path[draw=drawColor,line width= 0.4pt,line join=round,line cap=round,fill=fillColor] (376.29,473.06) circle (  1.96);
\path[draw=drawColor,line width= 0.4pt,line join=round,line cap=round,fill=fillColor] (376.50,473.08) circle (  1.96);
\path[draw=drawColor,line width= 0.4pt,line join=round,line cap=round,fill=fillColor] (376.71,473.11) circle (  1.96);
\path[draw=drawColor,line width= 0.4pt,line join=round,line cap=round,fill=fillColor] (376.92,473.13) circle (  1.96);
\path[draw=drawColor,line width= 0.4pt,line join=round,line cap=round,fill=fillColor] (377.13,473.15) circle (  1.96);
\path[draw=drawColor,line width= 0.4pt,line join=round,line cap=round,fill=fillColor] (377.34,473.18) circle (  1.96);
\path[draw=drawColor,line width= 0.4pt,line join=round,line cap=round,fill=fillColor] (377.55,473.20) circle (  1.96);
\path[draw=drawColor,line width= 0.4pt,line join=round,line cap=round,fill=fillColor] (377.76,473.22) circle (  1.96);
\path[draw=drawColor,line width= 0.4pt,line join=round,line cap=round,fill=fillColor] (377.97,473.24) circle (  1.96);
\path[draw=drawColor,line width= 0.4pt,line join=round,line cap=round,fill=fillColor] (378.18,473.27) circle (  1.96);
\path[draw=drawColor,line width= 0.4pt,line join=round,line cap=round,fill=fillColor] (378.39,473.29) circle (  1.96);
\path[draw=drawColor,line width= 0.4pt,line join=round,line cap=round,fill=fillColor] (378.60,473.31) circle (  1.96);
\path[draw=drawColor,line width= 0.4pt,line join=round,line cap=round,fill=fillColor] (378.81,473.33) circle (  1.96);
\path[draw=drawColor,line width= 0.4pt,line join=round,line cap=round,fill=fillColor] (379.02,473.36) circle (  1.96);
\path[draw=drawColor,line width= 0.4pt,line join=round,line cap=round,fill=fillColor] (379.23,473.38) circle (  1.96);
\path[draw=drawColor,line width= 0.4pt,line join=round,line cap=round,fill=fillColor] (379.44,473.40) circle (  1.96);
\path[draw=drawColor,line width= 0.4pt,line join=round,line cap=round,fill=fillColor] (379.65,473.42) circle (  1.96);
\path[draw=drawColor,line width= 0.4pt,line join=round,line cap=round,fill=fillColor] (379.86,473.44) circle (  1.96);
\path[draw=drawColor,line width= 0.4pt,line join=round,line cap=round,fill=fillColor] (380.07,473.47) circle (  1.96);
\path[draw=drawColor,line width= 0.4pt,line join=round,line cap=round,fill=fillColor] (380.28,473.49) circle (  1.96);
\path[draw=drawColor,line width= 0.4pt,line join=round,line cap=round,fill=fillColor] (380.49,473.51) circle (  1.96);
\path[draw=drawColor,line width= 0.4pt,line join=round,line cap=round,fill=fillColor] (380.70,473.53) circle (  1.96);
\path[draw=drawColor,line width= 0.4pt,line join=round,line cap=round,fill=fillColor] (380.91,473.55) circle (  1.96);
\path[draw=drawColor,line width= 0.4pt,line join=round,line cap=round,fill=fillColor] (381.12,473.57) circle (  1.96);
\path[draw=drawColor,line width= 0.4pt,line join=round,line cap=round,fill=fillColor] (381.32,473.59) circle (  1.96);
\path[draw=drawColor,line width= 0.4pt,line join=round,line cap=round,fill=fillColor] (381.53,473.61) circle (  1.96);
\path[draw=drawColor,line width= 0.4pt,line join=round,line cap=round,fill=fillColor] (381.74,473.64) circle (  1.96);
\path[draw=drawColor,line width= 0.4pt,line join=round,line cap=round,fill=fillColor] (381.95,473.66) circle (  1.96);
\path[draw=drawColor,line width= 0.4pt,line join=round,line cap=round,fill=fillColor] (382.16,473.68) circle (  1.96);
\path[draw=drawColor,line width= 0.4pt,line join=round,line cap=round,fill=fillColor] (382.37,473.70) circle (  1.96);
\path[draw=drawColor,line width= 0.4pt,line join=round,line cap=round,fill=fillColor] (382.58,473.72) circle (  1.96);
\path[draw=drawColor,line width= 0.4pt,line join=round,line cap=round,fill=fillColor] (382.79,473.74) circle (  1.96);
\path[draw=drawColor,line width= 0.4pt,line join=round,line cap=round,fill=fillColor] (383.00,473.76) circle (  1.96);
\path[draw=drawColor,line width= 0.4pt,line join=round,line cap=round,fill=fillColor] (383.21,473.78) circle (  1.96);
\path[draw=drawColor,line width= 0.4pt,line join=round,line cap=round,fill=fillColor] (383.42,473.80) circle (  1.96);
\path[draw=drawColor,line width= 0.4pt,line join=round,line cap=round,fill=fillColor] (383.63,473.82) circle (  1.96);
\path[draw=drawColor,line width= 0.4pt,line join=round,line cap=round,fill=fillColor] (383.84,473.84) circle (  1.96);
\path[draw=drawColor,line width= 0.4pt,line join=round,line cap=round,fill=fillColor] (384.05,473.86) circle (  1.96);
\path[draw=drawColor,line width= 0.4pt,line join=round,line cap=round,fill=fillColor] (384.26,473.88) circle (  1.96);
\path[draw=drawColor,line width= 0.4pt,line join=round,line cap=round,fill=fillColor] (384.47,473.90) circle (  1.96);
\path[draw=drawColor,line width= 0.4pt,line join=round,line cap=round,fill=fillColor] (384.68,473.92) circle (  1.96);
\path[draw=drawColor,line width= 0.4pt,line join=round,line cap=round,fill=fillColor] (384.89,473.94) circle (  1.96);
\path[draw=drawColor,line width= 0.4pt,line join=round,line cap=round,fill=fillColor] (385.10,473.96) circle (  1.96);
\path[draw=drawColor,line width= 0.4pt,line join=round,line cap=round,fill=fillColor] (385.31,473.98) circle (  1.96);
\path[draw=drawColor,line width= 0.4pt,line join=round,line cap=round,fill=fillColor] (385.52,474.00) circle (  1.96);
\path[draw=drawColor,line width= 0.4pt,line join=round,line cap=round,fill=fillColor] (385.73,474.02) circle (  1.96);
\path[draw=drawColor,line width= 0.4pt,line join=round,line cap=round,fill=fillColor] (385.94,474.04) circle (  1.96);
\path[draw=drawColor,line width= 0.4pt,line join=round,line cap=round,fill=fillColor] (386.15,474.06) circle (  1.96);
\path[draw=drawColor,line width= 0.4pt,line join=round,line cap=round,fill=fillColor] (386.36,474.08) circle (  1.96);
\path[draw=drawColor,line width= 0.4pt,line join=round,line cap=round,fill=fillColor] (386.57,474.10) circle (  1.96);
\path[draw=drawColor,line width= 0.4pt,line join=round,line cap=round,fill=fillColor] (386.78,474.12) circle (  1.96);
\path[draw=drawColor,line width= 0.4pt,line join=round,line cap=round,fill=fillColor] (386.98,474.14) circle (  1.96);
\path[draw=drawColor,line width= 0.4pt,line join=round,line cap=round,fill=fillColor] (387.19,474.16) circle (  1.96);
\path[draw=drawColor,line width= 0.4pt,line join=round,line cap=round,fill=fillColor] (387.40,474.18) circle (  1.96);
\path[draw=drawColor,line width= 0.4pt,line join=round,line cap=round,fill=fillColor] (387.61,474.20) circle (  1.96);
\path[draw=drawColor,line width= 0.4pt,line join=round,line cap=round,fill=fillColor] (387.82,474.22) circle (  1.96);
\path[draw=drawColor,line width= 0.4pt,line join=round,line cap=round,fill=fillColor] (388.03,474.23) circle (  1.96);
\path[draw=drawColor,line width= 0.4pt,line join=round,line cap=round,fill=fillColor] (388.24,474.25) circle (  1.96);
\path[draw=drawColor,line width= 0.4pt,line join=round,line cap=round,fill=fillColor] (388.45,474.27) circle (  1.96);
\path[draw=drawColor,line width= 0.4pt,line join=round,line cap=round,fill=fillColor] (388.66,474.29) circle (  1.96);
\path[draw=drawColor,line width= 0.4pt,line join=round,line cap=round,fill=fillColor] (388.87,474.31) circle (  1.96);
\path[draw=drawColor,line width= 0.4pt,line join=round,line cap=round,fill=fillColor] (389.08,474.33) circle (  1.96);
\path[draw=drawColor,line width= 0.4pt,line join=round,line cap=round,fill=fillColor] (389.29,474.35) circle (  1.96);
\path[draw=drawColor,line width= 0.4pt,line join=round,line cap=round,fill=fillColor] (389.50,474.37) circle (  1.96);
\path[draw=drawColor,line width= 0.4pt,line join=round,line cap=round,fill=fillColor] (389.71,474.38) circle (  1.96);
\path[draw=drawColor,line width= 0.4pt,line join=round,line cap=round,fill=fillColor] (389.92,474.40) circle (  1.96);
\path[draw=drawColor,line width= 0.4pt,line join=round,line cap=round,fill=fillColor] (390.13,474.42) circle (  1.96);
\path[draw=drawColor,line width= 0.4pt,line join=round,line cap=round,fill=fillColor] (390.34,474.44) circle (  1.96);
\path[draw=drawColor,line width= 0.4pt,line join=round,line cap=round,fill=fillColor] (390.55,474.46) circle (  1.96);
\path[draw=drawColor,line width= 0.4pt,line join=round,line cap=round,fill=fillColor] (390.76,474.47) circle (  1.96);
\path[draw=drawColor,line width= 0.4pt,line join=round,line cap=round,fill=fillColor] (390.97,474.49) circle (  1.96);
\path[draw=drawColor,line width= 0.4pt,line join=round,line cap=round,fill=fillColor] (391.18,474.51) circle (  1.96);
\path[draw=drawColor,line width= 0.4pt,line join=round,line cap=round,fill=fillColor] (391.39,474.53) circle (  1.96);
\path[draw=drawColor,line width= 0.4pt,line join=round,line cap=round,fill=fillColor] (391.60,474.55) circle (  1.96);
\path[draw=drawColor,line width= 0.4pt,line join=round,line cap=round,fill=fillColor] (391.81,474.56) circle (  1.96);
\path[draw=drawColor,line width= 0.4pt,line join=round,line cap=round,fill=fillColor] (392.02,474.58) circle (  1.96);
\path[draw=drawColor,line width= 0.4pt,line join=round,line cap=round,fill=fillColor] (392.23,474.60) circle (  1.96);
\path[draw=drawColor,line width= 0.4pt,line join=round,line cap=round,fill=fillColor] (392.43,474.62) circle (  1.96);
\path[draw=drawColor,line width= 0.4pt,line join=round,line cap=round,fill=fillColor] (392.64,474.63) circle (  1.96);
\path[draw=drawColor,line width= 0.4pt,line join=round,line cap=round,fill=fillColor] (392.85,474.65) circle (  1.96);
\path[draw=drawColor,line width= 0.4pt,line join=round,line cap=round,fill=fillColor] (393.06,474.67) circle (  1.96);
\path[draw=drawColor,line width= 0.4pt,line join=round,line cap=round,fill=fillColor] (393.27,474.69) circle (  1.96);
\path[draw=drawColor,line width= 0.4pt,line join=round,line cap=round,fill=fillColor] (393.48,474.70) circle (  1.96);
\path[draw=drawColor,line width= 0.4pt,line join=round,line cap=round,fill=fillColor] (393.69,474.72) circle (  1.96);
\path[draw=drawColor,line width= 0.4pt,line join=round,line cap=round,fill=fillColor] (393.90,474.74) circle (  1.96);
\path[draw=drawColor,line width= 0.4pt,line join=round,line cap=round,fill=fillColor] (394.11,474.76) circle (  1.96);
\path[draw=drawColor,line width= 0.4pt,line join=round,line cap=round,fill=fillColor] (394.32,474.77) circle (  1.96);
\path[draw=drawColor,line width= 0.4pt,line join=round,line cap=round,fill=fillColor] (394.53,474.79) circle (  1.96);
\path[draw=drawColor,line width= 0.4pt,line join=round,line cap=round,fill=fillColor] (394.74,474.81) circle (  1.96);
\path[draw=drawColor,line width= 0.4pt,line join=round,line cap=round,fill=fillColor] (394.95,474.82) circle (  1.96);
\path[draw=drawColor,line width= 0.4pt,line join=round,line cap=round,fill=fillColor] (395.16,474.84) circle (  1.96);
\path[draw=drawColor,line width= 0.4pt,line join=round,line cap=round,fill=fillColor] (395.37,474.86) circle (  1.96);
\path[draw=drawColor,line width= 0.4pt,line join=round,line cap=round,fill=fillColor] (395.58,474.87) circle (  1.96);
\path[draw=drawColor,line width= 0.4pt,line join=round,line cap=round,fill=fillColor] (395.79,474.89) circle (  1.96);
\path[draw=drawColor,line width= 0.4pt,line join=round,line cap=round,fill=fillColor] (396.00,474.91) circle (  1.96);
\path[draw=drawColor,line width= 0.4pt,line join=round,line cap=round,fill=fillColor] (396.21,474.92) circle (  1.96);
\path[draw=drawColor,line width= 0.4pt,line join=round,line cap=round,fill=fillColor] (396.42,474.94) circle (  1.96);
\path[draw=drawColor,line width= 0.4pt,line join=round,line cap=round,fill=fillColor] (396.63,474.96) circle (  1.96);
\path[draw=drawColor,line width= 0.4pt,line join=round,line cap=round,fill=fillColor] (396.84,474.97) circle (  1.96);
\path[draw=drawColor,line width= 0.4pt,line join=round,line cap=round,fill=fillColor] (397.05,474.99) circle (  1.96);
\path[draw=drawColor,line width= 0.4pt,line join=round,line cap=round,fill=fillColor] (397.26,475.00) circle (  1.96);
\path[draw=drawColor,line width= 0.4pt,line join=round,line cap=round,fill=fillColor] (397.47,475.02) circle (  1.96);
\path[draw=drawColor,line width= 0.4pt,line join=round,line cap=round,fill=fillColor] (397.68,475.04) circle (  1.96);
\path[draw=drawColor,line width= 0.4pt,line join=round,line cap=round,fill=fillColor] (397.88,475.05) circle (  1.96);
\path[draw=drawColor,line width= 0.4pt,line join=round,line cap=round,fill=fillColor] (398.09,475.07) circle (  1.96);
\path[draw=drawColor,line width= 0.4pt,line join=round,line cap=round,fill=fillColor] (398.30,475.08) circle (  1.96);
\path[draw=drawColor,line width= 0.4pt,line join=round,line cap=round,fill=fillColor] (398.51,475.10) circle (  1.96);
\path[draw=drawColor,line width= 0.4pt,line join=round,line cap=round,fill=fillColor] (398.72,475.11) circle (  1.96);
\path[draw=drawColor,line width= 0.4pt,line join=round,line cap=round,fill=fillColor] (398.93,475.13) circle (  1.96);
\path[draw=drawColor,line width= 0.4pt,line join=round,line cap=round,fill=fillColor] (399.14,475.15) circle (  1.96);
\path[draw=drawColor,line width= 0.4pt,line join=round,line cap=round,fill=fillColor] (399.35,475.16) circle (  1.96);
\path[draw=drawColor,line width= 0.4pt,line join=round,line cap=round,fill=fillColor] (399.56,475.18) circle (  1.96);
\path[draw=drawColor,line width= 0.4pt,line join=round,line cap=round,fill=fillColor] (399.77,475.19) circle (  1.96);
\path[draw=drawColor,line width= 0.4pt,line join=round,line cap=round,fill=fillColor] (399.98,475.21) circle (  1.96);
\path[draw=drawColor,line width= 0.4pt,line join=round,line cap=round,fill=fillColor] (400.19,475.22) circle (  1.96);
\path[draw=drawColor,line width= 0.4pt,line join=round,line cap=round,fill=fillColor] (400.40,475.24) circle (  1.96);
\path[draw=drawColor,line width= 0.4pt,line join=round,line cap=round,fill=fillColor] (400.61,475.25) circle (  1.96);
\path[draw=drawColor,line width= 0.4pt,line join=round,line cap=round,fill=fillColor] (400.82,475.27) circle (  1.96);
\path[draw=drawColor,line width= 0.4pt,line join=round,line cap=round,fill=fillColor] (401.03,475.28) circle (  1.96);
\path[draw=drawColor,line width= 0.4pt,line join=round,line cap=round,fill=fillColor] (401.24,475.30) circle (  1.96);
\path[draw=drawColor,line width= 0.4pt,line join=round,line cap=round,fill=fillColor] (401.45,475.31) circle (  1.96);
\path[draw=drawColor,line width= 0.4pt,line join=round,line cap=round,fill=fillColor] (401.66,475.33) circle (  1.96);
\path[draw=drawColor,line width= 0.4pt,line join=round,line cap=round,fill=fillColor] (401.87,475.34) circle (  1.96);
\path[draw=drawColor,line width= 0.4pt,line join=round,line cap=round,fill=fillColor] (402.08,475.36) circle (  1.96);
\path[draw=drawColor,line width= 0.4pt,line join=round,line cap=round,fill=fillColor] (402.29,475.37) circle (  1.96);
\path[draw=drawColor,line width= 0.4pt,line join=round,line cap=round,fill=fillColor] (402.50,475.39) circle (  1.96);
\path[draw=drawColor,line width= 0.4pt,line join=round,line cap=round,fill=fillColor] (402.71,475.40) circle (  1.96);
\path[draw=drawColor,line width= 0.4pt,line join=round,line cap=round,fill=fillColor] (402.92,475.42) circle (  1.96);
\path[draw=drawColor,line width= 0.4pt,line join=round,line cap=round,fill=fillColor] (403.13,475.43) circle (  1.96);
\path[draw=drawColor,line width= 0.4pt,line join=round,line cap=round,fill=fillColor] (403.34,475.45) circle (  1.96);
\path[draw=drawColor,line width= 0.4pt,line join=round,line cap=round,fill=fillColor] (403.54,475.46) circle (  1.96);
\path[draw=drawColor,line width= 0.4pt,line join=round,line cap=round,fill=fillColor] (403.75,475.47) circle (  1.96);
\path[draw=drawColor,line width= 0.4pt,line join=round,line cap=round,fill=fillColor] (403.96,475.49) circle (  1.96);
\path[draw=drawColor,line width= 0.4pt,line join=round,line cap=round,fill=fillColor] (404.17,475.50) circle (  1.96);
\path[draw=drawColor,line width= 0.4pt,line join=round,line cap=round,fill=fillColor] (404.38,475.52) circle (  1.96);
\path[draw=drawColor,line width= 0.4pt,line join=round,line cap=round,fill=fillColor] (404.59,475.53) circle (  1.96);
\path[draw=drawColor,line width= 0.4pt,line join=round,line cap=round,fill=fillColor] (404.80,475.55) circle (  1.96);
\path[draw=drawColor,line width= 0.4pt,line join=round,line cap=round,fill=fillColor] (405.01,475.56) circle (  1.96);
\path[draw=drawColor,line width= 0.4pt,line join=round,line cap=round,fill=fillColor] (405.22,475.57) circle (  1.96);
\path[draw=drawColor,line width= 0.4pt,line join=round,line cap=round,fill=fillColor] (405.43,475.59) circle (  1.96);
\path[draw=drawColor,line width= 0.4pt,line join=round,line cap=round,fill=fillColor] (405.64,475.60) circle (  1.96);
\path[draw=drawColor,line width= 0.4pt,line join=round,line cap=round,fill=fillColor] (405.85,475.62) circle (  1.96);
\path[draw=drawColor,line width= 0.4pt,line join=round,line cap=round,fill=fillColor] (406.06,475.63) circle (  1.96);
\path[draw=drawColor,line width= 0.4pt,line join=round,line cap=round,fill=fillColor] (406.27,475.64) circle (  1.96);
\path[draw=drawColor,line width= 0.4pt,line join=round,line cap=round,fill=fillColor] (406.48,475.66) circle (  1.96);
\path[draw=drawColor,line width= 0.4pt,line join=round,line cap=round,fill=fillColor] (406.69,475.67) circle (  1.96);
\path[draw=drawColor,line width= 0.4pt,line join=round,line cap=round,fill=fillColor] (406.90,475.68) circle (  1.96);
\path[draw=drawColor,line width= 0.4pt,line join=round,line cap=round,fill=fillColor] (407.11,475.70) circle (  1.96);
\path[draw=drawColor,line width= 0.4pt,line join=round,line cap=round,fill=fillColor] (407.32,475.71) circle (  1.96);
\path[draw=drawColor,line width= 0.4pt,line join=round,line cap=round,fill=fillColor] (407.53,475.72) circle (  1.96);
\path[draw=drawColor,line width= 0.4pt,line join=round,line cap=round,fill=fillColor] (407.74,475.74) circle (  1.96);
\path[draw=drawColor,line width= 0.4pt,line join=round,line cap=round,fill=fillColor] (407.95,475.75) circle (  1.96);
\path[draw=drawColor,line width= 0.4pt,line join=round,line cap=round,fill=fillColor] (408.16,475.76) circle (  1.96);
\path[draw=drawColor,line width= 0.4pt,line join=round,line cap=round,fill=fillColor] (408.37,475.78) circle (  1.96);
\path[draw=drawColor,line width= 0.4pt,line join=round,line cap=round,fill=fillColor] (408.58,475.79) circle (  1.96);
\path[draw=drawColor,line width= 0.4pt,line join=round,line cap=round,fill=fillColor] (408.79,475.80) circle (  1.96);
\path[draw=drawColor,line width= 0.4pt,line join=round,line cap=round,fill=fillColor] (408.99,475.82) circle (  1.96);
\path[draw=drawColor,line width= 0.4pt,line join=round,line cap=round,fill=fillColor] (409.20,475.83) circle (  1.96);
\path[draw=drawColor,line width= 0.4pt,line join=round,line cap=round,fill=fillColor] (409.41,475.84) circle (  1.96);
\path[draw=drawColor,line width= 0.4pt,line join=round,line cap=round,fill=fillColor] (409.62,475.86) circle (  1.96);
\path[draw=drawColor,line width= 0.4pt,line join=round,line cap=round,fill=fillColor] (409.83,475.87) circle (  1.96);
\path[draw=drawColor,line width= 0.4pt,line join=round,line cap=round,fill=fillColor] (410.04,475.88) circle (  1.96);
\path[draw=drawColor,line width= 0.4pt,line join=round,line cap=round,fill=fillColor] (410.25,475.90) circle (  1.96);
\path[draw=drawColor,line width= 0.4pt,line join=round,line cap=round,fill=fillColor] (410.46,475.91) circle (  1.96);
\path[draw=drawColor,line width= 0.4pt,line join=round,line cap=round,fill=fillColor] (410.67,475.92) circle (  1.96);
\path[draw=drawColor,line width= 0.4pt,line join=round,line cap=round,fill=fillColor] (410.88,475.93) circle (  1.96);
\path[draw=drawColor,line width= 0.4pt,line join=round,line cap=round,fill=fillColor] (411.09,475.95) circle (  1.96);
\path[draw=drawColor,line width= 0.4pt,line join=round,line cap=round,fill=fillColor] (411.30,475.96) circle (  1.96);
\path[draw=drawColor,line width= 0.4pt,line join=round,line cap=round,fill=fillColor] (411.51,475.97) circle (  1.96);
\path[draw=drawColor,line width= 0.4pt,line join=round,line cap=round,fill=fillColor] (411.72,475.98) circle (  1.96);
\path[draw=drawColor,line width= 0.4pt,line join=round,line cap=round,fill=fillColor] (411.93,476.00) circle (  1.96);
\path[draw=drawColor,line width= 0.4pt,line join=round,line cap=round,fill=fillColor] (412.14,476.01) circle (  1.96);
\path[draw=drawColor,line width= 0.4pt,line join=round,line cap=round,fill=fillColor] (412.35,476.02) circle (  1.96);
\path[draw=drawColor,line width= 0.4pt,line join=round,line cap=round,fill=fillColor] (412.56,476.03) circle (  1.96);
\path[draw=drawColor,line width= 0.4pt,line join=round,line cap=round,fill=fillColor] (412.77,476.05) circle (  1.96);
\path[draw=drawColor,line width= 0.4pt,line join=round,line cap=round,fill=fillColor] (412.98,476.06) circle (  1.96);
\path[draw=drawColor,line width= 0.4pt,line join=round,line cap=round,fill=fillColor] (413.19,476.07) circle (  1.96);
\path[draw=drawColor,line width= 0.4pt,line join=round,line cap=round,fill=fillColor] (413.40,476.08) circle (  1.96);
\path[draw=drawColor,line width= 0.4pt,line join=round,line cap=round,fill=fillColor] (413.61,476.09) circle (  1.96);
\path[draw=drawColor,line width= 0.4pt,line join=round,line cap=round,fill=fillColor] (413.82,476.11) circle (  1.96);
\path[draw=drawColor,line width= 0.4pt,line join=round,line cap=round,fill=fillColor] (414.03,476.12) circle (  1.96);
\path[draw=drawColor,line width= 0.4pt,line join=round,line cap=round,fill=fillColor] (414.24,476.13) circle (  1.96);
\path[draw=drawColor,line width= 0.4pt,line join=round,line cap=round,fill=fillColor] (414.45,476.14) circle (  1.96);
\path[draw=drawColor,line width= 0.4pt,line join=round,line cap=round,fill=fillColor] (414.65,476.15) circle (  1.96);
\path[draw=drawColor,line width= 0.4pt,line join=round,line cap=round,fill=fillColor] (414.86,476.17) circle (  1.96);
\path[draw=drawColor,line width= 0.4pt,line join=round,line cap=round,fill=fillColor] (415.07,476.18) circle (  1.96);
\path[draw=drawColor,line width= 0.4pt,line join=round,line cap=round,fill=fillColor] (415.28,476.19) circle (  1.96);
\path[draw=drawColor,line width= 0.4pt,line join=round,line cap=round,fill=fillColor] (415.49,476.20) circle (  1.96);
\path[draw=drawColor,line width= 0.4pt,line join=round,line cap=round,fill=fillColor] (415.70,476.21) circle (  1.96);
\path[draw=drawColor,line width= 0.4pt,line join=round,line cap=round,fill=fillColor] (415.91,476.22) circle (  1.96);
\path[draw=drawColor,line width= 0.4pt,line join=round,line cap=round,fill=fillColor] (416.12,476.23) circle (  1.96);
\path[draw=drawColor,line width= 0.4pt,line join=round,line cap=round,fill=fillColor] (416.33,476.25) circle (  1.96);
\path[draw=drawColor,line width= 0.4pt,line join=round,line cap=round,fill=fillColor] (416.54,476.26) circle (  1.96);
\path[draw=drawColor,line width= 0.4pt,line join=round,line cap=round,fill=fillColor] (416.75,476.27) circle (  1.96);
\path[draw=drawColor,line width= 0.4pt,line join=round,line cap=round,fill=fillColor] (416.96,476.28) circle (  1.96);
\path[draw=drawColor,line width= 0.4pt,line join=round,line cap=round,fill=fillColor] (417.17,476.29) circle (  1.96);
\path[draw=drawColor,line width= 0.4pt,line join=round,line cap=round,fill=fillColor] (417.38,476.30) circle (  1.96);
\path[draw=drawColor,line width= 0.4pt,line join=round,line cap=round,fill=fillColor] (417.59,476.31) circle (  1.96);
\path[draw=drawColor,line width= 0.4pt,line join=round,line cap=round,fill=fillColor] (417.80,476.33) circle (  1.96);
\path[draw=drawColor,line width= 0.4pt,line join=round,line cap=round,fill=fillColor] (418.01,476.34) circle (  1.96);
\path[draw=drawColor,line width= 0.4pt,line join=round,line cap=round,fill=fillColor] (418.22,476.35) circle (  1.96);
\path[draw=drawColor,line width= 0.4pt,line join=round,line cap=round,fill=fillColor] (418.43,476.36) circle (  1.96);
\path[draw=drawColor,line width= 0.4pt,line join=round,line cap=round,fill=fillColor] (418.64,476.37) circle (  1.96);
\path[draw=drawColor,line width= 0.4pt,line join=round,line cap=round,fill=fillColor] (418.85,476.38) circle (  1.96);
\path[draw=drawColor,line width= 0.4pt,line join=round,line cap=round,fill=fillColor] (419.06,476.39) circle (  1.96);
\path[draw=drawColor,line width= 0.4pt,line join=round,line cap=round,fill=fillColor] (419.27,476.40) circle (  1.96);
\path[draw=drawColor,line width= 0.4pt,line join=round,line cap=round,fill=fillColor] (419.48,476.41) circle (  1.96);
\path[draw=drawColor,line width= 0.4pt,line join=round,line cap=round,fill=fillColor] (419.69,476.42) circle (  1.96);
\path[draw=drawColor,line width= 0.4pt,line join=round,line cap=round,fill=fillColor] (419.90,476.44) circle (  1.96);
\path[draw=drawColor,line width= 0.4pt,line join=round,line cap=round,fill=fillColor] (420.10,476.45) circle (  1.96);
\path[draw=drawColor,line width= 0.4pt,line join=round,line cap=round,fill=fillColor] (420.31,476.46) circle (  1.96);
\path[draw=drawColor,line width= 0.4pt,line join=round,line cap=round,fill=fillColor] (420.52,476.47) circle (  1.96);
\path[draw=drawColor,line width= 0.4pt,line join=round,line cap=round,fill=fillColor] (420.73,476.48) circle (  1.96);
\path[draw=drawColor,line width= 0.4pt,line join=round,line cap=round,fill=fillColor] (420.94,476.49) circle (  1.96);
\path[draw=drawColor,line width= 0.4pt,line join=round,line cap=round,fill=fillColor] (421.15,476.50) circle (  1.96);
\path[draw=drawColor,line width= 0.4pt,line join=round,line cap=round,fill=fillColor] (421.36,476.51) circle (  1.96);
\path[draw=drawColor,line width= 0.4pt,line join=round,line cap=round,fill=fillColor] (421.57,476.52) circle (  1.96);
\path[draw=drawColor,line width= 0.4pt,line join=round,line cap=round,fill=fillColor] (421.78,476.53) circle (  1.96);
\path[draw=drawColor,line width= 0.4pt,line join=round,line cap=round,fill=fillColor] (421.99,476.54) circle (  1.96);
\path[draw=drawColor,line width= 0.4pt,line join=round,line cap=round,fill=fillColor] (422.20,476.55) circle (  1.96);
\path[draw=drawColor,line width= 0.4pt,line join=round,line cap=round,fill=fillColor] (422.41,476.56) circle (  1.96);
\path[draw=drawColor,line width= 0.4pt,line join=round,line cap=round,fill=fillColor] (422.62,476.57) circle (  1.96);
\path[draw=drawColor,line width= 0.4pt,line join=round,line cap=round,fill=fillColor] (422.83,476.58) circle (  1.96);
\path[draw=drawColor,line width= 0.4pt,line join=round,line cap=round,fill=fillColor] (423.04,476.59) circle (  1.96);
\path[draw=drawColor,line width= 0.4pt,line join=round,line cap=round,fill=fillColor] (423.25,476.60) circle (  1.96);
\path[draw=drawColor,line width= 0.4pt,line join=round,line cap=round,fill=fillColor] (423.46,476.61) circle (  1.96);
\path[draw=drawColor,line width= 0.4pt,line join=round,line cap=round,fill=fillColor] (423.67,476.62) circle (  1.96);
\path[draw=drawColor,line width= 0.4pt,line join=round,line cap=round,fill=fillColor] (423.88,476.63) circle (  1.96);
\path[draw=drawColor,line width= 0.4pt,line join=round,line cap=round,fill=fillColor] (424.09,476.64) circle (  1.96);
\path[draw=drawColor,line width= 0.4pt,line join=round,line cap=round,fill=fillColor] (424.30,476.65) circle (  1.96);
\path[draw=drawColor,line width= 0.4pt,line join=round,line cap=round,fill=fillColor] (424.51,476.66) circle (  1.96);
\path[draw=drawColor,line width= 0.4pt,line join=round,line cap=round,fill=fillColor] (424.72,476.67) circle (  1.96);
\path[draw=drawColor,line width= 0.4pt,line join=round,line cap=round,fill=fillColor] (424.93,476.68) circle (  1.96);
\path[draw=drawColor,line width= 0.4pt,line join=round,line cap=round,fill=fillColor] (425.14,476.69) circle (  1.96);
\path[draw=drawColor,line width= 0.4pt,line join=round,line cap=round,fill=fillColor] (425.35,476.70) circle (  1.96);
\path[draw=drawColor,line width= 0.4pt,line join=round,line cap=round,fill=fillColor] (425.56,476.71) circle (  1.96);
\path[draw=drawColor,line width= 0.4pt,line join=round,line cap=round,fill=fillColor] (425.76,476.72) circle (  1.96);
\path[draw=drawColor,line width= 0.4pt,line join=round,line cap=round,fill=fillColor] (425.97,476.73) circle (  1.96);
\path[draw=drawColor,line width= 0.4pt,line join=round,line cap=round,fill=fillColor] (426.18,476.74) circle (  1.96);
\path[draw=drawColor,line width= 0.4pt,line join=round,line cap=round,fill=fillColor] (426.39,476.75) circle (  1.96);
\path[draw=drawColor,line width= 0.4pt,line join=round,line cap=round,fill=fillColor] (426.60,476.76) circle (  1.96);
\path[draw=drawColor,line width= 0.4pt,line join=round,line cap=round,fill=fillColor] (426.81,476.77) circle (  1.96);
\path[draw=drawColor,line width= 0.4pt,line join=round,line cap=round,fill=fillColor] (427.02,476.78) circle (  1.96);
\path[draw=drawColor,line width= 0.4pt,line join=round,line cap=round,fill=fillColor] (427.23,476.79) circle (  1.96);
\path[draw=drawColor,line width= 0.4pt,line join=round,line cap=round,fill=fillColor] (427.44,476.80) circle (  1.96);
\path[draw=drawColor,line width= 0.4pt,line join=round,line cap=round,fill=fillColor] (427.65,476.81) circle (  1.96);
\path[draw=drawColor,line width= 0.4pt,line join=round,line cap=round,fill=fillColor] (427.86,476.82) circle (  1.96);
\path[draw=drawColor,line width= 0.4pt,line join=round,line cap=round,fill=fillColor] (428.07,476.82) circle (  1.96);
\path[draw=drawColor,line width= 0.4pt,line join=round,line cap=round,fill=fillColor] (428.28,476.83) circle (  1.96);
\path[draw=drawColor,line width= 0.4pt,line join=round,line cap=round,fill=fillColor] (428.49,476.84) circle (  1.96);
\path[draw=drawColor,line width= 0.4pt,line join=round,line cap=round,fill=fillColor] (428.70,476.85) circle (  1.96);
\path[draw=drawColor,line width= 0.4pt,line join=round,line cap=round,fill=fillColor] (428.91,476.86) circle (  1.96);
\path[draw=drawColor,line width= 0.4pt,line join=round,line cap=round,fill=fillColor] (429.12,476.87) circle (  1.96);
\path[draw=drawColor,line width= 0.4pt,line join=round,line cap=round,fill=fillColor] (429.33,476.88) circle (  1.96);
\path[draw=drawColor,line width= 0.4pt,line join=round,line cap=round,fill=fillColor] (429.54,476.89) circle (  1.96);
\path[draw=drawColor,line width= 0.4pt,line join=round,line cap=round,fill=fillColor] (429.75,476.90) circle (  1.96);
\path[draw=drawColor,line width= 0.4pt,line join=round,line cap=round,fill=fillColor] (429.96,476.91) circle (  1.96);
\path[draw=drawColor,line width= 0.4pt,line join=round,line cap=round,fill=fillColor] (430.17,476.92) circle (  1.96);
\path[draw=drawColor,line width= 0.4pt,line join=round,line cap=round,fill=fillColor] (430.38,476.92) circle (  1.96);
\path[draw=drawColor,line width= 0.4pt,line join=round,line cap=round,fill=fillColor] (430.59,476.93) circle (  1.96);
\path[draw=drawColor,line width= 0.4pt,line join=round,line cap=round,fill=fillColor] (430.80,476.94) circle (  1.96);
\path[draw=drawColor,line width= 0.4pt,line join=round,line cap=round,fill=fillColor] (431.01,476.95) circle (  1.96);
\path[draw=drawColor,line width= 0.4pt,line join=round,line cap=round,fill=fillColor] (431.21,476.96) circle (  1.96);
\path[draw=drawColor,line width= 0.4pt,line join=round,line cap=round,fill=fillColor] (431.42,476.97) circle (  1.96);
\path[draw=drawColor,line width= 0.4pt,line join=round,line cap=round,fill=fillColor] (431.63,476.98) circle (  1.96);
\path[draw=drawColor,line width= 0.4pt,line join=round,line cap=round,fill=fillColor] (431.84,476.99) circle (  1.96);
\path[draw=drawColor,line width= 0.4pt,line join=round,line cap=round,fill=fillColor] (432.05,476.99) circle (  1.96);
\path[draw=drawColor,line width= 0.4pt,line join=round,line cap=round,fill=fillColor] (432.26,477.00) circle (  1.96);
\path[draw=drawColor,line width= 0.4pt,line join=round,line cap=round,fill=fillColor] (432.47,477.01) circle (  1.96);
\path[draw=drawColor,line width= 0.4pt,line join=round,line cap=round,fill=fillColor] (432.68,477.02) circle (  1.96);
\path[draw=drawColor,line width= 0.4pt,line join=round,line cap=round,fill=fillColor] (432.89,477.03) circle (  1.96);
\path[draw=drawColor,line width= 0.4pt,line join=round,line cap=round,fill=fillColor] (433.10,477.04) circle (  1.96);
\path[draw=drawColor,line width= 0.4pt,line join=round,line cap=round,fill=fillColor] (433.31,477.05) circle (  1.96);
\path[draw=drawColor,line width= 0.4pt,line join=round,line cap=round,fill=fillColor] (433.52,477.05) circle (  1.96);
\path[draw=drawColor,line width= 0.4pt,line join=round,line cap=round,fill=fillColor] (433.73,477.06) circle (  1.96);
\path[draw=drawColor,line width= 0.4pt,line join=round,line cap=round,fill=fillColor] (433.94,477.07) circle (  1.96);
\path[draw=drawColor,line width= 0.4pt,line join=round,line cap=round,fill=fillColor] (434.15,477.08) circle (  1.96);
\path[draw=drawColor,line width= 0.4pt,line join=round,line cap=round,fill=fillColor] (434.36,477.09) circle (  1.96);
\path[draw=drawColor,line width= 0.4pt,line join=round,line cap=round,fill=fillColor] (434.57,477.10) circle (  1.96);
\path[draw=drawColor,line width= 0.4pt,line join=round,line cap=round,fill=fillColor] (434.78,477.10) circle (  1.96);
\path[draw=drawColor,line width= 0.4pt,line join=round,line cap=round,fill=fillColor] (434.99,477.11) circle (  1.96);
\path[draw=drawColor,line width= 0.4pt,line join=round,line cap=round,fill=fillColor] (435.20,477.12) circle (  1.96);
\path[draw=drawColor,line width= 0.4pt,line join=round,line cap=round,fill=fillColor] (435.41,477.13) circle (  1.96);
\path[draw=drawColor,line width= 0.4pt,line join=round,line cap=round,fill=fillColor] (435.62,477.14) circle (  1.96);
\path[draw=drawColor,line width= 0.4pt,line join=round,line cap=round,fill=fillColor] (435.83,477.14) circle (  1.96);
\path[draw=drawColor,line width= 0.4pt,line join=round,line cap=round,fill=fillColor] (436.04,477.15) circle (  1.96);
\path[draw=drawColor,line width= 0.4pt,line join=round,line cap=round,fill=fillColor] (436.25,477.16) circle (  1.96);
\path[draw=drawColor,line width= 0.4pt,line join=round,line cap=round,fill=fillColor] (436.46,477.17) circle (  1.96);
\path[draw=drawColor,line width= 0.4pt,line join=round,line cap=round,fill=fillColor] (436.66,477.18) circle (  1.96);
\path[draw=drawColor,line width= 0.4pt,line join=round,line cap=round,fill=fillColor] (436.87,477.18) circle (  1.96);
\path[draw=drawColor,line width= 0.4pt,line join=round,line cap=round,fill=fillColor] (437.08,477.19) circle (  1.96);
\path[draw=drawColor,line width= 0.4pt,line join=round,line cap=round,fill=fillColor] (437.29,477.20) circle (  1.96);
\path[draw=drawColor,line width= 0.4pt,line join=round,line cap=round,fill=fillColor] (437.50,477.21) circle (  1.96);
\path[draw=drawColor,line width= 0.4pt,line join=round,line cap=round,fill=fillColor] (437.71,477.22) circle (  1.96);
\path[draw=drawColor,line width= 0.4pt,line join=round,line cap=round,fill=fillColor] (437.92,477.22) circle (  1.96);
\path[draw=drawColor,line width= 0.4pt,line join=round,line cap=round,fill=fillColor] (438.13,477.23) circle (  1.96);
\path[draw=drawColor,line width= 0.4pt,line join=round,line cap=round,fill=fillColor] (438.34,477.24) circle (  1.96);
\path[draw=drawColor,line width= 0.4pt,line join=round,line cap=round,fill=fillColor] (438.55,477.25) circle (  1.96);
\path[draw=drawColor,line width= 0.4pt,line join=round,line cap=round,fill=fillColor] (438.76,477.25) circle (  1.96);
\path[draw=drawColor,line width= 0.4pt,line join=round,line cap=round,fill=fillColor] (438.97,477.26) circle (  1.96);
\path[draw=drawColor,line width= 0.4pt,line join=round,line cap=round,fill=fillColor] (439.18,477.27) circle (  1.96);
\path[draw=drawColor,line width= 0.4pt,line join=round,line cap=round,fill=fillColor] (439.39,477.28) circle (  1.96);
\path[draw=drawColor,line width= 0.4pt,line join=round,line cap=round,fill=fillColor] (439.60,477.28) circle (  1.96);
\path[draw=drawColor,line width= 0.4pt,line join=round,line cap=round,fill=fillColor] (439.81,477.29) circle (  1.96);
\path[draw=drawColor,line width= 0.4pt,line join=round,line cap=round,fill=fillColor] (440.02,477.30) circle (  1.96);
\path[draw=drawColor,line width= 0.4pt,line join=round,line cap=round,fill=fillColor] (440.23,477.31) circle (  1.96);
\path[draw=drawColor,line width= 0.4pt,line join=round,line cap=round,fill=fillColor] (440.44,477.31) circle (  1.96);
\path[draw=drawColor,line width= 0.4pt,line join=round,line cap=round,fill=fillColor] (440.65,477.32) circle (  1.96);
\path[draw=drawColor,line width= 0.4pt,line join=round,line cap=round,fill=fillColor] (440.86,477.33) circle (  1.96);
\path[draw=drawColor,line width= 0.4pt,line join=round,line cap=round,fill=fillColor] (441.07,477.34) circle (  1.96);
\path[draw=drawColor,line width= 0.4pt,line join=round,line cap=round,fill=fillColor] (441.28,477.34) circle (  1.96);
\path[draw=drawColor,line width= 0.4pt,line join=round,line cap=round,fill=fillColor] (441.49,477.35) circle (  1.96);
\path[draw=drawColor,line width= 0.4pt,line join=round,line cap=round,fill=fillColor] (441.70,477.36) circle (  1.96);
\path[draw=drawColor,line width= 0.4pt,line join=round,line cap=round,fill=fillColor] (441.91,477.36) circle (  1.96);
\path[draw=drawColor,line width= 0.4pt,line join=round,line cap=round,fill=fillColor] (442.12,477.37) circle (  1.96);
\path[draw=drawColor,line width= 0.4pt,line join=round,line cap=round,fill=fillColor] (442.32,477.38) circle (  1.96);
\path[draw=drawColor,line width= 0.4pt,line join=round,line cap=round,fill=fillColor] (442.53,477.39) circle (  1.96);
\path[draw=drawColor,line width= 0.4pt,line join=round,line cap=round,fill=fillColor] (442.74,477.39) circle (  1.96);
\path[draw=drawColor,line width= 0.4pt,line join=round,line cap=round,fill=fillColor] (442.95,477.40) circle (  1.96);
\path[draw=drawColor,line width= 0.4pt,line join=round,line cap=round,fill=fillColor] (443.16,477.41) circle (  1.96);
\path[draw=drawColor,line width= 0.4pt,line join=round,line cap=round,fill=fillColor] (443.37,477.41) circle (  1.96);
\path[draw=drawColor,line width= 0.4pt,line join=round,line cap=round,fill=fillColor] (443.58,477.42) circle (  1.96);
\path[draw=drawColor,line width= 0.4pt,line join=round,line cap=round,fill=fillColor] (443.79,477.43) circle (  1.96);
\path[draw=drawColor,line width= 0.4pt,line join=round,line cap=round,fill=fillColor] (444.00,477.44) circle (  1.96);
\path[draw=drawColor,line width= 0.4pt,line join=round,line cap=round,fill=fillColor] (444.21,477.44) circle (  1.96);
\path[draw=drawColor,line width= 0.4pt,line join=round,line cap=round,fill=fillColor] (444.42,477.45) circle (  1.96);
\path[draw=drawColor,line width= 0.4pt,line join=round,line cap=round,fill=fillColor] (444.63,477.46) circle (  1.96);
\path[draw=drawColor,line width= 0.4pt,line join=round,line cap=round,fill=fillColor] (444.84,477.46) circle (  1.96);
\path[draw=drawColor,line width= 0.4pt,line join=round,line cap=round,fill=fillColor] (445.05,477.47) circle (  1.96);
\path[draw=drawColor,line width= 0.4pt,line join=round,line cap=round,fill=fillColor] (445.26,477.48) circle (  1.96);
\path[draw=drawColor,line width= 0.4pt,line join=round,line cap=round,fill=fillColor] (445.47,477.48) circle (  1.96);
\path[draw=drawColor,line width= 0.4pt,line join=round,line cap=round,fill=fillColor] (445.68,477.49) circle (  1.96);
\path[draw=drawColor,line width= 0.4pt,line join=round,line cap=round,fill=fillColor] (445.89,477.50) circle (  1.96);
\path[draw=drawColor,line width= 0.4pt,line join=round,line cap=round,fill=fillColor] (446.10,477.50) circle (  1.96);
\path[draw=drawColor,line width= 0.4pt,line join=round,line cap=round,fill=fillColor] (446.31,477.51) circle (  1.96);
\path[draw=drawColor,line width= 0.4pt,line join=round,line cap=round,fill=fillColor] (446.52,477.52) circle (  1.96);
\path[draw=drawColor,line width= 0.4pt,line join=round,line cap=round,fill=fillColor] (446.73,477.52) circle (  1.96);
\path[draw=drawColor,line width= 0.4pt,line join=round,line cap=round,fill=fillColor] (446.94,477.53) circle (  1.96);
\path[draw=drawColor,line width= 0.4pt,line join=round,line cap=round,fill=fillColor] (447.15,477.54) circle (  1.96);
\path[draw=drawColor,line width= 0.4pt,line join=round,line cap=round,fill=fillColor] (447.36,477.54) circle (  1.96);
\path[draw=drawColor,line width= 0.4pt,line join=round,line cap=round,fill=fillColor] (447.57,477.55) circle (  1.96);
\path[draw=drawColor,line width= 0.4pt,line join=round,line cap=round,fill=fillColor] (447.77,477.55) circle (  1.96);
\path[draw=drawColor,line width= 0.4pt,line join=round,line cap=round,fill=fillColor] (447.98,477.56) circle (  1.96);
\path[draw=drawColor,line width= 0.4pt,line join=round,line cap=round,fill=fillColor] (448.19,477.57) circle (  1.96);
\path[draw=drawColor,line width= 0.4pt,line join=round,line cap=round,fill=fillColor] (448.40,477.57) circle (  1.96);
\path[draw=drawColor,line width= 0.4pt,line join=round,line cap=round,fill=fillColor] (448.61,477.58) circle (  1.96);
\path[draw=drawColor,line width= 0.4pt,line join=round,line cap=round,fill=fillColor] (448.82,477.59) circle (  1.96);
\path[draw=drawColor,line width= 0.4pt,line join=round,line cap=round,fill=fillColor] (449.03,477.59) circle (  1.96);
\path[draw=drawColor,line width= 0.4pt,line join=round,line cap=round,fill=fillColor] (449.24,477.60) circle (  1.96);
\path[draw=drawColor,line width= 0.4pt,line join=round,line cap=round,fill=fillColor] (449.45,477.61) circle (  1.96);
\path[draw=drawColor,line width= 0.4pt,line join=round,line cap=round,fill=fillColor] (449.66,477.61) circle (  1.96);
\path[draw=drawColor,line width= 0.4pt,line join=round,line cap=round,fill=fillColor] (449.87,477.62) circle (  1.96);
\path[draw=drawColor,line width= 0.4pt,line join=round,line cap=round,fill=fillColor] (450.08,477.62) circle (  1.96);
\path[draw=drawColor,line width= 0.4pt,line join=round,line cap=round,fill=fillColor] (450.29,477.63) circle (  1.96);
\path[draw=drawColor,line width= 0.4pt,line join=round,line cap=round,fill=fillColor] (450.50,477.64) circle (  1.96);
\path[draw=drawColor,line width= 0.4pt,line join=round,line cap=round,fill=fillColor] (450.71,477.64) circle (  1.96);
\path[draw=drawColor,line width= 0.4pt,line join=round,line cap=round,fill=fillColor] (450.92,477.65) circle (  1.96);
\path[draw=drawColor,line width= 0.4pt,line join=round,line cap=round,fill=fillColor] (451.13,477.65) circle (  1.96);
\path[draw=drawColor,line width= 0.4pt,line join=round,line cap=round,fill=fillColor] (451.34,477.66) circle (  1.96);
\path[draw=drawColor,line width= 0.4pt,line join=round,line cap=round,fill=fillColor] (451.55,477.67) circle (  1.96);
\path[draw=drawColor,line width= 0.4pt,line join=round,line cap=round,fill=fillColor] (451.76,477.67) circle (  1.96);
\path[draw=drawColor,line width= 0.4pt,line join=round,line cap=round,fill=fillColor] (451.97,477.68) circle (  1.96);
\path[draw=drawColor,line width= 0.4pt,line join=round,line cap=round,fill=fillColor] (452.18,477.68) circle (  1.96);
\path[draw=drawColor,line width= 0.4pt,line join=round,line cap=round,fill=fillColor] (452.39,477.69) circle (  1.96);
\path[draw=drawColor,line width= 0.4pt,line join=round,line cap=round,fill=fillColor] (452.60,477.70) circle (  1.96);
\path[draw=drawColor,line width= 0.4pt,line join=round,line cap=round,fill=fillColor] (452.81,477.70) circle (  1.96);
\path[draw=drawColor,line width= 0.4pt,line join=round,line cap=round,fill=fillColor] (453.02,477.71) circle (  1.96);
\path[draw=drawColor,line width= 0.4pt,line join=round,line cap=round,fill=fillColor] (453.23,477.71) circle (  1.96);
\path[draw=drawColor,line width= 0.4pt,line join=round,line cap=round,fill=fillColor] (453.43,477.72) circle (  1.96);
\path[draw=drawColor,line width= 0.4pt,line join=round,line cap=round,fill=fillColor] (453.64,477.72) circle (  1.96);
\path[draw=drawColor,line width= 0.4pt,line join=round,line cap=round,fill=fillColor] (453.85,477.73) circle (  1.96);
\path[draw=drawColor,line width= 0.4pt,line join=round,line cap=round,fill=fillColor] (454.06,477.74) circle (  1.96);
\path[draw=drawColor,line width= 0.4pt,line join=round,line cap=round,fill=fillColor] (454.27,477.74) circle (  1.96);
\path[draw=drawColor,line width= 0.4pt,line join=round,line cap=round,fill=fillColor] (454.48,477.75) circle (  1.96);
\path[draw=drawColor,line width= 0.4pt,line join=round,line cap=round,fill=fillColor] (454.69,477.75) circle (  1.96);
\path[draw=drawColor,line width= 0.4pt,line join=round,line cap=round,fill=fillColor] (454.90,477.76) circle (  1.96);
\path[draw=drawColor,line width= 0.4pt,line join=round,line cap=round,fill=fillColor] (455.11,477.76) circle (  1.96);
\path[draw=drawColor,line width= 0.4pt,line join=round,line cap=round,fill=fillColor] (455.32,477.77) circle (  1.96);
\path[draw=drawColor,line width= 0.4pt,line join=round,line cap=round,fill=fillColor] (455.53,477.78) circle (  1.96);
\path[draw=drawColor,line width= 0.4pt,line join=round,line cap=round,fill=fillColor] (455.74,477.78) circle (  1.96);
\path[draw=drawColor,line width= 0.4pt,line join=round,line cap=round,fill=fillColor] (455.95,477.79) circle (  1.96);
\path[draw=drawColor,line width= 0.4pt,line join=round,line cap=round,fill=fillColor] (456.16,477.79) circle (  1.96);
\path[draw=drawColor,line width= 0.4pt,line join=round,line cap=round,fill=fillColor] (456.37,477.80) circle (  1.96);
\path[draw=drawColor,line width= 0.4pt,line join=round,line cap=round,fill=fillColor] (456.58,477.80) circle (  1.96);
\path[draw=drawColor,line width= 0.4pt,line join=round,line cap=round,fill=fillColor] (456.79,477.81) circle (  1.96);
\path[draw=drawColor,line width= 0.4pt,line join=round,line cap=round,fill=fillColor] (457.00,477.81) circle (  1.96);
\path[draw=drawColor,line width= 0.4pt,line join=round,line cap=round,fill=fillColor] (457.21,477.82) circle (  1.96);
\path[draw=drawColor,line width= 0.4pt,line join=round,line cap=round,fill=fillColor] (457.42,477.83) circle (  1.96);
\path[draw=drawColor,line width= 0.4pt,line join=round,line cap=round,fill=fillColor] (457.63,477.83) circle (  1.96);
\path[draw=drawColor,line width= 0.4pt,line join=round,line cap=round,fill=fillColor] (457.84,477.84) circle (  1.96);
\path[draw=drawColor,line width= 0.4pt,line join=round,line cap=round,fill=fillColor] (458.05,477.84) circle (  1.96);
\path[draw=drawColor,line width= 0.4pt,line join=round,line cap=round,fill=fillColor] (458.26,477.85) circle (  1.96);
\path[draw=drawColor,line width= 0.4pt,line join=round,line cap=round,fill=fillColor] (458.47,477.85) circle (  1.96);
\path[draw=drawColor,line width= 0.4pt,line join=round,line cap=round,fill=fillColor] (458.68,477.86) circle (  1.96);
\path[draw=drawColor,line width= 0.4pt,line join=round,line cap=round,fill=fillColor] (458.88,477.86) circle (  1.96);
\path[draw=drawColor,line width= 0.4pt,line join=round,line cap=round,fill=fillColor] (459.09,477.87) circle (  1.96);
\path[draw=drawColor,line width= 0.4pt,line join=round,line cap=round,fill=fillColor] (459.30,477.87) circle (  1.96);
\path[draw=drawColor,line width= 0.4pt,line join=round,line cap=round,fill=fillColor] (459.51,477.88) circle (  1.96);
\path[draw=drawColor,line width= 0.4pt,line join=round,line cap=round,fill=fillColor] (459.72,477.88) circle (  1.96);
\path[draw=drawColor,line width= 0.4pt,line join=round,line cap=round,fill=fillColor] (459.93,477.89) circle (  1.96);
\path[draw=drawColor,line width= 0.4pt,line join=round,line cap=round,fill=fillColor] (460.14,477.89) circle (  1.96);
\path[draw=drawColor,line width= 0.4pt,line join=round,line cap=round,fill=fillColor] (460.35,477.90) circle (  1.96);
\path[draw=drawColor,line width= 0.4pt,line join=round,line cap=round,fill=fillColor] (460.56,477.90) circle (  1.96);
\path[draw=drawColor,line width= 0.4pt,line join=round,line cap=round,fill=fillColor] (460.77,477.91) circle (  1.96);
\path[draw=drawColor,line width= 0.4pt,line join=round,line cap=round,fill=fillColor] (460.98,477.91) circle (  1.96);
\path[draw=drawColor,line width= 0.4pt,line join=round,line cap=round,fill=fillColor] (461.19,477.92) circle (  1.96);
\path[draw=drawColor,line width= 0.4pt,line join=round,line cap=round,fill=fillColor] (461.40,477.92) circle (  1.96);
\path[draw=drawColor,line width= 0.4pt,line join=round,line cap=round,fill=fillColor] (461.61,477.93) circle (  1.96);
\path[draw=drawColor,line width= 0.4pt,line join=round,line cap=round,fill=fillColor] (461.82,477.93) circle (  1.96);
\path[draw=drawColor,line width= 0.4pt,line join=round,line cap=round,fill=fillColor] (462.03,477.94) circle (  1.96);
\path[draw=drawColor,line width= 0.4pt,line join=round,line cap=round,fill=fillColor] (462.24,477.94) circle (  1.96);
\path[draw=drawColor,line width= 0.4pt,line join=round,line cap=round,fill=fillColor] (462.45,477.95) circle (  1.96);
\path[draw=drawColor,line width= 0.4pt,line join=round,line cap=round,fill=fillColor] (462.66,477.95) circle (  1.96);
\path[draw=drawColor,line width= 0.4pt,line join=round,line cap=round,fill=fillColor] (462.87,477.96) circle (  1.96);
\path[draw=drawColor,line width= 0.4pt,line join=round,line cap=round,fill=fillColor] (463.08,477.96) circle (  1.96);
\path[draw=drawColor,line width= 0.4pt,line join=round,line cap=round,fill=fillColor] (463.29,477.97) circle (  1.96);
\path[draw=drawColor,line width= 0.4pt,line join=round,line cap=round,fill=fillColor] (463.50,477.97) circle (  1.96);
\path[draw=drawColor,line width= 0.4pt,line join=round,line cap=round,fill=fillColor] (463.71,477.98) circle (  1.96);
\path[draw=drawColor,line width= 0.4pt,line join=round,line cap=round,fill=fillColor] (463.92,477.98) circle (  1.96);
\path[draw=drawColor,line width= 0.4pt,line join=round,line cap=round,fill=fillColor] (464.13,477.99) circle (  1.96);
\path[draw=drawColor,line width= 0.4pt,line join=round,line cap=round,fill=fillColor] (464.34,477.99) circle (  1.96);
\path[draw=drawColor,line width= 0.4pt,line join=round,line cap=round,fill=fillColor] (464.54,478.00) circle (  1.96);
\path[draw=drawColor,line width= 0.4pt,line join=round,line cap=round,fill=fillColor] (464.75,478.00) circle (  1.96);
\path[draw=drawColor,line width= 0.4pt,line join=round,line cap=round,fill=fillColor] (464.96,478.01) circle (  1.96);
\path[draw=drawColor,line width= 0.4pt,line join=round,line cap=round,fill=fillColor] (465.17,478.01) circle (  1.96);
\path[draw=drawColor,line width= 0.4pt,line join=round,line cap=round,fill=fillColor] (465.38,478.01) circle (  1.96);
\path[draw=drawColor,line width= 0.4pt,line join=round,line cap=round,fill=fillColor] (465.59,478.02) circle (  1.96);
\path[draw=drawColor,line width= 0.4pt,line join=round,line cap=round,fill=fillColor] (465.80,478.02) circle (  1.96);
\path[draw=drawColor,line width= 0.4pt,line join=round,line cap=round,fill=fillColor] (466.01,478.03) circle (  1.96);
\path[draw=drawColor,line width= 0.4pt,line join=round,line cap=round,fill=fillColor] (466.22,478.03) circle (  1.96);
\path[draw=drawColor,line width= 0.4pt,line join=round,line cap=round,fill=fillColor] (466.43,478.04) circle (  1.96);
\path[draw=drawColor,line width= 0.4pt,line join=round,line cap=round,fill=fillColor] (466.64,478.04) circle (  1.96);
\path[draw=drawColor,line width= 0.4pt,line join=round,line cap=round,fill=fillColor] (466.85,478.05) circle (  1.96);
\path[draw=drawColor,line width= 0.4pt,line join=round,line cap=round,fill=fillColor] (467.06,478.05) circle (  1.96);
\path[draw=drawColor,line width= 0.4pt,line join=round,line cap=round,fill=fillColor] (467.27,478.05) circle (  1.96);
\path[draw=drawColor,line width= 0.4pt,line join=round,line cap=round,fill=fillColor] (467.48,478.06) circle (  1.96);
\path[draw=drawColor,line width= 0.4pt,line join=round,line cap=round,fill=fillColor] (467.69,478.06) circle (  1.96);
\path[draw=drawColor,line width= 0.4pt,line join=round,line cap=round,fill=fillColor] (467.90,478.07) circle (  1.96);
\path[draw=drawColor,line width= 0.4pt,line join=round,line cap=round,fill=fillColor] (468.11,478.07) circle (  1.96);
\path[draw=drawColor,line width= 0.4pt,line join=round,line cap=round,fill=fillColor] (468.32,478.08) circle (  1.96);
\path[draw=drawColor,line width= 0.4pt,line join=round,line cap=round,fill=fillColor] (468.53,478.08) circle (  1.96);
\path[draw=drawColor,line width= 0.4pt,line join=round,line cap=round,fill=fillColor] (468.74,478.09) circle (  1.96);
\path[draw=drawColor,line width= 0.4pt,line join=round,line cap=round,fill=fillColor] (468.95,478.09) circle (  1.96);
\path[draw=drawColor,line width= 0.4pt,line join=round,line cap=round,fill=fillColor] (469.16,478.09) circle (  1.96);
\path[draw=drawColor,line width= 0.4pt,line join=round,line cap=round,fill=fillColor] (469.37,478.10) circle (  1.96);
\path[draw=drawColor,line width= 0.4pt,line join=round,line cap=round,fill=fillColor] (469.58,478.10) circle (  1.96);
\path[draw=drawColor,line width= 0.4pt,line join=round,line cap=round,fill=fillColor] (469.79,478.11) circle (  1.96);
\path[draw=drawColor,line width= 0.4pt,line join=round,line cap=round,fill=fillColor] (469.99,478.11) circle (  1.96);
\path[draw=drawColor,line width= 0.4pt,line join=round,line cap=round,fill=fillColor] (470.20,478.11) circle (  1.96);
\path[draw=drawColor,line width= 0.4pt,line join=round,line cap=round,fill=fillColor] (470.41,478.12) circle (  1.96);
\path[draw=drawColor,line width= 0.4pt,line join=round,line cap=round,fill=fillColor] (470.62,478.12) circle (  1.96);
\path[draw=drawColor,line width= 0.4pt,line join=round,line cap=round,fill=fillColor] (470.83,478.13) circle (  1.96);
\path[draw=drawColor,line width= 0.4pt,line join=round,line cap=round,fill=fillColor] (471.04,478.13) circle (  1.96);
\path[draw=drawColor,line width= 0.4pt,line join=round,line cap=round,fill=fillColor] (471.25,478.14) circle (  1.96);
\path[draw=drawColor,line width= 0.4pt,line join=round,line cap=round,fill=fillColor] (471.46,478.14) circle (  1.96);
\path[draw=drawColor,line width= 0.4pt,line join=round,line cap=round,fill=fillColor] (471.67,478.14) circle (  1.96);
\path[draw=drawColor,line width= 0.4pt,line join=round,line cap=round,fill=fillColor] (471.88,478.15) circle (  1.96);
\path[draw=drawColor,line width= 0.4pt,line join=round,line cap=round,fill=fillColor] (472.09,478.15) circle (  1.96);
\path[draw=drawColor,line width= 0.4pt,line join=round,line cap=round,fill=fillColor] (472.30,478.16) circle (  1.96);
\path[draw=drawColor,line width= 0.4pt,line join=round,line cap=round,fill=fillColor] (472.51,478.16) circle (  1.96);
\path[draw=drawColor,line width= 0.4pt,line join=round,line cap=round,fill=fillColor] (472.72,478.16) circle (  1.96);
\path[draw=drawColor,line width= 0.4pt,line join=round,line cap=round,fill=fillColor] (472.93,478.17) circle (  1.96);
\path[draw=drawColor,line width= 0.4pt,line join=round,line cap=round,fill=fillColor] (473.14,478.17) circle (  1.96);
\path[draw=drawColor,line width= 0.4pt,line join=round,line cap=round,fill=fillColor] (473.35,478.18) circle (  1.96);
\path[draw=drawColor,line width= 0.4pt,line join=round,line cap=round,fill=fillColor] (473.56,478.18) circle (  1.96);
\path[draw=drawColor,line width= 0.4pt,line join=round,line cap=round,fill=fillColor] (473.77,478.18) circle (  1.96);
\path[draw=drawColor,line width= 0.4pt,line join=round,line cap=round,fill=fillColor] (473.98,478.19) circle (  1.96);
\path[draw=drawColor,line width= 0.4pt,line join=round,line cap=round,fill=fillColor] (474.19,478.19) circle (  1.96);
\path[draw=drawColor,line width= 0.4pt,line join=round,line cap=round,fill=fillColor] (474.40,478.20) circle (  1.96);
\path[draw=drawColor,line width= 0.4pt,line join=round,line cap=round,fill=fillColor] (474.61,478.20) circle (  1.96);
\path[draw=drawColor,line width= 0.4pt,line join=round,line cap=round,fill=fillColor] (474.82,478.20) circle (  1.96);
\path[draw=drawColor,line width= 0.4pt,line join=round,line cap=round,fill=fillColor] (475.03,478.21) circle (  1.96);
\path[draw=drawColor,line width= 0.4pt,line join=round,line cap=round,fill=fillColor] (475.24,478.21) circle (  1.96);
\path[draw=drawColor,line width= 0.4pt,line join=round,line cap=round,fill=fillColor] (475.45,478.21) circle (  1.96);
\path[draw=drawColor,line width= 0.4pt,line join=round,line cap=round,fill=fillColor] (475.65,478.22) circle (  1.96);
\path[draw=drawColor,line width= 0.4pt,line join=round,line cap=round,fill=fillColor] (475.86,478.22) circle (  1.96);
\path[draw=drawColor,line width= 0.4pt,line join=round,line cap=round,fill=fillColor] (476.07,478.23) circle (  1.96);
\path[draw=drawColor,line width= 0.4pt,line join=round,line cap=round,fill=fillColor] (476.28,478.23) circle (  1.96);
\path[draw=drawColor,line width= 0.4pt,line join=round,line cap=round,fill=fillColor] (476.49,478.23) circle (  1.96);
\path[draw=drawColor,line width= 0.4pt,line join=round,line cap=round,fill=fillColor] (476.70,478.24) circle (  1.96);
\path[draw=drawColor,line width= 0.4pt,line join=round,line cap=round,fill=fillColor] (476.91,478.24) circle (  1.96);
\path[draw=drawColor,line width= 0.4pt,line join=round,line cap=round,fill=fillColor] (477.12,478.24) circle (  1.96);
\path[draw=drawColor,line width= 0.4pt,line join=round,line cap=round,fill=fillColor] (477.33,478.25) circle (  1.96);
\end{scope}
\begin{scope}
\path[clip] (  0.00,  0.00) rectangle (505.89,505.89);
\definecolor{drawColor}{gray}{0.30}
\node[text=drawColor,anchor=base east,inner sep=0pt, outer sep=0pt, scale=  0.88] at ( 34.27, 50.34) {50};
\node[text=drawColor,anchor=base east,inner sep=0pt, outer sep=0pt, scale=  0.88] at ( 34.27,135.49) {60};
\node[text=drawColor,anchor=base east,inner sep=0pt, outer sep=0pt, scale=  0.88] at ( 34.27,220.63) {70};
\node[text=drawColor,anchor=base east,inner sep=0pt, outer sep=0pt, scale=  0.88] at ( 34.27,305.78) {80};
\node[text=drawColor,anchor=base east,inner sep=0pt, outer sep=0pt, scale=  0.88] at ( 34.27,390.93) {90};
\node[text=drawColor,anchor=base east,inner sep=0pt, outer sep=0pt, scale=  0.88] at ( 34.27,476.07) {100};
\end{scope}
\begin{scope}
\path[clip] (  0.00,  0.00) rectangle (505.89,505.89);
\definecolor{drawColor}{gray}{0.20}
\path[draw=drawColor,line width= 0.6pt,line join=round] ( 36.47, 53.37) --
	( 39.22, 53.37);
\path[draw=drawColor,line width= 0.6pt,line join=round] ( 36.47,138.52) --
	( 39.22,138.52);
\path[draw=drawColor,line width= 0.6pt,line join=round] ( 36.47,223.66) --
	( 39.22,223.66);
\path[draw=drawColor,line width= 0.6pt,line join=round] ( 36.47,308.81) --
	( 39.22,308.81);
\path[draw=drawColor,line width= 0.6pt,line join=round] ( 36.47,393.96) --
	( 39.22,393.96);
\path[draw=drawColor,line width= 0.6pt,line join=round] ( 36.47,479.10) --
	( 39.22,479.10);
\end{scope}
\begin{scope}
\path[clip] (  0.00,  0.00) rectangle (505.89,505.89);
\definecolor{drawColor}{gray}{0.20}
\path[draw=drawColor,line width= 0.6pt,line join=round] ( 60.18, 29.34) --
	( 60.18, 32.09);
\path[draw=drawColor,line width= 0.6pt,line join=round] (144.03, 29.34) --
	(144.03, 32.09);
\path[draw=drawColor,line width= 0.6pt,line join=round] (227.88, 29.34) --
	(227.88, 32.09);
\path[draw=drawColor,line width= 0.6pt,line join=round] (311.73, 29.34) --
	(311.73, 32.09);
\path[draw=drawColor,line width= 0.6pt,line join=round] (395.58, 29.34) --
	(395.58, 32.09);
\path[draw=drawColor,line width= 0.6pt,line join=round] (479.43, 29.34) --
	(479.43, 32.09);
\end{scope}
\begin{scope}
\path[clip] (  0.00,  0.00) rectangle (505.89,505.89);
\definecolor{drawColor}{gray}{0.30}
\node[text=drawColor,anchor=base,inner sep=0pt, outer sep=0pt, scale=  0.88] at ( 60.18, 21.08) {2000};
\node[text=drawColor,anchor=base,inner sep=0pt, outer sep=0pt, scale=  0.88] at (144.03, 21.08) {2400};
\node[text=drawColor,anchor=base,inner sep=0pt, outer sep=0pt, scale=  0.88] at (227.88, 21.08) {2800};
\node[text=drawColor,anchor=base,inner sep=0pt, outer sep=0pt, scale=  0.88] at (311.73, 21.08) {3200};
\node[text=drawColor,anchor=base,inner sep=0pt, outer sep=0pt, scale=  0.88] at (395.58, 21.08) {3600};
\node[text=drawColor,anchor=base,inner sep=0pt, outer sep=0pt, scale=  0.88] at (479.43, 21.08) {4000};
\end{scope}
\begin{scope}
\path[clip] (  0.00,  0.00) rectangle (505.89,505.89);
\definecolor{drawColor}{RGB}{0,0,0}
\node[text=drawColor,anchor=base,inner sep=0pt, outer sep=0pt, scale=  1.10] at (269.81,  8.00) {Anzahl an Adressen im Pool (x)};
\end{scope}
\begin{scope}
\path[clip] (  0.00,  0.00) rectangle (505.89,505.89);
\definecolor{drawColor}{RGB}{0,0,0}
\node[text=drawColor,rotate= 90.00,anchor=base,inner sep=0pt, outer sep=0pt, scale=  1.10] at ( 13.08,266.24) {Wahrscheinlichkeit ein fixes Eviction Sets zu finden (\%)};
\end{scope}
\end{tikzpicture}

\caption{Beschreibt den Zusammenhang zwischen der Poolgröße und der Wahrscheinlichkeit ein Eviction Set zu einer vorher festgelegten Adresse zu finden}
\end{scaletikzpicturetowidth}
\end{center}
Im zweiten Szenario möchte die Angreiferin in einem komplexen Angriff eine Vielzahl von Adressen in unterschiedlichen Cache-Sets überwachen. 
Sie interessiert sich nun dafür, wie wahrscheinlich es ist alle 8192 Cache-Sets zu finden und somit auch die für sie relevanten. 

Die Wahrscheinlichkeit für alle 128 möglichen Cache-Sets jeweils 16 Zuordnungen und damit gleichbedeutend alle 8192 möglichen Eviction Sets konstruieren zu können, ist mit 
\begin{align*}
P(count(cs)>=16|\#add = x)^{128} = (1-P(count(cs)<16|\#add = x))^{128}
\end{align*}
gegeben.  

Das Diagramm \ref{fig:prob_all_es} gibt die Erfolgswahrscheinlichkeiten für verschiedene $x$-Werte im zweiten Szenario an.
\label{fig:prob_all_es}
\begin{center}
\begin{scaletikzpicturetowidth}{\textwidth}
% Created by tikzDevice version 0.12 on 2018-09-02 02:53:13
% !TEX encoding = UTF-8 Unicode
\begin{tikzpicture}[scale=\tikzscale,y=20pt]
\definecolor{fillColor}{RGB}{255,255,255}
\path[use as bounding box,fill=fillColor,fill opacity=0.00] (0,0) rectangle (505.89,505.89);
\begin{scope}
\path[clip] (  0.00,  0.00) rectangle (505.89,505.89);
\definecolor{drawColor}{RGB}{255,255,255}
\definecolor{fillColor}{RGB}{255,255,255}
\path[draw=drawColor,line width= 0.6pt,line join=round,line cap=round,fill=fillColor] (  0.00,  0.00) rectangle (505.89,505.89);
\end{scope}
\begin{scope}
\path[clip] ( 39.22, 32.09) rectangle (500.39,500.39);
\definecolor{fillColor}{gray}{0.92}
\path[fill=fillColor] ( 39.22, 32.09) rectangle (500.39,500.39);
\definecolor{drawColor}{RGB}{255,255,255}
\path[draw=drawColor,line width= 0.3pt,line join=round] ( 39.22, 95.95) --
	(500.39, 95.95);
\path[draw=drawColor,line width= 0.3pt,line join=round] ( 39.22,181.09) --
	(500.39,181.09);
\path[draw=drawColor,line width= 0.3pt,line join=round] ( 39.22,266.24) --
	(500.39,266.24);
\path[draw=drawColor,line width= 0.3pt,line join=round] ( 39.22,351.38) --
	(500.39,351.38);
\path[draw=drawColor,line width= 0.3pt,line join=round] ( 39.22,436.53) --
	(500.39,436.53);
\path[draw=drawColor,line width= 0.3pt,line join=round] (102.11, 32.09) --
	(102.11,500.39);
\path[draw=drawColor,line width= 0.3pt,line join=round] (185.96, 32.09) --
	(185.96,500.39);
\path[draw=drawColor,line width= 0.3pt,line join=round] (269.81, 32.09) --
	(269.81,500.39);
\path[draw=drawColor,line width= 0.3pt,line join=round] (353.65, 32.09) --
	(353.65,500.39);
\path[draw=drawColor,line width= 0.3pt,line join=round] (437.50, 32.09) --
	(437.50,500.39);
\path[draw=drawColor,line width= 0.6pt,line join=round] ( 39.22, 53.37) --
	(500.39, 53.37);
\path[draw=drawColor,line width= 0.6pt,line join=round] ( 39.22,138.52) --
	(500.39,138.52);
\path[draw=drawColor,line width= 0.6pt,line join=round] ( 39.22,223.66) --
	(500.39,223.66);
\path[draw=drawColor,line width= 0.6pt,line join=round] ( 39.22,308.81) --
	(500.39,308.81);
\path[draw=drawColor,line width= 0.6pt,line join=round] ( 39.22,393.96) --
	(500.39,393.96);
\path[draw=drawColor,line width= 0.6pt,line join=round] ( 39.22,479.10) --
	(500.39,479.10);
\path[draw=drawColor,line width= 0.6pt,line join=round] ( 60.18, 32.09) --
	( 60.18,500.39);
\path[draw=drawColor,line width= 0.6pt,line join=round] (144.03, 32.09) --
	(144.03,500.39);
\path[draw=drawColor,line width= 0.6pt,line join=round] (227.88, 32.09) --
	(227.88,500.39);
\path[draw=drawColor,line width= 0.6pt,line join=round] (311.73, 32.09) --
	(311.73,500.39);
\path[draw=drawColor,line width= 0.6pt,line join=round] (395.58, 32.09) --
	(395.58,500.39);
\path[draw=drawColor,line width= 0.6pt,line join=round] (479.43, 32.09) --
	(479.43,500.39);
\definecolor{drawColor}{RGB}{0,0,0}
\definecolor{fillColor}{RGB}{0,0,0}
\path[draw=drawColor,line width= 0.4pt,line join=round,line cap=round,fill=fillColor] ( 75.70, 54.20) circle (  1.96);
\path[draw=drawColor,line width= 0.4pt,line join=round,line cap=round,fill=fillColor] ( 76.12, 55.94) circle (  1.96);
\path[draw=drawColor,line width= 0.4pt,line join=round,line cap=round,fill=fillColor] ( 76.53, 57.66) circle (  1.96);
\path[draw=drawColor,line width= 0.4pt,line join=round,line cap=round,fill=fillColor] ( 76.95, 59.38) circle (  1.96);
\path[draw=drawColor,line width= 0.4pt,line join=round,line cap=round,fill=fillColor] ( 77.37, 61.10) circle (  1.96);
\path[draw=drawColor,line width= 0.4pt,line join=round,line cap=round,fill=fillColor] ( 77.79, 62.81) circle (  1.96);
\path[draw=drawColor,line width= 0.4pt,line join=round,line cap=round,fill=fillColor] ( 78.21, 64.51) circle (  1.96);
\path[draw=drawColor,line width= 0.4pt,line join=round,line cap=round,fill=fillColor] ( 78.63, 66.20) circle (  1.96);
\path[draw=drawColor,line width= 0.4pt,line join=round,line cap=round,fill=fillColor] ( 79.05, 67.89) circle (  1.96);
\path[draw=drawColor,line width= 0.4pt,line join=round,line cap=round,fill=fillColor] ( 79.47, 69.58) circle (  1.96);
\path[draw=drawColor,line width= 0.4pt,line join=round,line cap=round,fill=fillColor] ( 79.89, 71.25) circle (  1.96);
\path[draw=drawColor,line width= 0.4pt,line join=round,line cap=round,fill=fillColor] ( 80.31, 72.92) circle (  1.96);
\path[draw=drawColor,line width= 0.4pt,line join=round,line cap=round,fill=fillColor] ( 80.73, 74.59) circle (  1.96);
\path[draw=drawColor,line width= 0.4pt,line join=round,line cap=round,fill=fillColor] ( 81.15, 76.25) circle (  1.96);
\path[draw=drawColor,line width= 0.4pt,line join=round,line cap=round,fill=fillColor] ( 81.57, 77.90) circle (  1.96);
\path[draw=drawColor,line width= 0.4pt,line join=round,line cap=round,fill=fillColor] ( 81.99, 79.54) circle (  1.96);
\path[draw=drawColor,line width= 0.4pt,line join=round,line cap=round,fill=fillColor] ( 82.40, 81.19) circle (  1.96);
\path[draw=drawColor,line width= 0.4pt,line join=round,line cap=round,fill=fillColor] ( 82.82, 82.82) circle (  1.96);
\path[draw=drawColor,line width= 0.4pt,line join=round,line cap=round,fill=fillColor] ( 83.24, 84.45) circle (  1.96);
\path[draw=drawColor,line width= 0.4pt,line join=round,line cap=round,fill=fillColor] ( 83.66, 86.07) circle (  1.96);
\path[draw=drawColor,line width= 0.4pt,line join=round,line cap=round,fill=fillColor] ( 84.08, 87.68) circle (  1.96);
\path[draw=drawColor,line width= 0.4pt,line join=round,line cap=round,fill=fillColor] ( 84.50, 89.29) circle (  1.96);
\path[draw=drawColor,line width= 0.4pt,line join=round,line cap=round,fill=fillColor] ( 84.92, 90.90) circle (  1.96);
\path[draw=drawColor,line width= 0.4pt,line join=round,line cap=round,fill=fillColor] ( 85.34, 92.49) circle (  1.96);
\path[draw=drawColor,line width= 0.4pt,line join=round,line cap=round,fill=fillColor] ( 85.76, 94.09) circle (  1.96);
\path[draw=drawColor,line width= 0.4pt,line join=round,line cap=round,fill=fillColor] ( 86.18, 95.67) circle (  1.96);
\path[draw=drawColor,line width= 0.4pt,line join=round,line cap=round,fill=fillColor] ( 86.60, 97.25) circle (  1.96);
\path[draw=drawColor,line width= 0.4pt,line join=round,line cap=round,fill=fillColor] ( 87.02, 98.83) circle (  1.96);
\path[draw=drawColor,line width= 0.4pt,line join=round,line cap=round,fill=fillColor] ( 87.44,100.39) circle (  1.96);
\path[draw=drawColor,line width= 0.4pt,line join=round,line cap=round,fill=fillColor] ( 87.85,101.96) circle (  1.96);
\path[draw=drawColor,line width= 0.4pt,line join=round,line cap=round,fill=fillColor] ( 88.27,103.52) circle (  1.96);
\path[draw=drawColor,line width= 0.4pt,line join=round,line cap=round,fill=fillColor] ( 88.69,105.06) circle (  1.96);
\path[draw=drawColor,line width= 0.4pt,line join=round,line cap=round,fill=fillColor] ( 89.11,106.61) circle (  1.96);
\path[draw=drawColor,line width= 0.4pt,line join=round,line cap=round,fill=fillColor] ( 89.53,108.15) circle (  1.96);
\path[draw=drawColor,line width= 0.4pt,line join=round,line cap=round,fill=fillColor] ( 89.95,109.68) circle (  1.96);
\path[draw=drawColor,line width= 0.4pt,line join=round,line cap=round,fill=fillColor] ( 90.37,111.21) circle (  1.96);
\path[draw=drawColor,line width= 0.4pt,line join=round,line cap=round,fill=fillColor] ( 90.79,112.73) circle (  1.96);
\path[draw=drawColor,line width= 0.4pt,line join=round,line cap=round,fill=fillColor] ( 91.21,114.25) circle (  1.96);
\path[draw=drawColor,line width= 0.4pt,line join=round,line cap=round,fill=fillColor] ( 91.63,115.76) circle (  1.96);
\path[draw=drawColor,line width= 0.4pt,line join=round,line cap=round,fill=fillColor] ( 92.05,117.26) circle (  1.96);
\path[draw=drawColor,line width= 0.4pt,line join=round,line cap=round,fill=fillColor] ( 92.47,118.76) circle (  1.96);
\path[draw=drawColor,line width= 0.4pt,line join=round,line cap=round,fill=fillColor] ( 92.89,120.25) circle (  1.96);
\path[draw=drawColor,line width= 0.4pt,line join=round,line cap=round,fill=fillColor] ( 93.30,121.75) circle (  1.96);
\path[draw=drawColor,line width= 0.4pt,line join=round,line cap=round,fill=fillColor] ( 93.72,123.23) circle (  1.96);
\path[draw=drawColor,line width= 0.4pt,line join=round,line cap=round,fill=fillColor] ( 94.14,124.70) circle (  1.96);
\path[draw=drawColor,line width= 0.4pt,line join=round,line cap=round,fill=fillColor] ( 94.56,126.17) circle (  1.96);
\path[draw=drawColor,line width= 0.4pt,line join=round,line cap=round,fill=fillColor] ( 94.98,127.64) circle (  1.96);
\path[draw=drawColor,line width= 0.4pt,line join=round,line cap=round,fill=fillColor] ( 95.40,129.10) circle (  1.96);
\path[draw=drawColor,line width= 0.4pt,line join=round,line cap=round,fill=fillColor] ( 95.82,130.55) circle (  1.96);
\path[draw=drawColor,line width= 0.4pt,line join=round,line cap=round,fill=fillColor] ( 96.24,132.00) circle (  1.96);
\path[draw=drawColor,line width= 0.4pt,line join=round,line cap=round,fill=fillColor] ( 96.66,133.44) circle (  1.96);
\path[draw=drawColor,line width= 0.4pt,line join=round,line cap=round,fill=fillColor] ( 97.08,134.88) circle (  1.96);
\path[draw=drawColor,line width= 0.4pt,line join=round,line cap=round,fill=fillColor] ( 97.50,136.32) circle (  1.96);
\path[draw=drawColor,line width= 0.4pt,line join=round,line cap=round,fill=fillColor] ( 97.92,137.74) circle (  1.96);
\path[draw=drawColor,line width= 0.4pt,line join=round,line cap=round,fill=fillColor] ( 98.34,139.17) circle (  1.96);
\path[draw=drawColor,line width= 0.4pt,line join=round,line cap=round,fill=fillColor] ( 98.75,140.58) circle (  1.96);
\path[draw=drawColor,line width= 0.4pt,line join=round,line cap=round,fill=fillColor] ( 99.17,141.99) circle (  1.96);
\path[draw=drawColor,line width= 0.4pt,line join=round,line cap=round,fill=fillColor] ( 99.59,143.40) circle (  1.96);
\path[draw=drawColor,line width= 0.4pt,line join=round,line cap=round,fill=fillColor] (100.01,144.80) circle (  1.96);
\path[draw=drawColor,line width= 0.4pt,line join=round,line cap=round,fill=fillColor] (100.43,146.19) circle (  1.96);
\path[draw=drawColor,line width= 0.4pt,line join=round,line cap=round,fill=fillColor] (100.85,147.58) circle (  1.96);
\path[draw=drawColor,line width= 0.4pt,line join=round,line cap=round,fill=fillColor] (101.27,148.97) circle (  1.96);
\path[draw=drawColor,line width= 0.4pt,line join=round,line cap=round,fill=fillColor] (101.69,150.35) circle (  1.96);
\path[draw=drawColor,line width= 0.4pt,line join=round,line cap=round,fill=fillColor] (102.11,151.72) circle (  1.96);
\path[draw=drawColor,line width= 0.4pt,line join=round,line cap=round,fill=fillColor] (102.53,153.09) circle (  1.96);
\path[draw=drawColor,line width= 0.4pt,line join=round,line cap=round,fill=fillColor] (102.95,154.45) circle (  1.96);
\path[draw=drawColor,line width= 0.4pt,line join=round,line cap=round,fill=fillColor] (103.37,155.81) circle (  1.96);
\path[draw=drawColor,line width= 0.4pt,line join=round,line cap=round,fill=fillColor] (103.79,157.16) circle (  1.96);
\path[draw=drawColor,line width= 0.4pt,line join=round,line cap=round,fill=fillColor] (104.20,158.51) circle (  1.96);
\path[draw=drawColor,line width= 0.4pt,line join=round,line cap=round,fill=fillColor] (104.62,159.85) circle (  1.96);
\path[draw=drawColor,line width= 0.4pt,line join=round,line cap=round,fill=fillColor] (105.04,161.18) circle (  1.96);
\path[draw=drawColor,line width= 0.4pt,line join=round,line cap=round,fill=fillColor] (105.46,162.52) circle (  1.96);
\path[draw=drawColor,line width= 0.4pt,line join=round,line cap=round,fill=fillColor] (105.88,163.85) circle (  1.96);
\path[draw=drawColor,line width= 0.4pt,line join=round,line cap=round,fill=fillColor] (106.30,165.17) circle (  1.96);
\path[draw=drawColor,line width= 0.4pt,line join=round,line cap=round,fill=fillColor] (106.72,166.49) circle (  1.96);
\path[draw=drawColor,line width= 0.4pt,line join=round,line cap=round,fill=fillColor] (107.14,167.80) circle (  1.96);
\path[draw=drawColor,line width= 0.4pt,line join=round,line cap=round,fill=fillColor] (107.56,169.10) circle (  1.96);
\path[draw=drawColor,line width= 0.4pt,line join=round,line cap=round,fill=fillColor] (107.98,170.40) circle (  1.96);
\path[draw=drawColor,line width= 0.4pt,line join=round,line cap=round,fill=fillColor] (108.40,171.70) circle (  1.96);
\path[draw=drawColor,line width= 0.4pt,line join=round,line cap=round,fill=fillColor] (108.82,172.99) circle (  1.96);
\path[draw=drawColor,line width= 0.4pt,line join=round,line cap=round,fill=fillColor] (109.24,174.28) circle (  1.96);
\path[draw=drawColor,line width= 0.4pt,line join=round,line cap=round,fill=fillColor] (109.66,175.56) circle (  1.96);
\path[draw=drawColor,line width= 0.4pt,line join=round,line cap=round,fill=fillColor] (110.07,176.83) circle (  1.96);
\path[draw=drawColor,line width= 0.4pt,line join=round,line cap=round,fill=fillColor] (110.49,178.11) circle (  1.96);
\path[draw=drawColor,line width= 0.4pt,line join=round,line cap=round,fill=fillColor] (110.91,179.38) circle (  1.96);
\path[draw=drawColor,line width= 0.4pt,line join=round,line cap=round,fill=fillColor] (111.33,180.64) circle (  1.96);
\path[draw=drawColor,line width= 0.4pt,line join=round,line cap=round,fill=fillColor] (111.75,181.89) circle (  1.96);
\path[draw=drawColor,line width= 0.4pt,line join=round,line cap=round,fill=fillColor] (112.17,183.14) circle (  1.96);
\path[draw=drawColor,line width= 0.4pt,line join=round,line cap=round,fill=fillColor] (112.59,184.39) circle (  1.96);
\path[draw=drawColor,line width= 0.4pt,line join=round,line cap=round,fill=fillColor] (113.01,185.63) circle (  1.96);
\path[draw=drawColor,line width= 0.4pt,line join=round,line cap=round,fill=fillColor] (113.43,186.87) circle (  1.96);
\path[draw=drawColor,line width= 0.4pt,line join=round,line cap=round,fill=fillColor] (113.85,188.10) circle (  1.96);
\path[draw=drawColor,line width= 0.4pt,line join=round,line cap=round,fill=fillColor] (114.27,189.33) circle (  1.96);
\path[draw=drawColor,line width= 0.4pt,line join=round,line cap=round,fill=fillColor] (114.69,190.56) circle (  1.96);
\path[draw=drawColor,line width= 0.4pt,line join=round,line cap=round,fill=fillColor] (115.11,191.77) circle (  1.96);
\path[draw=drawColor,line width= 0.4pt,line join=round,line cap=round,fill=fillColor] (115.52,192.99) circle (  1.96);
\path[draw=drawColor,line width= 0.4pt,line join=round,line cap=round,fill=fillColor] (115.94,194.20) circle (  1.96);
\path[draw=drawColor,line width= 0.4pt,line join=round,line cap=round,fill=fillColor] (116.36,195.40) circle (  1.96);
\path[draw=drawColor,line width= 0.4pt,line join=round,line cap=round,fill=fillColor] (116.78,196.60) circle (  1.96);
\path[draw=drawColor,line width= 0.4pt,line join=round,line cap=round,fill=fillColor] (117.20,197.80) circle (  1.96);
\path[draw=drawColor,line width= 0.4pt,line join=round,line cap=round,fill=fillColor] (117.62,198.99) circle (  1.96);
\path[draw=drawColor,line width= 0.4pt,line join=round,line cap=round,fill=fillColor] (118.04,200.17) circle (  1.96);
\path[draw=drawColor,line width= 0.4pt,line join=round,line cap=round,fill=fillColor] (118.46,201.35) circle (  1.96);
\path[draw=drawColor,line width= 0.4pt,line join=round,line cap=round,fill=fillColor] (118.88,202.53) circle (  1.96);
\path[draw=drawColor,line width= 0.4pt,line join=round,line cap=round,fill=fillColor] (119.30,203.70) circle (  1.96);
\path[draw=drawColor,line width= 0.4pt,line join=round,line cap=round,fill=fillColor] (119.72,204.87) circle (  1.96);
\path[draw=drawColor,line width= 0.4pt,line join=round,line cap=round,fill=fillColor] (120.14,206.03) circle (  1.96);
\path[draw=drawColor,line width= 0.4pt,line join=round,line cap=round,fill=fillColor] (120.56,207.19) circle (  1.96);
\path[draw=drawColor,line width= 0.4pt,line join=round,line cap=round,fill=fillColor] (120.97,208.34) circle (  1.96);
\path[draw=drawColor,line width= 0.4pt,line join=round,line cap=round,fill=fillColor] (121.39,209.49) circle (  1.96);
\path[draw=drawColor,line width= 0.4pt,line join=round,line cap=round,fill=fillColor] (121.81,210.63) circle (  1.96);
\path[draw=drawColor,line width= 0.4pt,line join=round,line cap=round,fill=fillColor] (122.23,211.77) circle (  1.96);
\path[draw=drawColor,line width= 0.4pt,line join=round,line cap=round,fill=fillColor] (122.65,212.91) circle (  1.96);
\path[draw=drawColor,line width= 0.4pt,line join=round,line cap=round,fill=fillColor] (123.07,214.04) circle (  1.96);
\path[draw=drawColor,line width= 0.4pt,line join=round,line cap=round,fill=fillColor] (123.49,215.17) circle (  1.96);
\path[draw=drawColor,line width= 0.4pt,line join=round,line cap=round,fill=fillColor] (123.91,216.29) circle (  1.96);
\path[draw=drawColor,line width= 0.4pt,line join=round,line cap=round,fill=fillColor] (124.33,217.41) circle (  1.96);
\path[draw=drawColor,line width= 0.4pt,line join=round,line cap=round,fill=fillColor] (124.75,218.52) circle (  1.96);
\path[draw=drawColor,line width= 0.4pt,line join=round,line cap=round,fill=fillColor] (125.17,219.63) circle (  1.96);
\path[draw=drawColor,line width= 0.4pt,line join=round,line cap=round,fill=fillColor] (125.59,220.74) circle (  1.96);
\path[draw=drawColor,line width= 0.4pt,line join=round,line cap=round,fill=fillColor] (126.01,221.83) circle (  1.96);
\path[draw=drawColor,line width= 0.4pt,line join=round,line cap=round,fill=fillColor] (126.42,222.93) circle (  1.96);
\path[draw=drawColor,line width= 0.4pt,line join=round,line cap=round,fill=fillColor] (126.84,224.02) circle (  1.96);
\path[draw=drawColor,line width= 0.4pt,line join=round,line cap=round,fill=fillColor] (127.26,225.11) circle (  1.96);
\path[draw=drawColor,line width= 0.4pt,line join=round,line cap=round,fill=fillColor] (127.68,226.19) circle (  1.96);
\path[draw=drawColor,line width= 0.4pt,line join=round,line cap=round,fill=fillColor] (128.10,227.27) circle (  1.96);
\path[draw=drawColor,line width= 0.4pt,line join=round,line cap=round,fill=fillColor] (128.52,228.34) circle (  1.96);
\path[draw=drawColor,line width= 0.4pt,line join=round,line cap=round,fill=fillColor] (128.94,229.41) circle (  1.96);
\path[draw=drawColor,line width= 0.4pt,line join=round,line cap=round,fill=fillColor] (129.36,230.48) circle (  1.96);
\path[draw=drawColor,line width= 0.4pt,line join=round,line cap=round,fill=fillColor] (129.78,231.54) circle (  1.96);
\path[draw=drawColor,line width= 0.4pt,line join=round,line cap=round,fill=fillColor] (130.20,232.60) circle (  1.96);
\path[draw=drawColor,line width= 0.4pt,line join=round,line cap=round,fill=fillColor] (130.62,233.65) circle (  1.96);
\path[draw=drawColor,line width= 0.4pt,line join=round,line cap=round,fill=fillColor] (131.04,234.70) circle (  1.96);
\path[draw=drawColor,line width= 0.4pt,line join=round,line cap=round,fill=fillColor] (131.46,235.75) circle (  1.96);
\path[draw=drawColor,line width= 0.4pt,line join=round,line cap=round,fill=fillColor] (131.87,236.79) circle (  1.96);
\path[draw=drawColor,line width= 0.4pt,line join=round,line cap=round,fill=fillColor] (132.29,237.82) circle (  1.96);
\path[draw=drawColor,line width= 0.4pt,line join=round,line cap=round,fill=fillColor] (132.71,238.86) circle (  1.96);
\path[draw=drawColor,line width= 0.4pt,line join=round,line cap=round,fill=fillColor] (133.13,239.89) circle (  1.96);
\path[draw=drawColor,line width= 0.4pt,line join=round,line cap=round,fill=fillColor] (133.55,240.91) circle (  1.96);
\path[draw=drawColor,line width= 0.4pt,line join=round,line cap=round,fill=fillColor] (133.97,241.93) circle (  1.96);
\path[draw=drawColor,line width= 0.4pt,line join=round,line cap=round,fill=fillColor] (134.39,242.94) circle (  1.96);
\path[draw=drawColor,line width= 0.4pt,line join=round,line cap=round,fill=fillColor] (134.81,243.96) circle (  1.96);
\path[draw=drawColor,line width= 0.4pt,line join=round,line cap=round,fill=fillColor] (135.23,244.96) circle (  1.96);
\path[draw=drawColor,line width= 0.4pt,line join=round,line cap=round,fill=fillColor] (135.65,245.97) circle (  1.96);
\path[draw=drawColor,line width= 0.4pt,line join=round,line cap=round,fill=fillColor] (136.07,246.97) circle (  1.96);
\path[draw=drawColor,line width= 0.4pt,line join=round,line cap=round,fill=fillColor] (136.49,247.97) circle (  1.96);
\path[draw=drawColor,line width= 0.4pt,line join=round,line cap=round,fill=fillColor] (136.91,248.96) circle (  1.96);
\path[draw=drawColor,line width= 0.4pt,line join=round,line cap=round,fill=fillColor] (137.33,249.95) circle (  1.96);
\path[draw=drawColor,line width= 0.4pt,line join=round,line cap=round,fill=fillColor] (137.74,250.93) circle (  1.96);
\path[draw=drawColor,line width= 0.4pt,line join=round,line cap=round,fill=fillColor] (138.16,251.91) circle (  1.96);
\path[draw=drawColor,line width= 0.4pt,line join=round,line cap=round,fill=fillColor] (138.58,252.88) circle (  1.96);
\path[draw=drawColor,line width= 0.4pt,line join=round,line cap=round,fill=fillColor] (139.00,253.86) circle (  1.96);
\path[draw=drawColor,line width= 0.4pt,line join=round,line cap=round,fill=fillColor] (139.42,254.82) circle (  1.96);
\path[draw=drawColor,line width= 0.4pt,line join=round,line cap=round,fill=fillColor] (139.84,255.79) circle (  1.96);
\path[draw=drawColor,line width= 0.4pt,line join=round,line cap=round,fill=fillColor] (140.26,256.75) circle (  1.96);
\path[draw=drawColor,line width= 0.4pt,line join=round,line cap=round,fill=fillColor] (140.68,257.71) circle (  1.96);
\path[draw=drawColor,line width= 0.4pt,line join=round,line cap=round,fill=fillColor] (141.10,258.66) circle (  1.96);
\path[draw=drawColor,line width= 0.4pt,line join=round,line cap=round,fill=fillColor] (141.52,259.61) circle (  1.96);
\path[draw=drawColor,line width= 0.4pt,line join=round,line cap=round,fill=fillColor] (141.94,260.55) circle (  1.96);
\path[draw=drawColor,line width= 0.4pt,line join=round,line cap=round,fill=fillColor] (142.36,261.50) circle (  1.96);
\path[draw=drawColor,line width= 0.4pt,line join=round,line cap=round,fill=fillColor] (142.78,262.43) circle (  1.96);
\path[draw=drawColor,line width= 0.4pt,line join=round,line cap=round,fill=fillColor] (143.19,263.37) circle (  1.96);
\path[draw=drawColor,line width= 0.4pt,line join=round,line cap=round,fill=fillColor] (143.61,264.30) circle (  1.96);
\path[draw=drawColor,line width= 0.4pt,line join=round,line cap=round,fill=fillColor] (144.03,265.22) circle (  1.96);
\path[draw=drawColor,line width= 0.4pt,line join=round,line cap=round,fill=fillColor] (144.45,266.14) circle (  1.96);
\path[draw=drawColor,line width= 0.4pt,line join=round,line cap=round,fill=fillColor] (144.87,267.06) circle (  1.96);
\path[draw=drawColor,line width= 0.4pt,line join=round,line cap=round,fill=fillColor] (145.29,267.98) circle (  1.96);
\path[draw=drawColor,line width= 0.4pt,line join=round,line cap=round,fill=fillColor] (145.71,268.89) circle (  1.96);
\path[draw=drawColor,line width= 0.4pt,line join=round,line cap=round,fill=fillColor] (146.13,269.80) circle (  1.96);
\path[draw=drawColor,line width= 0.4pt,line join=round,line cap=round,fill=fillColor] (146.55,270.70) circle (  1.96);
\path[draw=drawColor,line width= 0.4pt,line join=round,line cap=round,fill=fillColor] (146.97,271.60) circle (  1.96);
\path[draw=drawColor,line width= 0.4pt,line join=round,line cap=round,fill=fillColor] (147.39,272.50) circle (  1.96);
\path[draw=drawColor,line width= 0.4pt,line join=round,line cap=round,fill=fillColor] (147.81,273.39) circle (  1.96);
\path[draw=drawColor,line width= 0.4pt,line join=round,line cap=round,fill=fillColor] (148.23,274.28) circle (  1.96);
\path[draw=drawColor,line width= 0.4pt,line join=round,line cap=round,fill=fillColor] (148.64,275.17) circle (  1.96);
\path[draw=drawColor,line width= 0.4pt,line join=round,line cap=round,fill=fillColor] (149.06,276.05) circle (  1.96);
\path[draw=drawColor,line width= 0.4pt,line join=round,line cap=round,fill=fillColor] (149.48,276.93) circle (  1.96);
\path[draw=drawColor,line width= 0.4pt,line join=round,line cap=round,fill=fillColor] (149.90,277.81) circle (  1.96);
\path[draw=drawColor,line width= 0.4pt,line join=round,line cap=round,fill=fillColor] (150.32,278.68) circle (  1.96);
\path[draw=drawColor,line width= 0.4pt,line join=round,line cap=round,fill=fillColor] (150.74,279.55) circle (  1.96);
\path[draw=drawColor,line width= 0.4pt,line join=round,line cap=round,fill=fillColor] (151.16,280.41) circle (  1.96);
\path[draw=drawColor,line width= 0.4pt,line join=round,line cap=round,fill=fillColor] (151.58,281.27) circle (  1.96);
\path[draw=drawColor,line width= 0.4pt,line join=round,line cap=round,fill=fillColor] (152.00,282.13) circle (  1.96);
\path[draw=drawColor,line width= 0.4pt,line join=round,line cap=round,fill=fillColor] (152.42,282.99) circle (  1.96);
\path[draw=drawColor,line width= 0.4pt,line join=round,line cap=round,fill=fillColor] (152.84,283.83) circle (  1.96);
\path[draw=drawColor,line width= 0.4pt,line join=round,line cap=round,fill=fillColor] (153.26,284.68) circle (  1.96);
\path[draw=drawColor,line width= 0.4pt,line join=round,line cap=round,fill=fillColor] (153.68,285.53) circle (  1.96);
\path[draw=drawColor,line width= 0.4pt,line join=round,line cap=round,fill=fillColor] (154.09,286.37) circle (  1.96);
\path[draw=drawColor,line width= 0.4pt,line join=round,line cap=round,fill=fillColor] (154.51,287.21) circle (  1.96);
\path[draw=drawColor,line width= 0.4pt,line join=round,line cap=round,fill=fillColor] (154.93,288.04) circle (  1.96);
\path[draw=drawColor,line width= 0.4pt,line join=round,line cap=round,fill=fillColor] (155.35,288.87) circle (  1.96);
\path[draw=drawColor,line width= 0.4pt,line join=round,line cap=round,fill=fillColor] (155.77,289.70) circle (  1.96);
\path[draw=drawColor,line width= 0.4pt,line join=round,line cap=round,fill=fillColor] (156.19,290.52) circle (  1.96);
\path[draw=drawColor,line width= 0.4pt,line join=round,line cap=round,fill=fillColor] (156.61,291.34) circle (  1.96);
\path[draw=drawColor,line width= 0.4pt,line join=round,line cap=round,fill=fillColor] (157.03,292.16) circle (  1.96);
\path[draw=drawColor,line width= 0.4pt,line join=round,line cap=round,fill=fillColor] (157.45,292.97) circle (  1.96);
\path[draw=drawColor,line width= 0.4pt,line join=round,line cap=round,fill=fillColor] (157.87,293.78) circle (  1.96);
\path[draw=drawColor,line width= 0.4pt,line join=round,line cap=round,fill=fillColor] (158.29,294.58) circle (  1.96);
\path[draw=drawColor,line width= 0.4pt,line join=round,line cap=round,fill=fillColor] (158.71,295.39) circle (  1.96);
\path[draw=drawColor,line width= 0.4pt,line join=round,line cap=round,fill=fillColor] (159.13,296.19) circle (  1.96);
\path[draw=drawColor,line width= 0.4pt,line join=round,line cap=round,fill=fillColor] (159.55,296.98) circle (  1.96);
\path[draw=drawColor,line width= 0.4pt,line join=round,line cap=round,fill=fillColor] (159.96,297.78) circle (  1.96);
\path[draw=drawColor,line width= 0.4pt,line join=round,line cap=round,fill=fillColor] (160.38,298.57) circle (  1.96);
\path[draw=drawColor,line width= 0.4pt,line join=round,line cap=round,fill=fillColor] (160.80,299.36) circle (  1.96);
\path[draw=drawColor,line width= 0.4pt,line join=round,line cap=round,fill=fillColor] (161.22,300.14) circle (  1.96);
\path[draw=drawColor,line width= 0.4pt,line join=round,line cap=round,fill=fillColor] (161.64,300.92) circle (  1.96);
\path[draw=drawColor,line width= 0.4pt,line join=round,line cap=round,fill=fillColor] (162.06,301.70) circle (  1.96);
\path[draw=drawColor,line width= 0.4pt,line join=round,line cap=round,fill=fillColor] (162.48,302.47) circle (  1.96);
\path[draw=drawColor,line width= 0.4pt,line join=round,line cap=round,fill=fillColor] (162.90,303.24) circle (  1.96);
\path[draw=drawColor,line width= 0.4pt,line join=round,line cap=round,fill=fillColor] (163.32,304.01) circle (  1.96);
\path[draw=drawColor,line width= 0.4pt,line join=round,line cap=round,fill=fillColor] (163.74,304.78) circle (  1.96);
\path[draw=drawColor,line width= 0.4pt,line join=round,line cap=round,fill=fillColor] (164.16,305.54) circle (  1.96);
\path[draw=drawColor,line width= 0.4pt,line join=round,line cap=round,fill=fillColor] (164.58,306.30) circle (  1.96);
\path[draw=drawColor,line width= 0.4pt,line join=round,line cap=round,fill=fillColor] (165.00,307.05) circle (  1.96);
\path[draw=drawColor,line width= 0.4pt,line join=round,line cap=round,fill=fillColor] (165.41,307.81) circle (  1.96);
\path[draw=drawColor,line width= 0.4pt,line join=round,line cap=round,fill=fillColor] (165.83,308.55) circle (  1.96);
\path[draw=drawColor,line width= 0.4pt,line join=round,line cap=round,fill=fillColor] (166.25,309.30) circle (  1.96);
\path[draw=drawColor,line width= 0.4pt,line join=round,line cap=round,fill=fillColor] (166.67,310.04) circle (  1.96);
\path[draw=drawColor,line width= 0.4pt,line join=round,line cap=round,fill=fillColor] (167.09,310.78) circle (  1.96);
\path[draw=drawColor,line width= 0.4pt,line join=round,line cap=round,fill=fillColor] (167.51,311.52) circle (  1.96);
\path[draw=drawColor,line width= 0.4pt,line join=round,line cap=round,fill=fillColor] (167.93,312.26) circle (  1.96);
\path[draw=drawColor,line width= 0.4pt,line join=round,line cap=round,fill=fillColor] (168.35,312.98) circle (  1.96);
\path[draw=drawColor,line width= 0.4pt,line join=round,line cap=round,fill=fillColor] (168.77,313.71) circle (  1.96);
\path[draw=drawColor,line width= 0.4pt,line join=round,line cap=round,fill=fillColor] (169.19,314.44) circle (  1.96);
\path[draw=drawColor,line width= 0.4pt,line join=round,line cap=round,fill=fillColor] (169.61,315.16) circle (  1.96);
\path[draw=drawColor,line width= 0.4pt,line join=round,line cap=round,fill=fillColor] (170.03,315.88) circle (  1.96);
\path[draw=drawColor,line width= 0.4pt,line join=round,line cap=round,fill=fillColor] (170.45,316.59) circle (  1.96);
\path[draw=drawColor,line width= 0.4pt,line join=round,line cap=round,fill=fillColor] (170.86,317.30) circle (  1.96);
\path[draw=drawColor,line width= 0.4pt,line join=round,line cap=round,fill=fillColor] (171.28,318.02) circle (  1.96);
\path[draw=drawColor,line width= 0.4pt,line join=round,line cap=round,fill=fillColor] (171.70,318.72) circle (  1.96);
\path[draw=drawColor,line width= 0.4pt,line join=round,line cap=round,fill=fillColor] (172.12,319.43) circle (  1.96);
\path[draw=drawColor,line width= 0.4pt,line join=round,line cap=round,fill=fillColor] (172.54,320.13) circle (  1.96);
\path[draw=drawColor,line width= 0.4pt,line join=round,line cap=round,fill=fillColor] (172.96,320.83) circle (  1.96);
\path[draw=drawColor,line width= 0.4pt,line join=round,line cap=round,fill=fillColor] (173.38,321.52) circle (  1.96);
\path[draw=drawColor,line width= 0.4pt,line join=round,line cap=round,fill=fillColor] (173.80,322.21) circle (  1.96);
\path[draw=drawColor,line width= 0.4pt,line join=round,line cap=round,fill=fillColor] (174.22,322.90) circle (  1.96);
\path[draw=drawColor,line width= 0.4pt,line join=round,line cap=round,fill=fillColor] (174.64,323.59) circle (  1.96);
\path[draw=drawColor,line width= 0.4pt,line join=round,line cap=round,fill=fillColor] (175.06,324.27) circle (  1.96);
\path[draw=drawColor,line width= 0.4pt,line join=round,line cap=round,fill=fillColor] (175.48,324.95) circle (  1.96);
\path[draw=drawColor,line width= 0.4pt,line join=round,line cap=round,fill=fillColor] (175.90,325.64) circle (  1.96);
\path[draw=drawColor,line width= 0.4pt,line join=round,line cap=round,fill=fillColor] (176.31,326.31) circle (  1.96);
\path[draw=drawColor,line width= 0.4pt,line join=round,line cap=round,fill=fillColor] (176.73,326.98) circle (  1.96);
\path[draw=drawColor,line width= 0.4pt,line join=round,line cap=round,fill=fillColor] (177.15,327.65) circle (  1.96);
\path[draw=drawColor,line width= 0.4pt,line join=round,line cap=round,fill=fillColor] (177.57,328.32) circle (  1.96);
\path[draw=drawColor,line width= 0.4pt,line join=round,line cap=round,fill=fillColor] (177.99,328.98) circle (  1.96);
\path[draw=drawColor,line width= 0.4pt,line join=round,line cap=round,fill=fillColor] (178.41,329.65) circle (  1.96);
\path[draw=drawColor,line width= 0.4pt,line join=round,line cap=round,fill=fillColor] (178.83,330.30) circle (  1.96);
\path[draw=drawColor,line width= 0.4pt,line join=round,line cap=round,fill=fillColor] (179.25,330.96) circle (  1.96);
\path[draw=drawColor,line width= 0.4pt,line join=round,line cap=round,fill=fillColor] (179.67,331.61) circle (  1.96);
\path[draw=drawColor,line width= 0.4pt,line join=round,line cap=round,fill=fillColor] (180.09,332.26) circle (  1.96);
\path[draw=drawColor,line width= 0.4pt,line join=round,line cap=round,fill=fillColor] (180.51,332.91) circle (  1.96);
\path[draw=drawColor,line width= 0.4pt,line join=round,line cap=round,fill=fillColor] (180.93,333.55) circle (  1.96);
\path[draw=drawColor,line width= 0.4pt,line join=round,line cap=round,fill=fillColor] (181.35,334.20) circle (  1.96);
\path[draw=drawColor,line width= 0.4pt,line join=round,line cap=round,fill=fillColor] (181.76,334.84) circle (  1.96);
\path[draw=drawColor,line width= 0.4pt,line join=round,line cap=round,fill=fillColor] (182.18,335.47) circle (  1.96);
\path[draw=drawColor,line width= 0.4pt,line join=round,line cap=round,fill=fillColor] (182.60,336.11) circle (  1.96);
\path[draw=drawColor,line width= 0.4pt,line join=round,line cap=round,fill=fillColor] (183.02,336.74) circle (  1.96);
\path[draw=drawColor,line width= 0.4pt,line join=round,line cap=round,fill=fillColor] (183.44,337.37) circle (  1.96);
\path[draw=drawColor,line width= 0.4pt,line join=round,line cap=round,fill=fillColor] (183.86,337.99) circle (  1.96);
\path[draw=drawColor,line width= 0.4pt,line join=round,line cap=round,fill=fillColor] (184.28,338.62) circle (  1.96);
\path[draw=drawColor,line width= 0.4pt,line join=round,line cap=round,fill=fillColor] (184.70,339.24) circle (  1.96);
\path[draw=drawColor,line width= 0.4pt,line join=round,line cap=round,fill=fillColor] (185.12,339.86) circle (  1.96);
\path[draw=drawColor,line width= 0.4pt,line join=round,line cap=round,fill=fillColor] (185.54,340.48) circle (  1.96);
\path[draw=drawColor,line width= 0.4pt,line join=round,line cap=round,fill=fillColor] (185.96,341.09) circle (  1.96);
\path[draw=drawColor,line width= 0.4pt,line join=round,line cap=round,fill=fillColor] (186.38,341.70) circle (  1.96);
\path[draw=drawColor,line width= 0.4pt,line join=round,line cap=round,fill=fillColor] (186.80,342.31) circle (  1.96);
\path[draw=drawColor,line width= 0.4pt,line join=round,line cap=round,fill=fillColor] (187.22,342.92) circle (  1.96);
\path[draw=drawColor,line width= 0.4pt,line join=round,line cap=round,fill=fillColor] (187.63,343.52) circle (  1.96);
\path[draw=drawColor,line width= 0.4pt,line join=round,line cap=round,fill=fillColor] (188.05,344.12) circle (  1.96);
\path[draw=drawColor,line width= 0.4pt,line join=round,line cap=round,fill=fillColor] (188.47,344.72) circle (  1.96);
\path[draw=drawColor,line width= 0.4pt,line join=round,line cap=round,fill=fillColor] (188.89,345.31) circle (  1.96);
\path[draw=drawColor,line width= 0.4pt,line join=round,line cap=round,fill=fillColor] (189.31,345.91) circle (  1.96);
\path[draw=drawColor,line width= 0.4pt,line join=round,line cap=round,fill=fillColor] (189.73,346.50) circle (  1.96);
\path[draw=drawColor,line width= 0.4pt,line join=round,line cap=round,fill=fillColor] (190.15,347.08) circle (  1.96);
\path[draw=drawColor,line width= 0.4pt,line join=round,line cap=round,fill=fillColor] (190.57,347.67) circle (  1.96);
\path[draw=drawColor,line width= 0.4pt,line join=round,line cap=round,fill=fillColor] (190.99,348.26) circle (  1.96);
\path[draw=drawColor,line width= 0.4pt,line join=round,line cap=round,fill=fillColor] (191.41,348.83) circle (  1.96);
\path[draw=drawColor,line width= 0.4pt,line join=round,line cap=round,fill=fillColor] (191.83,349.41) circle (  1.96);
\path[draw=drawColor,line width= 0.4pt,line join=round,line cap=round,fill=fillColor] (192.25,349.99) circle (  1.96);
\path[draw=drawColor,line width= 0.4pt,line join=round,line cap=round,fill=fillColor] (192.67,350.56) circle (  1.96);
\path[draw=drawColor,line width= 0.4pt,line join=round,line cap=round,fill=fillColor] (193.08,351.13) circle (  1.96);
\path[draw=drawColor,line width= 0.4pt,line join=round,line cap=round,fill=fillColor] (193.50,351.70) circle (  1.96);
\path[draw=drawColor,line width= 0.4pt,line join=round,line cap=round,fill=fillColor] (193.92,352.27) circle (  1.96);
\path[draw=drawColor,line width= 0.4pt,line join=round,line cap=round,fill=fillColor] (194.34,352.83) circle (  1.96);
\path[draw=drawColor,line width= 0.4pt,line join=round,line cap=round,fill=fillColor] (194.76,353.39) circle (  1.96);
\path[draw=drawColor,line width= 0.4pt,line join=round,line cap=round,fill=fillColor] (195.18,353.95) circle (  1.96);
\path[draw=drawColor,line width= 0.4pt,line join=round,line cap=round,fill=fillColor] (195.60,354.51) circle (  1.96);
\path[draw=drawColor,line width= 0.4pt,line join=round,line cap=round,fill=fillColor] (196.02,355.06) circle (  1.96);
\path[draw=drawColor,line width= 0.4pt,line join=round,line cap=round,fill=fillColor] (196.44,355.62) circle (  1.96);
\path[draw=drawColor,line width= 0.4pt,line join=round,line cap=round,fill=fillColor] (196.86,356.17) circle (  1.96);
\path[draw=drawColor,line width= 0.4pt,line join=round,line cap=round,fill=fillColor] (197.28,356.71) circle (  1.96);
\path[draw=drawColor,line width= 0.4pt,line join=round,line cap=round,fill=fillColor] (197.70,357.26) circle (  1.96);
\path[draw=drawColor,line width= 0.4pt,line join=round,line cap=round,fill=fillColor] (198.12,357.80) circle (  1.96);
\path[draw=drawColor,line width= 0.4pt,line join=round,line cap=round,fill=fillColor] (198.53,358.34) circle (  1.96);
\path[draw=drawColor,line width= 0.4pt,line join=round,line cap=round,fill=fillColor] (198.95,358.88) circle (  1.96);
\path[draw=drawColor,line width= 0.4pt,line join=round,line cap=round,fill=fillColor] (199.37,359.41) circle (  1.96);
\path[draw=drawColor,line width= 0.4pt,line join=round,line cap=round,fill=fillColor] (199.79,359.95) circle (  1.96);
\path[draw=drawColor,line width= 0.4pt,line join=round,line cap=round,fill=fillColor] (200.21,360.48) circle (  1.96);
\path[draw=drawColor,line width= 0.4pt,line join=round,line cap=round,fill=fillColor] (200.63,361.01) circle (  1.96);
\path[draw=drawColor,line width= 0.4pt,line join=round,line cap=round,fill=fillColor] (201.05,361.54) circle (  1.96);
\path[draw=drawColor,line width= 0.4pt,line join=round,line cap=round,fill=fillColor] (201.47,362.06) circle (  1.96);
\path[draw=drawColor,line width= 0.4pt,line join=round,line cap=round,fill=fillColor] (201.89,362.58) circle (  1.96);
\path[draw=drawColor,line width= 0.4pt,line join=round,line cap=round,fill=fillColor] (202.31,363.10) circle (  1.96);
\path[draw=drawColor,line width= 0.4pt,line join=round,line cap=round,fill=fillColor] (202.73,363.62) circle (  1.96);
\path[draw=drawColor,line width= 0.4pt,line join=round,line cap=round,fill=fillColor] (203.15,364.13) circle (  1.96);
\path[draw=drawColor,line width= 0.4pt,line join=round,line cap=round,fill=fillColor] (203.57,364.65) circle (  1.96);
\path[draw=drawColor,line width= 0.4pt,line join=round,line cap=round,fill=fillColor] (203.98,365.16) circle (  1.96);
\path[draw=drawColor,line width= 0.4pt,line join=round,line cap=round,fill=fillColor] (204.40,365.67) circle (  1.96);
\path[draw=drawColor,line width= 0.4pt,line join=round,line cap=round,fill=fillColor] (204.82,366.18) circle (  1.96);
\path[draw=drawColor,line width= 0.4pt,line join=round,line cap=round,fill=fillColor] (205.24,366.68) circle (  1.96);
\path[draw=drawColor,line width= 0.4pt,line join=round,line cap=round,fill=fillColor] (205.66,367.19) circle (  1.96);
\path[draw=drawColor,line width= 0.4pt,line join=round,line cap=round,fill=fillColor] (206.08,367.69) circle (  1.96);
\path[draw=drawColor,line width= 0.4pt,line join=round,line cap=round,fill=fillColor] (206.50,368.18) circle (  1.96);
\path[draw=drawColor,line width= 0.4pt,line join=round,line cap=round,fill=fillColor] (206.92,368.68) circle (  1.96);
\path[draw=drawColor,line width= 0.4pt,line join=round,line cap=round,fill=fillColor] (207.34,369.18) circle (  1.96);
\path[draw=drawColor,line width= 0.4pt,line join=round,line cap=round,fill=fillColor] (207.76,369.67) circle (  1.96);
\path[draw=drawColor,line width= 0.4pt,line join=round,line cap=round,fill=fillColor] (208.18,370.16) circle (  1.96);
\path[draw=drawColor,line width= 0.4pt,line join=round,line cap=round,fill=fillColor] (208.60,370.64) circle (  1.96);
\path[draw=drawColor,line width= 0.4pt,line join=round,line cap=round,fill=fillColor] (209.02,371.13) circle (  1.96);
\path[draw=drawColor,line width= 0.4pt,line join=round,line cap=round,fill=fillColor] (209.44,371.61) circle (  1.96);
\path[draw=drawColor,line width= 0.4pt,line join=round,line cap=round,fill=fillColor] (209.85,372.10) circle (  1.96);
\path[draw=drawColor,line width= 0.4pt,line join=round,line cap=round,fill=fillColor] (210.27,372.58) circle (  1.96);
\path[draw=drawColor,line width= 0.4pt,line join=round,line cap=round,fill=fillColor] (210.69,373.05) circle (  1.96);
\path[draw=drawColor,line width= 0.4pt,line join=round,line cap=round,fill=fillColor] (211.11,373.53) circle (  1.96);
\path[draw=drawColor,line width= 0.4pt,line join=round,line cap=round,fill=fillColor] (211.53,374.00) circle (  1.96);
\path[draw=drawColor,line width= 0.4pt,line join=round,line cap=round,fill=fillColor] (211.95,374.48) circle (  1.96);
\path[draw=drawColor,line width= 0.4pt,line join=round,line cap=round,fill=fillColor] (212.37,374.94) circle (  1.96);
\path[draw=drawColor,line width= 0.4pt,line join=round,line cap=round,fill=fillColor] (212.79,375.41) circle (  1.96);
\path[draw=drawColor,line width= 0.4pt,line join=round,line cap=round,fill=fillColor] (213.21,375.88) circle (  1.96);
\path[draw=drawColor,line width= 0.4pt,line join=round,line cap=round,fill=fillColor] (213.63,376.34) circle (  1.96);
\path[draw=drawColor,line width= 0.4pt,line join=round,line cap=round,fill=fillColor] (214.05,376.80) circle (  1.96);
\path[draw=drawColor,line width= 0.4pt,line join=round,line cap=round,fill=fillColor] (214.47,377.26) circle (  1.96);
\path[draw=drawColor,line width= 0.4pt,line join=round,line cap=round,fill=fillColor] (214.89,377.72) circle (  1.96);
\path[draw=drawColor,line width= 0.4pt,line join=round,line cap=round,fill=fillColor] (215.30,378.18) circle (  1.96);
\path[draw=drawColor,line width= 0.4pt,line join=round,line cap=round,fill=fillColor] (215.72,378.63) circle (  1.96);
\path[draw=drawColor,line width= 0.4pt,line join=round,line cap=round,fill=fillColor] (216.14,379.08) circle (  1.96);
\path[draw=drawColor,line width= 0.4pt,line join=round,line cap=round,fill=fillColor] (216.56,379.53) circle (  1.96);
\path[draw=drawColor,line width= 0.4pt,line join=round,line cap=round,fill=fillColor] (216.98,379.98) circle (  1.96);
\path[draw=drawColor,line width= 0.4pt,line join=round,line cap=round,fill=fillColor] (217.40,380.42) circle (  1.96);
\path[draw=drawColor,line width= 0.4pt,line join=round,line cap=round,fill=fillColor] (217.82,380.87) circle (  1.96);
\path[draw=drawColor,line width= 0.4pt,line join=round,line cap=round,fill=fillColor] (218.24,381.31) circle (  1.96);
\path[draw=drawColor,line width= 0.4pt,line join=round,line cap=round,fill=fillColor] (218.66,381.75) circle (  1.96);
\path[draw=drawColor,line width= 0.4pt,line join=round,line cap=round,fill=fillColor] (219.08,382.19) circle (  1.96);
\path[draw=drawColor,line width= 0.4pt,line join=round,line cap=round,fill=fillColor] (219.50,382.62) circle (  1.96);
\path[draw=drawColor,line width= 0.4pt,line join=round,line cap=round,fill=fillColor] (219.92,383.06) circle (  1.96);
\path[draw=drawColor,line width= 0.4pt,line join=round,line cap=round,fill=fillColor] (220.34,383.49) circle (  1.96);
\path[draw=drawColor,line width= 0.4pt,line join=round,line cap=round,fill=fillColor] (220.75,383.92) circle (  1.96);
\path[draw=drawColor,line width= 0.4pt,line join=round,line cap=round,fill=fillColor] (221.17,384.35) circle (  1.96);
\path[draw=drawColor,line width= 0.4pt,line join=round,line cap=round,fill=fillColor] (221.59,384.78) circle (  1.96);
\path[draw=drawColor,line width= 0.4pt,line join=round,line cap=round,fill=fillColor] (222.01,385.20) circle (  1.96);
\path[draw=drawColor,line width= 0.4pt,line join=round,line cap=round,fill=fillColor] (222.43,385.63) circle (  1.96);
\path[draw=drawColor,line width= 0.4pt,line join=round,line cap=round,fill=fillColor] (222.85,386.05) circle (  1.96);
\path[draw=drawColor,line width= 0.4pt,line join=round,line cap=round,fill=fillColor] (223.27,386.47) circle (  1.96);
\path[draw=drawColor,line width= 0.4pt,line join=round,line cap=round,fill=fillColor] (223.69,386.89) circle (  1.96);
\path[draw=drawColor,line width= 0.4pt,line join=round,line cap=round,fill=fillColor] (224.11,387.30) circle (  1.96);
\path[draw=drawColor,line width= 0.4pt,line join=round,line cap=round,fill=fillColor] (224.53,387.72) circle (  1.96);
\path[draw=drawColor,line width= 0.4pt,line join=round,line cap=round,fill=fillColor] (224.95,388.13) circle (  1.96);
\path[draw=drawColor,line width= 0.4pt,line join=round,line cap=round,fill=fillColor] (225.37,388.54) circle (  1.96);
\path[draw=drawColor,line width= 0.4pt,line join=round,line cap=round,fill=fillColor] (225.79,388.95) circle (  1.96);
\path[draw=drawColor,line width= 0.4pt,line join=round,line cap=round,fill=fillColor] (226.20,389.36) circle (  1.96);
\path[draw=drawColor,line width= 0.4pt,line join=round,line cap=round,fill=fillColor] (226.62,389.76) circle (  1.96);
\path[draw=drawColor,line width= 0.4pt,line join=round,line cap=round,fill=fillColor] (227.04,390.17) circle (  1.96);
\path[draw=drawColor,line width= 0.4pt,line join=round,line cap=round,fill=fillColor] (227.46,390.57) circle (  1.96);
\path[draw=drawColor,line width= 0.4pt,line join=round,line cap=round,fill=fillColor] (227.88,390.97) circle (  1.96);
\path[draw=drawColor,line width= 0.4pt,line join=round,line cap=round,fill=fillColor] (228.30,391.37) circle (  1.96);
\path[draw=drawColor,line width= 0.4pt,line join=round,line cap=round,fill=fillColor] (228.72,391.76) circle (  1.96);
\path[draw=drawColor,line width= 0.4pt,line join=round,line cap=round,fill=fillColor] (229.14,392.16) circle (  1.96);
\path[draw=drawColor,line width= 0.4pt,line join=round,line cap=round,fill=fillColor] (229.56,392.55) circle (  1.96);
\path[draw=drawColor,line width= 0.4pt,line join=round,line cap=round,fill=fillColor] (229.98,392.94) circle (  1.96);
\path[draw=drawColor,line width= 0.4pt,line join=round,line cap=round,fill=fillColor] (230.40,393.34) circle (  1.96);
\path[draw=drawColor,line width= 0.4pt,line join=round,line cap=round,fill=fillColor] (230.82,393.72) circle (  1.96);
\path[draw=drawColor,line width= 0.4pt,line join=round,line cap=round,fill=fillColor] (231.24,394.11) circle (  1.96);
\path[draw=drawColor,line width= 0.4pt,line join=round,line cap=round,fill=fillColor] (231.65,394.49) circle (  1.96);
\path[draw=drawColor,line width= 0.4pt,line join=round,line cap=round,fill=fillColor] (232.07,394.88) circle (  1.96);
\path[draw=drawColor,line width= 0.4pt,line join=round,line cap=round,fill=fillColor] (232.49,395.26) circle (  1.96);
\path[draw=drawColor,line width= 0.4pt,line join=round,line cap=round,fill=fillColor] (232.91,395.64) circle (  1.96);
\path[draw=drawColor,line width= 0.4pt,line join=round,line cap=round,fill=fillColor] (233.33,396.02) circle (  1.96);
\path[draw=drawColor,line width= 0.4pt,line join=round,line cap=round,fill=fillColor] (233.75,396.40) circle (  1.96);
\path[draw=drawColor,line width= 0.4pt,line join=round,line cap=round,fill=fillColor] (234.17,396.77) circle (  1.96);
\path[draw=drawColor,line width= 0.4pt,line join=round,line cap=round,fill=fillColor] (234.59,397.15) circle (  1.96);
\path[draw=drawColor,line width= 0.4pt,line join=round,line cap=round,fill=fillColor] (235.01,397.52) circle (  1.96);
\path[draw=drawColor,line width= 0.4pt,line join=round,line cap=round,fill=fillColor] (235.43,397.89) circle (  1.96);
\path[draw=drawColor,line width= 0.4pt,line join=round,line cap=round,fill=fillColor] (235.85,398.25) circle (  1.96);
\path[draw=drawColor,line width= 0.4pt,line join=round,line cap=round,fill=fillColor] (236.27,398.62) circle (  1.96);
\path[draw=drawColor,line width= 0.4pt,line join=round,line cap=round,fill=fillColor] (236.69,398.99) circle (  1.96);
\path[draw=drawColor,line width= 0.4pt,line join=round,line cap=round,fill=fillColor] (237.11,399.35) circle (  1.96);
\path[draw=drawColor,line width= 0.4pt,line join=round,line cap=round,fill=fillColor] (237.52,399.71) circle (  1.96);
\path[draw=drawColor,line width= 0.4pt,line join=round,line cap=round,fill=fillColor] (237.94,400.08) circle (  1.96);
\path[draw=drawColor,line width= 0.4pt,line join=round,line cap=round,fill=fillColor] (238.36,400.43) circle (  1.96);
\path[draw=drawColor,line width= 0.4pt,line join=round,line cap=round,fill=fillColor] (238.78,400.79) circle (  1.96);
\path[draw=drawColor,line width= 0.4pt,line join=round,line cap=round,fill=fillColor] (239.20,401.15) circle (  1.96);
\path[draw=drawColor,line width= 0.4pt,line join=round,line cap=round,fill=fillColor] (239.62,401.50) circle (  1.96);
\path[draw=drawColor,line width= 0.4pt,line join=round,line cap=round,fill=fillColor] (240.04,401.85) circle (  1.96);
\path[draw=drawColor,line width= 0.4pt,line join=round,line cap=round,fill=fillColor] (240.46,402.20) circle (  1.96);
\path[draw=drawColor,line width= 0.4pt,line join=round,line cap=round,fill=fillColor] (240.88,402.55) circle (  1.96);
\path[draw=drawColor,line width= 0.4pt,line join=round,line cap=round,fill=fillColor] (241.30,402.90) circle (  1.96);
\path[draw=drawColor,line width= 0.4pt,line join=round,line cap=round,fill=fillColor] (241.72,403.25) circle (  1.96);
\path[draw=drawColor,line width= 0.4pt,line join=round,line cap=round,fill=fillColor] (242.14,403.60) circle (  1.96);
\path[draw=drawColor,line width= 0.4pt,line join=round,line cap=round,fill=fillColor] (242.56,403.94) circle (  1.96);
\path[draw=drawColor,line width= 0.4pt,line join=round,line cap=round,fill=fillColor] (242.97,404.28) circle (  1.96);
\path[draw=drawColor,line width= 0.4pt,line join=round,line cap=round,fill=fillColor] (243.39,404.62) circle (  1.96);
\path[draw=drawColor,line width= 0.4pt,line join=round,line cap=round,fill=fillColor] (243.81,404.96) circle (  1.96);
\path[draw=drawColor,line width= 0.4pt,line join=round,line cap=round,fill=fillColor] (244.23,405.30) circle (  1.96);
\path[draw=drawColor,line width= 0.4pt,line join=round,line cap=round,fill=fillColor] (244.65,405.64) circle (  1.96);
\path[draw=drawColor,line width= 0.4pt,line join=round,line cap=round,fill=fillColor] (245.07,405.97) circle (  1.96);
\path[draw=drawColor,line width= 0.4pt,line join=round,line cap=round,fill=fillColor] (245.49,406.30) circle (  1.96);
\path[draw=drawColor,line width= 0.4pt,line join=round,line cap=round,fill=fillColor] (245.91,406.64) circle (  1.96);
\path[draw=drawColor,line width= 0.4pt,line join=round,line cap=round,fill=fillColor] (246.33,406.96) circle (  1.96);
\path[draw=drawColor,line width= 0.4pt,line join=round,line cap=round,fill=fillColor] (246.75,407.30) circle (  1.96);
\path[draw=drawColor,line width= 0.4pt,line join=round,line cap=round,fill=fillColor] (247.17,407.62) circle (  1.96);
\path[draw=drawColor,line width= 0.4pt,line join=round,line cap=round,fill=fillColor] (247.59,407.95) circle (  1.96);
\path[draw=drawColor,line width= 0.4pt,line join=round,line cap=round,fill=fillColor] (248.01,408.27) circle (  1.96);
\path[draw=drawColor,line width= 0.4pt,line join=round,line cap=round,fill=fillColor] (248.42,408.59) circle (  1.96);
\path[draw=drawColor,line width= 0.4pt,line join=round,line cap=round,fill=fillColor] (248.84,408.92) circle (  1.96);
\path[draw=drawColor,line width= 0.4pt,line join=round,line cap=round,fill=fillColor] (249.26,409.24) circle (  1.96);
\path[draw=drawColor,line width= 0.4pt,line join=round,line cap=round,fill=fillColor] (249.68,409.56) circle (  1.96);
\path[draw=drawColor,line width= 0.4pt,line join=round,line cap=round,fill=fillColor] (250.10,409.88) circle (  1.96);
\path[draw=drawColor,line width= 0.4pt,line join=round,line cap=round,fill=fillColor] (250.52,410.19) circle (  1.96);
\path[draw=drawColor,line width= 0.4pt,line join=round,line cap=round,fill=fillColor] (250.94,410.51) circle (  1.96);
\path[draw=drawColor,line width= 0.4pt,line join=round,line cap=round,fill=fillColor] (251.36,410.82) circle (  1.96);
\path[draw=drawColor,line width= 0.4pt,line join=round,line cap=round,fill=fillColor] (251.78,411.13) circle (  1.96);
\path[draw=drawColor,line width= 0.4pt,line join=round,line cap=round,fill=fillColor] (252.20,411.44) circle (  1.96);
\path[draw=drawColor,line width= 0.4pt,line join=round,line cap=round,fill=fillColor] (252.62,411.75) circle (  1.96);
\path[draw=drawColor,line width= 0.4pt,line join=round,line cap=round,fill=fillColor] (253.04,412.06) circle (  1.96);
\path[draw=drawColor,line width= 0.4pt,line join=round,line cap=round,fill=fillColor] (253.46,412.37) circle (  1.96);
\path[draw=drawColor,line width= 0.4pt,line join=round,line cap=round,fill=fillColor] (253.87,412.67) circle (  1.96);
\path[draw=drawColor,line width= 0.4pt,line join=round,line cap=round,fill=fillColor] (254.29,412.97) circle (  1.96);
\path[draw=drawColor,line width= 0.4pt,line join=round,line cap=round,fill=fillColor] (254.71,413.28) circle (  1.96);
\path[draw=drawColor,line width= 0.4pt,line join=round,line cap=round,fill=fillColor] (255.13,413.58) circle (  1.96);
\path[draw=drawColor,line width= 0.4pt,line join=round,line cap=round,fill=fillColor] (255.55,413.88) circle (  1.96);
\path[draw=drawColor,line width= 0.4pt,line join=round,line cap=round,fill=fillColor] (255.97,414.18) circle (  1.96);
\path[draw=drawColor,line width= 0.4pt,line join=round,line cap=round,fill=fillColor] (256.39,414.48) circle (  1.96);
\path[draw=drawColor,line width= 0.4pt,line join=round,line cap=round,fill=fillColor] (256.81,414.77) circle (  1.96);
\path[draw=drawColor,line width= 0.4pt,line join=round,line cap=round,fill=fillColor] (257.23,415.07) circle (  1.96);
\path[draw=drawColor,line width= 0.4pt,line join=round,line cap=round,fill=fillColor] (257.65,415.36) circle (  1.96);
\path[draw=drawColor,line width= 0.4pt,line join=round,line cap=round,fill=fillColor] (258.07,415.65) circle (  1.96);
\path[draw=drawColor,line width= 0.4pt,line join=round,line cap=round,fill=fillColor] (258.49,415.94) circle (  1.96);
\path[draw=drawColor,line width= 0.4pt,line join=round,line cap=round,fill=fillColor] (258.91,416.23) circle (  1.96);
\path[draw=drawColor,line width= 0.4pt,line join=round,line cap=round,fill=fillColor] (259.33,416.52) circle (  1.96);
\path[draw=drawColor,line width= 0.4pt,line join=round,line cap=round,fill=fillColor] (259.74,416.81) circle (  1.96);
\path[draw=drawColor,line width= 0.4pt,line join=round,line cap=round,fill=fillColor] (260.16,417.10) circle (  1.96);
\path[draw=drawColor,line width= 0.4pt,line join=round,line cap=round,fill=fillColor] (260.58,417.38) circle (  1.96);
\path[draw=drawColor,line width= 0.4pt,line join=round,line cap=round,fill=fillColor] (261.00,417.66) circle (  1.96);
\path[draw=drawColor,line width= 0.4pt,line join=round,line cap=round,fill=fillColor] (261.42,417.94) circle (  1.96);
\path[draw=drawColor,line width= 0.4pt,line join=round,line cap=round,fill=fillColor] (261.84,418.22) circle (  1.96);
\path[draw=drawColor,line width= 0.4pt,line join=round,line cap=round,fill=fillColor] (262.26,418.50) circle (  1.96);
\path[draw=drawColor,line width= 0.4pt,line join=round,line cap=round,fill=fillColor] (262.68,418.78) circle (  1.96);
\path[draw=drawColor,line width= 0.4pt,line join=round,line cap=round,fill=fillColor] (263.10,419.06) circle (  1.96);
\path[draw=drawColor,line width= 0.4pt,line join=round,line cap=round,fill=fillColor] (263.52,419.34) circle (  1.96);
\path[draw=drawColor,line width= 0.4pt,line join=round,line cap=round,fill=fillColor] (263.94,419.61) circle (  1.96);
\path[draw=drawColor,line width= 0.4pt,line join=round,line cap=round,fill=fillColor] (264.36,419.88) circle (  1.96);
\path[draw=drawColor,line width= 0.4pt,line join=round,line cap=round,fill=fillColor] (264.78,420.16) circle (  1.96);
\path[draw=drawColor,line width= 0.4pt,line join=round,line cap=round,fill=fillColor] (265.19,420.42) circle (  1.96);
\path[draw=drawColor,line width= 0.4pt,line join=round,line cap=round,fill=fillColor] (265.61,420.70) circle (  1.96);
\path[draw=drawColor,line width= 0.4pt,line join=round,line cap=round,fill=fillColor] (266.03,420.97) circle (  1.96);
\path[draw=drawColor,line width= 0.4pt,line join=round,line cap=round,fill=fillColor] (266.45,421.23) circle (  1.96);
\path[draw=drawColor,line width= 0.4pt,line join=round,line cap=round,fill=fillColor] (266.87,421.50) circle (  1.96);
\path[draw=drawColor,line width= 0.4pt,line join=round,line cap=round,fill=fillColor] (267.29,421.76) circle (  1.96);
\path[draw=drawColor,line width= 0.4pt,line join=round,line cap=round,fill=fillColor] (267.71,422.03) circle (  1.96);
\path[draw=drawColor,line width= 0.4pt,line join=round,line cap=round,fill=fillColor] (268.13,422.29) circle (  1.96);
\path[draw=drawColor,line width= 0.4pt,line join=round,line cap=round,fill=fillColor] (268.55,422.55) circle (  1.96);
\path[draw=drawColor,line width= 0.4pt,line join=round,line cap=round,fill=fillColor] (268.97,422.81) circle (  1.96);
\path[draw=drawColor,line width= 0.4pt,line join=round,line cap=round,fill=fillColor] (269.39,423.07) circle (  1.96);
\path[draw=drawColor,line width= 0.4pt,line join=round,line cap=round,fill=fillColor] (269.81,423.33) circle (  1.96);
\path[draw=drawColor,line width= 0.4pt,line join=round,line cap=round,fill=fillColor] (270.23,423.59) circle (  1.96);
\path[draw=drawColor,line width= 0.4pt,line join=round,line cap=round,fill=fillColor] (270.64,423.84) circle (  1.96);
\path[draw=drawColor,line width= 0.4pt,line join=round,line cap=round,fill=fillColor] (271.06,424.09) circle (  1.96);
\path[draw=drawColor,line width= 0.4pt,line join=round,line cap=round,fill=fillColor] (271.48,424.35) circle (  1.96);
\path[draw=drawColor,line width= 0.4pt,line join=round,line cap=round,fill=fillColor] (271.90,424.60) circle (  1.96);
\path[draw=drawColor,line width= 0.4pt,line join=round,line cap=round,fill=fillColor] (272.32,424.85) circle (  1.96);
\path[draw=drawColor,line width= 0.4pt,line join=round,line cap=round,fill=fillColor] (272.74,425.10) circle (  1.96);
\path[draw=drawColor,line width= 0.4pt,line join=round,line cap=round,fill=fillColor] (273.16,425.35) circle (  1.96);
\path[draw=drawColor,line width= 0.4pt,line join=round,line cap=round,fill=fillColor] (273.58,425.60) circle (  1.96);
\path[draw=drawColor,line width= 0.4pt,line join=round,line cap=round,fill=fillColor] (274.00,425.85) circle (  1.96);
\path[draw=drawColor,line width= 0.4pt,line join=round,line cap=round,fill=fillColor] (274.42,426.10) circle (  1.96);
\path[draw=drawColor,line width= 0.4pt,line join=round,line cap=round,fill=fillColor] (274.84,426.34) circle (  1.96);
\path[draw=drawColor,line width= 0.4pt,line join=round,line cap=round,fill=fillColor] (275.26,426.58) circle (  1.96);
\path[draw=drawColor,line width= 0.4pt,line join=round,line cap=round,fill=fillColor] (275.68,426.82) circle (  1.96);
\path[draw=drawColor,line width= 0.4pt,line join=round,line cap=round,fill=fillColor] (276.09,427.07) circle (  1.96);
\path[draw=drawColor,line width= 0.4pt,line join=round,line cap=round,fill=fillColor] (276.51,427.30) circle (  1.96);
\path[draw=drawColor,line width= 0.4pt,line join=round,line cap=round,fill=fillColor] (276.93,427.55) circle (  1.96);
\path[draw=drawColor,line width= 0.4pt,line join=round,line cap=round,fill=fillColor] (277.35,427.79) circle (  1.96);
\path[draw=drawColor,line width= 0.4pt,line join=round,line cap=round,fill=fillColor] (277.77,428.02) circle (  1.96);
\path[draw=drawColor,line width= 0.4pt,line join=round,line cap=round,fill=fillColor] (278.19,428.26) circle (  1.96);
\path[draw=drawColor,line width= 0.4pt,line join=round,line cap=round,fill=fillColor] (278.61,428.49) circle (  1.96);
\path[draw=drawColor,line width= 0.4pt,line join=round,line cap=round,fill=fillColor] (279.03,428.73) circle (  1.96);
\path[draw=drawColor,line width= 0.4pt,line join=round,line cap=round,fill=fillColor] (279.45,428.96) circle (  1.96);
\path[draw=drawColor,line width= 0.4pt,line join=round,line cap=round,fill=fillColor] (279.87,429.19) circle (  1.96);
\path[draw=drawColor,line width= 0.4pt,line join=round,line cap=round,fill=fillColor] (280.29,429.42) circle (  1.96);
\path[draw=drawColor,line width= 0.4pt,line join=round,line cap=round,fill=fillColor] (280.71,429.65) circle (  1.96);
\path[draw=drawColor,line width= 0.4pt,line join=round,line cap=round,fill=fillColor] (281.13,429.88) circle (  1.96);
\path[draw=drawColor,line width= 0.4pt,line join=round,line cap=round,fill=fillColor] (281.54,430.11) circle (  1.96);
\path[draw=drawColor,line width= 0.4pt,line join=round,line cap=round,fill=fillColor] (281.96,430.34) circle (  1.96);
\path[draw=drawColor,line width= 0.4pt,line join=round,line cap=round,fill=fillColor] (282.38,430.56) circle (  1.96);
\path[draw=drawColor,line width= 0.4pt,line join=round,line cap=round,fill=fillColor] (282.80,430.79) circle (  1.96);
\path[draw=drawColor,line width= 0.4pt,line join=round,line cap=round,fill=fillColor] (283.22,431.01) circle (  1.96);
\path[draw=drawColor,line width= 0.4pt,line join=round,line cap=round,fill=fillColor] (283.64,431.23) circle (  1.96);
\path[draw=drawColor,line width= 0.4pt,line join=round,line cap=round,fill=fillColor] (284.06,431.46) circle (  1.96);
\path[draw=drawColor,line width= 0.4pt,line join=round,line cap=round,fill=fillColor] (284.48,431.67) circle (  1.96);
\path[draw=drawColor,line width= 0.4pt,line join=round,line cap=round,fill=fillColor] (284.90,431.89) circle (  1.96);
\path[draw=drawColor,line width= 0.4pt,line join=round,line cap=round,fill=fillColor] (285.32,432.11) circle (  1.96);
\path[draw=drawColor,line width= 0.4pt,line join=round,line cap=round,fill=fillColor] (285.74,432.33) circle (  1.96);
\path[draw=drawColor,line width= 0.4pt,line join=round,line cap=round,fill=fillColor] (286.16,432.55) circle (  1.96);
\path[draw=drawColor,line width= 0.4pt,line join=round,line cap=round,fill=fillColor] (286.58,432.76) circle (  1.96);
\path[draw=drawColor,line width= 0.4pt,line join=round,line cap=round,fill=fillColor] (287.00,432.98) circle (  1.96);
\path[draw=drawColor,line width= 0.4pt,line join=round,line cap=round,fill=fillColor] (287.41,433.19) circle (  1.96);
\path[draw=drawColor,line width= 0.4pt,line join=round,line cap=round,fill=fillColor] (287.83,433.41) circle (  1.96);
\path[draw=drawColor,line width= 0.4pt,line join=round,line cap=round,fill=fillColor] (288.25,433.62) circle (  1.96);
\path[draw=drawColor,line width= 0.4pt,line join=round,line cap=round,fill=fillColor] (288.67,433.83) circle (  1.96);
\path[draw=drawColor,line width= 0.4pt,line join=round,line cap=round,fill=fillColor] (289.09,434.04) circle (  1.96);
\path[draw=drawColor,line width= 0.4pt,line join=round,line cap=round,fill=fillColor] (289.51,434.25) circle (  1.96);
\path[draw=drawColor,line width= 0.4pt,line join=round,line cap=round,fill=fillColor] (289.93,434.46) circle (  1.96);
\path[draw=drawColor,line width= 0.4pt,line join=round,line cap=round,fill=fillColor] (290.35,434.67) circle (  1.96);
\path[draw=drawColor,line width= 0.4pt,line join=round,line cap=round,fill=fillColor] (290.77,434.87) circle (  1.96);
\path[draw=drawColor,line width= 0.4pt,line join=round,line cap=round,fill=fillColor] (291.19,435.07) circle (  1.96);
\path[draw=drawColor,line width= 0.4pt,line join=round,line cap=round,fill=fillColor] (291.61,435.28) circle (  1.96);
\path[draw=drawColor,line width= 0.4pt,line join=round,line cap=round,fill=fillColor] (292.03,435.48) circle (  1.96);
\path[draw=drawColor,line width= 0.4pt,line join=round,line cap=round,fill=fillColor] (292.45,435.69) circle (  1.96);
\path[draw=drawColor,line width= 0.4pt,line join=round,line cap=round,fill=fillColor] (292.86,435.89) circle (  1.96);
\path[draw=drawColor,line width= 0.4pt,line join=round,line cap=round,fill=fillColor] (293.28,436.09) circle (  1.96);
\path[draw=drawColor,line width= 0.4pt,line join=round,line cap=round,fill=fillColor] (293.70,436.29) circle (  1.96);
\path[draw=drawColor,line width= 0.4pt,line join=round,line cap=round,fill=fillColor] (294.12,436.49) circle (  1.96);
\path[draw=drawColor,line width= 0.4pt,line join=round,line cap=round,fill=fillColor] (294.54,436.69) circle (  1.96);
\path[draw=drawColor,line width= 0.4pt,line join=round,line cap=round,fill=fillColor] (294.96,436.88) circle (  1.96);
\path[draw=drawColor,line width= 0.4pt,line join=round,line cap=round,fill=fillColor] (295.38,437.08) circle (  1.96);
\path[draw=drawColor,line width= 0.4pt,line join=round,line cap=round,fill=fillColor] (295.80,437.28) circle (  1.96);
\path[draw=drawColor,line width= 0.4pt,line join=round,line cap=round,fill=fillColor] (296.22,437.47) circle (  1.96);
\path[draw=drawColor,line width= 0.4pt,line join=round,line cap=round,fill=fillColor] (296.64,437.66) circle (  1.96);
\path[draw=drawColor,line width= 0.4pt,line join=round,line cap=round,fill=fillColor] (297.06,437.86) circle (  1.96);
\path[draw=drawColor,line width= 0.4pt,line join=round,line cap=round,fill=fillColor] (297.48,438.05) circle (  1.96);
\path[draw=drawColor,line width= 0.4pt,line join=round,line cap=round,fill=fillColor] (297.90,438.24) circle (  1.96);
\path[draw=drawColor,line width= 0.4pt,line join=round,line cap=round,fill=fillColor] (298.31,438.43) circle (  1.96);
\path[draw=drawColor,line width= 0.4pt,line join=round,line cap=round,fill=fillColor] (298.73,438.62) circle (  1.96);
\path[draw=drawColor,line width= 0.4pt,line join=round,line cap=round,fill=fillColor] (299.15,438.81) circle (  1.96);
\path[draw=drawColor,line width= 0.4pt,line join=round,line cap=round,fill=fillColor] (299.57,439.00) circle (  1.96);
\path[draw=drawColor,line width= 0.4pt,line join=round,line cap=round,fill=fillColor] (299.99,439.18) circle (  1.96);
\path[draw=drawColor,line width= 0.4pt,line join=round,line cap=round,fill=fillColor] (300.41,439.37) circle (  1.96);
\path[draw=drawColor,line width= 0.4pt,line join=round,line cap=round,fill=fillColor] (300.83,439.56) circle (  1.96);
\path[draw=drawColor,line width= 0.4pt,line join=round,line cap=round,fill=fillColor] (301.25,439.74) circle (  1.96);
\path[draw=drawColor,line width= 0.4pt,line join=round,line cap=round,fill=fillColor] (301.67,439.92) circle (  1.96);
\path[draw=drawColor,line width= 0.4pt,line join=round,line cap=round,fill=fillColor] (302.09,440.11) circle (  1.96);
\path[draw=drawColor,line width= 0.4pt,line join=round,line cap=round,fill=fillColor] (302.51,440.29) circle (  1.96);
\path[draw=drawColor,line width= 0.4pt,line join=round,line cap=round,fill=fillColor] (302.93,440.47) circle (  1.96);
\path[draw=drawColor,line width= 0.4pt,line join=round,line cap=round,fill=fillColor] (303.35,440.65) circle (  1.96);
\path[draw=drawColor,line width= 0.4pt,line join=round,line cap=round,fill=fillColor] (303.76,440.83) circle (  1.96);
\path[draw=drawColor,line width= 0.4pt,line join=round,line cap=round,fill=fillColor] (304.18,441.01) circle (  1.96);
\path[draw=drawColor,line width= 0.4pt,line join=round,line cap=round,fill=fillColor] (304.60,441.19) circle (  1.96);
\path[draw=drawColor,line width= 0.4pt,line join=round,line cap=round,fill=fillColor] (305.02,441.36) circle (  1.96);
\path[draw=drawColor,line width= 0.4pt,line join=round,line cap=round,fill=fillColor] (305.44,441.54) circle (  1.96);
\path[draw=drawColor,line width= 0.4pt,line join=round,line cap=round,fill=fillColor] (305.86,441.72) circle (  1.96);
\path[draw=drawColor,line width= 0.4pt,line join=round,line cap=round,fill=fillColor] (306.28,441.89) circle (  1.96);
\path[draw=drawColor,line width= 0.4pt,line join=round,line cap=round,fill=fillColor] (306.70,442.06) circle (  1.96);
\path[draw=drawColor,line width= 0.4pt,line join=round,line cap=round,fill=fillColor] (307.12,442.24) circle (  1.96);
\path[draw=drawColor,line width= 0.4pt,line join=round,line cap=round,fill=fillColor] (307.54,442.41) circle (  1.96);
\path[draw=drawColor,line width= 0.4pt,line join=round,line cap=round,fill=fillColor] (307.96,442.58) circle (  1.96);
\path[draw=drawColor,line width= 0.4pt,line join=round,line cap=round,fill=fillColor] (308.38,442.75) circle (  1.96);
\path[draw=drawColor,line width= 0.4pt,line join=round,line cap=round,fill=fillColor] (308.80,442.92) circle (  1.96);
\path[draw=drawColor,line width= 0.4pt,line join=round,line cap=round,fill=fillColor] (309.21,443.09) circle (  1.96);
\path[draw=drawColor,line width= 0.4pt,line join=round,line cap=round,fill=fillColor] (309.63,443.26) circle (  1.96);
\path[draw=drawColor,line width= 0.4pt,line join=round,line cap=round,fill=fillColor] (310.05,443.43) circle (  1.96);
\path[draw=drawColor,line width= 0.4pt,line join=round,line cap=round,fill=fillColor] (310.47,443.59) circle (  1.96);
\path[draw=drawColor,line width= 0.4pt,line join=round,line cap=round,fill=fillColor] (310.89,443.76) circle (  1.96);
\path[draw=drawColor,line width= 0.4pt,line join=round,line cap=round,fill=fillColor] (311.31,443.93) circle (  1.96);
\path[draw=drawColor,line width= 0.4pt,line join=round,line cap=round,fill=fillColor] (311.73,444.09) circle (  1.96);
\path[draw=drawColor,line width= 0.4pt,line join=round,line cap=round,fill=fillColor] (312.15,444.25) circle (  1.96);
\path[draw=drawColor,line width= 0.4pt,line join=round,line cap=round,fill=fillColor] (312.57,444.42) circle (  1.96);
\path[draw=drawColor,line width= 0.4pt,line join=round,line cap=round,fill=fillColor] (312.99,444.58) circle (  1.96);
\path[draw=drawColor,line width= 0.4pt,line join=round,line cap=round,fill=fillColor] (313.41,444.74) circle (  1.96);
\path[draw=drawColor,line width= 0.4pt,line join=round,line cap=round,fill=fillColor] (313.83,444.90) circle (  1.96);
\path[draw=drawColor,line width= 0.4pt,line join=round,line cap=round,fill=fillColor] (314.25,445.06) circle (  1.96);
\path[draw=drawColor,line width= 0.4pt,line join=round,line cap=round,fill=fillColor] (314.67,445.22) circle (  1.96);
\path[draw=drawColor,line width= 0.4pt,line join=round,line cap=round,fill=fillColor] (315.08,445.38) circle (  1.96);
\path[draw=drawColor,line width= 0.4pt,line join=round,line cap=round,fill=fillColor] (315.50,445.54) circle (  1.96);
\path[draw=drawColor,line width= 0.4pt,line join=round,line cap=round,fill=fillColor] (315.92,445.70) circle (  1.96);
\path[draw=drawColor,line width= 0.4pt,line join=round,line cap=round,fill=fillColor] (316.34,445.85) circle (  1.96);
\path[draw=drawColor,line width= 0.4pt,line join=round,line cap=round,fill=fillColor] (316.76,446.01) circle (  1.96);
\path[draw=drawColor,line width= 0.4pt,line join=round,line cap=round,fill=fillColor] (317.18,446.16) circle (  1.96);
\path[draw=drawColor,line width= 0.4pt,line join=round,line cap=round,fill=fillColor] (317.60,446.32) circle (  1.96);
\path[draw=drawColor,line width= 0.4pt,line join=round,line cap=round,fill=fillColor] (318.02,446.47) circle (  1.96);
\path[draw=drawColor,line width= 0.4pt,line join=round,line cap=round,fill=fillColor] (318.44,446.62) circle (  1.96);
\path[draw=drawColor,line width= 0.4pt,line join=round,line cap=round,fill=fillColor] (318.86,446.78) circle (  1.96);
\path[draw=drawColor,line width= 0.4pt,line join=round,line cap=round,fill=fillColor] (319.28,446.93) circle (  1.96);
\path[draw=drawColor,line width= 0.4pt,line join=round,line cap=round,fill=fillColor] (319.70,447.08) circle (  1.96);
\path[draw=drawColor,line width= 0.4pt,line join=round,line cap=round,fill=fillColor] (320.12,447.23) circle (  1.96);
\path[draw=drawColor,line width= 0.4pt,line join=round,line cap=round,fill=fillColor] (320.53,447.38) circle (  1.96);
\path[draw=drawColor,line width= 0.4pt,line join=round,line cap=round,fill=fillColor] (320.95,447.53) circle (  1.96);
\path[draw=drawColor,line width= 0.4pt,line join=round,line cap=round,fill=fillColor] (321.37,447.68) circle (  1.96);
\path[draw=drawColor,line width= 0.4pt,line join=round,line cap=round,fill=fillColor] (321.79,447.83) circle (  1.96);
\path[draw=drawColor,line width= 0.4pt,line join=round,line cap=round,fill=fillColor] (322.21,447.97) circle (  1.96);
\path[draw=drawColor,line width= 0.4pt,line join=round,line cap=round,fill=fillColor] (322.63,448.12) circle (  1.96);
\path[draw=drawColor,line width= 0.4pt,line join=round,line cap=round,fill=fillColor] (323.05,448.26) circle (  1.96);
\path[draw=drawColor,line width= 0.4pt,line join=round,line cap=round,fill=fillColor] (323.47,448.41) circle (  1.96);
\path[draw=drawColor,line width= 0.4pt,line join=round,line cap=round,fill=fillColor] (323.89,448.55) circle (  1.96);
\path[draw=drawColor,line width= 0.4pt,line join=round,line cap=round,fill=fillColor] (324.31,448.70) circle (  1.96);
\path[draw=drawColor,line width= 0.4pt,line join=round,line cap=round,fill=fillColor] (324.73,448.84) circle (  1.96);
\path[draw=drawColor,line width= 0.4pt,line join=round,line cap=round,fill=fillColor] (325.15,448.98) circle (  1.96);
\path[draw=drawColor,line width= 0.4pt,line join=round,line cap=round,fill=fillColor] (325.57,449.13) circle (  1.96);
\path[draw=drawColor,line width= 0.4pt,line join=round,line cap=round,fill=fillColor] (325.98,449.27) circle (  1.96);
\path[draw=drawColor,line width= 0.4pt,line join=round,line cap=round,fill=fillColor] (326.40,449.41) circle (  1.96);
\path[draw=drawColor,line width= 0.4pt,line join=round,line cap=round,fill=fillColor] (326.82,449.54) circle (  1.96);
\path[draw=drawColor,line width= 0.4pt,line join=round,line cap=round,fill=fillColor] (327.24,449.69) circle (  1.96);
\path[draw=drawColor,line width= 0.4pt,line join=round,line cap=round,fill=fillColor] (327.66,449.83) circle (  1.96);
\path[draw=drawColor,line width= 0.4pt,line join=round,line cap=round,fill=fillColor] (328.08,449.96) circle (  1.96);
\path[draw=drawColor,line width= 0.4pt,line join=round,line cap=round,fill=fillColor] (328.50,450.10) circle (  1.96);
\path[draw=drawColor,line width= 0.4pt,line join=round,line cap=round,fill=fillColor] (328.92,450.23) circle (  1.96);
\path[draw=drawColor,line width= 0.4pt,line join=round,line cap=round,fill=fillColor] (329.34,450.37) circle (  1.96);
\path[draw=drawColor,line width= 0.4pt,line join=round,line cap=round,fill=fillColor] (329.76,450.51) circle (  1.96);
\path[draw=drawColor,line width= 0.4pt,line join=round,line cap=round,fill=fillColor] (330.18,450.64) circle (  1.96);
\path[draw=drawColor,line width= 0.4pt,line join=round,line cap=round,fill=fillColor] (330.60,450.78) circle (  1.96);
\path[draw=drawColor,line width= 0.4pt,line join=round,line cap=round,fill=fillColor] (331.02,450.91) circle (  1.96);
\path[draw=drawColor,line width= 0.4pt,line join=round,line cap=round,fill=fillColor] (331.43,451.04) circle (  1.96);
\path[draw=drawColor,line width= 0.4pt,line join=round,line cap=round,fill=fillColor] (331.85,451.18) circle (  1.96);
\path[draw=drawColor,line width= 0.4pt,line join=round,line cap=round,fill=fillColor] (332.27,451.31) circle (  1.96);
\path[draw=drawColor,line width= 0.4pt,line join=round,line cap=round,fill=fillColor] (332.69,451.44) circle (  1.96);
\path[draw=drawColor,line width= 0.4pt,line join=round,line cap=round,fill=fillColor] (333.11,451.57) circle (  1.96);
\path[draw=drawColor,line width= 0.4pt,line join=round,line cap=round,fill=fillColor] (333.53,451.70) circle (  1.96);
\path[draw=drawColor,line width= 0.4pt,line join=round,line cap=round,fill=fillColor] (333.95,451.83) circle (  1.96);
\path[draw=drawColor,line width= 0.4pt,line join=round,line cap=round,fill=fillColor] (334.37,451.96) circle (  1.96);
\path[draw=drawColor,line width= 0.4pt,line join=round,line cap=round,fill=fillColor] (334.79,452.09) circle (  1.96);
\path[draw=drawColor,line width= 0.4pt,line join=round,line cap=round,fill=fillColor] (335.21,452.21) circle (  1.96);
\path[draw=drawColor,line width= 0.4pt,line join=round,line cap=round,fill=fillColor] (335.63,452.34) circle (  1.96);
\path[draw=drawColor,line width= 0.4pt,line join=round,line cap=round,fill=fillColor] (336.05,452.47) circle (  1.96);
\path[draw=drawColor,line width= 0.4pt,line join=round,line cap=round,fill=fillColor] (336.47,452.59) circle (  1.96);
\path[draw=drawColor,line width= 0.4pt,line join=round,line cap=round,fill=fillColor] (336.89,452.72) circle (  1.96);
\path[draw=drawColor,line width= 0.4pt,line join=round,line cap=round,fill=fillColor] (337.30,452.84) circle (  1.96);
\path[draw=drawColor,line width= 0.4pt,line join=round,line cap=round,fill=fillColor] (337.72,452.97) circle (  1.96);
\path[draw=drawColor,line width= 0.4pt,line join=round,line cap=round,fill=fillColor] (338.14,453.09) circle (  1.96);
\path[draw=drawColor,line width= 0.4pt,line join=round,line cap=round,fill=fillColor] (338.56,453.21) circle (  1.96);
\path[draw=drawColor,line width= 0.4pt,line join=round,line cap=round,fill=fillColor] (338.98,453.33) circle (  1.96);
\path[draw=drawColor,line width= 0.4pt,line join=round,line cap=round,fill=fillColor] (339.40,453.46) circle (  1.96);
\path[draw=drawColor,line width= 0.4pt,line join=round,line cap=round,fill=fillColor] (339.82,453.58) circle (  1.96);
\path[draw=drawColor,line width= 0.4pt,line join=round,line cap=round,fill=fillColor] (340.24,453.70) circle (  1.96);
\path[draw=drawColor,line width= 0.4pt,line join=round,line cap=round,fill=fillColor] (340.66,453.82) circle (  1.96);
\path[draw=drawColor,line width= 0.4pt,line join=round,line cap=round,fill=fillColor] (341.08,453.94) circle (  1.96);
\path[draw=drawColor,line width= 0.4pt,line join=round,line cap=round,fill=fillColor] (341.50,454.06) circle (  1.96);
\path[draw=drawColor,line width= 0.4pt,line join=round,line cap=round,fill=fillColor] (341.92,454.18) circle (  1.96);
\path[draw=drawColor,line width= 0.4pt,line join=round,line cap=round,fill=fillColor] (342.34,454.30) circle (  1.96);
\path[draw=drawColor,line width= 0.4pt,line join=round,line cap=round,fill=fillColor] (342.75,454.41) circle (  1.96);
\path[draw=drawColor,line width= 0.4pt,line join=round,line cap=round,fill=fillColor] (343.17,454.53) circle (  1.96);
\path[draw=drawColor,line width= 0.4pt,line join=round,line cap=round,fill=fillColor] (343.59,454.65) circle (  1.96);
\path[draw=drawColor,line width= 0.4pt,line join=round,line cap=round,fill=fillColor] (344.01,454.76) circle (  1.96);
\path[draw=drawColor,line width= 0.4pt,line join=round,line cap=round,fill=fillColor] (344.43,454.88) circle (  1.96);
\path[draw=drawColor,line width= 0.4pt,line join=round,line cap=round,fill=fillColor] (344.85,454.99) circle (  1.96);
\path[draw=drawColor,line width= 0.4pt,line join=round,line cap=round,fill=fillColor] (345.27,455.11) circle (  1.96);
\path[draw=drawColor,line width= 0.4pt,line join=round,line cap=round,fill=fillColor] (345.69,455.22) circle (  1.96);
\path[draw=drawColor,line width= 0.4pt,line join=round,line cap=round,fill=fillColor] (346.11,455.33) circle (  1.96);
\path[draw=drawColor,line width= 0.4pt,line join=round,line cap=round,fill=fillColor] (346.53,455.45) circle (  1.96);
\path[draw=drawColor,line width= 0.4pt,line join=round,line cap=round,fill=fillColor] (346.95,455.56) circle (  1.96);
\path[draw=drawColor,line width= 0.4pt,line join=round,line cap=round,fill=fillColor] (347.37,455.67) circle (  1.96);
\path[draw=drawColor,line width= 0.4pt,line join=round,line cap=round,fill=fillColor] (347.79,455.78) circle (  1.96);
\path[draw=drawColor,line width= 0.4pt,line join=round,line cap=round,fill=fillColor] (348.20,455.89) circle (  1.96);
\path[draw=drawColor,line width= 0.4pt,line join=round,line cap=round,fill=fillColor] (348.62,456.00) circle (  1.96);
\path[draw=drawColor,line width= 0.4pt,line join=round,line cap=round,fill=fillColor] (349.04,456.11) circle (  1.96);
\path[draw=drawColor,line width= 0.4pt,line join=round,line cap=round,fill=fillColor] (349.46,456.22) circle (  1.96);
\path[draw=drawColor,line width= 0.4pt,line join=round,line cap=round,fill=fillColor] (349.88,456.33) circle (  1.96);
\path[draw=drawColor,line width= 0.4pt,line join=round,line cap=round,fill=fillColor] (350.30,456.44) circle (  1.96);
\path[draw=drawColor,line width= 0.4pt,line join=round,line cap=round,fill=fillColor] (350.72,456.54) circle (  1.96);
\path[draw=drawColor,line width= 0.4pt,line join=round,line cap=round,fill=fillColor] (351.14,456.65) circle (  1.96);
\path[draw=drawColor,line width= 0.4pt,line join=round,line cap=round,fill=fillColor] (351.56,456.76) circle (  1.96);
\path[draw=drawColor,line width= 0.4pt,line join=round,line cap=round,fill=fillColor] (351.98,456.86) circle (  1.96);
\path[draw=drawColor,line width= 0.4pt,line join=round,line cap=round,fill=fillColor] (352.40,456.97) circle (  1.96);
\path[draw=drawColor,line width= 0.4pt,line join=round,line cap=round,fill=fillColor] (352.82,457.07) circle (  1.96);
\path[draw=drawColor,line width= 0.4pt,line join=round,line cap=round,fill=fillColor] (353.24,457.18) circle (  1.96);
\path[draw=drawColor,line width= 0.4pt,line join=round,line cap=round,fill=fillColor] (353.65,457.28) circle (  1.96);
\path[draw=drawColor,line width= 0.4pt,line join=round,line cap=round,fill=fillColor] (354.07,457.39) circle (  1.96);
\path[draw=drawColor,line width= 0.4pt,line join=round,line cap=round,fill=fillColor] (354.49,457.49) circle (  1.96);
\path[draw=drawColor,line width= 0.4pt,line join=round,line cap=round,fill=fillColor] (354.91,457.59) circle (  1.96);
\path[draw=drawColor,line width= 0.4pt,line join=round,line cap=round,fill=fillColor] (355.33,457.69) circle (  1.96);
\path[draw=drawColor,line width= 0.4pt,line join=round,line cap=round,fill=fillColor] (355.75,457.80) circle (  1.96);
\path[draw=drawColor,line width= 0.4pt,line join=round,line cap=round,fill=fillColor] (356.17,457.90) circle (  1.96);
\path[draw=drawColor,line width= 0.4pt,line join=round,line cap=round,fill=fillColor] (356.59,458.00) circle (  1.96);
\path[draw=drawColor,line width= 0.4pt,line join=round,line cap=round,fill=fillColor] (357.01,458.10) circle (  1.96);
\path[draw=drawColor,line width= 0.4pt,line join=round,line cap=round,fill=fillColor] (357.43,458.20) circle (  1.96);
\path[draw=drawColor,line width= 0.4pt,line join=round,line cap=round,fill=fillColor] (357.85,458.30) circle (  1.96);
\path[draw=drawColor,line width= 0.4pt,line join=round,line cap=round,fill=fillColor] (358.27,458.40) circle (  1.96);
\path[draw=drawColor,line width= 0.4pt,line join=round,line cap=round,fill=fillColor] (358.69,458.50) circle (  1.96);
\path[draw=drawColor,line width= 0.4pt,line join=round,line cap=round,fill=fillColor] (359.10,458.60) circle (  1.96);
\path[draw=drawColor,line width= 0.4pt,line join=round,line cap=round,fill=fillColor] (359.52,458.69) circle (  1.96);
\path[draw=drawColor,line width= 0.4pt,line join=round,line cap=round,fill=fillColor] (359.94,458.79) circle (  1.96);
\path[draw=drawColor,line width= 0.4pt,line join=round,line cap=round,fill=fillColor] (360.36,458.89) circle (  1.96);
\path[draw=drawColor,line width= 0.4pt,line join=round,line cap=round,fill=fillColor] (360.78,458.98) circle (  1.96);
\path[draw=drawColor,line width= 0.4pt,line join=round,line cap=round,fill=fillColor] (361.20,459.08) circle (  1.96);
\path[draw=drawColor,line width= 0.4pt,line join=round,line cap=round,fill=fillColor] (361.62,459.18) circle (  1.96);
\path[draw=drawColor,line width= 0.4pt,line join=round,line cap=round,fill=fillColor] (362.04,459.27) circle (  1.96);
\path[draw=drawColor,line width= 0.4pt,line join=round,line cap=round,fill=fillColor] (362.46,459.36) circle (  1.96);
\path[draw=drawColor,line width= 0.4pt,line join=round,line cap=round,fill=fillColor] (362.88,459.46) circle (  1.96);
\path[draw=drawColor,line width= 0.4pt,line join=round,line cap=round,fill=fillColor] (363.30,459.55) circle (  1.96);
\path[draw=drawColor,line width= 0.4pt,line join=round,line cap=round,fill=fillColor] (363.72,459.64) circle (  1.96);
\path[draw=drawColor,line width= 0.4pt,line join=round,line cap=round,fill=fillColor] (364.14,459.74) circle (  1.96);
\path[draw=drawColor,line width= 0.4pt,line join=round,line cap=round,fill=fillColor] (364.56,459.83) circle (  1.96);
\path[draw=drawColor,line width= 0.4pt,line join=round,line cap=round,fill=fillColor] (364.97,459.92) circle (  1.96);
\path[draw=drawColor,line width= 0.4pt,line join=round,line cap=round,fill=fillColor] (365.39,460.01) circle (  1.96);
\path[draw=drawColor,line width= 0.4pt,line join=round,line cap=round,fill=fillColor] (365.81,460.10) circle (  1.96);
\path[draw=drawColor,line width= 0.4pt,line join=round,line cap=round,fill=fillColor] (366.23,460.19) circle (  1.96);
\path[draw=drawColor,line width= 0.4pt,line join=round,line cap=round,fill=fillColor] (366.65,460.29) circle (  1.96);
\path[draw=drawColor,line width= 0.4pt,line join=round,line cap=round,fill=fillColor] (367.07,460.38) circle (  1.96);
\path[draw=drawColor,line width= 0.4pt,line join=round,line cap=round,fill=fillColor] (367.49,460.46) circle (  1.96);
\path[draw=drawColor,line width= 0.4pt,line join=round,line cap=round,fill=fillColor] (367.91,460.55) circle (  1.96);
\path[draw=drawColor,line width= 0.4pt,line join=round,line cap=round,fill=fillColor] (368.33,460.64) circle (  1.96);
\path[draw=drawColor,line width= 0.4pt,line join=round,line cap=round,fill=fillColor] (368.75,460.73) circle (  1.96);
\path[draw=drawColor,line width= 0.4pt,line join=round,line cap=round,fill=fillColor] (369.17,460.82) circle (  1.96);
\path[draw=drawColor,line width= 0.4pt,line join=round,line cap=round,fill=fillColor] (369.59,460.90) circle (  1.96);
\path[draw=drawColor,line width= 0.4pt,line join=round,line cap=round,fill=fillColor] (370.01,460.99) circle (  1.96);
\path[draw=drawColor,line width= 0.4pt,line join=round,line cap=round,fill=fillColor] (370.42,461.08) circle (  1.96);
\path[draw=drawColor,line width= 0.4pt,line join=round,line cap=round,fill=fillColor] (370.84,461.16) circle (  1.96);
\path[draw=drawColor,line width= 0.4pt,line join=round,line cap=round,fill=fillColor] (371.26,461.25) circle (  1.96);
\path[draw=drawColor,line width= 0.4pt,line join=round,line cap=round,fill=fillColor] (371.68,461.33) circle (  1.96);
\path[draw=drawColor,line width= 0.4pt,line join=round,line cap=round,fill=fillColor] (372.10,461.42) circle (  1.96);
\path[draw=drawColor,line width= 0.4pt,line join=round,line cap=round,fill=fillColor] (372.52,461.50) circle (  1.96);
\path[draw=drawColor,line width= 0.4pt,line join=round,line cap=round,fill=fillColor] (372.94,461.59) circle (  1.96);
\path[draw=drawColor,line width= 0.4pt,line join=round,line cap=round,fill=fillColor] (373.36,461.67) circle (  1.96);
\path[draw=drawColor,line width= 0.4pt,line join=round,line cap=round,fill=fillColor] (373.78,461.75) circle (  1.96);
\path[draw=drawColor,line width= 0.4pt,line join=round,line cap=round,fill=fillColor] (374.20,461.84) circle (  1.96);
\path[draw=drawColor,line width= 0.4pt,line join=round,line cap=round,fill=fillColor] (374.62,461.92) circle (  1.96);
\path[draw=drawColor,line width= 0.4pt,line join=round,line cap=round,fill=fillColor] (375.04,462.00) circle (  1.96);
\path[draw=drawColor,line width= 0.4pt,line join=round,line cap=round,fill=fillColor] (375.46,462.09) circle (  1.96);
\path[draw=drawColor,line width= 0.4pt,line join=round,line cap=round,fill=fillColor] (375.87,462.17) circle (  1.96);
\path[draw=drawColor,line width= 0.4pt,line join=round,line cap=round,fill=fillColor] (376.29,462.25) circle (  1.96);
\path[draw=drawColor,line width= 0.4pt,line join=round,line cap=round,fill=fillColor] (376.71,462.33) circle (  1.96);
\path[draw=drawColor,line width= 0.4pt,line join=round,line cap=round,fill=fillColor] (377.13,462.41) circle (  1.96);
\path[draw=drawColor,line width= 0.4pt,line join=round,line cap=round,fill=fillColor] (377.55,462.49) circle (  1.96);
\path[draw=drawColor,line width= 0.4pt,line join=round,line cap=round,fill=fillColor] (377.97,462.57) circle (  1.96);
\path[draw=drawColor,line width= 0.4pt,line join=round,line cap=round,fill=fillColor] (378.39,462.65) circle (  1.96);
\path[draw=drawColor,line width= 0.4pt,line join=round,line cap=round,fill=fillColor] (378.81,462.73) circle (  1.96);
\path[draw=drawColor,line width= 0.4pt,line join=round,line cap=round,fill=fillColor] (379.23,462.81) circle (  1.96);
\path[draw=drawColor,line width= 0.4pt,line join=round,line cap=round,fill=fillColor] (379.65,462.88) circle (  1.96);
\path[draw=drawColor,line width= 0.4pt,line join=round,line cap=round,fill=fillColor] (380.07,462.96) circle (  1.96);
\path[draw=drawColor,line width= 0.4pt,line join=round,line cap=round,fill=fillColor] (380.49,463.04) circle (  1.96);
\path[draw=drawColor,line width= 0.4pt,line join=round,line cap=round,fill=fillColor] (380.91,463.11) circle (  1.96);
\path[draw=drawColor,line width= 0.4pt,line join=round,line cap=round,fill=fillColor] (381.32,463.19) circle (  1.96);
\path[draw=drawColor,line width= 0.4pt,line join=round,line cap=round,fill=fillColor] (381.74,463.27) circle (  1.96);
\path[draw=drawColor,line width= 0.4pt,line join=round,line cap=round,fill=fillColor] (382.16,463.34) circle (  1.96);
\path[draw=drawColor,line width= 0.4pt,line join=round,line cap=round,fill=fillColor] (382.58,463.42) circle (  1.96);
\path[draw=drawColor,line width= 0.4pt,line join=round,line cap=round,fill=fillColor] (383.00,463.50) circle (  1.96);
\path[draw=drawColor,line width= 0.4pt,line join=round,line cap=round,fill=fillColor] (383.42,463.57) circle (  1.96);
\path[draw=drawColor,line width= 0.4pt,line join=round,line cap=round,fill=fillColor] (383.84,463.65) circle (  1.96);
\path[draw=drawColor,line width= 0.4pt,line join=round,line cap=round,fill=fillColor] (384.26,463.72) circle (  1.96);
\path[draw=drawColor,line width= 0.4pt,line join=round,line cap=round,fill=fillColor] (384.68,463.79) circle (  1.96);
\path[draw=drawColor,line width= 0.4pt,line join=round,line cap=round,fill=fillColor] (385.10,463.87) circle (  1.96);
\path[draw=drawColor,line width= 0.4pt,line join=round,line cap=round,fill=fillColor] (385.52,463.94) circle (  1.96);
\path[draw=drawColor,line width= 0.4pt,line join=round,line cap=round,fill=fillColor] (385.94,464.01) circle (  1.96);
\path[draw=drawColor,line width= 0.4pt,line join=round,line cap=round,fill=fillColor] (386.36,464.08) circle (  1.96);
\path[draw=drawColor,line width= 0.4pt,line join=round,line cap=round,fill=fillColor] (386.78,464.16) circle (  1.96);
\path[draw=drawColor,line width= 0.4pt,line join=round,line cap=round,fill=fillColor] (387.19,464.23) circle (  1.96);
\path[draw=drawColor,line width= 0.4pt,line join=round,line cap=round,fill=fillColor] (387.61,464.30) circle (  1.96);
\path[draw=drawColor,line width= 0.4pt,line join=round,line cap=round,fill=fillColor] (388.03,464.37) circle (  1.96);
\path[draw=drawColor,line width= 0.4pt,line join=round,line cap=round,fill=fillColor] (388.45,464.44) circle (  1.96);
\path[draw=drawColor,line width= 0.4pt,line join=round,line cap=round,fill=fillColor] (388.87,464.51) circle (  1.96);
\path[draw=drawColor,line width= 0.4pt,line join=round,line cap=round,fill=fillColor] (389.29,464.58) circle (  1.96);
\path[draw=drawColor,line width= 0.4pt,line join=round,line cap=round,fill=fillColor] (389.71,464.65) circle (  1.96);
\path[draw=drawColor,line width= 0.4pt,line join=round,line cap=round,fill=fillColor] (390.13,464.72) circle (  1.96);
\path[draw=drawColor,line width= 0.4pt,line join=round,line cap=round,fill=fillColor] (390.55,464.79) circle (  1.96);
\path[draw=drawColor,line width= 0.4pt,line join=round,line cap=round,fill=fillColor] (390.97,464.86) circle (  1.96);
\path[draw=drawColor,line width= 0.4pt,line join=round,line cap=round,fill=fillColor] (391.39,464.93) circle (  1.96);
\path[draw=drawColor,line width= 0.4pt,line join=round,line cap=round,fill=fillColor] (391.81,464.99) circle (  1.96);
\path[draw=drawColor,line width= 0.4pt,line join=round,line cap=round,fill=fillColor] (392.23,465.06) circle (  1.96);
\path[draw=drawColor,line width= 0.4pt,line join=round,line cap=round,fill=fillColor] (392.64,465.13) circle (  1.96);
\path[draw=drawColor,line width= 0.4pt,line join=round,line cap=round,fill=fillColor] (393.06,465.20) circle (  1.96);
\path[draw=drawColor,line width= 0.4pt,line join=round,line cap=round,fill=fillColor] (393.48,465.27) circle (  1.96);
\path[draw=drawColor,line width= 0.4pt,line join=round,line cap=round,fill=fillColor] (393.90,465.33) circle (  1.96);
\path[draw=drawColor,line width= 0.4pt,line join=round,line cap=round,fill=fillColor] (394.32,465.40) circle (  1.96);
\path[draw=drawColor,line width= 0.4pt,line join=round,line cap=round,fill=fillColor] (394.74,465.46) circle (  1.96);
\path[draw=drawColor,line width= 0.4pt,line join=round,line cap=round,fill=fillColor] (395.16,465.53) circle (  1.96);
\path[draw=drawColor,line width= 0.4pt,line join=round,line cap=round,fill=fillColor] (395.58,465.60) circle (  1.96);
\path[draw=drawColor,line width= 0.4pt,line join=round,line cap=round,fill=fillColor] (396.00,465.66) circle (  1.96);
\path[draw=drawColor,line width= 0.4pt,line join=round,line cap=round,fill=fillColor] (396.42,465.73) circle (  1.96);
\path[draw=drawColor,line width= 0.4pt,line join=round,line cap=round,fill=fillColor] (396.84,465.79) circle (  1.96);
\path[draw=drawColor,line width= 0.4pt,line join=round,line cap=round,fill=fillColor] (397.26,465.85) circle (  1.96);
\path[draw=drawColor,line width= 0.4pt,line join=round,line cap=round,fill=fillColor] (397.68,465.92) circle (  1.96);
\path[draw=drawColor,line width= 0.4pt,line join=round,line cap=round,fill=fillColor] (398.09,465.98) circle (  1.96);
\path[draw=drawColor,line width= 0.4pt,line join=round,line cap=round,fill=fillColor] (398.51,466.05) circle (  1.96);
\path[draw=drawColor,line width= 0.4pt,line join=round,line cap=round,fill=fillColor] (398.93,466.11) circle (  1.96);
\path[draw=drawColor,line width= 0.4pt,line join=round,line cap=round,fill=fillColor] (399.35,466.17) circle (  1.96);
\path[draw=drawColor,line width= 0.4pt,line join=round,line cap=round,fill=fillColor] (399.77,466.23) circle (  1.96);
\path[draw=drawColor,line width= 0.4pt,line join=round,line cap=round,fill=fillColor] (400.19,466.29) circle (  1.96);
\path[draw=drawColor,line width= 0.4pt,line join=round,line cap=round,fill=fillColor] (400.61,466.36) circle (  1.96);
\path[draw=drawColor,line width= 0.4pt,line join=round,line cap=round,fill=fillColor] (401.03,466.42) circle (  1.96);
\path[draw=drawColor,line width= 0.4pt,line join=round,line cap=round,fill=fillColor] (401.45,466.48) circle (  1.96);
\path[draw=drawColor,line width= 0.4pt,line join=round,line cap=round,fill=fillColor] (401.87,466.54) circle (  1.96);
\path[draw=drawColor,line width= 0.4pt,line join=round,line cap=round,fill=fillColor] (402.29,466.60) circle (  1.96);
\path[draw=drawColor,line width= 0.4pt,line join=round,line cap=round,fill=fillColor] (402.71,466.66) circle (  1.96);
\path[draw=drawColor,line width= 0.4pt,line join=round,line cap=round,fill=fillColor] (403.13,466.72) circle (  1.96);
\path[draw=drawColor,line width= 0.4pt,line join=round,line cap=round,fill=fillColor] (403.54,466.78) circle (  1.96);
\path[draw=drawColor,line width= 0.4pt,line join=round,line cap=round,fill=fillColor] (403.96,466.84) circle (  1.96);
\path[draw=drawColor,line width= 0.4pt,line join=round,line cap=round,fill=fillColor] (404.38,466.90) circle (  1.96);
\path[draw=drawColor,line width= 0.4pt,line join=round,line cap=round,fill=fillColor] (404.80,466.96) circle (  1.96);
\path[draw=drawColor,line width= 0.4pt,line join=round,line cap=round,fill=fillColor] (405.22,467.02) circle (  1.96);
\path[draw=drawColor,line width= 0.4pt,line join=round,line cap=round,fill=fillColor] (405.64,467.08) circle (  1.96);
\path[draw=drawColor,line width= 0.4pt,line join=round,line cap=round,fill=fillColor] (406.06,467.13) circle (  1.96);
\path[draw=drawColor,line width= 0.4pt,line join=round,line cap=round,fill=fillColor] (406.48,467.19) circle (  1.96);
\path[draw=drawColor,line width= 0.4pt,line join=round,line cap=round,fill=fillColor] (406.90,467.25) circle (  1.96);
\path[draw=drawColor,line width= 0.4pt,line join=round,line cap=round,fill=fillColor] (407.32,467.31) circle (  1.96);
\path[draw=drawColor,line width= 0.4pt,line join=round,line cap=round,fill=fillColor] (407.74,467.36) circle (  1.96);
\path[draw=drawColor,line width= 0.4pt,line join=round,line cap=round,fill=fillColor] (408.16,467.42) circle (  1.96);
\path[draw=drawColor,line width= 0.4pt,line join=round,line cap=round,fill=fillColor] (408.58,467.48) circle (  1.96);
\path[draw=drawColor,line width= 0.4pt,line join=round,line cap=round,fill=fillColor] (408.99,467.53) circle (  1.96);
\path[draw=drawColor,line width= 0.4pt,line join=round,line cap=round,fill=fillColor] (409.41,467.59) circle (  1.96);
\path[draw=drawColor,line width= 0.4pt,line join=round,line cap=round,fill=fillColor] (409.83,467.64) circle (  1.96);
\path[draw=drawColor,line width= 0.4pt,line join=round,line cap=round,fill=fillColor] (410.25,467.70) circle (  1.96);
\path[draw=drawColor,line width= 0.4pt,line join=round,line cap=round,fill=fillColor] (410.67,467.75) circle (  1.96);
\path[draw=drawColor,line width= 0.4pt,line join=round,line cap=round,fill=fillColor] (411.09,467.81) circle (  1.96);
\path[draw=drawColor,line width= 0.4pt,line join=round,line cap=round,fill=fillColor] (411.51,467.86) circle (  1.96);
\path[draw=drawColor,line width= 0.4pt,line join=round,line cap=round,fill=fillColor] (411.93,467.92) circle (  1.96);
\path[draw=drawColor,line width= 0.4pt,line join=round,line cap=round,fill=fillColor] (412.35,467.97) circle (  1.96);
\path[draw=drawColor,line width= 0.4pt,line join=round,line cap=round,fill=fillColor] (412.77,468.03) circle (  1.96);
\path[draw=drawColor,line width= 0.4pt,line join=round,line cap=round,fill=fillColor] (413.19,468.08) circle (  1.96);
\path[draw=drawColor,line width= 0.4pt,line join=round,line cap=round,fill=fillColor] (413.61,468.13) circle (  1.96);
\path[draw=drawColor,line width= 0.4pt,line join=round,line cap=round,fill=fillColor] (414.03,468.19) circle (  1.96);
\path[draw=drawColor,line width= 0.4pt,line join=round,line cap=round,fill=fillColor] (414.45,468.24) circle (  1.96);
\path[draw=drawColor,line width= 0.4pt,line join=round,line cap=round,fill=fillColor] (414.86,468.29) circle (  1.96);
\path[draw=drawColor,line width= 0.4pt,line join=round,line cap=round,fill=fillColor] (415.28,468.35) circle (  1.96);
\path[draw=drawColor,line width= 0.4pt,line join=round,line cap=round,fill=fillColor] (415.70,468.40) circle (  1.96);
\path[draw=drawColor,line width= 0.4pt,line join=round,line cap=round,fill=fillColor] (416.12,468.45) circle (  1.96);
\path[draw=drawColor,line width= 0.4pt,line join=round,line cap=round,fill=fillColor] (416.54,468.50) circle (  1.96);
\path[draw=drawColor,line width= 0.4pt,line join=round,line cap=round,fill=fillColor] (416.96,468.55) circle (  1.96);
\path[draw=drawColor,line width= 0.4pt,line join=round,line cap=round,fill=fillColor] (417.38,468.60) circle (  1.96);
\path[draw=drawColor,line width= 0.4pt,line join=round,line cap=round,fill=fillColor] (417.80,468.66) circle (  1.96);
\path[draw=drawColor,line width= 0.4pt,line join=round,line cap=round,fill=fillColor] (418.22,468.71) circle (  1.96);
\path[draw=drawColor,line width= 0.4pt,line join=round,line cap=round,fill=fillColor] (418.64,468.75) circle (  1.96);
\path[draw=drawColor,line width= 0.4pt,line join=round,line cap=round,fill=fillColor] (419.06,468.81) circle (  1.96);
\path[draw=drawColor,line width= 0.4pt,line join=round,line cap=round,fill=fillColor] (419.48,468.86) circle (  1.96);
\path[draw=drawColor,line width= 0.4pt,line join=round,line cap=round,fill=fillColor] (419.90,468.91) circle (  1.96);
\path[draw=drawColor,line width= 0.4pt,line join=round,line cap=round,fill=fillColor] (420.31,468.95) circle (  1.96);
\path[draw=drawColor,line width= 0.4pt,line join=round,line cap=round,fill=fillColor] (420.73,469.01) circle (  1.96);
\path[draw=drawColor,line width= 0.4pt,line join=round,line cap=round,fill=fillColor] (421.15,469.05) circle (  1.96);
\path[draw=drawColor,line width= 0.4pt,line join=round,line cap=round,fill=fillColor] (421.57,469.10) circle (  1.96);
\path[draw=drawColor,line width= 0.4pt,line join=round,line cap=round,fill=fillColor] (421.99,469.15) circle (  1.96);
\path[draw=drawColor,line width= 0.4pt,line join=round,line cap=round,fill=fillColor] (422.41,469.20) circle (  1.96);
\path[draw=drawColor,line width= 0.4pt,line join=round,line cap=round,fill=fillColor] (422.83,469.25) circle (  1.96);
\path[draw=drawColor,line width= 0.4pt,line join=round,line cap=round,fill=fillColor] (423.25,469.29) circle (  1.96);
\path[draw=drawColor,line width= 0.4pt,line join=round,line cap=round,fill=fillColor] (423.67,469.34) circle (  1.96);
\path[draw=drawColor,line width= 0.4pt,line join=round,line cap=round,fill=fillColor] (424.09,469.39) circle (  1.96);
\path[draw=drawColor,line width= 0.4pt,line join=round,line cap=round,fill=fillColor] (424.51,469.44) circle (  1.96);
\path[draw=drawColor,line width= 0.4pt,line join=round,line cap=round,fill=fillColor] (424.93,469.48) circle (  1.96);
\path[draw=drawColor,line width= 0.4pt,line join=round,line cap=round,fill=fillColor] (425.35,469.53) circle (  1.96);
\path[draw=drawColor,line width= 0.4pt,line join=round,line cap=round,fill=fillColor] (425.76,469.58) circle (  1.96);
\path[draw=drawColor,line width= 0.4pt,line join=round,line cap=round,fill=fillColor] (426.18,469.62) circle (  1.96);
\path[draw=drawColor,line width= 0.4pt,line join=round,line cap=round,fill=fillColor] (426.60,469.67) circle (  1.96);
\path[draw=drawColor,line width= 0.4pt,line join=round,line cap=round,fill=fillColor] (427.02,469.72) circle (  1.96);
\path[draw=drawColor,line width= 0.4pt,line join=round,line cap=round,fill=fillColor] (427.44,469.76) circle (  1.96);
\path[draw=drawColor,line width= 0.4pt,line join=round,line cap=round,fill=fillColor] (427.86,469.81) circle (  1.96);
\path[draw=drawColor,line width= 0.4pt,line join=round,line cap=round,fill=fillColor] (428.28,469.85) circle (  1.96);
\path[draw=drawColor,line width= 0.4pt,line join=round,line cap=round,fill=fillColor] (428.70,469.89) circle (  1.96);
\path[draw=drawColor,line width= 0.4pt,line join=round,line cap=round,fill=fillColor] (429.12,469.94) circle (  1.96);
\path[draw=drawColor,line width= 0.4pt,line join=round,line cap=round,fill=fillColor] (429.54,469.98) circle (  1.96);
\path[draw=drawColor,line width= 0.4pt,line join=round,line cap=round,fill=fillColor] (429.96,470.03) circle (  1.96);
\path[draw=drawColor,line width= 0.4pt,line join=round,line cap=round,fill=fillColor] (430.38,470.07) circle (  1.96);
\path[draw=drawColor,line width= 0.4pt,line join=round,line cap=round,fill=fillColor] (430.80,470.12) circle (  1.96);
\path[draw=drawColor,line width= 0.4pt,line join=round,line cap=round,fill=fillColor] (431.21,470.16) circle (  1.96);
\path[draw=drawColor,line width= 0.4pt,line join=round,line cap=round,fill=fillColor] (431.63,470.21) circle (  1.96);
\path[draw=drawColor,line width= 0.4pt,line join=round,line cap=round,fill=fillColor] (432.05,470.25) circle (  1.96);
\path[draw=drawColor,line width= 0.4pt,line join=round,line cap=round,fill=fillColor] (432.47,470.29) circle (  1.96);
\path[draw=drawColor,line width= 0.4pt,line join=round,line cap=round,fill=fillColor] (432.89,470.33) circle (  1.96);
\path[draw=drawColor,line width= 0.4pt,line join=round,line cap=round,fill=fillColor] (433.31,470.38) circle (  1.96);
\path[draw=drawColor,line width= 0.4pt,line join=round,line cap=round,fill=fillColor] (433.73,470.42) circle (  1.96);
\path[draw=drawColor,line width= 0.4pt,line join=round,line cap=round,fill=fillColor] (434.15,470.46) circle (  1.96);
\path[draw=drawColor,line width= 0.4pt,line join=round,line cap=round,fill=fillColor] (434.57,470.50) circle (  1.96);
\path[draw=drawColor,line width= 0.4pt,line join=round,line cap=round,fill=fillColor] (434.99,470.55) circle (  1.96);
\path[draw=drawColor,line width= 0.4pt,line join=round,line cap=round,fill=fillColor] (435.41,470.59) circle (  1.96);
\path[draw=drawColor,line width= 0.4pt,line join=round,line cap=round,fill=fillColor] (435.83,470.63) circle (  1.96);
\path[draw=drawColor,line width= 0.4pt,line join=round,line cap=round,fill=fillColor] (436.25,470.67) circle (  1.96);
\path[draw=drawColor,line width= 0.4pt,line join=round,line cap=round,fill=fillColor] (436.66,470.71) circle (  1.96);
\path[draw=drawColor,line width= 0.4pt,line join=round,line cap=round,fill=fillColor] (437.08,470.75) circle (  1.96);
\path[draw=drawColor,line width= 0.4pt,line join=round,line cap=round,fill=fillColor] (437.50,470.79) circle (  1.96);
\path[draw=drawColor,line width= 0.4pt,line join=round,line cap=round,fill=fillColor] (437.92,470.84) circle (  1.96);
\path[draw=drawColor,line width= 0.4pt,line join=round,line cap=round,fill=fillColor] (438.34,470.87) circle (  1.96);
\path[draw=drawColor,line width= 0.4pt,line join=round,line cap=round,fill=fillColor] (438.76,470.92) circle (  1.96);
\path[draw=drawColor,line width= 0.4pt,line join=round,line cap=round,fill=fillColor] (439.18,470.95) circle (  1.96);
\path[draw=drawColor,line width= 0.4pt,line join=round,line cap=round,fill=fillColor] (439.60,470.99) circle (  1.96);
\path[draw=drawColor,line width= 0.4pt,line join=round,line cap=round,fill=fillColor] (440.02,471.04) circle (  1.96);
\path[draw=drawColor,line width= 0.4pt,line join=round,line cap=round,fill=fillColor] (440.44,471.07) circle (  1.96);
\path[draw=drawColor,line width= 0.4pt,line join=round,line cap=round,fill=fillColor] (440.86,471.11) circle (  1.96);
\path[draw=drawColor,line width= 0.4pt,line join=round,line cap=round,fill=fillColor] (441.28,471.15) circle (  1.96);
\path[draw=drawColor,line width= 0.4pt,line join=round,line cap=round,fill=fillColor] (441.70,471.19) circle (  1.96);
\path[draw=drawColor,line width= 0.4pt,line join=round,line cap=round,fill=fillColor] (442.12,471.23) circle (  1.96);
\path[draw=drawColor,line width= 0.4pt,line join=round,line cap=round,fill=fillColor] (442.53,471.27) circle (  1.96);
\path[draw=drawColor,line width= 0.4pt,line join=round,line cap=round,fill=fillColor] (442.95,471.30) circle (  1.96);
\path[draw=drawColor,line width= 0.4pt,line join=round,line cap=round,fill=fillColor] (443.37,471.34) circle (  1.96);
\path[draw=drawColor,line width= 0.4pt,line join=round,line cap=round,fill=fillColor] (443.79,471.38) circle (  1.96);
\path[draw=drawColor,line width= 0.4pt,line join=round,line cap=round,fill=fillColor] (444.21,471.42) circle (  1.96);
\path[draw=drawColor,line width= 0.4pt,line join=round,line cap=round,fill=fillColor] (444.63,471.46) circle (  1.96);
\path[draw=drawColor,line width= 0.4pt,line join=round,line cap=round,fill=fillColor] (445.05,471.50) circle (  1.96);
\path[draw=drawColor,line width= 0.4pt,line join=round,line cap=round,fill=fillColor] (445.47,471.53) circle (  1.96);
\path[draw=drawColor,line width= 0.4pt,line join=round,line cap=round,fill=fillColor] (445.89,471.57) circle (  1.96);
\path[draw=drawColor,line width= 0.4pt,line join=round,line cap=round,fill=fillColor] (446.31,471.61) circle (  1.96);
\path[draw=drawColor,line width= 0.4pt,line join=round,line cap=round,fill=fillColor] (446.73,471.64) circle (  1.96);
\path[draw=drawColor,line width= 0.4pt,line join=round,line cap=round,fill=fillColor] (447.15,471.68) circle (  1.96);
\path[draw=drawColor,line width= 0.4pt,line join=round,line cap=round,fill=fillColor] (447.57,471.72) circle (  1.96);
\path[draw=drawColor,line width= 0.4pt,line join=round,line cap=round,fill=fillColor] (447.98,471.75) circle (  1.96);
\path[draw=drawColor,line width= 0.4pt,line join=round,line cap=round,fill=fillColor] (448.40,471.79) circle (  1.96);
\path[draw=drawColor,line width= 0.4pt,line join=round,line cap=round,fill=fillColor] (448.82,471.82) circle (  1.96);
\path[draw=drawColor,line width= 0.4pt,line join=round,line cap=round,fill=fillColor] (449.24,471.86) circle (  1.96);
\path[draw=drawColor,line width= 0.4pt,line join=round,line cap=round,fill=fillColor] (449.66,471.90) circle (  1.96);
\path[draw=drawColor,line width= 0.4pt,line join=round,line cap=round,fill=fillColor] (450.08,471.93) circle (  1.96);
\path[draw=drawColor,line width= 0.4pt,line join=round,line cap=round,fill=fillColor] (450.50,471.96) circle (  1.96);
\path[draw=drawColor,line width= 0.4pt,line join=round,line cap=round,fill=fillColor] (450.92,472.00) circle (  1.96);
\path[draw=drawColor,line width= 0.4pt,line join=round,line cap=round,fill=fillColor] (451.34,472.04) circle (  1.96);
\path[draw=drawColor,line width= 0.4pt,line join=round,line cap=round,fill=fillColor] (451.76,472.07) circle (  1.96);
\path[draw=drawColor,line width= 0.4pt,line join=round,line cap=round,fill=fillColor] (452.18,472.10) circle (  1.96);
\path[draw=drawColor,line width= 0.4pt,line join=round,line cap=round,fill=fillColor] (452.60,472.14) circle (  1.96);
\path[draw=drawColor,line width= 0.4pt,line join=round,line cap=round,fill=fillColor] (453.02,472.17) circle (  1.96);
\path[draw=drawColor,line width= 0.4pt,line join=round,line cap=round,fill=fillColor] (453.43,472.21) circle (  1.96);
\path[draw=drawColor,line width= 0.4pt,line join=round,line cap=round,fill=fillColor] (453.85,472.24) circle (  1.96);
\path[draw=drawColor,line width= 0.4pt,line join=round,line cap=round,fill=fillColor] (454.27,472.27) circle (  1.96);
\path[draw=drawColor,line width= 0.4pt,line join=round,line cap=round,fill=fillColor] (454.69,472.30) circle (  1.96);
\path[draw=drawColor,line width= 0.4pt,line join=round,line cap=round,fill=fillColor] (455.11,472.34) circle (  1.96);
\path[draw=drawColor,line width= 0.4pt,line join=round,line cap=round,fill=fillColor] (455.53,472.37) circle (  1.96);
\path[draw=drawColor,line width= 0.4pt,line join=round,line cap=round,fill=fillColor] (455.95,472.41) circle (  1.96);
\path[draw=drawColor,line width= 0.4pt,line join=round,line cap=round,fill=fillColor] (456.37,472.44) circle (  1.96);
\path[draw=drawColor,line width= 0.4pt,line join=round,line cap=round,fill=fillColor] (456.79,472.47) circle (  1.96);
\path[draw=drawColor,line width= 0.4pt,line join=round,line cap=round,fill=fillColor] (457.21,472.50) circle (  1.96);
\path[draw=drawColor,line width= 0.4pt,line join=round,line cap=round,fill=fillColor] (457.63,472.53) circle (  1.96);
\path[draw=drawColor,line width= 0.4pt,line join=round,line cap=round,fill=fillColor] (458.05,472.57) circle (  1.96);
\path[draw=drawColor,line width= 0.4pt,line join=round,line cap=round,fill=fillColor] (458.47,472.60) circle (  1.96);
\path[draw=drawColor,line width= 0.4pt,line join=round,line cap=round,fill=fillColor] (458.88,472.63) circle (  1.96);
\path[draw=drawColor,line width= 0.4pt,line join=round,line cap=round,fill=fillColor] (459.30,472.67) circle (  1.96);
\path[draw=drawColor,line width= 0.4pt,line join=round,line cap=round,fill=fillColor] (459.72,472.70) circle (  1.96);
\path[draw=drawColor,line width= 0.4pt,line join=round,line cap=round,fill=fillColor] (460.14,472.73) circle (  1.96);
\path[draw=drawColor,line width= 0.4pt,line join=round,line cap=round,fill=fillColor] (460.56,472.76) circle (  1.96);
\path[draw=drawColor,line width= 0.4pt,line join=round,line cap=round,fill=fillColor] (460.98,472.79) circle (  1.96);
\path[draw=drawColor,line width= 0.4pt,line join=round,line cap=round,fill=fillColor] (461.40,472.82) circle (  1.96);
\path[draw=drawColor,line width= 0.4pt,line join=round,line cap=round,fill=fillColor] (461.82,472.85) circle (  1.96);
\path[draw=drawColor,line width= 0.4pt,line join=round,line cap=round,fill=fillColor] (462.24,472.88) circle (  1.96);
\path[draw=drawColor,line width= 0.4pt,line join=round,line cap=round,fill=fillColor] (462.66,472.91) circle (  1.96);
\path[draw=drawColor,line width= 0.4pt,line join=round,line cap=round,fill=fillColor] (463.08,472.94) circle (  1.96);
\path[draw=drawColor,line width= 0.4pt,line join=round,line cap=round,fill=fillColor] (463.50,472.97) circle (  1.96);
\path[draw=drawColor,line width= 0.4pt,line join=round,line cap=round,fill=fillColor] (463.92,473.00) circle (  1.96);
\path[draw=drawColor,line width= 0.4pt,line join=round,line cap=round,fill=fillColor] (464.34,473.03) circle (  1.96);
\path[draw=drawColor,line width= 0.4pt,line join=round,line cap=round,fill=fillColor] (464.75,473.06) circle (  1.96);
\path[draw=drawColor,line width= 0.4pt,line join=round,line cap=round,fill=fillColor] (465.17,473.09) circle (  1.96);
\path[draw=drawColor,line width= 0.4pt,line join=round,line cap=round,fill=fillColor] (465.59,473.12) circle (  1.96);
\path[draw=drawColor,line width= 0.4pt,line join=round,line cap=round,fill=fillColor] (466.01,473.15) circle (  1.96);
\path[draw=drawColor,line width= 0.4pt,line join=round,line cap=round,fill=fillColor] (466.43,473.18) circle (  1.96);
\path[draw=drawColor,line width= 0.4pt,line join=round,line cap=round,fill=fillColor] (466.85,473.21) circle (  1.96);
\path[draw=drawColor,line width= 0.4pt,line join=round,line cap=round,fill=fillColor] (467.27,473.24) circle (  1.96);
\path[draw=drawColor,line width= 0.4pt,line join=round,line cap=round,fill=fillColor] (467.69,473.27) circle (  1.96);
\path[draw=drawColor,line width= 0.4pt,line join=round,line cap=round,fill=fillColor] (468.11,473.30) circle (  1.96);
\path[draw=drawColor,line width= 0.4pt,line join=round,line cap=round,fill=fillColor] (468.53,473.33) circle (  1.96);
\path[draw=drawColor,line width= 0.4pt,line join=round,line cap=round,fill=fillColor] (468.95,473.36) circle (  1.96);
\path[draw=drawColor,line width= 0.4pt,line join=round,line cap=round,fill=fillColor] (469.37,473.38) circle (  1.96);
\path[draw=drawColor,line width= 0.4pt,line join=round,line cap=round,fill=fillColor] (469.79,473.41) circle (  1.96);
\path[draw=drawColor,line width= 0.4pt,line join=round,line cap=round,fill=fillColor] (470.20,473.44) circle (  1.96);
\path[draw=drawColor,line width= 0.4pt,line join=round,line cap=round,fill=fillColor] (470.62,473.47) circle (  1.96);
\path[draw=drawColor,line width= 0.4pt,line join=round,line cap=round,fill=fillColor] (471.04,473.49) circle (  1.96);
\path[draw=drawColor,line width= 0.4pt,line join=round,line cap=round,fill=fillColor] (471.46,473.52) circle (  1.96);
\path[draw=drawColor,line width= 0.4pt,line join=round,line cap=round,fill=fillColor] (471.88,473.55) circle (  1.96);
\path[draw=drawColor,line width= 0.4pt,line join=round,line cap=round,fill=fillColor] (472.30,473.58) circle (  1.96);
\path[draw=drawColor,line width= 0.4pt,line join=round,line cap=round,fill=fillColor] (472.72,473.60) circle (  1.96);
\path[draw=drawColor,line width= 0.4pt,line join=round,line cap=round,fill=fillColor] (473.14,473.63) circle (  1.96);
\path[draw=drawColor,line width= 0.4pt,line join=round,line cap=round,fill=fillColor] (473.56,473.66) circle (  1.96);
\path[draw=drawColor,line width= 0.4pt,line join=round,line cap=round,fill=fillColor] (473.98,473.68) circle (  1.96);
\path[draw=drawColor,line width= 0.4pt,line join=round,line cap=round,fill=fillColor] (474.40,473.71) circle (  1.96);
\path[draw=drawColor,line width= 0.4pt,line join=round,line cap=round,fill=fillColor] (474.82,473.74) circle (  1.96);
\path[draw=drawColor,line width= 0.4pt,line join=round,line cap=round,fill=fillColor] (475.24,473.76) circle (  1.96);
\path[draw=drawColor,line width= 0.4pt,line join=round,line cap=round,fill=fillColor] (475.65,473.79) circle (  1.96);
\path[draw=drawColor,line width= 0.4pt,line join=round,line cap=round,fill=fillColor] (476.07,473.82) circle (  1.96);
\path[draw=drawColor,line width= 0.4pt,line join=round,line cap=round,fill=fillColor] (476.49,473.84) circle (  1.96);
\path[draw=drawColor,line width= 0.4pt,line join=round,line cap=round,fill=fillColor] (476.91,473.87) circle (  1.96);
\path[draw=drawColor,line width= 0.4pt,line join=round,line cap=round,fill=fillColor] (477.33,473.90) circle (  1.96);
\path[draw=drawColor,line width= 0.4pt,line join=round,line cap=round,fill=fillColor] (477.75,473.92) circle (  1.96);
\path[draw=drawColor,line width= 0.4pt,line join=round,line cap=round,fill=fillColor] (478.17,473.95) circle (  1.96);
\path[draw=drawColor,line width= 0.4pt,line join=round,line cap=round,fill=fillColor] (478.59,473.97) circle (  1.96);
\path[draw=drawColor,line width= 0.4pt,line join=round,line cap=round,fill=fillColor] (479.01,473.99) circle (  1.96);
\path[draw=drawColor,line width= 0.4pt,line join=round,line cap=round,fill=fillColor] (479.43,474.02) circle (  1.96);
\end{scope}
\begin{scope}
\path[clip] (  0.00,  0.00) rectangle (505.89,505.89);
\definecolor{drawColor}{gray}{0.30}
\node[text=drawColor,anchor=base east,inner sep=0pt, outer sep=0pt, scale=  0.88] at ( 34.27, 50.34) {90};
\node[text=drawColor,anchor=base east,inner sep=0pt, outer sep=0pt, scale=  0.88] at ( 34.27,135.49) {92};
\node[text=drawColor,anchor=base east,inner sep=0pt, outer sep=0pt, scale=  0.88] at ( 34.27,220.63) {94};
\node[text=drawColor,anchor=base east,inner sep=0pt, outer sep=0pt, scale=  0.88] at ( 34.27,305.78) {96};
\node[text=drawColor,anchor=base east,inner sep=0pt, outer sep=0pt, scale=  0.88] at ( 34.27,390.93) {98};
\node[text=drawColor,anchor=base east,inner sep=0pt, outer sep=0pt, scale=  0.88] at ( 34.27,476.07) {100};
\end{scope}
\begin{scope}
\path[clip] (  0.00,  0.00) rectangle (505.89,505.89);
\definecolor{drawColor}{gray}{0.20}
\path[draw=drawColor,line width= 0.6pt,line join=round] ( 36.47, 53.37) --
	( 39.22, 53.37);
\path[draw=drawColor,line width= 0.6pt,line join=round] ( 36.47,138.52) --
	( 39.22,138.52);
\path[draw=drawColor,line width= 0.6pt,line join=round] ( 36.47,223.66) --
	( 39.22,223.66);
\path[draw=drawColor,line width= 0.6pt,line join=round] ( 36.47,308.81) --
	( 39.22,308.81);
\path[draw=drawColor,line width= 0.6pt,line join=round] ( 36.47,393.96) --
	( 39.22,393.96);
\path[draw=drawColor,line width= 0.6pt,line join=round] ( 36.47,479.10) --
	( 39.22,479.10);
\end{scope}
\begin{scope}
\path[clip] (  0.00,  0.00) rectangle (505.89,505.89);
\definecolor{drawColor}{gray}{0.20}
\path[draw=drawColor,line width= 0.6pt,line join=round] ( 60.18, 29.34) --
	( 60.18, 32.09);
\path[draw=drawColor,line width= 0.6pt,line join=round] (144.03, 29.34) --
	(144.03, 32.09);
\path[draw=drawColor,line width= 0.6pt,line join=round] (227.88, 29.34) --
	(227.88, 32.09);
\path[draw=drawColor,line width= 0.6pt,line join=round] (311.73, 29.34) --
	(311.73, 32.09);
\path[draw=drawColor,line width= 0.6pt,line join=round] (395.58, 29.34) --
	(395.58, 32.09);
\path[draw=drawColor,line width= 0.6pt,line join=round] (479.43, 29.34) --
	(479.43, 32.09);
\end{scope}
\begin{scope}
\path[clip] (  0.00,  0.00) rectangle (505.89,505.89);
\definecolor{drawColor}{gray}{0.30}
\node[text=drawColor,anchor=base,inner sep=0pt, outer sep=0pt, scale=  0.88] at ( 60.18, 21.08) {4000};
\node[text=drawColor,anchor=base,inner sep=0pt, outer sep=0pt, scale=  0.88] at (144.03, 21.08) {4200};
\node[text=drawColor,anchor=base,inner sep=0pt, outer sep=0pt, scale=  0.88] at (227.88, 21.08) {4400};
\node[text=drawColor,anchor=base,inner sep=0pt, outer sep=0pt, scale=  0.88] at (311.73, 21.08) {4600};
\node[text=drawColor,anchor=base,inner sep=0pt, outer sep=0pt, scale=  0.88] at (395.58, 21.08) {4800};
\node[text=drawColor,anchor=base,inner sep=0pt, outer sep=0pt, scale=  0.88] at (479.43, 21.08) {5000};
\end{scope}
\begin{scope}
\path[clip] (  0.00,  0.00) rectangle (505.89,505.89);
\definecolor{drawColor}{RGB}{0,0,0}
\node[text=drawColor,anchor=base,inner sep=0pt, outer sep=0pt, scale=  1.10] at (269.81,  8.00) {Anzahl an Adressen im Pool (x)};
\end{scope}
\begin{scope}
\path[clip] (  0.00,  0.00) rectangle (505.89,505.89);
\definecolor{drawColor}{RGB}{0,0,0}
\node[text=drawColor,rotate= 90.00,anchor=base,inner sep=0pt, outer sep=0pt, scale=  1.10] at ( 13.08,266.24) {Wahrscheinlichkeit alle Eviction Sets zu finden (\%)};
\end{scope}
\end{tikzpicture}

\caption{Beschreibt den Zusammenhang zwischen der Poolgröße und der Wahrscheinlichkeit alle Eviction Sets zu finden}
\end{scaletikzpicturetowidth}
\end{center}
Häufig wird in Benchmarks angegeben wie viele Eviction-Sets überhaupt gefunden wurden. Die Wahrscheinlichkeit 
\begin{align*}
\sum\limits_{i=1}^{128} i {x \choose i} P(count(cs)>=16|\#add = x)^i \cdot P(count(cs)<16|\#add = x)^{128-i} 
\end{align*}
gibt den Erwartungswert der gefundenen Eviction-Sets an.

Das Diagramm \ref{fig:avg_es} gibt die Erfolgswahrscheinlichkeiten für verschiedene $x$-Werte im zweiten Szenario an.
\label{fig:avg_es}
\begin{center}
\begin{scaletikzpicturetowidth}{\textwidth}
% Created by tikzDevice version 0.12 on 2018-09-06 15:30:23
% !TEX encoding = UTF-8 Unicode
\begin{tikzpicture}[scale=\tikzscale,y=20pt]
\definecolor{fillColor}{RGB}{255,255,255}
\path[use as bounding box,fill=fillColor,fill opacity=0.00] (0,0) rectangle (505.89,505.89);
\begin{scope}
\path[clip] (  0.00,  0.00) rectangle (505.89,505.89);
\definecolor{drawColor}{RGB}{255,255,255}
\definecolor{fillColor}{RGB}{255,255,255}
\path[draw=drawColor,line width= 0.6pt,line join=round,line cap=round,fill=fillColor] (  0.00,  0.00) rectangle (505.89,505.89);
\end{scope}
\begin{scope}
\path[clip] ( 43.07, 32.09) rectangle (500.39,500.39);
\definecolor{fillColor}{gray}{0.92}
\path[fill=fillColor] ( 43.07, 32.09) rectangle (500.39,500.39);
\definecolor{drawColor}{RGB}{255,255,255}
\path[draw=drawColor,line width= 0.3pt,line join=round] ( 43.07, 32.09) --
	(500.39, 32.09);
\path[draw=drawColor,line width= 0.3pt,line join=round] ( 43.07, 74.66) --
	(500.39, 74.66);
\path[draw=drawColor,line width= 0.3pt,line join=round] ( 43.07,117.23) --
	(500.39,117.23);
\path[draw=drawColor,line width= 0.3pt,line join=round] ( 43.07,159.81) --
	(500.39,159.81);
\path[draw=drawColor,line width= 0.3pt,line join=round] ( 43.07,202.38) --
	(500.39,202.38);
\path[draw=drawColor,line width= 0.3pt,line join=round] ( 43.07,244.95) --
	(500.39,244.95);
\path[draw=drawColor,line width= 0.3pt,line join=round] ( 43.07,287.52) --
	(500.39,287.52);
\path[draw=drawColor,line width= 0.3pt,line join=round] ( 43.07,330.10) --
	(500.39,330.10);
\path[draw=drawColor,line width= 0.3pt,line join=round] ( 43.07,372.67) --
	(500.39,372.67);
\path[draw=drawColor,line width= 0.3pt,line join=round] ( 43.07,415.24) --
	(500.39,415.24);
\path[draw=drawColor,line width= 0.3pt,line join=round] ( 43.07,457.82) --
	(500.39,457.82);
\path[draw=drawColor,line width= 0.3pt,line join=round] ( 43.07,500.39) --
	(500.39,500.39);
\path[draw=drawColor,line width= 0.3pt,line join=round] ( 89.84, 32.09) --
	( 89.84,500.39);
\path[draw=drawColor,line width= 0.3pt,line join=round] (141.81, 32.09) --
	(141.81,500.39);
\path[draw=drawColor,line width= 0.3pt,line join=round] (193.77, 32.09) --
	(193.77,500.39);
\path[draw=drawColor,line width= 0.3pt,line join=round] (245.74, 32.09) --
	(245.74,500.39);
\path[draw=drawColor,line width= 0.3pt,line join=round] (297.71, 32.09) --
	(297.71,500.39);
\path[draw=drawColor,line width= 0.3pt,line join=round] (349.68, 32.09) --
	(349.68,500.39);
\path[draw=drawColor,line width= 0.3pt,line join=round] (401.65, 32.09) --
	(401.65,500.39);
\path[draw=drawColor,line width= 0.3pt,line join=round] (453.62, 32.09) --
	(453.62,500.39);
\path[draw=drawColor,line width= 0.6pt,line join=round] ( 43.07, 53.37) --
	(500.39, 53.37);
\path[draw=drawColor,line width= 0.6pt,line join=round] ( 43.07, 95.95) --
	(500.39, 95.95);
\path[draw=drawColor,line width= 0.6pt,line join=round] ( 43.07,138.52) --
	(500.39,138.52);
\path[draw=drawColor,line width= 0.6pt,line join=round] ( 43.07,181.09) --
	(500.39,181.09);
\path[draw=drawColor,line width= 0.6pt,line join=round] ( 43.07,223.66) --
	(500.39,223.66);
\path[draw=drawColor,line width= 0.6pt,line join=round] ( 43.07,266.24) --
	(500.39,266.24);
\path[draw=drawColor,line width= 0.6pt,line join=round] ( 43.07,308.81) --
	(500.39,308.81);
\path[draw=drawColor,line width= 0.6pt,line join=round] ( 43.07,351.38) --
	(500.39,351.38);
\path[draw=drawColor,line width= 0.6pt,line join=round] ( 43.07,393.96) --
	(500.39,393.96);
\path[draw=drawColor,line width= 0.6pt,line join=round] ( 43.07,436.53) --
	(500.39,436.53);
\path[draw=drawColor,line width= 0.6pt,line join=round] ( 43.07,479.10) --
	(500.39,479.10);
\path[draw=drawColor,line width= 0.6pt,line join=round] ( 43.07,479.10) --
	(500.39,479.10);
\path[draw=drawColor,line width= 0.6pt,line join=round] ( 63.85, 32.09) --
	( 63.85,500.39);
\path[draw=drawColor,line width= 0.6pt,line join=round] (115.82, 32.09) --
	(115.82,500.39);
\path[draw=drawColor,line width= 0.6pt,line join=round] (167.79, 32.09) --
	(167.79,500.39);
\path[draw=drawColor,line width= 0.6pt,line join=round] (219.76, 32.09) --
	(219.76,500.39);
\path[draw=drawColor,line width= 0.6pt,line join=round] (271.73, 32.09) --
	(271.73,500.39);
\path[draw=drawColor,line width= 0.6pt,line join=round] (323.70, 32.09) --
	(323.70,500.39);
\path[draw=drawColor,line width= 0.6pt,line join=round] (375.67, 32.09) --
	(375.67,500.39);
\path[draw=drawColor,line width= 0.6pt,line join=round] (427.63, 32.09) --
	(427.63,500.39);
\path[draw=drawColor,line width= 0.6pt,line join=round] (479.60, 32.09) --
	(479.60,500.39);
\definecolor{drawColor}{RGB}{0,0,0}
\definecolor{fillColor}{RGB}{0,0,0}
\path[draw=drawColor,line width= 0.4pt,line join=round,line cap=round,fill=fillColor] ( 79.96, 53.70) circle (  1.96);
\path[draw=drawColor,line width= 0.4pt,line join=round,line cap=round,fill=fillColor] ( 80.22, 54.58) circle (  1.96);
\path[draw=drawColor,line width= 0.4pt,line join=round,line cap=round,fill=fillColor] ( 80.48, 55.46) circle (  1.96);
\path[draw=drawColor,line width= 0.4pt,line join=round,line cap=round,fill=fillColor] ( 80.74, 56.34) circle (  1.96);
\path[draw=drawColor,line width= 0.4pt,line join=round,line cap=round,fill=fillColor] ( 81.00, 57.22) circle (  1.96);
\path[draw=drawColor,line width= 0.4pt,line join=round,line cap=round,fill=fillColor] ( 81.26, 58.10) circle (  1.96);
\path[draw=drawColor,line width= 0.4pt,line join=round,line cap=round,fill=fillColor] ( 81.52, 58.98) circle (  1.96);
\path[draw=drawColor,line width= 0.4pt,line join=round,line cap=round,fill=fillColor] ( 81.78, 59.85) circle (  1.96);
\path[draw=drawColor,line width= 0.4pt,line join=round,line cap=round,fill=fillColor] ( 82.04, 60.73) circle (  1.96);
\path[draw=drawColor,line width= 0.4pt,line join=round,line cap=round,fill=fillColor] ( 82.30, 61.60) circle (  1.96);
\path[draw=drawColor,line width= 0.4pt,line join=round,line cap=round,fill=fillColor] ( 82.56, 62.47) circle (  1.96);
\path[draw=drawColor,line width= 0.4pt,line join=round,line cap=round,fill=fillColor] ( 82.82, 63.35) circle (  1.96);
\path[draw=drawColor,line width= 0.4pt,line join=round,line cap=round,fill=fillColor] ( 83.08, 64.22) circle (  1.96);
\path[draw=drawColor,line width= 0.4pt,line join=round,line cap=round,fill=fillColor] ( 83.34, 65.09) circle (  1.96);
\path[draw=drawColor,line width= 0.4pt,line join=round,line cap=round,fill=fillColor] ( 83.60, 65.96) circle (  1.96);
\path[draw=drawColor,line width= 0.4pt,line join=round,line cap=round,fill=fillColor] ( 83.86, 66.82) circle (  1.96);
\path[draw=drawColor,line width= 0.4pt,line join=round,line cap=round,fill=fillColor] ( 84.12, 67.69) circle (  1.96);
\path[draw=drawColor,line width= 0.4pt,line join=round,line cap=round,fill=fillColor] ( 84.38, 68.56) circle (  1.96);
\path[draw=drawColor,line width= 0.4pt,line join=round,line cap=round,fill=fillColor] ( 84.64, 69.42) circle (  1.96);
\path[draw=drawColor,line width= 0.4pt,line join=round,line cap=round,fill=fillColor] ( 84.90, 70.28) circle (  1.96);
\path[draw=drawColor,line width= 0.4pt,line join=round,line cap=round,fill=fillColor] ( 85.16, 71.15) circle (  1.96);
\path[draw=drawColor,line width= 0.4pt,line join=round,line cap=round,fill=fillColor] ( 85.42, 72.01) circle (  1.96);
\path[draw=drawColor,line width= 0.4pt,line join=round,line cap=round,fill=fillColor] ( 85.68, 72.87) circle (  1.96);
\path[draw=drawColor,line width= 0.4pt,line join=round,line cap=round,fill=fillColor] ( 85.94, 73.73) circle (  1.96);
\path[draw=drawColor,line width= 0.4pt,line join=round,line cap=round,fill=fillColor] ( 86.20, 74.59) circle (  1.96);
\path[draw=drawColor,line width= 0.4pt,line join=round,line cap=round,fill=fillColor] ( 86.46, 75.44) circle (  1.96);
\path[draw=drawColor,line width= 0.4pt,line join=round,line cap=round,fill=fillColor] ( 86.72, 76.30) circle (  1.96);
\path[draw=drawColor,line width= 0.4pt,line join=round,line cap=round,fill=fillColor] ( 86.98, 77.15) circle (  1.96);
\path[draw=drawColor,line width= 0.4pt,line join=round,line cap=round,fill=fillColor] ( 87.24, 78.01) circle (  1.96);
\path[draw=drawColor,line width= 0.4pt,line join=round,line cap=round,fill=fillColor] ( 87.50, 78.86) circle (  1.96);
\path[draw=drawColor,line width= 0.4pt,line join=round,line cap=round,fill=fillColor] ( 87.76, 79.71) circle (  1.96);
\path[draw=drawColor,line width= 0.4pt,line join=round,line cap=round,fill=fillColor] ( 88.02, 80.56) circle (  1.96);
\path[draw=drawColor,line width= 0.4pt,line join=round,line cap=round,fill=fillColor] ( 88.28, 81.41) circle (  1.96);
\path[draw=drawColor,line width= 0.4pt,line join=round,line cap=round,fill=fillColor] ( 88.54, 82.26) circle (  1.96);
\path[draw=drawColor,line width= 0.4pt,line join=round,line cap=round,fill=fillColor] ( 88.80, 83.11) circle (  1.96);
\path[draw=drawColor,line width= 0.4pt,line join=round,line cap=round,fill=fillColor] ( 89.06, 83.95) circle (  1.96);
\path[draw=drawColor,line width= 0.4pt,line join=round,line cap=round,fill=fillColor] ( 89.32, 84.80) circle (  1.96);
\path[draw=drawColor,line width= 0.4pt,line join=round,line cap=round,fill=fillColor] ( 89.58, 85.64) circle (  1.96);
\path[draw=drawColor,line width= 0.4pt,line join=round,line cap=round,fill=fillColor] ( 89.84, 86.48) circle (  1.96);
\path[draw=drawColor,line width= 0.4pt,line join=round,line cap=round,fill=fillColor] ( 90.10, 87.33) circle (  1.96);
\path[draw=drawColor,line width= 0.4pt,line join=round,line cap=round,fill=fillColor] ( 90.36, 88.17) circle (  1.96);
\path[draw=drawColor,line width= 0.4pt,line join=round,line cap=round,fill=fillColor] ( 90.62, 89.01) circle (  1.96);
\path[draw=drawColor,line width= 0.4pt,line join=round,line cap=round,fill=fillColor] ( 90.88, 89.85) circle (  1.96);
\path[draw=drawColor,line width= 0.4pt,line join=round,line cap=round,fill=fillColor] ( 91.14, 90.68) circle (  1.96);
\path[draw=drawColor,line width= 0.4pt,line join=round,line cap=round,fill=fillColor] ( 91.40, 91.52) circle (  1.96);
\path[draw=drawColor,line width= 0.4pt,line join=round,line cap=round,fill=fillColor] ( 91.66, 92.35) circle (  1.96);
\path[draw=drawColor,line width= 0.4pt,line join=round,line cap=round,fill=fillColor] ( 91.92, 93.19) circle (  1.96);
\path[draw=drawColor,line width= 0.4pt,line join=round,line cap=round,fill=fillColor] ( 92.18, 94.02) circle (  1.96);
\path[draw=drawColor,line width= 0.4pt,line join=round,line cap=round,fill=fillColor] ( 92.44, 94.85) circle (  1.96);
\path[draw=drawColor,line width= 0.4pt,line join=round,line cap=round,fill=fillColor] ( 92.70, 95.68) circle (  1.96);
\path[draw=drawColor,line width= 0.4pt,line join=round,line cap=round,fill=fillColor] ( 92.96, 96.51) circle (  1.96);
\path[draw=drawColor,line width= 0.4pt,line join=round,line cap=round,fill=fillColor] ( 93.22, 97.34) circle (  1.96);
\path[draw=drawColor,line width= 0.4pt,line join=round,line cap=round,fill=fillColor] ( 93.48, 98.17) circle (  1.96);
\path[draw=drawColor,line width= 0.4pt,line join=round,line cap=round,fill=fillColor] ( 93.74, 99.00) circle (  1.96);
\path[draw=drawColor,line width= 0.4pt,line join=round,line cap=round,fill=fillColor] ( 94.00, 99.82) circle (  1.96);
\path[draw=drawColor,line width= 0.4pt,line join=round,line cap=round,fill=fillColor] ( 94.25,100.64) circle (  1.96);
\path[draw=drawColor,line width= 0.4pt,line join=round,line cap=round,fill=fillColor] ( 94.51,101.47) circle (  1.96);
\path[draw=drawColor,line width= 0.4pt,line join=round,line cap=round,fill=fillColor] ( 94.77,102.29) circle (  1.96);
\path[draw=drawColor,line width= 0.4pt,line join=round,line cap=round,fill=fillColor] ( 95.03,103.11) circle (  1.96);
\path[draw=drawColor,line width= 0.4pt,line join=round,line cap=round,fill=fillColor] ( 95.29,103.93) circle (  1.96);
\path[draw=drawColor,line width= 0.4pt,line join=round,line cap=round,fill=fillColor] ( 95.55,104.75) circle (  1.96);
\path[draw=drawColor,line width= 0.4pt,line join=round,line cap=round,fill=fillColor] ( 95.81,105.57) circle (  1.96);
\path[draw=drawColor,line width= 0.4pt,line join=round,line cap=round,fill=fillColor] ( 96.07,106.38) circle (  1.96);
\path[draw=drawColor,line width= 0.4pt,line join=round,line cap=round,fill=fillColor] ( 96.33,107.20) circle (  1.96);
\path[draw=drawColor,line width= 0.4pt,line join=round,line cap=round,fill=fillColor] ( 96.59,108.01) circle (  1.96);
\path[draw=drawColor,line width= 0.4pt,line join=round,line cap=round,fill=fillColor] ( 96.85,108.83) circle (  1.96);
\path[draw=drawColor,line width= 0.4pt,line join=round,line cap=round,fill=fillColor] ( 97.11,109.64) circle (  1.96);
\path[draw=drawColor,line width= 0.4pt,line join=round,line cap=round,fill=fillColor] ( 97.37,110.45) circle (  1.96);
\path[draw=drawColor,line width= 0.4pt,line join=round,line cap=round,fill=fillColor] ( 97.63,111.26) circle (  1.96);
\path[draw=drawColor,line width= 0.4pt,line join=round,line cap=round,fill=fillColor] ( 97.89,112.07) circle (  1.96);
\path[draw=drawColor,line width= 0.4pt,line join=round,line cap=round,fill=fillColor] ( 98.15,112.87) circle (  1.96);
\path[draw=drawColor,line width= 0.4pt,line join=round,line cap=round,fill=fillColor] ( 98.41,113.68) circle (  1.96);
\path[draw=drawColor,line width= 0.4pt,line join=round,line cap=round,fill=fillColor] ( 98.67,114.49) circle (  1.96);
\path[draw=drawColor,line width= 0.4pt,line join=round,line cap=round,fill=fillColor] ( 98.93,115.29) circle (  1.96);
\path[draw=drawColor,line width= 0.4pt,line join=round,line cap=round,fill=fillColor] ( 99.19,116.09) circle (  1.96);
\path[draw=drawColor,line width= 0.4pt,line join=round,line cap=round,fill=fillColor] ( 99.45,116.89) circle (  1.96);
\path[draw=drawColor,line width= 0.4pt,line join=round,line cap=round,fill=fillColor] ( 99.71,117.69) circle (  1.96);
\path[draw=drawColor,line width= 0.4pt,line join=round,line cap=round,fill=fillColor] ( 99.97,118.49) circle (  1.96);
\path[draw=drawColor,line width= 0.4pt,line join=round,line cap=round,fill=fillColor] (100.23,119.29) circle (  1.96);
\path[draw=drawColor,line width= 0.4pt,line join=round,line cap=round,fill=fillColor] (100.49,120.09) circle (  1.96);
\path[draw=drawColor,line width= 0.4pt,line join=round,line cap=round,fill=fillColor] (100.75,120.89) circle (  1.96);
\path[draw=drawColor,line width= 0.4pt,line join=round,line cap=round,fill=fillColor] (101.01,121.68) circle (  1.96);
\path[draw=drawColor,line width= 0.4pt,line join=round,line cap=round,fill=fillColor] (101.27,122.47) circle (  1.96);
\path[draw=drawColor,line width= 0.4pt,line join=round,line cap=round,fill=fillColor] (101.53,123.27) circle (  1.96);
\path[draw=drawColor,line width= 0.4pt,line join=round,line cap=round,fill=fillColor] (101.79,124.06) circle (  1.96);
\path[draw=drawColor,line width= 0.4pt,line join=round,line cap=round,fill=fillColor] (102.05,124.85) circle (  1.96);
\path[draw=drawColor,line width= 0.4pt,line join=round,line cap=round,fill=fillColor] (102.31,125.64) circle (  1.96);
\path[draw=drawColor,line width= 0.4pt,line join=round,line cap=round,fill=fillColor] (102.57,126.43) circle (  1.96);
\path[draw=drawColor,line width= 0.4pt,line join=round,line cap=round,fill=fillColor] (102.83,127.21) circle (  1.96);
\path[draw=drawColor,line width= 0.4pt,line join=round,line cap=round,fill=fillColor] (103.09,128.00) circle (  1.96);
\path[draw=drawColor,line width= 0.4pt,line join=round,line cap=round,fill=fillColor] (103.35,128.78) circle (  1.96);
\path[draw=drawColor,line width= 0.4pt,line join=round,line cap=round,fill=fillColor] (103.61,129.57) circle (  1.96);
\path[draw=drawColor,line width= 0.4pt,line join=round,line cap=round,fill=fillColor] (103.87,130.35) circle (  1.96);
\path[draw=drawColor,line width= 0.4pt,line join=round,line cap=round,fill=fillColor] (104.13,131.13) circle (  1.96);
\path[draw=drawColor,line width= 0.4pt,line join=round,line cap=round,fill=fillColor] (104.39,131.91) circle (  1.96);
\path[draw=drawColor,line width= 0.4pt,line join=round,line cap=round,fill=fillColor] (104.65,132.69) circle (  1.96);
\path[draw=drawColor,line width= 0.4pt,line join=round,line cap=round,fill=fillColor] (104.91,133.47) circle (  1.96);
\path[draw=drawColor,line width= 0.4pt,line join=round,line cap=round,fill=fillColor] (105.17,134.25) circle (  1.96);
\path[draw=drawColor,line width= 0.4pt,line join=round,line cap=round,fill=fillColor] (105.43,135.02) circle (  1.96);
\path[draw=drawColor,line width= 0.4pt,line join=round,line cap=round,fill=fillColor] (105.69,135.80) circle (  1.96);
\path[draw=drawColor,line width= 0.4pt,line join=round,line cap=round,fill=fillColor] (105.95,136.57) circle (  1.96);
\path[draw=drawColor,line width= 0.4pt,line join=round,line cap=round,fill=fillColor] (106.21,137.34) circle (  1.96);
\path[draw=drawColor,line width= 0.4pt,line join=round,line cap=round,fill=fillColor] (106.47,138.11) circle (  1.96);
\path[draw=drawColor,line width= 0.4pt,line join=round,line cap=round,fill=fillColor] (106.73,138.88) circle (  1.96);
\path[draw=drawColor,line width= 0.4pt,line join=round,line cap=round,fill=fillColor] (106.99,139.65) circle (  1.96);
\path[draw=drawColor,line width= 0.4pt,line join=round,line cap=round,fill=fillColor] (107.25,140.42) circle (  1.96);
\path[draw=drawColor,line width= 0.4pt,line join=round,line cap=round,fill=fillColor] (107.51,141.19) circle (  1.96);
\path[draw=drawColor,line width= 0.4pt,line join=round,line cap=round,fill=fillColor] (107.77,141.95) circle (  1.96);
\path[draw=drawColor,line width= 0.4pt,line join=round,line cap=round,fill=fillColor] (108.03,142.72) circle (  1.96);
\path[draw=drawColor,line width= 0.4pt,line join=round,line cap=round,fill=fillColor] (108.29,143.48) circle (  1.96);
\path[draw=drawColor,line width= 0.4pt,line join=round,line cap=round,fill=fillColor] (108.55,144.24) circle (  1.96);
\path[draw=drawColor,line width= 0.4pt,line join=round,line cap=round,fill=fillColor] (108.81,145.00) circle (  1.96);
\path[draw=drawColor,line width= 0.4pt,line join=round,line cap=round,fill=fillColor] (109.07,145.76) circle (  1.96);
\path[draw=drawColor,line width= 0.4pt,line join=round,line cap=round,fill=fillColor] (109.33,146.52) circle (  1.96);
\path[draw=drawColor,line width= 0.4pt,line join=round,line cap=round,fill=fillColor] (109.59,147.28) circle (  1.96);
\path[draw=drawColor,line width= 0.4pt,line join=round,line cap=round,fill=fillColor] (109.85,148.04) circle (  1.96);
\path[draw=drawColor,line width= 0.4pt,line join=round,line cap=round,fill=fillColor] (110.11,148.79) circle (  1.96);
\path[draw=drawColor,line width= 0.4pt,line join=round,line cap=round,fill=fillColor] (110.37,149.55) circle (  1.96);
\path[draw=drawColor,line width= 0.4pt,line join=round,line cap=round,fill=fillColor] (110.62,150.30) circle (  1.96);
\path[draw=drawColor,line width= 0.4pt,line join=round,line cap=round,fill=fillColor] (110.88,151.05) circle (  1.96);
\path[draw=drawColor,line width= 0.4pt,line join=round,line cap=round,fill=fillColor] (111.14,151.80) circle (  1.96);
\path[draw=drawColor,line width= 0.4pt,line join=round,line cap=round,fill=fillColor] (111.40,152.55) circle (  1.96);
\path[draw=drawColor,line width= 0.4pt,line join=round,line cap=round,fill=fillColor] (111.66,153.30) circle (  1.96);
\path[draw=drawColor,line width= 0.4pt,line join=round,line cap=round,fill=fillColor] (111.92,154.05) circle (  1.96);
\path[draw=drawColor,line width= 0.4pt,line join=round,line cap=round,fill=fillColor] (112.18,154.79) circle (  1.96);
\path[draw=drawColor,line width= 0.4pt,line join=round,line cap=round,fill=fillColor] (112.44,155.54) circle (  1.96);
\path[draw=drawColor,line width= 0.4pt,line join=round,line cap=round,fill=fillColor] (112.70,156.28) circle (  1.96);
\path[draw=drawColor,line width= 0.4pt,line join=round,line cap=round,fill=fillColor] (112.96,157.02) circle (  1.96);
\path[draw=drawColor,line width= 0.4pt,line join=round,line cap=round,fill=fillColor] (113.22,157.77) circle (  1.96);
\path[draw=drawColor,line width= 0.4pt,line join=round,line cap=round,fill=fillColor] (113.48,158.51) circle (  1.96);
\path[draw=drawColor,line width= 0.4pt,line join=round,line cap=round,fill=fillColor] (113.74,159.24) circle (  1.96);
\path[draw=drawColor,line width= 0.4pt,line join=round,line cap=round,fill=fillColor] (114.00,159.98) circle (  1.96);
\path[draw=drawColor,line width= 0.4pt,line join=round,line cap=round,fill=fillColor] (114.26,160.72) circle (  1.96);
\path[draw=drawColor,line width= 0.4pt,line join=round,line cap=round,fill=fillColor] (114.52,161.46) circle (  1.96);
\path[draw=drawColor,line width= 0.4pt,line join=round,line cap=round,fill=fillColor] (114.78,162.19) circle (  1.96);
\path[draw=drawColor,line width= 0.4pt,line join=round,line cap=round,fill=fillColor] (115.04,162.92) circle (  1.96);
\path[draw=drawColor,line width= 0.4pt,line join=round,line cap=round,fill=fillColor] (115.30,163.66) circle (  1.96);
\path[draw=drawColor,line width= 0.4pt,line join=round,line cap=round,fill=fillColor] (115.56,164.39) circle (  1.96);
\path[draw=drawColor,line width= 0.4pt,line join=round,line cap=round,fill=fillColor] (115.82,165.12) circle (  1.96);
\path[draw=drawColor,line width= 0.4pt,line join=round,line cap=round,fill=fillColor] (116.08,165.85) circle (  1.96);
\path[draw=drawColor,line width= 0.4pt,line join=round,line cap=round,fill=fillColor] (116.34,166.57) circle (  1.96);
\path[draw=drawColor,line width= 0.4pt,line join=round,line cap=round,fill=fillColor] (116.60,167.30) circle (  1.96);
\path[draw=drawColor,line width= 0.4pt,line join=round,line cap=round,fill=fillColor] (116.86,168.03) circle (  1.96);
\path[draw=drawColor,line width= 0.4pt,line join=round,line cap=round,fill=fillColor] (117.12,168.75) circle (  1.96);
\path[draw=drawColor,line width= 0.4pt,line join=round,line cap=round,fill=fillColor] (117.38,169.47) circle (  1.96);
\path[draw=drawColor,line width= 0.4pt,line join=round,line cap=round,fill=fillColor] (117.64,170.20) circle (  1.96);
\path[draw=drawColor,line width= 0.4pt,line join=round,line cap=round,fill=fillColor] (117.90,170.92) circle (  1.96);
\path[draw=drawColor,line width= 0.4pt,line join=round,line cap=round,fill=fillColor] (118.16,171.64) circle (  1.96);
\path[draw=drawColor,line width= 0.4pt,line join=round,line cap=round,fill=fillColor] (118.42,172.35) circle (  1.96);
\path[draw=drawColor,line width= 0.4pt,line join=round,line cap=round,fill=fillColor] (118.68,173.07) circle (  1.96);
\path[draw=drawColor,line width= 0.4pt,line join=round,line cap=round,fill=fillColor] (118.94,173.79) circle (  1.96);
\path[draw=drawColor,line width= 0.4pt,line join=round,line cap=round,fill=fillColor] (119.20,174.50) circle (  1.96);
\path[draw=drawColor,line width= 0.4pt,line join=round,line cap=round,fill=fillColor] (119.46,175.22) circle (  1.96);
\path[draw=drawColor,line width= 0.4pt,line join=round,line cap=round,fill=fillColor] (119.72,175.93) circle (  1.96);
\path[draw=drawColor,line width= 0.4pt,line join=round,line cap=round,fill=fillColor] (119.98,176.64) circle (  1.96);
\path[draw=drawColor,line width= 0.4pt,line join=round,line cap=round,fill=fillColor] (120.24,177.35) circle (  1.96);
\path[draw=drawColor,line width= 0.4pt,line join=round,line cap=round,fill=fillColor] (120.50,178.06) circle (  1.96);
\path[draw=drawColor,line width= 0.4pt,line join=round,line cap=round,fill=fillColor] (120.76,178.77) circle (  1.96);
\path[draw=drawColor,line width= 0.4pt,line join=round,line cap=round,fill=fillColor] (121.02,179.48) circle (  1.96);
\path[draw=drawColor,line width= 0.4pt,line join=round,line cap=round,fill=fillColor] (121.28,180.18) circle (  1.96);
\path[draw=drawColor,line width= 0.4pt,line join=round,line cap=round,fill=fillColor] (121.54,180.89) circle (  1.96);
\path[draw=drawColor,line width= 0.4pt,line join=round,line cap=round,fill=fillColor] (121.80,181.59) circle (  1.96);
\path[draw=drawColor,line width= 0.4pt,line join=round,line cap=round,fill=fillColor] (122.06,182.29) circle (  1.96);
\path[draw=drawColor,line width= 0.4pt,line join=round,line cap=round,fill=fillColor] (122.32,183.00) circle (  1.96);
\path[draw=drawColor,line width= 0.4pt,line join=round,line cap=round,fill=fillColor] (122.58,183.70) circle (  1.96);
\path[draw=drawColor,line width= 0.4pt,line join=round,line cap=round,fill=fillColor] (122.84,184.39) circle (  1.96);
\path[draw=drawColor,line width= 0.4pt,line join=round,line cap=round,fill=fillColor] (123.10,185.09) circle (  1.96);
\path[draw=drawColor,line width= 0.4pt,line join=round,line cap=round,fill=fillColor] (123.36,185.79) circle (  1.96);
\path[draw=drawColor,line width= 0.4pt,line join=round,line cap=round,fill=fillColor] (123.62,186.49) circle (  1.96);
\path[draw=drawColor,line width= 0.4pt,line join=round,line cap=round,fill=fillColor] (123.88,187.18) circle (  1.96);
\path[draw=drawColor,line width= 0.4pt,line join=round,line cap=round,fill=fillColor] (124.14,187.87) circle (  1.96);
\path[draw=drawColor,line width= 0.4pt,line join=round,line cap=round,fill=fillColor] (124.40,188.56) circle (  1.96);
\path[draw=drawColor,line width= 0.4pt,line join=round,line cap=round,fill=fillColor] (124.66,189.26) circle (  1.96);
\path[draw=drawColor,line width= 0.4pt,line join=round,line cap=round,fill=fillColor] (124.92,189.95) circle (  1.96);
\path[draw=drawColor,line width= 0.4pt,line join=round,line cap=round,fill=fillColor] (125.18,190.63) circle (  1.96);
\path[draw=drawColor,line width= 0.4pt,line join=round,line cap=round,fill=fillColor] (125.44,191.32) circle (  1.96);
\path[draw=drawColor,line width= 0.4pt,line join=round,line cap=round,fill=fillColor] (125.70,192.01) circle (  1.96);
\path[draw=drawColor,line width= 0.4pt,line join=round,line cap=round,fill=fillColor] (125.96,192.69) circle (  1.96);
\path[draw=drawColor,line width= 0.4pt,line join=round,line cap=round,fill=fillColor] (126.22,193.37) circle (  1.96);
\path[draw=drawColor,line width= 0.4pt,line join=round,line cap=round,fill=fillColor] (126.48,194.06) circle (  1.96);
\path[draw=drawColor,line width= 0.4pt,line join=round,line cap=round,fill=fillColor] (126.74,194.74) circle (  1.96);
\path[draw=drawColor,line width= 0.4pt,line join=round,line cap=round,fill=fillColor] (127.00,195.42) circle (  1.96);
\path[draw=drawColor,line width= 0.4pt,line join=round,line cap=round,fill=fillColor] (127.25,196.10) circle (  1.96);
\path[draw=drawColor,line width= 0.4pt,line join=round,line cap=round,fill=fillColor] (127.51,196.78) circle (  1.96);
\path[draw=drawColor,line width= 0.4pt,line join=round,line cap=round,fill=fillColor] (127.77,197.46) circle (  1.96);
\path[draw=drawColor,line width= 0.4pt,line join=round,line cap=round,fill=fillColor] (128.03,198.13) circle (  1.96);
\path[draw=drawColor,line width= 0.4pt,line join=round,line cap=round,fill=fillColor] (128.29,198.81) circle (  1.96);
\path[draw=drawColor,line width= 0.4pt,line join=round,line cap=round,fill=fillColor] (128.55,199.48) circle (  1.96);
\path[draw=drawColor,line width= 0.4pt,line join=round,line cap=round,fill=fillColor] (128.81,200.15) circle (  1.96);
\path[draw=drawColor,line width= 0.4pt,line join=round,line cap=round,fill=fillColor] (129.07,200.82) circle (  1.96);
\path[draw=drawColor,line width= 0.4pt,line join=round,line cap=round,fill=fillColor] (129.33,201.50) circle (  1.96);
\path[draw=drawColor,line width= 0.4pt,line join=round,line cap=round,fill=fillColor] (129.59,202.17) circle (  1.96);
\path[draw=drawColor,line width= 0.4pt,line join=round,line cap=round,fill=fillColor] (129.85,202.84) circle (  1.96);
\path[draw=drawColor,line width= 0.4pt,line join=round,line cap=round,fill=fillColor] (130.11,203.50) circle (  1.96);
\path[draw=drawColor,line width= 0.4pt,line join=round,line cap=round,fill=fillColor] (130.37,204.17) circle (  1.96);
\path[draw=drawColor,line width= 0.4pt,line join=round,line cap=round,fill=fillColor] (130.63,204.83) circle (  1.96);
\path[draw=drawColor,line width= 0.4pt,line join=round,line cap=round,fill=fillColor] (130.89,205.50) circle (  1.96);
\path[draw=drawColor,line width= 0.4pt,line join=round,line cap=round,fill=fillColor] (131.15,206.16) circle (  1.96);
\path[draw=drawColor,line width= 0.4pt,line join=round,line cap=round,fill=fillColor] (131.41,206.82) circle (  1.96);
\path[draw=drawColor,line width= 0.4pt,line join=round,line cap=round,fill=fillColor] (131.67,207.48) circle (  1.96);
\path[draw=drawColor,line width= 0.4pt,line join=round,line cap=round,fill=fillColor] (131.93,208.14) circle (  1.96);
\path[draw=drawColor,line width= 0.4pt,line join=round,line cap=round,fill=fillColor] (132.19,208.80) circle (  1.96);
\path[draw=drawColor,line width= 0.4pt,line join=round,line cap=round,fill=fillColor] (132.45,209.46) circle (  1.96);
\path[draw=drawColor,line width= 0.4pt,line join=round,line cap=round,fill=fillColor] (132.71,210.11) circle (  1.96);
\path[draw=drawColor,line width= 0.4pt,line join=round,line cap=round,fill=fillColor] (132.97,210.77) circle (  1.96);
\path[draw=drawColor,line width= 0.4pt,line join=round,line cap=round,fill=fillColor] (133.23,211.41) circle (  1.96);
\path[draw=drawColor,line width= 0.4pt,line join=round,line cap=round,fill=fillColor] (133.49,212.06) circle (  1.96);
\path[draw=drawColor,line width= 0.4pt,line join=round,line cap=round,fill=fillColor] (133.75,212.71) circle (  1.96);
\path[draw=drawColor,line width= 0.4pt,line join=round,line cap=round,fill=fillColor] (134.01,213.37) circle (  1.96);
\path[draw=drawColor,line width= 0.4pt,line join=round,line cap=round,fill=fillColor] (134.27,214.01) circle (  1.96);
\path[draw=drawColor,line width= 0.4pt,line join=round,line cap=round,fill=fillColor] (134.53,214.66) circle (  1.96);
\path[draw=drawColor,line width= 0.4pt,line join=round,line cap=round,fill=fillColor] (134.79,215.31) circle (  1.96);
\path[draw=drawColor,line width= 0.4pt,line join=round,line cap=round,fill=fillColor] (135.05,215.96) circle (  1.96);
\path[draw=drawColor,line width= 0.4pt,line join=round,line cap=round,fill=fillColor] (135.31,216.60) circle (  1.96);
\path[draw=drawColor,line width= 0.4pt,line join=round,line cap=round,fill=fillColor] (135.57,217.25) circle (  1.96);
\path[draw=drawColor,line width= 0.4pt,line join=round,line cap=round,fill=fillColor] (135.83,217.89) circle (  1.96);
\path[draw=drawColor,line width= 0.4pt,line join=round,line cap=round,fill=fillColor] (136.09,218.52) circle (  1.96);
\path[draw=drawColor,line width= 0.4pt,line join=round,line cap=round,fill=fillColor] (136.35,219.16) circle (  1.96);
\path[draw=drawColor,line width= 0.4pt,line join=round,line cap=round,fill=fillColor] (136.61,219.80) circle (  1.96);
\path[draw=drawColor,line width= 0.4pt,line join=round,line cap=round,fill=fillColor] (136.87,220.44) circle (  1.96);
\path[draw=drawColor,line width= 0.4pt,line join=round,line cap=round,fill=fillColor] (137.13,221.08) circle (  1.96);
\path[draw=drawColor,line width= 0.4pt,line join=round,line cap=round,fill=fillColor] (137.39,221.72) circle (  1.96);
\path[draw=drawColor,line width= 0.4pt,line join=round,line cap=round,fill=fillColor] (137.65,222.35) circle (  1.96);
\path[draw=drawColor,line width= 0.4pt,line join=round,line cap=round,fill=fillColor] (137.91,222.98) circle (  1.96);
\path[draw=drawColor,line width= 0.4pt,line join=round,line cap=round,fill=fillColor] (138.17,223.61) circle (  1.96);
\path[draw=drawColor,line width= 0.4pt,line join=round,line cap=round,fill=fillColor] (138.43,224.24) circle (  1.96);
\path[draw=drawColor,line width= 0.4pt,line join=round,line cap=round,fill=fillColor] (138.69,224.87) circle (  1.96);
\path[draw=drawColor,line width= 0.4pt,line join=round,line cap=round,fill=fillColor] (138.95,225.50) circle (  1.96);
\path[draw=drawColor,line width= 0.4pt,line join=round,line cap=round,fill=fillColor] (139.21,226.12) circle (  1.96);
\path[draw=drawColor,line width= 0.4pt,line join=round,line cap=round,fill=fillColor] (139.47,226.75) circle (  1.96);
\path[draw=drawColor,line width= 0.4pt,line join=round,line cap=round,fill=fillColor] (139.73,227.38) circle (  1.96);
\path[draw=drawColor,line width= 0.4pt,line join=round,line cap=round,fill=fillColor] (139.99,228.00) circle (  1.96);
\path[draw=drawColor,line width= 0.4pt,line join=round,line cap=round,fill=fillColor] (140.25,228.62) circle (  1.96);
\path[draw=drawColor,line width= 0.4pt,line join=round,line cap=round,fill=fillColor] (140.51,229.24) circle (  1.96);
\path[draw=drawColor,line width= 0.4pt,line join=round,line cap=round,fill=fillColor] (140.77,229.87) circle (  1.96);
\path[draw=drawColor,line width= 0.4pt,line join=round,line cap=round,fill=fillColor] (141.03,230.49) circle (  1.96);
\path[draw=drawColor,line width= 0.4pt,line join=round,line cap=round,fill=fillColor] (141.29,231.10) circle (  1.96);
\path[draw=drawColor,line width= 0.4pt,line join=round,line cap=round,fill=fillColor] (141.55,231.72) circle (  1.96);
\path[draw=drawColor,line width= 0.4pt,line join=round,line cap=round,fill=fillColor] (141.81,232.34) circle (  1.96);
\path[draw=drawColor,line width= 0.4pt,line join=round,line cap=round,fill=fillColor] (142.07,232.95) circle (  1.96);
\path[draw=drawColor,line width= 0.4pt,line join=round,line cap=round,fill=fillColor] (142.33,233.56) circle (  1.96);
\path[draw=drawColor,line width= 0.4pt,line join=round,line cap=round,fill=fillColor] (142.59,234.18) circle (  1.96);
\path[draw=drawColor,line width= 0.4pt,line join=round,line cap=round,fill=fillColor] (142.85,234.79) circle (  1.96);
\path[draw=drawColor,line width= 0.4pt,line join=round,line cap=round,fill=fillColor] (143.11,235.39) circle (  1.96);
\path[draw=drawColor,line width= 0.4pt,line join=round,line cap=round,fill=fillColor] (143.37,236.01) circle (  1.96);
\path[draw=drawColor,line width= 0.4pt,line join=round,line cap=round,fill=fillColor] (143.63,236.62) circle (  1.96);
\path[draw=drawColor,line width= 0.4pt,line join=round,line cap=round,fill=fillColor] (143.88,237.22) circle (  1.96);
\path[draw=drawColor,line width= 0.4pt,line join=round,line cap=round,fill=fillColor] (144.14,237.83) circle (  1.96);
\path[draw=drawColor,line width= 0.4pt,line join=round,line cap=round,fill=fillColor] (144.40,238.44) circle (  1.96);
\path[draw=drawColor,line width= 0.4pt,line join=round,line cap=round,fill=fillColor] (144.66,239.03) circle (  1.96);
\path[draw=drawColor,line width= 0.4pt,line join=round,line cap=round,fill=fillColor] (144.92,239.64) circle (  1.96);
\path[draw=drawColor,line width= 0.4pt,line join=round,line cap=round,fill=fillColor] (145.18,240.24) circle (  1.96);
\path[draw=drawColor,line width= 0.4pt,line join=round,line cap=round,fill=fillColor] (145.44,240.84) circle (  1.96);
\path[draw=drawColor,line width= 0.4pt,line join=round,line cap=round,fill=fillColor] (145.70,241.44) circle (  1.96);
\path[draw=drawColor,line width= 0.4pt,line join=round,line cap=round,fill=fillColor] (145.96,242.04) circle (  1.96);
\path[draw=drawColor,line width= 0.4pt,line join=round,line cap=round,fill=fillColor] (146.22,242.63) circle (  1.96);
\path[draw=drawColor,line width= 0.4pt,line join=round,line cap=round,fill=fillColor] (146.48,243.23) circle (  1.96);
\path[draw=drawColor,line width= 0.4pt,line join=round,line cap=round,fill=fillColor] (146.74,243.82) circle (  1.96);
\path[draw=drawColor,line width= 0.4pt,line join=round,line cap=round,fill=fillColor] (147.00,244.42) circle (  1.96);
\path[draw=drawColor,line width= 0.4pt,line join=round,line cap=round,fill=fillColor] (147.26,245.02) circle (  1.96);
\path[draw=drawColor,line width= 0.4pt,line join=round,line cap=round,fill=fillColor] (147.52,245.60) circle (  1.96);
\path[draw=drawColor,line width= 0.4pt,line join=round,line cap=round,fill=fillColor] (147.78,246.20) circle (  1.96);
\path[draw=drawColor,line width= 0.4pt,line join=round,line cap=round,fill=fillColor] (148.04,246.78) circle (  1.96);
\path[draw=drawColor,line width= 0.4pt,line join=round,line cap=round,fill=fillColor] (148.30,247.37) circle (  1.96);
\path[draw=drawColor,line width= 0.4pt,line join=round,line cap=round,fill=fillColor] (148.56,247.95) circle (  1.96);
\path[draw=drawColor,line width= 0.4pt,line join=round,line cap=round,fill=fillColor] (148.82,248.54) circle (  1.96);
\path[draw=drawColor,line width= 0.4pt,line join=round,line cap=round,fill=fillColor] (149.08,249.12) circle (  1.96);
\path[draw=drawColor,line width= 0.4pt,line join=round,line cap=round,fill=fillColor] (149.34,249.71) circle (  1.96);
\path[draw=drawColor,line width= 0.4pt,line join=round,line cap=round,fill=fillColor] (149.60,250.29) circle (  1.96);
\path[draw=drawColor,line width= 0.4pt,line join=round,line cap=round,fill=fillColor] (149.86,250.88) circle (  1.96);
\path[draw=drawColor,line width= 0.4pt,line join=round,line cap=round,fill=fillColor] (150.12,251.45) circle (  1.96);
\path[draw=drawColor,line width= 0.4pt,line join=round,line cap=round,fill=fillColor] (150.38,252.03) circle (  1.96);
\path[draw=drawColor,line width= 0.4pt,line join=round,line cap=round,fill=fillColor] (150.64,252.61) circle (  1.96);
\path[draw=drawColor,line width= 0.4pt,line join=round,line cap=round,fill=fillColor] (150.90,253.19) circle (  1.96);
\path[draw=drawColor,line width= 0.4pt,line join=round,line cap=round,fill=fillColor] (151.16,253.76) circle (  1.96);
\path[draw=drawColor,line width= 0.4pt,line join=round,line cap=round,fill=fillColor] (151.42,254.34) circle (  1.96);
\path[draw=drawColor,line width= 0.4pt,line join=round,line cap=round,fill=fillColor] (151.68,254.91) circle (  1.96);
\path[draw=drawColor,line width= 0.4pt,line join=round,line cap=round,fill=fillColor] (151.94,255.49) circle (  1.96);
\path[draw=drawColor,line width= 0.4pt,line join=round,line cap=round,fill=fillColor] (152.20,256.05) circle (  1.96);
\path[draw=drawColor,line width= 0.4pt,line join=round,line cap=round,fill=fillColor] (152.46,256.63) circle (  1.96);
\path[draw=drawColor,line width= 0.4pt,line join=round,line cap=round,fill=fillColor] (152.72,257.20) circle (  1.96);
\path[draw=drawColor,line width= 0.4pt,line join=round,line cap=round,fill=fillColor] (152.98,257.77) circle (  1.96);
\path[draw=drawColor,line width= 0.4pt,line join=round,line cap=round,fill=fillColor] (153.24,258.33) circle (  1.96);
\path[draw=drawColor,line width= 0.4pt,line join=round,line cap=round,fill=fillColor] (153.50,258.89) circle (  1.96);
\path[draw=drawColor,line width= 0.4pt,line join=round,line cap=round,fill=fillColor] (153.76,259.46) circle (  1.96);
\path[draw=drawColor,line width= 0.4pt,line join=round,line cap=round,fill=fillColor] (154.02,260.02) circle (  1.96);
\path[draw=drawColor,line width= 0.4pt,line join=round,line cap=round,fill=fillColor] (154.28,260.59) circle (  1.96);
\path[draw=drawColor,line width= 0.4pt,line join=round,line cap=round,fill=fillColor] (154.54,261.15) circle (  1.96);
\path[draw=drawColor,line width= 0.4pt,line join=round,line cap=round,fill=fillColor] (154.80,261.71) circle (  1.96);
\path[draw=drawColor,line width= 0.4pt,line join=round,line cap=round,fill=fillColor] (155.06,262.27) circle (  1.96);
\path[draw=drawColor,line width= 0.4pt,line join=round,line cap=round,fill=fillColor] (155.32,262.83) circle (  1.96);
\path[draw=drawColor,line width= 0.4pt,line join=round,line cap=round,fill=fillColor] (155.58,263.39) circle (  1.96);
\path[draw=drawColor,line width= 0.4pt,line join=round,line cap=round,fill=fillColor] (155.84,263.94) circle (  1.96);
\path[draw=drawColor,line width= 0.4pt,line join=round,line cap=round,fill=fillColor] (156.10,264.50) circle (  1.96);
\path[draw=drawColor,line width= 0.4pt,line join=round,line cap=round,fill=fillColor] (156.36,265.06) circle (  1.96);
\path[draw=drawColor,line width= 0.4pt,line join=round,line cap=round,fill=fillColor] (156.62,265.61) circle (  1.96);
\path[draw=drawColor,line width= 0.4pt,line join=round,line cap=round,fill=fillColor] (156.88,266.15) circle (  1.96);
\path[draw=drawColor,line width= 0.4pt,line join=round,line cap=round,fill=fillColor] (157.14,266.71) circle (  1.96);
\path[draw=drawColor,line width= 0.4pt,line join=round,line cap=round,fill=fillColor] (157.40,267.26) circle (  1.96);
\path[draw=drawColor,line width= 0.4pt,line join=round,line cap=round,fill=fillColor] (157.66,267.80) circle (  1.96);
\path[draw=drawColor,line width= 0.4pt,line join=round,line cap=round,fill=fillColor] (157.92,268.36) circle (  1.96);
\path[draw=drawColor,line width= 0.4pt,line join=round,line cap=round,fill=fillColor] (158.18,268.90) circle (  1.96);
\path[draw=drawColor,line width= 0.4pt,line join=round,line cap=round,fill=fillColor] (158.44,269.44) circle (  1.96);
\path[draw=drawColor,line width= 0.4pt,line join=round,line cap=round,fill=fillColor] (158.70,270.00) circle (  1.96);
\path[draw=drawColor,line width= 0.4pt,line join=round,line cap=round,fill=fillColor] (158.96,270.54) circle (  1.96);
\path[draw=drawColor,line width= 0.4pt,line join=round,line cap=round,fill=fillColor] (159.22,271.08) circle (  1.96);
\path[draw=drawColor,line width= 0.4pt,line join=round,line cap=round,fill=fillColor] (159.48,271.61) circle (  1.96);
\path[draw=drawColor,line width= 0.4pt,line join=round,line cap=round,fill=fillColor] (159.74,272.16) circle (  1.96);
\path[draw=drawColor,line width= 0.4pt,line join=round,line cap=round,fill=fillColor] (160.00,272.70) circle (  1.96);
\path[draw=drawColor,line width= 0.4pt,line join=round,line cap=round,fill=fillColor] (160.26,273.23) circle (  1.96);
\path[draw=drawColor,line width= 0.4pt,line join=round,line cap=round,fill=fillColor] (160.51,273.77) circle (  1.96);
\path[draw=drawColor,line width= 0.4pt,line join=round,line cap=round,fill=fillColor] (160.77,274.31) circle (  1.96);
\path[draw=drawColor,line width= 0.4pt,line join=round,line cap=round,fill=fillColor] (161.03,274.84) circle (  1.96);
\path[draw=drawColor,line width= 0.4pt,line join=round,line cap=round,fill=fillColor] (161.29,275.37) circle (  1.96);
\path[draw=drawColor,line width= 0.4pt,line join=round,line cap=round,fill=fillColor] (161.55,275.90) circle (  1.96);
\path[draw=drawColor,line width= 0.4pt,line join=round,line cap=round,fill=fillColor] (161.81,276.43) circle (  1.96);
\path[draw=drawColor,line width= 0.4pt,line join=round,line cap=round,fill=fillColor] (162.07,276.97) circle (  1.96);
\path[draw=drawColor,line width= 0.4pt,line join=round,line cap=round,fill=fillColor] (162.33,277.50) circle (  1.96);
\path[draw=drawColor,line width= 0.4pt,line join=round,line cap=round,fill=fillColor] (162.59,278.03) circle (  1.96);
\path[draw=drawColor,line width= 0.4pt,line join=round,line cap=round,fill=fillColor] (162.85,278.55) circle (  1.96);
\path[draw=drawColor,line width= 0.4pt,line join=round,line cap=round,fill=fillColor] (163.11,279.08) circle (  1.96);
\path[draw=drawColor,line width= 0.4pt,line join=round,line cap=round,fill=fillColor] (163.37,279.61) circle (  1.96);
\path[draw=drawColor,line width= 0.4pt,line join=round,line cap=round,fill=fillColor] (163.63,280.13) circle (  1.96);
\path[draw=drawColor,line width= 0.4pt,line join=round,line cap=round,fill=fillColor] (163.89,280.65) circle (  1.96);
\path[draw=drawColor,line width= 0.4pt,line join=round,line cap=round,fill=fillColor] (164.15,281.17) circle (  1.96);
\path[draw=drawColor,line width= 0.4pt,line join=round,line cap=round,fill=fillColor] (164.41,281.69) circle (  1.96);
\path[draw=drawColor,line width= 0.4pt,line join=round,line cap=round,fill=fillColor] (164.67,282.21) circle (  1.96);
\path[draw=drawColor,line width= 0.4pt,line join=round,line cap=round,fill=fillColor] (164.93,282.74) circle (  1.96);
\path[draw=drawColor,line width= 0.4pt,line join=round,line cap=round,fill=fillColor] (165.19,283.25) circle (  1.96);
\path[draw=drawColor,line width= 0.4pt,line join=round,line cap=round,fill=fillColor] (165.45,283.77) circle (  1.96);
\path[draw=drawColor,line width= 0.4pt,line join=round,line cap=round,fill=fillColor] (165.71,284.28) circle (  1.96);
\path[draw=drawColor,line width= 0.4pt,line join=round,line cap=round,fill=fillColor] (165.97,284.80) circle (  1.96);
\path[draw=drawColor,line width= 0.4pt,line join=round,line cap=round,fill=fillColor] (166.23,285.31) circle (  1.96);
\path[draw=drawColor,line width= 0.4pt,line join=round,line cap=round,fill=fillColor] (166.49,285.82) circle (  1.96);
\path[draw=drawColor,line width= 0.4pt,line join=round,line cap=round,fill=fillColor] (166.75,286.33) circle (  1.96);
\path[draw=drawColor,line width= 0.4pt,line join=round,line cap=round,fill=fillColor] (167.01,286.84) circle (  1.96);
\path[draw=drawColor,line width= 0.4pt,line join=round,line cap=round,fill=fillColor] (167.27,287.35) circle (  1.96);
\path[draw=drawColor,line width= 0.4pt,line join=round,line cap=round,fill=fillColor] (167.53,287.87) circle (  1.96);
\path[draw=drawColor,line width= 0.4pt,line join=round,line cap=round,fill=fillColor] (167.79,288.37) circle (  1.96);
\path[draw=drawColor,line width= 0.4pt,line join=round,line cap=round,fill=fillColor] (168.05,288.88) circle (  1.96);
\path[draw=drawColor,line width= 0.4pt,line join=round,line cap=round,fill=fillColor] (168.31,289.38) circle (  1.96);
\path[draw=drawColor,line width= 0.4pt,line join=round,line cap=round,fill=fillColor] (168.57,289.89) circle (  1.96);
\path[draw=drawColor,line width= 0.4pt,line join=round,line cap=round,fill=fillColor] (168.83,290.39) circle (  1.96);
\path[draw=drawColor,line width= 0.4pt,line join=round,line cap=round,fill=fillColor] (169.09,290.89) circle (  1.96);
\path[draw=drawColor,line width= 0.4pt,line join=round,line cap=round,fill=fillColor] (169.35,291.39) circle (  1.96);
\path[draw=drawColor,line width= 0.4pt,line join=round,line cap=round,fill=fillColor] (169.61,291.89) circle (  1.96);
\path[draw=drawColor,line width= 0.4pt,line join=round,line cap=round,fill=fillColor] (169.87,292.39) circle (  1.96);
\path[draw=drawColor,line width= 0.4pt,line join=round,line cap=round,fill=fillColor] (170.13,292.89) circle (  1.96);
\path[draw=drawColor,line width= 0.4pt,line join=round,line cap=round,fill=fillColor] (170.39,293.39) circle (  1.96);
\path[draw=drawColor,line width= 0.4pt,line join=round,line cap=round,fill=fillColor] (170.65,293.88) circle (  1.96);
\path[draw=drawColor,line width= 0.4pt,line join=round,line cap=round,fill=fillColor] (170.91,294.38) circle (  1.96);
\path[draw=drawColor,line width= 0.4pt,line join=round,line cap=round,fill=fillColor] (171.17,294.87) circle (  1.96);
\path[draw=drawColor,line width= 0.4pt,line join=round,line cap=round,fill=fillColor] (171.43,295.36) circle (  1.96);
\path[draw=drawColor,line width= 0.4pt,line join=round,line cap=round,fill=fillColor] (171.69,295.86) circle (  1.96);
\path[draw=drawColor,line width= 0.4pt,line join=round,line cap=round,fill=fillColor] (171.95,296.35) circle (  1.96);
\path[draw=drawColor,line width= 0.4pt,line join=round,line cap=round,fill=fillColor] (172.21,296.84) circle (  1.96);
\path[draw=drawColor,line width= 0.4pt,line join=round,line cap=round,fill=fillColor] (172.47,297.33) circle (  1.96);
\path[draw=drawColor,line width= 0.4pt,line join=round,line cap=round,fill=fillColor] (172.73,297.81) circle (  1.96);
\path[draw=drawColor,line width= 0.4pt,line join=round,line cap=round,fill=fillColor] (172.99,298.30) circle (  1.96);
\path[draw=drawColor,line width= 0.4pt,line join=round,line cap=round,fill=fillColor] (173.25,298.79) circle (  1.96);
\path[draw=drawColor,line width= 0.4pt,line join=round,line cap=round,fill=fillColor] (173.51,299.26) circle (  1.96);
\path[draw=drawColor,line width= 0.4pt,line join=round,line cap=round,fill=fillColor] (173.77,299.75) circle (  1.96);
\path[draw=drawColor,line width= 0.4pt,line join=round,line cap=round,fill=fillColor] (174.03,300.23) circle (  1.96);
\path[draw=drawColor,line width= 0.4pt,line join=round,line cap=round,fill=fillColor] (174.29,300.71) circle (  1.96);
\path[draw=drawColor,line width= 0.4pt,line join=round,line cap=round,fill=fillColor] (174.55,301.19) circle (  1.96);
\path[draw=drawColor,line width= 0.4pt,line join=round,line cap=round,fill=fillColor] (174.81,301.67) circle (  1.96);
\path[draw=drawColor,line width= 0.4pt,line join=round,line cap=round,fill=fillColor] (175.07,302.15) circle (  1.96);
\path[draw=drawColor,line width= 0.4pt,line join=round,line cap=round,fill=fillColor] (175.33,302.63) circle (  1.96);
\path[draw=drawColor,line width= 0.4pt,line join=round,line cap=round,fill=fillColor] (175.59,303.11) circle (  1.96);
\path[draw=drawColor,line width= 0.4pt,line join=round,line cap=round,fill=fillColor] (175.85,303.59) circle (  1.96);
\path[draw=drawColor,line width= 0.4pt,line join=round,line cap=round,fill=fillColor] (176.11,304.05) circle (  1.96);
\path[draw=drawColor,line width= 0.4pt,line join=round,line cap=round,fill=fillColor] (176.37,304.53) circle (  1.96);
\path[draw=drawColor,line width= 0.4pt,line join=round,line cap=round,fill=fillColor] (176.63,305.00) circle (  1.96);
\path[draw=drawColor,line width= 0.4pt,line join=round,line cap=round,fill=fillColor] (176.89,305.47) circle (  1.96);
\path[draw=drawColor,line width= 0.4pt,line join=round,line cap=round,fill=fillColor] (177.14,305.95) circle (  1.96);
\path[draw=drawColor,line width= 0.4pt,line join=round,line cap=round,fill=fillColor] (177.40,306.42) circle (  1.96);
\path[draw=drawColor,line width= 0.4pt,line join=round,line cap=round,fill=fillColor] (177.66,306.88) circle (  1.96);
\path[draw=drawColor,line width= 0.4pt,line join=round,line cap=round,fill=fillColor] (177.92,307.35) circle (  1.96);
\path[draw=drawColor,line width= 0.4pt,line join=round,line cap=round,fill=fillColor] (178.18,307.81) circle (  1.96);
\path[draw=drawColor,line width= 0.4pt,line join=round,line cap=round,fill=fillColor] (178.44,308.28) circle (  1.96);
\path[draw=drawColor,line width= 0.4pt,line join=round,line cap=round,fill=fillColor] (178.70,308.75) circle (  1.96);
\path[draw=drawColor,line width= 0.4pt,line join=round,line cap=round,fill=fillColor] (178.96,309.20) circle (  1.96);
\path[draw=drawColor,line width= 0.4pt,line join=round,line cap=round,fill=fillColor] (179.22,309.67) circle (  1.96);
\path[draw=drawColor,line width= 0.4pt,line join=round,line cap=round,fill=fillColor] (179.48,310.13) circle (  1.96);
\path[draw=drawColor,line width= 0.4pt,line join=round,line cap=round,fill=fillColor] (179.74,310.59) circle (  1.96);
\path[draw=drawColor,line width= 0.4pt,line join=round,line cap=round,fill=fillColor] (180.00,311.06) circle (  1.96);
\path[draw=drawColor,line width= 0.4pt,line join=round,line cap=round,fill=fillColor] (180.26,311.51) circle (  1.96);
\path[draw=drawColor,line width= 0.4pt,line join=round,line cap=round,fill=fillColor] (180.52,311.97) circle (  1.96);
\path[draw=drawColor,line width= 0.4pt,line join=round,line cap=round,fill=fillColor] (180.78,312.42) circle (  1.96);
\path[draw=drawColor,line width= 0.4pt,line join=round,line cap=round,fill=fillColor] (181.04,312.88) circle (  1.96);
\path[draw=drawColor,line width= 0.4pt,line join=round,line cap=round,fill=fillColor] (181.30,313.33) circle (  1.96);
\path[draw=drawColor,line width= 0.4pt,line join=round,line cap=round,fill=fillColor] (181.56,313.79) circle (  1.96);
\path[draw=drawColor,line width= 0.4pt,line join=round,line cap=round,fill=fillColor] (181.82,314.24) circle (  1.96);
\path[draw=drawColor,line width= 0.4pt,line join=round,line cap=round,fill=fillColor] (182.08,314.69) circle (  1.96);
\path[draw=drawColor,line width= 0.4pt,line join=round,line cap=round,fill=fillColor] (182.34,315.14) circle (  1.96);
\path[draw=drawColor,line width= 0.4pt,line join=round,line cap=round,fill=fillColor] (182.60,315.59) circle (  1.96);
\path[draw=drawColor,line width= 0.4pt,line join=round,line cap=round,fill=fillColor] (182.86,316.04) circle (  1.96);
\path[draw=drawColor,line width= 0.4pt,line join=round,line cap=round,fill=fillColor] (183.12,316.48) circle (  1.96);
\path[draw=drawColor,line width= 0.4pt,line join=round,line cap=round,fill=fillColor] (183.38,316.93) circle (  1.96);
\path[draw=drawColor,line width= 0.4pt,line join=round,line cap=round,fill=fillColor] (183.64,317.38) circle (  1.96);
\path[draw=drawColor,line width= 0.4pt,line join=round,line cap=round,fill=fillColor] (183.90,317.83) circle (  1.96);
\path[draw=drawColor,line width= 0.4pt,line join=round,line cap=round,fill=fillColor] (184.16,318.27) circle (  1.96);
\path[draw=drawColor,line width= 0.4pt,line join=round,line cap=round,fill=fillColor] (184.42,318.71) circle (  1.96);
\path[draw=drawColor,line width= 0.4pt,line join=round,line cap=round,fill=fillColor] (184.68,319.16) circle (  1.96);
\path[draw=drawColor,line width= 0.4pt,line join=round,line cap=round,fill=fillColor] (184.94,319.59) circle (  1.96);
\path[draw=drawColor,line width= 0.4pt,line join=round,line cap=round,fill=fillColor] (185.20,320.03) circle (  1.96);
\path[draw=drawColor,line width= 0.4pt,line join=round,line cap=round,fill=fillColor] (185.46,320.48) circle (  1.96);
\path[draw=drawColor,line width= 0.4pt,line join=round,line cap=round,fill=fillColor] (185.72,320.91) circle (  1.96);
\path[draw=drawColor,line width= 0.4pt,line join=round,line cap=round,fill=fillColor] (185.98,321.35) circle (  1.96);
\path[draw=drawColor,line width= 0.4pt,line join=round,line cap=round,fill=fillColor] (186.24,321.79) circle (  1.96);
\path[draw=drawColor,line width= 0.4pt,line join=round,line cap=round,fill=fillColor] (186.50,322.22) circle (  1.96);
\path[draw=drawColor,line width= 0.4pt,line join=round,line cap=round,fill=fillColor] (186.76,322.65) circle (  1.96);
\path[draw=drawColor,line width= 0.4pt,line join=round,line cap=round,fill=fillColor] (187.02,323.08) circle (  1.96);
\path[draw=drawColor,line width= 0.4pt,line join=round,line cap=round,fill=fillColor] (187.28,323.52) circle (  1.96);
\path[draw=drawColor,line width= 0.4pt,line join=round,line cap=round,fill=fillColor] (187.54,323.95) circle (  1.96);
\path[draw=drawColor,line width= 0.4pt,line join=round,line cap=round,fill=fillColor] (187.80,324.38) circle (  1.96);
\path[draw=drawColor,line width= 0.4pt,line join=round,line cap=round,fill=fillColor] (188.06,324.81) circle (  1.96);
\path[draw=drawColor,line width= 0.4pt,line join=round,line cap=round,fill=fillColor] (188.32,325.23) circle (  1.96);
\path[draw=drawColor,line width= 0.4pt,line join=round,line cap=round,fill=fillColor] (188.58,325.66) circle (  1.96);
\path[draw=drawColor,line width= 0.4pt,line join=round,line cap=round,fill=fillColor] (188.84,326.09) circle (  1.96);
\path[draw=drawColor,line width= 0.4pt,line join=round,line cap=round,fill=fillColor] (189.10,326.51) circle (  1.96);
\path[draw=drawColor,line width= 0.4pt,line join=round,line cap=round,fill=fillColor] (189.36,326.94) circle (  1.96);
\path[draw=drawColor,line width= 0.4pt,line join=round,line cap=round,fill=fillColor] (189.62,327.36) circle (  1.96);
\path[draw=drawColor,line width= 0.4pt,line join=round,line cap=round,fill=fillColor] (189.88,327.79) circle (  1.96);
\path[draw=drawColor,line width= 0.4pt,line join=round,line cap=round,fill=fillColor] (190.14,328.20) circle (  1.96);
\path[draw=drawColor,line width= 0.4pt,line join=round,line cap=round,fill=fillColor] (190.40,328.63) circle (  1.96);
\path[draw=drawColor,line width= 0.4pt,line join=round,line cap=round,fill=fillColor] (190.66,329.04) circle (  1.96);
\path[draw=drawColor,line width= 0.4pt,line join=round,line cap=round,fill=fillColor] (190.92,329.47) circle (  1.96);
\path[draw=drawColor,line width= 0.4pt,line join=round,line cap=round,fill=fillColor] (191.18,329.88) circle (  1.96);
\path[draw=drawColor,line width= 0.4pt,line join=round,line cap=round,fill=fillColor] (191.44,330.30) circle (  1.96);
\path[draw=drawColor,line width= 0.4pt,line join=round,line cap=round,fill=fillColor] (191.70,330.71) circle (  1.96);
\path[draw=drawColor,line width= 0.4pt,line join=round,line cap=round,fill=fillColor] (191.96,331.13) circle (  1.96);
\path[draw=drawColor,line width= 0.4pt,line join=round,line cap=round,fill=fillColor] (192.22,331.55) circle (  1.96);
\path[draw=drawColor,line width= 0.4pt,line join=round,line cap=round,fill=fillColor] (192.48,331.96) circle (  1.96);
\path[draw=drawColor,line width= 0.4pt,line join=round,line cap=round,fill=fillColor] (192.74,332.38) circle (  1.96);
\path[draw=drawColor,line width= 0.4pt,line join=round,line cap=round,fill=fillColor] (193.00,332.78) circle (  1.96);
\path[draw=drawColor,line width= 0.4pt,line join=round,line cap=round,fill=fillColor] (193.26,333.19) circle (  1.96);
\path[draw=drawColor,line width= 0.4pt,line join=round,line cap=round,fill=fillColor] (193.52,333.60) circle (  1.96);
\path[draw=drawColor,line width= 0.4pt,line join=round,line cap=round,fill=fillColor] (193.77,334.01) circle (  1.96);
\path[draw=drawColor,line width= 0.4pt,line join=round,line cap=round,fill=fillColor] (194.03,334.42) circle (  1.96);
\path[draw=drawColor,line width= 0.4pt,line join=round,line cap=round,fill=fillColor] (194.29,334.82) circle (  1.96);
\path[draw=drawColor,line width= 0.4pt,line join=round,line cap=round,fill=fillColor] (194.55,335.23) circle (  1.96);
\path[draw=drawColor,line width= 0.4pt,line join=round,line cap=round,fill=fillColor] (194.81,335.63) circle (  1.96);
\path[draw=drawColor,line width= 0.4pt,line join=round,line cap=round,fill=fillColor] (195.07,336.04) circle (  1.96);
\path[draw=drawColor,line width= 0.4pt,line join=round,line cap=round,fill=fillColor] (195.33,336.44) circle (  1.96);
\path[draw=drawColor,line width= 0.4pt,line join=round,line cap=round,fill=fillColor] (195.59,336.85) circle (  1.96);
\path[draw=drawColor,line width= 0.4pt,line join=round,line cap=round,fill=fillColor] (195.85,337.24) circle (  1.96);
\path[draw=drawColor,line width= 0.4pt,line join=round,line cap=round,fill=fillColor] (196.11,337.64) circle (  1.96);
\path[draw=drawColor,line width= 0.4pt,line join=round,line cap=round,fill=fillColor] (196.37,338.05) circle (  1.96);
\path[draw=drawColor,line width= 0.4pt,line join=round,line cap=round,fill=fillColor] (196.63,338.44) circle (  1.96);
\path[draw=drawColor,line width= 0.4pt,line join=round,line cap=round,fill=fillColor] (196.89,338.84) circle (  1.96);
\path[draw=drawColor,line width= 0.4pt,line join=round,line cap=round,fill=fillColor] (197.15,339.24) circle (  1.96);
\path[draw=drawColor,line width= 0.4pt,line join=round,line cap=round,fill=fillColor] (197.41,339.63) circle (  1.96);
\path[draw=drawColor,line width= 0.4pt,line join=round,line cap=round,fill=fillColor] (197.67,340.03) circle (  1.96);
\path[draw=drawColor,line width= 0.4pt,line join=round,line cap=round,fill=fillColor] (197.93,340.42) circle (  1.96);
\path[draw=drawColor,line width= 0.4pt,line join=round,line cap=round,fill=fillColor] (198.19,340.82) circle (  1.96);
\path[draw=drawColor,line width= 0.4pt,line join=round,line cap=round,fill=fillColor] (198.45,341.21) circle (  1.96);
\path[draw=drawColor,line width= 0.4pt,line join=round,line cap=round,fill=fillColor] (198.71,341.59) circle (  1.96);
\path[draw=drawColor,line width= 0.4pt,line join=round,line cap=round,fill=fillColor] (198.97,341.99) circle (  1.96);
\path[draw=drawColor,line width= 0.4pt,line join=round,line cap=round,fill=fillColor] (199.23,342.38) circle (  1.96);
\path[draw=drawColor,line width= 0.4pt,line join=round,line cap=round,fill=fillColor] (199.49,342.76) circle (  1.96);
\path[draw=drawColor,line width= 0.4pt,line join=round,line cap=round,fill=fillColor] (199.75,343.15) circle (  1.96);
\path[draw=drawColor,line width= 0.4pt,line join=round,line cap=round,fill=fillColor] (200.01,343.54) circle (  1.96);
\path[draw=drawColor,line width= 0.4pt,line join=round,line cap=round,fill=fillColor] (200.27,343.92) circle (  1.96);
\path[draw=drawColor,line width= 0.4pt,line join=round,line cap=round,fill=fillColor] (200.53,344.31) circle (  1.96);
\path[draw=drawColor,line width= 0.4pt,line join=round,line cap=round,fill=fillColor] (200.79,344.69) circle (  1.96);
\path[draw=drawColor,line width= 0.4pt,line join=round,line cap=round,fill=fillColor] (201.05,345.07) circle (  1.96);
\path[draw=drawColor,line width= 0.4pt,line join=round,line cap=round,fill=fillColor] (201.31,345.46) circle (  1.96);
\path[draw=drawColor,line width= 0.4pt,line join=round,line cap=round,fill=fillColor] (201.57,345.84) circle (  1.96);
\path[draw=drawColor,line width= 0.4pt,line join=round,line cap=round,fill=fillColor] (201.83,346.22) circle (  1.96);
\path[draw=drawColor,line width= 0.4pt,line join=round,line cap=round,fill=fillColor] (202.09,346.59) circle (  1.96);
\path[draw=drawColor,line width= 0.4pt,line join=round,line cap=round,fill=fillColor] (202.35,346.98) circle (  1.96);
\path[draw=drawColor,line width= 0.4pt,line join=round,line cap=round,fill=fillColor] (202.61,347.35) circle (  1.96);
\path[draw=drawColor,line width= 0.4pt,line join=round,line cap=round,fill=fillColor] (202.87,347.73) circle (  1.96);
\path[draw=drawColor,line width= 0.4pt,line join=round,line cap=round,fill=fillColor] (203.13,348.11) circle (  1.96);
\path[draw=drawColor,line width= 0.4pt,line join=round,line cap=round,fill=fillColor] (203.39,348.48) circle (  1.96);
\path[draw=drawColor,line width= 0.4pt,line join=round,line cap=round,fill=fillColor] (203.65,348.86) circle (  1.96);
\path[draw=drawColor,line width= 0.4pt,line join=round,line cap=round,fill=fillColor] (203.91,349.23) circle (  1.96);
\path[draw=drawColor,line width= 0.4pt,line join=round,line cap=round,fill=fillColor] (204.17,349.61) circle (  1.96);
\path[draw=drawColor,line width= 0.4pt,line join=round,line cap=round,fill=fillColor] (204.43,349.97) circle (  1.96);
\path[draw=drawColor,line width= 0.4pt,line join=round,line cap=round,fill=fillColor] (204.69,350.34) circle (  1.96);
\path[draw=drawColor,line width= 0.4pt,line join=round,line cap=round,fill=fillColor] (204.95,350.71) circle (  1.96);
\path[draw=drawColor,line width= 0.4pt,line join=round,line cap=round,fill=fillColor] (205.21,351.09) circle (  1.96);
\path[draw=drawColor,line width= 0.4pt,line join=round,line cap=round,fill=fillColor] (205.47,351.45) circle (  1.96);
\path[draw=drawColor,line width= 0.4pt,line join=round,line cap=round,fill=fillColor] (205.73,351.82) circle (  1.96);
\path[draw=drawColor,line width= 0.4pt,line join=round,line cap=round,fill=fillColor] (205.99,352.18) circle (  1.96);
\path[draw=drawColor,line width= 0.4pt,line join=round,line cap=round,fill=fillColor] (206.25,352.55) circle (  1.96);
\path[draw=drawColor,line width= 0.4pt,line join=round,line cap=round,fill=fillColor] (206.51,352.92) circle (  1.96);
\path[draw=drawColor,line width= 0.4pt,line join=round,line cap=round,fill=fillColor] (206.77,353.28) circle (  1.96);
\path[draw=drawColor,line width= 0.4pt,line join=round,line cap=round,fill=fillColor] (207.03,353.64) circle (  1.96);
\path[draw=drawColor,line width= 0.4pt,line join=round,line cap=round,fill=fillColor] (207.29,354.00) circle (  1.96);
\path[draw=drawColor,line width= 0.4pt,line join=round,line cap=round,fill=fillColor] (207.55,354.36) circle (  1.96);
\path[draw=drawColor,line width= 0.4pt,line join=round,line cap=round,fill=fillColor] (207.81,354.73) circle (  1.96);
\path[draw=drawColor,line width= 0.4pt,line join=round,line cap=round,fill=fillColor] (208.07,355.09) circle (  1.96);
\path[draw=drawColor,line width= 0.4pt,line join=round,line cap=round,fill=fillColor] (208.33,355.44) circle (  1.96);
\path[draw=drawColor,line width= 0.4pt,line join=round,line cap=round,fill=fillColor] (208.59,355.80) circle (  1.96);
\path[draw=drawColor,line width= 0.4pt,line join=round,line cap=round,fill=fillColor] (208.85,356.15) circle (  1.96);
\path[draw=drawColor,line width= 0.4pt,line join=round,line cap=round,fill=fillColor] (209.11,356.51) circle (  1.96);
\path[draw=drawColor,line width= 0.4pt,line join=round,line cap=round,fill=fillColor] (209.37,356.87) circle (  1.96);
\path[draw=drawColor,line width= 0.4pt,line join=round,line cap=round,fill=fillColor] (209.63,357.23) circle (  1.96);
\path[draw=drawColor,line width= 0.4pt,line join=round,line cap=round,fill=fillColor] (209.89,357.58) circle (  1.96);
\path[draw=drawColor,line width= 0.4pt,line join=round,line cap=round,fill=fillColor] (210.14,357.93) circle (  1.96);
\path[draw=drawColor,line width= 0.4pt,line join=round,line cap=round,fill=fillColor] (210.40,358.28) circle (  1.96);
\path[draw=drawColor,line width= 0.4pt,line join=round,line cap=round,fill=fillColor] (210.66,358.63) circle (  1.96);
\path[draw=drawColor,line width= 0.4pt,line join=round,line cap=round,fill=fillColor] (210.92,358.98) circle (  1.96);
\path[draw=drawColor,line width= 0.4pt,line join=round,line cap=round,fill=fillColor] (211.18,359.33) circle (  1.96);
\path[draw=drawColor,line width= 0.4pt,line join=round,line cap=round,fill=fillColor] (211.44,359.68) circle (  1.96);
\path[draw=drawColor,line width= 0.4pt,line join=round,line cap=round,fill=fillColor] (211.70,360.03) circle (  1.96);
\path[draw=drawColor,line width= 0.4pt,line join=round,line cap=round,fill=fillColor] (211.96,360.38) circle (  1.96);
\path[draw=drawColor,line width= 0.4pt,line join=round,line cap=round,fill=fillColor] (212.22,360.72) circle (  1.96);
\path[draw=drawColor,line width= 0.4pt,line join=round,line cap=round,fill=fillColor] (212.48,361.06) circle (  1.96);
\path[draw=drawColor,line width= 0.4pt,line join=round,line cap=round,fill=fillColor] (212.74,361.41) circle (  1.96);
\path[draw=drawColor,line width= 0.4pt,line join=round,line cap=round,fill=fillColor] (213.00,361.75) circle (  1.96);
\path[draw=drawColor,line width= 0.4pt,line join=round,line cap=round,fill=fillColor] (213.26,362.09) circle (  1.96);
\path[draw=drawColor,line width= 0.4pt,line join=round,line cap=round,fill=fillColor] (213.52,362.43) circle (  1.96);
\path[draw=drawColor,line width= 0.4pt,line join=round,line cap=round,fill=fillColor] (213.78,362.77) circle (  1.96);
\path[draw=drawColor,line width= 0.4pt,line join=round,line cap=round,fill=fillColor] (214.04,363.11) circle (  1.96);
\path[draw=drawColor,line width= 0.4pt,line join=round,line cap=round,fill=fillColor] (214.30,363.45) circle (  1.96);
\path[draw=drawColor,line width= 0.4pt,line join=round,line cap=round,fill=fillColor] (214.56,363.79) circle (  1.96);
\path[draw=drawColor,line width= 0.4pt,line join=round,line cap=round,fill=fillColor] (214.82,364.13) circle (  1.96);
\path[draw=drawColor,line width= 0.4pt,line join=round,line cap=round,fill=fillColor] (215.08,364.46) circle (  1.96);
\path[draw=drawColor,line width= 0.4pt,line join=round,line cap=round,fill=fillColor] (215.34,364.81) circle (  1.96);
\path[draw=drawColor,line width= 0.4pt,line join=round,line cap=round,fill=fillColor] (215.60,365.14) circle (  1.96);
\path[draw=drawColor,line width= 0.4pt,line join=round,line cap=round,fill=fillColor] (215.86,365.48) circle (  1.96);
\path[draw=drawColor,line width= 0.4pt,line join=round,line cap=round,fill=fillColor] (216.12,365.81) circle (  1.96);
\path[draw=drawColor,line width= 0.4pt,line join=round,line cap=round,fill=fillColor] (216.38,366.14) circle (  1.96);
\path[draw=drawColor,line width= 0.4pt,line join=round,line cap=round,fill=fillColor] (216.64,366.48) circle (  1.96);
\path[draw=drawColor,line width= 0.4pt,line join=round,line cap=round,fill=fillColor] (216.90,366.81) circle (  1.96);
\path[draw=drawColor,line width= 0.4pt,line join=round,line cap=round,fill=fillColor] (217.16,367.14) circle (  1.96);
\path[draw=drawColor,line width= 0.4pt,line join=round,line cap=round,fill=fillColor] (217.42,367.47) circle (  1.96);
\path[draw=drawColor,line width= 0.4pt,line join=round,line cap=round,fill=fillColor] (217.68,367.80) circle (  1.96);
\path[draw=drawColor,line width= 0.4pt,line join=round,line cap=round,fill=fillColor] (217.94,368.12) circle (  1.96);
\path[draw=drawColor,line width= 0.4pt,line join=round,line cap=round,fill=fillColor] (218.20,368.45) circle (  1.96);
\path[draw=drawColor,line width= 0.4pt,line join=round,line cap=round,fill=fillColor] (218.46,368.78) circle (  1.96);
\path[draw=drawColor,line width= 0.4pt,line join=round,line cap=round,fill=fillColor] (218.72,369.09) circle (  1.96);
\path[draw=drawColor,line width= 0.4pt,line join=round,line cap=round,fill=fillColor] (218.98,369.42) circle (  1.96);
\path[draw=drawColor,line width= 0.4pt,line join=round,line cap=round,fill=fillColor] (219.24,369.74) circle (  1.96);
\path[draw=drawColor,line width= 0.4pt,line join=round,line cap=round,fill=fillColor] (219.50,370.07) circle (  1.96);
\path[draw=drawColor,line width= 0.4pt,line join=round,line cap=round,fill=fillColor] (219.76,370.39) circle (  1.96);
\path[draw=drawColor,line width= 0.4pt,line join=round,line cap=round,fill=fillColor] (220.02,370.71) circle (  1.96);
\path[draw=drawColor,line width= 0.4pt,line join=round,line cap=round,fill=fillColor] (220.28,371.03) circle (  1.96);
\path[draw=drawColor,line width= 0.4pt,line join=round,line cap=round,fill=fillColor] (220.54,371.35) circle (  1.96);
\path[draw=drawColor,line width= 0.4pt,line join=round,line cap=round,fill=fillColor] (220.80,371.67) circle (  1.96);
\path[draw=drawColor,line width= 0.4pt,line join=round,line cap=round,fill=fillColor] (221.06,371.99) circle (  1.96);
\path[draw=drawColor,line width= 0.4pt,line join=round,line cap=round,fill=fillColor] (221.32,372.31) circle (  1.96);
\path[draw=drawColor,line width= 0.4pt,line join=round,line cap=round,fill=fillColor] (221.58,372.63) circle (  1.96);
\path[draw=drawColor,line width= 0.4pt,line join=round,line cap=round,fill=fillColor] (221.84,372.94) circle (  1.96);
\path[draw=drawColor,line width= 0.4pt,line join=round,line cap=round,fill=fillColor] (222.10,373.26) circle (  1.96);
\path[draw=drawColor,line width= 0.4pt,line join=round,line cap=round,fill=fillColor] (222.36,373.58) circle (  1.96);
\path[draw=drawColor,line width= 0.4pt,line join=round,line cap=round,fill=fillColor] (222.62,373.88) circle (  1.96);
\path[draw=drawColor,line width= 0.4pt,line join=round,line cap=round,fill=fillColor] (222.88,374.19) circle (  1.96);
\path[draw=drawColor,line width= 0.4pt,line join=round,line cap=round,fill=fillColor] (223.14,374.51) circle (  1.96);
\path[draw=drawColor,line width= 0.4pt,line join=round,line cap=round,fill=fillColor] (223.40,374.82) circle (  1.96);
\path[draw=drawColor,line width= 0.4pt,line join=round,line cap=round,fill=fillColor] (223.66,375.13) circle (  1.96);
\path[draw=drawColor,line width= 0.4pt,line join=round,line cap=round,fill=fillColor] (223.92,375.44) circle (  1.96);
\path[draw=drawColor,line width= 0.4pt,line join=round,line cap=round,fill=fillColor] (224.18,375.75) circle (  1.96);
\path[draw=drawColor,line width= 0.4pt,line join=round,line cap=round,fill=fillColor] (224.44,376.06) circle (  1.96);
\path[draw=drawColor,line width= 0.4pt,line join=round,line cap=round,fill=fillColor] (224.70,376.36) circle (  1.96);
\path[draw=drawColor,line width= 0.4pt,line join=round,line cap=round,fill=fillColor] (224.96,376.67) circle (  1.96);
\path[draw=drawColor,line width= 0.4pt,line join=round,line cap=round,fill=fillColor] (225.22,376.98) circle (  1.96);
\path[draw=drawColor,line width= 0.4pt,line join=round,line cap=round,fill=fillColor] (225.48,377.28) circle (  1.96);
\path[draw=drawColor,line width= 0.4pt,line join=round,line cap=round,fill=fillColor] (225.74,377.59) circle (  1.96);
\path[draw=drawColor,line width= 0.4pt,line join=round,line cap=round,fill=fillColor] (226.00,377.89) circle (  1.96);
\path[draw=drawColor,line width= 0.4pt,line join=round,line cap=round,fill=fillColor] (226.26,378.19) circle (  1.96);
\path[draw=drawColor,line width= 0.4pt,line join=round,line cap=round,fill=fillColor] (226.52,378.49) circle (  1.96);
\path[draw=drawColor,line width= 0.4pt,line join=round,line cap=round,fill=fillColor] (226.77,378.79) circle (  1.96);
\path[draw=drawColor,line width= 0.4pt,line join=round,line cap=round,fill=fillColor] (227.03,379.10) circle (  1.96);
\path[draw=drawColor,line width= 0.4pt,line join=round,line cap=round,fill=fillColor] (227.29,379.40) circle (  1.96);
\path[draw=drawColor,line width= 0.4pt,line join=round,line cap=round,fill=fillColor] (227.55,379.70) circle (  1.96);
\path[draw=drawColor,line width= 0.4pt,line join=round,line cap=round,fill=fillColor] (227.81,379.99) circle (  1.96);
\path[draw=drawColor,line width= 0.4pt,line join=round,line cap=round,fill=fillColor] (228.07,380.29) circle (  1.96);
\path[draw=drawColor,line width= 0.4pt,line join=round,line cap=round,fill=fillColor] (228.33,380.59) circle (  1.96);
\path[draw=drawColor,line width= 0.4pt,line join=round,line cap=round,fill=fillColor] (228.59,380.89) circle (  1.96);
\path[draw=drawColor,line width= 0.4pt,line join=round,line cap=round,fill=fillColor] (228.85,381.17) circle (  1.96);
\path[draw=drawColor,line width= 0.4pt,line join=round,line cap=round,fill=fillColor] (229.11,381.47) circle (  1.96);
\path[draw=drawColor,line width= 0.4pt,line join=round,line cap=round,fill=fillColor] (229.37,381.77) circle (  1.96);
\path[draw=drawColor,line width= 0.4pt,line join=round,line cap=round,fill=fillColor] (229.63,382.06) circle (  1.96);
\path[draw=drawColor,line width= 0.4pt,line join=round,line cap=round,fill=fillColor] (229.89,382.36) circle (  1.96);
\path[draw=drawColor,line width= 0.4pt,line join=round,line cap=round,fill=fillColor] (230.15,382.64) circle (  1.96);
\path[draw=drawColor,line width= 0.4pt,line join=round,line cap=round,fill=fillColor] (230.41,382.93) circle (  1.96);
\path[draw=drawColor,line width= 0.4pt,line join=round,line cap=round,fill=fillColor] (230.67,383.22) circle (  1.96);
\path[draw=drawColor,line width= 0.4pt,line join=round,line cap=round,fill=fillColor] (230.93,383.52) circle (  1.96);
\path[draw=drawColor,line width= 0.4pt,line join=round,line cap=round,fill=fillColor] (231.19,383.80) circle (  1.96);
\path[draw=drawColor,line width= 0.4pt,line join=round,line cap=round,fill=fillColor] (231.45,384.09) circle (  1.96);
\path[draw=drawColor,line width= 0.4pt,line join=round,line cap=round,fill=fillColor] (231.71,384.38) circle (  1.96);
\path[draw=drawColor,line width= 0.4pt,line join=round,line cap=round,fill=fillColor] (231.97,384.67) circle (  1.96);
\path[draw=drawColor,line width= 0.4pt,line join=round,line cap=round,fill=fillColor] (232.23,384.94) circle (  1.96);
\path[draw=drawColor,line width= 0.4pt,line join=round,line cap=round,fill=fillColor] (232.49,385.23) circle (  1.96);
\path[draw=drawColor,line width= 0.4pt,line join=round,line cap=round,fill=fillColor] (232.75,385.52) circle (  1.96);
\path[draw=drawColor,line width= 0.4pt,line join=round,line cap=round,fill=fillColor] (233.01,385.80) circle (  1.96);
\path[draw=drawColor,line width= 0.4pt,line join=round,line cap=round,fill=fillColor] (233.27,386.08) circle (  1.96);
\path[draw=drawColor,line width= 0.4pt,line join=round,line cap=round,fill=fillColor] (233.53,386.37) circle (  1.96);
\path[draw=drawColor,line width= 0.4pt,line join=round,line cap=round,fill=fillColor] (233.79,386.65) circle (  1.96);
\path[draw=drawColor,line width= 0.4pt,line join=round,line cap=round,fill=fillColor] (234.05,386.92) circle (  1.96);
\path[draw=drawColor,line width= 0.4pt,line join=round,line cap=round,fill=fillColor] (234.31,387.21) circle (  1.96);
\path[draw=drawColor,line width= 0.4pt,line join=round,line cap=round,fill=fillColor] (234.57,387.49) circle (  1.96);
\path[draw=drawColor,line width= 0.4pt,line join=round,line cap=round,fill=fillColor] (234.83,387.76) circle (  1.96);
\path[draw=drawColor,line width= 0.4pt,line join=round,line cap=round,fill=fillColor] (235.09,388.04) circle (  1.96);
\path[draw=drawColor,line width= 0.4pt,line join=round,line cap=round,fill=fillColor] (235.35,388.32) circle (  1.96);
\path[draw=drawColor,line width= 0.4pt,line join=round,line cap=round,fill=fillColor] (235.61,388.59) circle (  1.96);
\path[draw=drawColor,line width= 0.4pt,line join=round,line cap=round,fill=fillColor] (235.87,388.87) circle (  1.96);
\path[draw=drawColor,line width= 0.4pt,line join=round,line cap=round,fill=fillColor] (236.13,389.15) circle (  1.96);
\path[draw=drawColor,line width= 0.4pt,line join=round,line cap=round,fill=fillColor] (236.39,389.42) circle (  1.96);
\path[draw=drawColor,line width= 0.4pt,line join=round,line cap=round,fill=fillColor] (236.65,389.69) circle (  1.96);
\path[draw=drawColor,line width= 0.4pt,line join=round,line cap=round,fill=fillColor] (236.91,389.97) circle (  1.96);
\path[draw=drawColor,line width= 0.4pt,line join=round,line cap=round,fill=fillColor] (237.17,390.23) circle (  1.96);
\path[draw=drawColor,line width= 0.4pt,line join=round,line cap=round,fill=fillColor] (237.43,390.51) circle (  1.96);
\path[draw=drawColor,line width= 0.4pt,line join=round,line cap=round,fill=fillColor] (237.69,390.77) circle (  1.96);
\path[draw=drawColor,line width= 0.4pt,line join=round,line cap=round,fill=fillColor] (237.95,391.05) circle (  1.96);
\path[draw=drawColor,line width= 0.4pt,line join=round,line cap=round,fill=fillColor] (238.21,391.32) circle (  1.96);
\path[draw=drawColor,line width= 0.4pt,line join=round,line cap=round,fill=fillColor] (238.47,391.58) circle (  1.96);
\path[draw=drawColor,line width= 0.4pt,line join=round,line cap=round,fill=fillColor] (238.73,391.85) circle (  1.96);
\path[draw=drawColor,line width= 0.4pt,line join=round,line cap=round,fill=fillColor] (238.99,392.12) circle (  1.96);
\path[draw=drawColor,line width= 0.4pt,line join=round,line cap=round,fill=fillColor] (239.25,392.38) circle (  1.96);
\path[draw=drawColor,line width= 0.4pt,line join=round,line cap=round,fill=fillColor] (239.51,392.65) circle (  1.96);
\path[draw=drawColor,line width= 0.4pt,line join=round,line cap=round,fill=fillColor] (239.77,392.91) circle (  1.96);
\path[draw=drawColor,line width= 0.4pt,line join=round,line cap=round,fill=fillColor] (240.03,393.18) circle (  1.96);
\path[draw=drawColor,line width= 0.4pt,line join=round,line cap=round,fill=fillColor] (240.29,393.45) circle (  1.96);
\path[draw=drawColor,line width= 0.4pt,line join=round,line cap=round,fill=fillColor] (240.55,393.70) circle (  1.96);
\path[draw=drawColor,line width= 0.4pt,line join=round,line cap=round,fill=fillColor] (240.81,393.97) circle (  1.96);
\path[draw=drawColor,line width= 0.4pt,line join=round,line cap=round,fill=fillColor] (241.07,394.23) circle (  1.96);
\path[draw=drawColor,line width= 0.4pt,line join=round,line cap=round,fill=fillColor] (241.33,394.49) circle (  1.96);
\path[draw=drawColor,line width= 0.4pt,line join=round,line cap=round,fill=fillColor] (241.59,394.76) circle (  1.96);
\path[draw=drawColor,line width= 0.4pt,line join=round,line cap=round,fill=fillColor] (241.85,395.01) circle (  1.96);
\path[draw=drawColor,line width= 0.4pt,line join=round,line cap=round,fill=fillColor] (242.11,395.27) circle (  1.96);
\path[draw=drawColor,line width= 0.4pt,line join=round,line cap=round,fill=fillColor] (242.37,395.53) circle (  1.96);
\path[draw=drawColor,line width= 0.4pt,line join=round,line cap=round,fill=fillColor] (242.63,395.79) circle (  1.96);
\path[draw=drawColor,line width= 0.4pt,line join=round,line cap=round,fill=fillColor] (242.89,396.04) circle (  1.96);
\path[draw=drawColor,line width= 0.4pt,line join=round,line cap=round,fill=fillColor] (243.15,396.30) circle (  1.96);
\path[draw=drawColor,line width= 0.4pt,line join=round,line cap=round,fill=fillColor] (243.40,396.55) circle (  1.96);
\path[draw=drawColor,line width= 0.4pt,line join=round,line cap=round,fill=fillColor] (243.66,396.81) circle (  1.96);
\path[draw=drawColor,line width= 0.4pt,line join=round,line cap=round,fill=fillColor] (243.92,397.07) circle (  1.96);
\path[draw=drawColor,line width= 0.4pt,line join=round,line cap=round,fill=fillColor] (244.18,397.31) circle (  1.96);
\path[draw=drawColor,line width= 0.4pt,line join=round,line cap=round,fill=fillColor] (244.44,397.57) circle (  1.96);
\path[draw=drawColor,line width= 0.4pt,line join=round,line cap=round,fill=fillColor] (244.70,397.82) circle (  1.96);
\path[draw=drawColor,line width= 0.4pt,line join=round,line cap=round,fill=fillColor] (244.96,398.08) circle (  1.96);
\path[draw=drawColor,line width= 0.4pt,line join=round,line cap=round,fill=fillColor] (245.22,398.32) circle (  1.96);
\path[draw=drawColor,line width= 0.4pt,line join=round,line cap=round,fill=fillColor] (245.48,398.58) circle (  1.96);
\path[draw=drawColor,line width= 0.4pt,line join=round,line cap=round,fill=fillColor] (245.74,398.82) circle (  1.96);
\path[draw=drawColor,line width= 0.4pt,line join=round,line cap=round,fill=fillColor] (246.00,399.07) circle (  1.96);
\path[draw=drawColor,line width= 0.4pt,line join=round,line cap=round,fill=fillColor] (246.26,399.32) circle (  1.96);
\path[draw=drawColor,line width= 0.4pt,line join=round,line cap=round,fill=fillColor] (246.52,399.57) circle (  1.96);
\path[draw=drawColor,line width= 0.4pt,line join=round,line cap=round,fill=fillColor] (246.78,399.81) circle (  1.96);
\path[draw=drawColor,line width= 0.4pt,line join=round,line cap=round,fill=fillColor] (247.04,400.06) circle (  1.96);
\path[draw=drawColor,line width= 0.4pt,line join=round,line cap=round,fill=fillColor] (247.30,400.30) circle (  1.96);
\path[draw=drawColor,line width= 0.4pt,line join=round,line cap=round,fill=fillColor] (247.56,400.55) circle (  1.96);
\path[draw=drawColor,line width= 0.4pt,line join=round,line cap=round,fill=fillColor] (247.82,400.79) circle (  1.96);
\path[draw=drawColor,line width= 0.4pt,line join=round,line cap=round,fill=fillColor] (248.08,401.04) circle (  1.96);
\path[draw=drawColor,line width= 0.4pt,line join=round,line cap=round,fill=fillColor] (248.34,401.28) circle (  1.96);
\path[draw=drawColor,line width= 0.4pt,line join=round,line cap=round,fill=fillColor] (248.60,401.52) circle (  1.96);
\path[draw=drawColor,line width= 0.4pt,line join=round,line cap=round,fill=fillColor] (248.86,401.76) circle (  1.96);
\path[draw=drawColor,line width= 0.4pt,line join=round,line cap=round,fill=fillColor] (249.12,402.00) circle (  1.96);
\path[draw=drawColor,line width= 0.4pt,line join=round,line cap=round,fill=fillColor] (249.38,402.24) circle (  1.96);
\path[draw=drawColor,line width= 0.4pt,line join=round,line cap=round,fill=fillColor] (249.64,402.48) circle (  1.96);
\path[draw=drawColor,line width= 0.4pt,line join=round,line cap=round,fill=fillColor] (249.90,402.72) circle (  1.96);
\path[draw=drawColor,line width= 0.4pt,line join=round,line cap=round,fill=fillColor] (250.16,402.96) circle (  1.96);
\path[draw=drawColor,line width= 0.4pt,line join=round,line cap=round,fill=fillColor] (250.42,403.20) circle (  1.96);
\path[draw=drawColor,line width= 0.4pt,line join=round,line cap=round,fill=fillColor] (250.68,403.43) circle (  1.96);
\path[draw=drawColor,line width= 0.4pt,line join=round,line cap=round,fill=fillColor] (250.94,403.67) circle (  1.96);
\path[draw=drawColor,line width= 0.4pt,line join=round,line cap=round,fill=fillColor] (251.20,403.91) circle (  1.96);
\path[draw=drawColor,line width= 0.4pt,line join=round,line cap=round,fill=fillColor] (251.46,404.14) circle (  1.96);
\path[draw=drawColor,line width= 0.4pt,line join=round,line cap=round,fill=fillColor] (251.72,404.38) circle (  1.96);
\path[draw=drawColor,line width= 0.4pt,line join=round,line cap=round,fill=fillColor] (251.98,404.61) circle (  1.96);
\path[draw=drawColor,line width= 0.4pt,line join=round,line cap=round,fill=fillColor] (252.24,404.85) circle (  1.96);
\path[draw=drawColor,line width= 0.4pt,line join=round,line cap=round,fill=fillColor] (252.50,405.08) circle (  1.96);
\path[draw=drawColor,line width= 0.4pt,line join=round,line cap=round,fill=fillColor] (252.76,405.30) circle (  1.96);
\path[draw=drawColor,line width= 0.4pt,line join=round,line cap=round,fill=fillColor] (253.02,405.54) circle (  1.96);
\path[draw=drawColor,line width= 0.4pt,line join=round,line cap=round,fill=fillColor] (253.28,405.77) circle (  1.96);
\path[draw=drawColor,line width= 0.4pt,line join=round,line cap=round,fill=fillColor] (253.54,405.99) circle (  1.96);
\path[draw=drawColor,line width= 0.4pt,line join=round,line cap=round,fill=fillColor] (253.80,406.23) circle (  1.96);
\path[draw=drawColor,line width= 0.4pt,line join=round,line cap=round,fill=fillColor] (254.06,406.45) circle (  1.96);
\path[draw=drawColor,line width= 0.4pt,line join=round,line cap=round,fill=fillColor] (254.32,406.69) circle (  1.96);
\path[draw=drawColor,line width= 0.4pt,line join=round,line cap=round,fill=fillColor] (254.58,406.91) circle (  1.96);
\path[draw=drawColor,line width= 0.4pt,line join=round,line cap=round,fill=fillColor] (254.84,407.14) circle (  1.96);
\path[draw=drawColor,line width= 0.4pt,line join=round,line cap=round,fill=fillColor] (255.10,407.37) circle (  1.96);
\path[draw=drawColor,line width= 0.4pt,line join=round,line cap=round,fill=fillColor] (255.36,407.59) circle (  1.96);
\path[draw=drawColor,line width= 0.4pt,line join=round,line cap=round,fill=fillColor] (255.62,407.81) circle (  1.96);
\path[draw=drawColor,line width= 0.4pt,line join=round,line cap=round,fill=fillColor] (255.88,408.04) circle (  1.96);
\path[draw=drawColor,line width= 0.4pt,line join=round,line cap=round,fill=fillColor] (256.14,408.26) circle (  1.96);
\path[draw=drawColor,line width= 0.4pt,line join=round,line cap=round,fill=fillColor] (256.40,408.49) circle (  1.96);
\path[draw=drawColor,line width= 0.4pt,line join=round,line cap=round,fill=fillColor] (256.66,408.71) circle (  1.96);
\path[draw=drawColor,line width= 0.4pt,line join=round,line cap=round,fill=fillColor] (256.92,408.93) circle (  1.96);
\path[draw=drawColor,line width= 0.4pt,line join=round,line cap=round,fill=fillColor] (257.18,409.16) circle (  1.96);
\path[draw=drawColor,line width= 0.4pt,line join=round,line cap=round,fill=fillColor] (257.44,409.38) circle (  1.96);
\path[draw=drawColor,line width= 0.4pt,line join=round,line cap=round,fill=fillColor] (257.70,409.59) circle (  1.96);
\path[draw=drawColor,line width= 0.4pt,line join=round,line cap=round,fill=fillColor] (257.96,409.82) circle (  1.96);
\path[draw=drawColor,line width= 0.4pt,line join=round,line cap=round,fill=fillColor] (258.22,410.04) circle (  1.96);
\path[draw=drawColor,line width= 0.4pt,line join=round,line cap=round,fill=fillColor] (258.48,410.25) circle (  1.96);
\path[draw=drawColor,line width= 0.4pt,line join=round,line cap=round,fill=fillColor] (258.74,410.48) circle (  1.96);
\path[draw=drawColor,line width= 0.4pt,line join=round,line cap=round,fill=fillColor] (259.00,410.69) circle (  1.96);
\path[draw=drawColor,line width= 0.4pt,line join=round,line cap=round,fill=fillColor] (259.26,410.90) circle (  1.96);
\path[draw=drawColor,line width= 0.4pt,line join=round,line cap=round,fill=fillColor] (259.52,411.12) circle (  1.96);
\path[draw=drawColor,line width= 0.4pt,line join=round,line cap=round,fill=fillColor] (259.78,411.34) circle (  1.96);
\path[draw=drawColor,line width= 0.4pt,line join=round,line cap=round,fill=fillColor] (260.03,411.55) circle (  1.96);
\path[draw=drawColor,line width= 0.4pt,line join=round,line cap=round,fill=fillColor] (260.29,411.76) circle (  1.96);
\path[draw=drawColor,line width= 0.4pt,line join=round,line cap=round,fill=fillColor] (260.55,411.98) circle (  1.96);
\path[draw=drawColor,line width= 0.4pt,line join=round,line cap=round,fill=fillColor] (260.81,412.19) circle (  1.96);
\path[draw=drawColor,line width= 0.4pt,line join=round,line cap=round,fill=fillColor] (261.07,412.40) circle (  1.96);
\path[draw=drawColor,line width= 0.4pt,line join=round,line cap=round,fill=fillColor] (261.33,412.61) circle (  1.96);
\path[draw=drawColor,line width= 0.4pt,line join=round,line cap=round,fill=fillColor] (261.59,412.83) circle (  1.96);
\path[draw=drawColor,line width= 0.4pt,line join=round,line cap=round,fill=fillColor] (261.85,413.04) circle (  1.96);
\path[draw=drawColor,line width= 0.4pt,line join=round,line cap=round,fill=fillColor] (262.11,413.25) circle (  1.96);
\path[draw=drawColor,line width= 0.4pt,line join=round,line cap=round,fill=fillColor] (262.37,413.46) circle (  1.96);
\path[draw=drawColor,line width= 0.4pt,line join=round,line cap=round,fill=fillColor] (262.63,413.67) circle (  1.96);
\path[draw=drawColor,line width= 0.4pt,line join=round,line cap=round,fill=fillColor] (262.89,413.87) circle (  1.96);
\path[draw=drawColor,line width= 0.4pt,line join=round,line cap=round,fill=fillColor] (263.15,414.08) circle (  1.96);
\path[draw=drawColor,line width= 0.4pt,line join=round,line cap=round,fill=fillColor] (263.41,414.29) circle (  1.96);
\path[draw=drawColor,line width= 0.4pt,line join=round,line cap=round,fill=fillColor] (263.67,414.50) circle (  1.96);
\path[draw=drawColor,line width= 0.4pt,line join=round,line cap=round,fill=fillColor] (263.93,414.70) circle (  1.96);
\path[draw=drawColor,line width= 0.4pt,line join=round,line cap=round,fill=fillColor] (264.19,414.91) circle (  1.96);
\path[draw=drawColor,line width= 0.4pt,line join=round,line cap=round,fill=fillColor] (264.45,415.12) circle (  1.96);
\path[draw=drawColor,line width= 0.4pt,line join=round,line cap=round,fill=fillColor] (264.71,415.32) circle (  1.96);
\path[draw=drawColor,line width= 0.4pt,line join=round,line cap=round,fill=fillColor] (264.97,415.52) circle (  1.96);
\path[draw=drawColor,line width= 0.4pt,line join=round,line cap=round,fill=fillColor] (265.23,415.72) circle (  1.96);
\path[draw=drawColor,line width= 0.4pt,line join=round,line cap=round,fill=fillColor] (265.49,415.93) circle (  1.96);
\path[draw=drawColor,line width= 0.4pt,line join=round,line cap=round,fill=fillColor] (265.75,416.13) circle (  1.96);
\path[draw=drawColor,line width= 0.4pt,line join=round,line cap=round,fill=fillColor] (266.01,416.33) circle (  1.96);
\path[draw=drawColor,line width= 0.4pt,line join=round,line cap=round,fill=fillColor] (266.27,416.53) circle (  1.96);
\path[draw=drawColor,line width= 0.4pt,line join=round,line cap=round,fill=fillColor] (266.53,416.73) circle (  1.96);
\path[draw=drawColor,line width= 0.4pt,line join=round,line cap=round,fill=fillColor] (266.79,416.94) circle (  1.96);
\path[draw=drawColor,line width= 0.4pt,line join=round,line cap=round,fill=fillColor] (267.05,417.14) circle (  1.96);
\path[draw=drawColor,line width= 0.4pt,line join=round,line cap=round,fill=fillColor] (267.31,417.33) circle (  1.96);
\path[draw=drawColor,line width= 0.4pt,line join=round,line cap=round,fill=fillColor] (267.57,417.53) circle (  1.96);
\path[draw=drawColor,line width= 0.4pt,line join=round,line cap=round,fill=fillColor] (267.83,417.72) circle (  1.96);
\path[draw=drawColor,line width= 0.4pt,line join=round,line cap=round,fill=fillColor] (268.09,417.93) circle (  1.96);
\path[draw=drawColor,line width= 0.4pt,line join=round,line cap=round,fill=fillColor] (268.35,418.12) circle (  1.96);
\path[draw=drawColor,line width= 0.4pt,line join=round,line cap=round,fill=fillColor] (268.61,418.32) circle (  1.96);
\path[draw=drawColor,line width= 0.4pt,line join=round,line cap=round,fill=fillColor] (268.87,418.51) circle (  1.96);
\path[draw=drawColor,line width= 0.4pt,line join=round,line cap=round,fill=fillColor] (269.13,418.70) circle (  1.96);
\path[draw=drawColor,line width= 0.4pt,line join=round,line cap=round,fill=fillColor] (269.39,418.91) circle (  1.96);
\path[draw=drawColor,line width= 0.4pt,line join=round,line cap=round,fill=fillColor] (269.65,419.10) circle (  1.96);
\path[draw=drawColor,line width= 0.4pt,line join=round,line cap=round,fill=fillColor] (269.91,419.29) circle (  1.96);
\path[draw=drawColor,line width= 0.4pt,line join=round,line cap=round,fill=fillColor] (270.17,419.48) circle (  1.96);
\path[draw=drawColor,line width= 0.4pt,line join=round,line cap=round,fill=fillColor] (270.43,419.67) circle (  1.96);
\path[draw=drawColor,line width= 0.4pt,line join=round,line cap=round,fill=fillColor] (270.69,419.86) circle (  1.96);
\path[draw=drawColor,line width= 0.4pt,line join=round,line cap=round,fill=fillColor] (270.95,420.05) circle (  1.96);
\path[draw=drawColor,line width= 0.4pt,line join=round,line cap=round,fill=fillColor] (271.21,420.25) circle (  1.96);
\path[draw=drawColor,line width= 0.4pt,line join=round,line cap=round,fill=fillColor] (271.47,420.44) circle (  1.96);
\path[draw=drawColor,line width= 0.4pt,line join=round,line cap=round,fill=fillColor] (271.73,420.63) circle (  1.96);
\path[draw=drawColor,line width= 0.4pt,line join=round,line cap=round,fill=fillColor] (271.99,420.82) circle (  1.96);
\path[draw=drawColor,line width= 0.4pt,line join=round,line cap=round,fill=fillColor] (272.25,421.00) circle (  1.96);
\path[draw=drawColor,line width= 0.4pt,line join=round,line cap=round,fill=fillColor] (272.51,421.19) circle (  1.96);
\path[draw=drawColor,line width= 0.4pt,line join=round,line cap=round,fill=fillColor] (272.77,421.39) circle (  1.96);
\path[draw=drawColor,line width= 0.4pt,line join=round,line cap=round,fill=fillColor] (273.03,421.57) circle (  1.96);
\path[draw=drawColor,line width= 0.4pt,line join=round,line cap=round,fill=fillColor] (273.29,421.76) circle (  1.96);
\path[draw=drawColor,line width= 0.4pt,line join=round,line cap=round,fill=fillColor] (273.55,421.94) circle (  1.96);
\path[draw=drawColor,line width= 0.4pt,line join=round,line cap=round,fill=fillColor] (273.81,422.12) circle (  1.96);
\path[draw=drawColor,line width= 0.4pt,line join=round,line cap=round,fill=fillColor] (274.07,422.31) circle (  1.96);
\path[draw=drawColor,line width= 0.4pt,line join=round,line cap=round,fill=fillColor] (274.33,422.49) circle (  1.96);
\path[draw=drawColor,line width= 0.4pt,line join=round,line cap=round,fill=fillColor] (274.59,422.67) circle (  1.96);
\path[draw=drawColor,line width= 0.4pt,line join=round,line cap=round,fill=fillColor] (274.85,422.86) circle (  1.96);
\path[draw=drawColor,line width= 0.4pt,line join=round,line cap=round,fill=fillColor] (275.11,423.05) circle (  1.96);
\path[draw=drawColor,line width= 0.4pt,line join=round,line cap=round,fill=fillColor] (275.37,423.23) circle (  1.96);
\path[draw=drawColor,line width= 0.4pt,line join=round,line cap=round,fill=fillColor] (275.63,423.41) circle (  1.96);
\path[draw=drawColor,line width= 0.4pt,line join=round,line cap=round,fill=fillColor] (275.89,423.59) circle (  1.96);
\path[draw=drawColor,line width= 0.4pt,line join=round,line cap=round,fill=fillColor] (276.15,423.77) circle (  1.96);
\path[draw=drawColor,line width= 0.4pt,line join=round,line cap=round,fill=fillColor] (276.41,423.95) circle (  1.96);
\path[draw=drawColor,line width= 0.4pt,line join=round,line cap=round,fill=fillColor] (276.66,424.13) circle (  1.96);
\path[draw=drawColor,line width= 0.4pt,line join=round,line cap=round,fill=fillColor] (276.92,424.30) circle (  1.96);
\path[draw=drawColor,line width= 0.4pt,line join=round,line cap=round,fill=fillColor] (277.18,424.48) circle (  1.96);
\path[draw=drawColor,line width= 0.4pt,line join=round,line cap=round,fill=fillColor] (277.44,424.66) circle (  1.96);
\path[draw=drawColor,line width= 0.4pt,line join=round,line cap=round,fill=fillColor] (277.70,424.84) circle (  1.96);
\path[draw=drawColor,line width= 0.4pt,line join=round,line cap=round,fill=fillColor] (277.96,425.01) circle (  1.96);
\path[draw=drawColor,line width= 0.4pt,line join=round,line cap=round,fill=fillColor] (278.22,425.20) circle (  1.96);
\path[draw=drawColor,line width= 0.4pt,line join=round,line cap=round,fill=fillColor] (278.48,425.37) circle (  1.96);
\path[draw=drawColor,line width= 0.4pt,line join=round,line cap=round,fill=fillColor] (278.74,425.55) circle (  1.96);
\path[draw=drawColor,line width= 0.4pt,line join=round,line cap=round,fill=fillColor] (279.00,425.72) circle (  1.96);
\path[draw=drawColor,line width= 0.4pt,line join=round,line cap=round,fill=fillColor] (279.26,425.90) circle (  1.96);
\path[draw=drawColor,line width= 0.4pt,line join=round,line cap=round,fill=fillColor] (279.52,426.07) circle (  1.96);
\path[draw=drawColor,line width= 0.4pt,line join=round,line cap=round,fill=fillColor] (279.78,426.24) circle (  1.96);
\path[draw=drawColor,line width= 0.4pt,line join=round,line cap=round,fill=fillColor] (280.04,426.42) circle (  1.96);
\path[draw=drawColor,line width= 0.4pt,line join=round,line cap=round,fill=fillColor] (280.30,426.59) circle (  1.96);
\path[draw=drawColor,line width= 0.4pt,line join=round,line cap=round,fill=fillColor] (280.56,426.76) circle (  1.96);
\path[draw=drawColor,line width= 0.4pt,line join=round,line cap=round,fill=fillColor] (280.82,426.93) circle (  1.96);
\path[draw=drawColor,line width= 0.4pt,line join=round,line cap=round,fill=fillColor] (281.08,427.10) circle (  1.96);
\path[draw=drawColor,line width= 0.4pt,line join=round,line cap=round,fill=fillColor] (281.34,427.27) circle (  1.96);
\path[draw=drawColor,line width= 0.4pt,line join=round,line cap=round,fill=fillColor] (281.60,427.44) circle (  1.96);
\path[draw=drawColor,line width= 0.4pt,line join=round,line cap=round,fill=fillColor] (281.86,427.61) circle (  1.96);
\path[draw=drawColor,line width= 0.4pt,line join=round,line cap=round,fill=fillColor] (282.12,427.78) circle (  1.96);
\path[draw=drawColor,line width= 0.4pt,line join=round,line cap=round,fill=fillColor] (282.38,427.95) circle (  1.96);
\path[draw=drawColor,line width= 0.4pt,line join=round,line cap=round,fill=fillColor] (282.64,428.12) circle (  1.96);
\path[draw=drawColor,line width= 0.4pt,line join=round,line cap=round,fill=fillColor] (282.90,428.28) circle (  1.96);
\path[draw=drawColor,line width= 0.4pt,line join=round,line cap=round,fill=fillColor] (283.16,428.45) circle (  1.96);
\path[draw=drawColor,line width= 0.4pt,line join=round,line cap=round,fill=fillColor] (283.42,428.62) circle (  1.96);
\path[draw=drawColor,line width= 0.4pt,line join=round,line cap=round,fill=fillColor] (283.68,428.78) circle (  1.96);
\path[draw=drawColor,line width= 0.4pt,line join=round,line cap=round,fill=fillColor] (283.94,428.95) circle (  1.96);
\path[draw=drawColor,line width= 0.4pt,line join=round,line cap=round,fill=fillColor] (284.20,429.11) circle (  1.96);
\path[draw=drawColor,line width= 0.4pt,line join=round,line cap=round,fill=fillColor] (284.46,429.28) circle (  1.96);
\path[draw=drawColor,line width= 0.4pt,line join=round,line cap=round,fill=fillColor] (284.72,429.44) circle (  1.96);
\path[draw=drawColor,line width= 0.4pt,line join=round,line cap=round,fill=fillColor] (284.98,429.61) circle (  1.96);
\path[draw=drawColor,line width= 0.4pt,line join=round,line cap=round,fill=fillColor] (285.24,429.77) circle (  1.96);
\path[draw=drawColor,line width= 0.4pt,line join=round,line cap=round,fill=fillColor] (285.50,429.93) circle (  1.96);
\path[draw=drawColor,line width= 0.4pt,line join=round,line cap=round,fill=fillColor] (285.76,430.09) circle (  1.96);
\path[draw=drawColor,line width= 0.4pt,line join=round,line cap=round,fill=fillColor] (286.02,430.26) circle (  1.96);
\path[draw=drawColor,line width= 0.4pt,line join=round,line cap=round,fill=fillColor] (286.28,430.42) circle (  1.96);
\path[draw=drawColor,line width= 0.4pt,line join=round,line cap=round,fill=fillColor] (286.54,430.58) circle (  1.96);
\path[draw=drawColor,line width= 0.4pt,line join=round,line cap=round,fill=fillColor] (286.80,430.74) circle (  1.96);
\path[draw=drawColor,line width= 0.4pt,line join=round,line cap=round,fill=fillColor] (287.06,430.90) circle (  1.96);
\path[draw=drawColor,line width= 0.4pt,line join=round,line cap=round,fill=fillColor] (287.32,431.06) circle (  1.96);
\path[draw=drawColor,line width= 0.4pt,line join=round,line cap=round,fill=fillColor] (287.58,431.22) circle (  1.96);
\path[draw=drawColor,line width= 0.4pt,line join=round,line cap=round,fill=fillColor] (287.84,431.38) circle (  1.96);
\path[draw=drawColor,line width= 0.4pt,line join=round,line cap=round,fill=fillColor] (288.10,431.54) circle (  1.96);
\path[draw=drawColor,line width= 0.4pt,line join=round,line cap=round,fill=fillColor] (288.36,431.69) circle (  1.96);
\path[draw=drawColor,line width= 0.4pt,line join=round,line cap=round,fill=fillColor] (288.62,431.85) circle (  1.96);
\path[draw=drawColor,line width= 0.4pt,line join=round,line cap=round,fill=fillColor] (288.88,432.01) circle (  1.96);
\path[draw=drawColor,line width= 0.4pt,line join=round,line cap=round,fill=fillColor] (289.14,432.17) circle (  1.96);
\path[draw=drawColor,line width= 0.4pt,line join=round,line cap=round,fill=fillColor] (289.40,432.32) circle (  1.96);
\path[draw=drawColor,line width= 0.4pt,line join=round,line cap=round,fill=fillColor] (289.66,432.48) circle (  1.96);
\path[draw=drawColor,line width= 0.4pt,line join=round,line cap=round,fill=fillColor] (289.92,432.62) circle (  1.96);
\path[draw=drawColor,line width= 0.4pt,line join=round,line cap=round,fill=fillColor] (290.18,432.78) circle (  1.96);
\path[draw=drawColor,line width= 0.4pt,line join=round,line cap=round,fill=fillColor] (290.44,432.93) circle (  1.96);
\path[draw=drawColor,line width= 0.4pt,line join=round,line cap=round,fill=fillColor] (290.70,433.09) circle (  1.96);
\path[draw=drawColor,line width= 0.4pt,line join=round,line cap=round,fill=fillColor] (290.96,433.24) circle (  1.96);
\path[draw=drawColor,line width= 0.4pt,line join=round,line cap=round,fill=fillColor] (291.22,433.39) circle (  1.96);
\path[draw=drawColor,line width= 0.4pt,line join=round,line cap=round,fill=fillColor] (291.48,433.55) circle (  1.96);
\path[draw=drawColor,line width= 0.4pt,line join=round,line cap=round,fill=fillColor] (291.74,433.70) circle (  1.96);
\path[draw=drawColor,line width= 0.4pt,line join=round,line cap=round,fill=fillColor] (292.00,433.85) circle (  1.96);
\path[draw=drawColor,line width= 0.4pt,line join=round,line cap=round,fill=fillColor] (292.26,434.00) circle (  1.96);
\path[draw=drawColor,line width= 0.4pt,line join=round,line cap=round,fill=fillColor] (292.52,434.15) circle (  1.96);
\path[draw=drawColor,line width= 0.4pt,line join=round,line cap=round,fill=fillColor] (292.78,434.31) circle (  1.96);
\path[draw=drawColor,line width= 0.4pt,line join=round,line cap=round,fill=fillColor] (293.04,434.45) circle (  1.96);
\path[draw=drawColor,line width= 0.4pt,line join=round,line cap=round,fill=fillColor] (293.29,434.60) circle (  1.96);
\path[draw=drawColor,line width= 0.4pt,line join=round,line cap=round,fill=fillColor] (293.55,434.75) circle (  1.96);
\path[draw=drawColor,line width= 0.4pt,line join=round,line cap=round,fill=fillColor] (293.81,434.89) circle (  1.96);
\path[draw=drawColor,line width= 0.4pt,line join=round,line cap=round,fill=fillColor] (294.07,435.04) circle (  1.96);
\path[draw=drawColor,line width= 0.4pt,line join=round,line cap=round,fill=fillColor] (294.33,435.19) circle (  1.96);
\path[draw=drawColor,line width= 0.4pt,line join=round,line cap=round,fill=fillColor] (294.59,435.34) circle (  1.96);
\path[draw=drawColor,line width= 0.4pt,line join=round,line cap=round,fill=fillColor] (294.85,435.49) circle (  1.96);
\path[draw=drawColor,line width= 0.4pt,line join=round,line cap=round,fill=fillColor] (295.11,435.63) circle (  1.96);
\path[draw=drawColor,line width= 0.4pt,line join=round,line cap=round,fill=fillColor] (295.37,435.77) circle (  1.96);
\path[draw=drawColor,line width= 0.4pt,line join=round,line cap=round,fill=fillColor] (295.63,435.92) circle (  1.96);
\path[draw=drawColor,line width= 0.4pt,line join=round,line cap=round,fill=fillColor] (295.89,436.06) circle (  1.96);
\path[draw=drawColor,line width= 0.4pt,line join=round,line cap=round,fill=fillColor] (296.15,436.21) circle (  1.96);
\path[draw=drawColor,line width= 0.4pt,line join=round,line cap=round,fill=fillColor] (296.41,436.35) circle (  1.96);
\path[draw=drawColor,line width= 0.4pt,line join=round,line cap=round,fill=fillColor] (296.67,436.50) circle (  1.96);
\path[draw=drawColor,line width= 0.4pt,line join=round,line cap=round,fill=fillColor] (296.93,436.64) circle (  1.96);
\path[draw=drawColor,line width= 0.4pt,line join=round,line cap=round,fill=fillColor] (297.19,436.79) circle (  1.96);
\path[draw=drawColor,line width= 0.4pt,line join=round,line cap=round,fill=fillColor] (297.45,436.92) circle (  1.96);
\path[draw=drawColor,line width= 0.4pt,line join=round,line cap=round,fill=fillColor] (297.71,437.06) circle (  1.96);
\path[draw=drawColor,line width= 0.4pt,line join=round,line cap=round,fill=fillColor] (297.97,437.21) circle (  1.96);
\path[draw=drawColor,line width= 0.4pt,line join=round,line cap=round,fill=fillColor] (298.23,437.35) circle (  1.96);
\path[draw=drawColor,line width= 0.4pt,line join=round,line cap=round,fill=fillColor] (298.49,437.49) circle (  1.96);
\path[draw=drawColor,line width= 0.4pt,line join=round,line cap=round,fill=fillColor] (298.75,437.63) circle (  1.96);
\path[draw=drawColor,line width= 0.4pt,line join=round,line cap=round,fill=fillColor] (299.01,437.77) circle (  1.96);
\path[draw=drawColor,line width= 0.4pt,line join=round,line cap=round,fill=fillColor] (299.27,437.90) circle (  1.96);
\path[draw=drawColor,line width= 0.4pt,line join=round,line cap=round,fill=fillColor] (299.53,438.04) circle (  1.96);
\path[draw=drawColor,line width= 0.4pt,line join=round,line cap=round,fill=fillColor] (299.79,438.18) circle (  1.96);
\path[draw=drawColor,line width= 0.4pt,line join=round,line cap=round,fill=fillColor] (300.05,438.32) circle (  1.96);
\path[draw=drawColor,line width= 0.4pt,line join=round,line cap=round,fill=fillColor] (300.31,438.46) circle (  1.96);
\path[draw=drawColor,line width= 0.4pt,line join=round,line cap=round,fill=fillColor] (300.57,438.60) circle (  1.96);
\path[draw=drawColor,line width= 0.4pt,line join=round,line cap=round,fill=fillColor] (300.83,438.73) circle (  1.96);
\path[draw=drawColor,line width= 0.4pt,line join=round,line cap=round,fill=fillColor] (301.09,438.87) circle (  1.96);
\path[draw=drawColor,line width= 0.4pt,line join=round,line cap=round,fill=fillColor] (301.35,439.00) circle (  1.96);
\path[draw=drawColor,line width= 0.4pt,line join=round,line cap=round,fill=fillColor] (301.61,439.14) circle (  1.96);
\path[draw=drawColor,line width= 0.4pt,line join=round,line cap=round,fill=fillColor] (301.87,439.28) circle (  1.96);
\path[draw=drawColor,line width= 0.4pt,line join=round,line cap=round,fill=fillColor] (302.13,439.40) circle (  1.96);
\path[draw=drawColor,line width= 0.4pt,line join=round,line cap=round,fill=fillColor] (302.39,439.54) circle (  1.96);
\path[draw=drawColor,line width= 0.4pt,line join=round,line cap=round,fill=fillColor] (302.65,439.68) circle (  1.96);
\path[draw=drawColor,line width= 0.4pt,line join=round,line cap=round,fill=fillColor] (302.91,439.81) circle (  1.96);
\path[draw=drawColor,line width= 0.4pt,line join=round,line cap=round,fill=fillColor] (303.17,439.95) circle (  1.96);
\path[draw=drawColor,line width= 0.4pt,line join=round,line cap=round,fill=fillColor] (303.43,440.07) circle (  1.96);
\path[draw=drawColor,line width= 0.4pt,line join=round,line cap=round,fill=fillColor] (303.69,440.21) circle (  1.96);
\path[draw=drawColor,line width= 0.4pt,line join=round,line cap=round,fill=fillColor] (303.95,440.34) circle (  1.96);
\path[draw=drawColor,line width= 0.4pt,line join=round,line cap=round,fill=fillColor] (304.21,440.47) circle (  1.96);
\path[draw=drawColor,line width= 0.4pt,line join=round,line cap=round,fill=fillColor] (304.47,440.61) circle (  1.96);
\path[draw=drawColor,line width= 0.4pt,line join=round,line cap=round,fill=fillColor] (304.73,440.73) circle (  1.96);
\path[draw=drawColor,line width= 0.4pt,line join=round,line cap=round,fill=fillColor] (304.99,440.86) circle (  1.96);
\path[draw=drawColor,line width= 0.4pt,line join=round,line cap=round,fill=fillColor] (305.25,440.99) circle (  1.96);
\path[draw=drawColor,line width= 0.4pt,line join=round,line cap=round,fill=fillColor] (305.51,441.12) circle (  1.96);
\path[draw=drawColor,line width= 0.4pt,line join=round,line cap=round,fill=fillColor] (305.77,441.26) circle (  1.96);
\path[draw=drawColor,line width= 0.4pt,line join=round,line cap=round,fill=fillColor] (306.03,441.38) circle (  1.96);
\path[draw=drawColor,line width= 0.4pt,line join=round,line cap=round,fill=fillColor] (306.29,441.51) circle (  1.96);
\path[draw=drawColor,line width= 0.4pt,line join=round,line cap=round,fill=fillColor] (306.55,441.64) circle (  1.96);
\path[draw=drawColor,line width= 0.4pt,line join=round,line cap=round,fill=fillColor] (306.81,441.77) circle (  1.96);
\path[draw=drawColor,line width= 0.4pt,line join=round,line cap=round,fill=fillColor] (307.07,441.89) circle (  1.96);
\path[draw=drawColor,line width= 0.4pt,line join=round,line cap=round,fill=fillColor] (307.33,442.01) circle (  1.96);
\path[draw=drawColor,line width= 0.4pt,line join=round,line cap=round,fill=fillColor] (307.59,442.14) circle (  1.96);
\path[draw=drawColor,line width= 0.4pt,line join=round,line cap=round,fill=fillColor] (307.85,442.27) circle (  1.96);
\path[draw=drawColor,line width= 0.4pt,line join=round,line cap=round,fill=fillColor] (308.11,442.39) circle (  1.96);
\path[draw=drawColor,line width= 0.4pt,line join=round,line cap=round,fill=fillColor] (308.37,442.52) circle (  1.96);
\path[draw=drawColor,line width= 0.4pt,line join=round,line cap=round,fill=fillColor] (308.63,442.64) circle (  1.96);
\path[draw=drawColor,line width= 0.4pt,line join=round,line cap=round,fill=fillColor] (308.89,442.77) circle (  1.96);
\path[draw=drawColor,line width= 0.4pt,line join=round,line cap=round,fill=fillColor] (309.15,442.90) circle (  1.96);
\path[draw=drawColor,line width= 0.4pt,line join=round,line cap=round,fill=fillColor] (309.41,443.01) circle (  1.96);
\path[draw=drawColor,line width= 0.4pt,line join=round,line cap=round,fill=fillColor] (309.66,443.14) circle (  1.96);
\path[draw=drawColor,line width= 0.4pt,line join=round,line cap=round,fill=fillColor] (309.92,443.27) circle (  1.96);
\path[draw=drawColor,line width= 0.4pt,line join=round,line cap=round,fill=fillColor] (310.18,443.38) circle (  1.96);
\path[draw=drawColor,line width= 0.4pt,line join=round,line cap=round,fill=fillColor] (310.44,443.50) circle (  1.96);
\path[draw=drawColor,line width= 0.4pt,line join=round,line cap=round,fill=fillColor] (310.70,443.63) circle (  1.96);
\path[draw=drawColor,line width= 0.4pt,line join=round,line cap=round,fill=fillColor] (310.96,443.75) circle (  1.96);
\path[draw=drawColor,line width= 0.4pt,line join=round,line cap=round,fill=fillColor] (311.22,443.87) circle (  1.96);
\path[draw=drawColor,line width= 0.4pt,line join=round,line cap=round,fill=fillColor] (311.48,443.99) circle (  1.96);
\path[draw=drawColor,line width= 0.4pt,line join=round,line cap=round,fill=fillColor] (311.74,444.11) circle (  1.96);
\path[draw=drawColor,line width= 0.4pt,line join=round,line cap=round,fill=fillColor] (312.00,444.24) circle (  1.96);
\path[draw=drawColor,line width= 0.4pt,line join=round,line cap=round,fill=fillColor] (312.26,444.35) circle (  1.96);
\path[draw=drawColor,line width= 0.4pt,line join=round,line cap=round,fill=fillColor] (312.52,444.47) circle (  1.96);
\path[draw=drawColor,line width= 0.4pt,line join=round,line cap=round,fill=fillColor] (312.78,444.59) circle (  1.96);
\path[draw=drawColor,line width= 0.4pt,line join=round,line cap=round,fill=fillColor] (313.04,444.70) circle (  1.96);
\path[draw=drawColor,line width= 0.4pt,line join=round,line cap=round,fill=fillColor] (313.30,444.82) circle (  1.96);
\path[draw=drawColor,line width= 0.4pt,line join=round,line cap=round,fill=fillColor] (313.56,444.94) circle (  1.96);
\path[draw=drawColor,line width= 0.4pt,line join=round,line cap=round,fill=fillColor] (313.82,445.06) circle (  1.96);
\path[draw=drawColor,line width= 0.4pt,line join=round,line cap=round,fill=fillColor] (314.08,445.17) circle (  1.96);
\path[draw=drawColor,line width= 0.4pt,line join=round,line cap=round,fill=fillColor] (314.34,445.29) circle (  1.96);
\path[draw=drawColor,line width= 0.4pt,line join=round,line cap=round,fill=fillColor] (314.60,445.41) circle (  1.96);
\path[draw=drawColor,line width= 0.4pt,line join=round,line cap=round,fill=fillColor] (314.86,445.52) circle (  1.96);
\path[draw=drawColor,line width= 0.4pt,line join=round,line cap=round,fill=fillColor] (315.12,445.64) circle (  1.96);
\path[draw=drawColor,line width= 0.4pt,line join=round,line cap=round,fill=fillColor] (315.38,445.76) circle (  1.96);
\path[draw=drawColor,line width= 0.4pt,line join=round,line cap=round,fill=fillColor] (315.64,445.88) circle (  1.96);
\path[draw=drawColor,line width= 0.4pt,line join=round,line cap=round,fill=fillColor] (315.90,445.98) circle (  1.96);
\path[draw=drawColor,line width= 0.4pt,line join=round,line cap=round,fill=fillColor] (316.16,446.10) circle (  1.96);
\path[draw=drawColor,line width= 0.4pt,line join=round,line cap=round,fill=fillColor] (316.42,446.22) circle (  1.96);
\path[draw=drawColor,line width= 0.4pt,line join=round,line cap=round,fill=fillColor] (316.68,446.32) circle (  1.96);
\path[draw=drawColor,line width= 0.4pt,line join=round,line cap=round,fill=fillColor] (316.94,446.44) circle (  1.96);
\path[draw=drawColor,line width= 0.4pt,line join=round,line cap=round,fill=fillColor] (317.20,446.56) circle (  1.96);
\path[draw=drawColor,line width= 0.4pt,line join=round,line cap=round,fill=fillColor] (317.46,446.66) circle (  1.96);
\path[draw=drawColor,line width= 0.4pt,line join=round,line cap=round,fill=fillColor] (317.72,446.78) circle (  1.96);
\path[draw=drawColor,line width= 0.4pt,line join=round,line cap=round,fill=fillColor] (317.98,446.89) circle (  1.96);
\path[draw=drawColor,line width= 0.4pt,line join=round,line cap=round,fill=fillColor] (318.24,447.00) circle (  1.96);
\path[draw=drawColor,line width= 0.4pt,line join=round,line cap=round,fill=fillColor] (318.50,447.11) circle (  1.96);
\path[draw=drawColor,line width= 0.4pt,line join=round,line cap=round,fill=fillColor] (318.76,447.22) circle (  1.96);
\path[draw=drawColor,line width= 0.4pt,line join=round,line cap=round,fill=fillColor] (319.02,447.33) circle (  1.96);
\path[draw=drawColor,line width= 0.4pt,line join=round,line cap=round,fill=fillColor] (319.28,447.44) circle (  1.96);
\path[draw=drawColor,line width= 0.4pt,line join=round,line cap=round,fill=fillColor] (319.54,447.55) circle (  1.96);
\path[draw=drawColor,line width= 0.4pt,line join=round,line cap=round,fill=fillColor] (319.80,447.66) circle (  1.96);
\path[draw=drawColor,line width= 0.4pt,line join=round,line cap=round,fill=fillColor] (320.06,447.77) circle (  1.96);
\path[draw=drawColor,line width= 0.4pt,line join=round,line cap=round,fill=fillColor] (320.32,447.88) circle (  1.96);
\path[draw=drawColor,line width= 0.4pt,line join=round,line cap=round,fill=fillColor] (320.58,447.98) circle (  1.96);
\path[draw=drawColor,line width= 0.4pt,line join=round,line cap=round,fill=fillColor] (320.84,448.09) circle (  1.96);
\path[draw=drawColor,line width= 0.4pt,line join=round,line cap=round,fill=fillColor] (321.10,448.21) circle (  1.96);
\path[draw=drawColor,line width= 0.4pt,line join=round,line cap=round,fill=fillColor] (321.36,448.31) circle (  1.96);
\path[draw=drawColor,line width= 0.4pt,line join=round,line cap=round,fill=fillColor] (321.62,448.42) circle (  1.96);
\path[draw=drawColor,line width= 0.4pt,line join=round,line cap=round,fill=fillColor] (321.88,448.53) circle (  1.96);
\path[draw=drawColor,line width= 0.4pt,line join=round,line cap=round,fill=fillColor] (322.14,448.63) circle (  1.96);
\path[draw=drawColor,line width= 0.4pt,line join=round,line cap=round,fill=fillColor] (322.40,448.74) circle (  1.96);
\path[draw=drawColor,line width= 0.4pt,line join=round,line cap=round,fill=fillColor] (322.66,448.83) circle (  1.96);
\path[draw=drawColor,line width= 0.4pt,line join=round,line cap=round,fill=fillColor] (322.92,448.94) circle (  1.96);
\path[draw=drawColor,line width= 0.4pt,line join=round,line cap=round,fill=fillColor] (323.18,449.05) circle (  1.96);
\path[draw=drawColor,line width= 0.4pt,line join=round,line cap=round,fill=fillColor] (323.44,449.15) circle (  1.96);
\path[draw=drawColor,line width= 0.4pt,line join=round,line cap=round,fill=fillColor] (323.70,449.26) circle (  1.96);
\path[draw=drawColor,line width= 0.4pt,line join=round,line cap=round,fill=fillColor] (323.96,449.37) circle (  1.96);
\path[draw=drawColor,line width= 0.4pt,line join=round,line cap=round,fill=fillColor] (324.22,449.46) circle (  1.96);
\path[draw=drawColor,line width= 0.4pt,line join=round,line cap=round,fill=fillColor] (324.48,449.57) circle (  1.96);
\path[draw=drawColor,line width= 0.4pt,line join=round,line cap=round,fill=fillColor] (324.74,449.67) circle (  1.96);
\path[draw=drawColor,line width= 0.4pt,line join=round,line cap=round,fill=fillColor] (325.00,449.77) circle (  1.96);
\path[draw=drawColor,line width= 0.4pt,line join=round,line cap=round,fill=fillColor] (325.26,449.88) circle (  1.96);
\path[draw=drawColor,line width= 0.4pt,line join=round,line cap=round,fill=fillColor] (325.52,449.98) circle (  1.96);
\path[draw=drawColor,line width= 0.4pt,line join=round,line cap=round,fill=fillColor] (325.78,450.08) circle (  1.96);
\path[draw=drawColor,line width= 0.4pt,line join=round,line cap=round,fill=fillColor] (326.04,450.19) circle (  1.96);
\path[draw=drawColor,line width= 0.4pt,line join=round,line cap=round,fill=fillColor] (326.29,450.28) circle (  1.96);
\path[draw=drawColor,line width= 0.4pt,line join=round,line cap=round,fill=fillColor] (326.55,450.39) circle (  1.96);
\path[draw=drawColor,line width= 0.4pt,line join=round,line cap=round,fill=fillColor] (326.81,450.48) circle (  1.96);
\path[draw=drawColor,line width= 0.4pt,line join=round,line cap=round,fill=fillColor] (327.07,450.58) circle (  1.96);
\path[draw=drawColor,line width= 0.4pt,line join=round,line cap=round,fill=fillColor] (327.33,450.69) circle (  1.96);
\path[draw=drawColor,line width= 0.4pt,line join=round,line cap=round,fill=fillColor] (327.59,450.78) circle (  1.96);
\path[draw=drawColor,line width= 0.4pt,line join=round,line cap=round,fill=fillColor] (327.85,450.88) circle (  1.96);
\path[draw=drawColor,line width= 0.4pt,line join=round,line cap=round,fill=fillColor] (328.11,450.98) circle (  1.96);
\path[draw=drawColor,line width= 0.4pt,line join=round,line cap=round,fill=fillColor] (328.37,451.08) circle (  1.96);
\path[draw=drawColor,line width= 0.4pt,line join=round,line cap=round,fill=fillColor] (328.63,451.18) circle (  1.96);
\path[draw=drawColor,line width= 0.4pt,line join=round,line cap=round,fill=fillColor] (328.89,451.27) circle (  1.96);
\path[draw=drawColor,line width= 0.4pt,line join=round,line cap=round,fill=fillColor] (329.15,451.37) circle (  1.96);
\path[draw=drawColor,line width= 0.4pt,line join=round,line cap=round,fill=fillColor] (329.41,451.47) circle (  1.96);
\path[draw=drawColor,line width= 0.4pt,line join=round,line cap=round,fill=fillColor] (329.67,451.57) circle (  1.96);
\path[draw=drawColor,line width= 0.4pt,line join=round,line cap=round,fill=fillColor] (329.93,451.67) circle (  1.96);
\path[draw=drawColor,line width= 0.4pt,line join=round,line cap=round,fill=fillColor] (330.19,451.76) circle (  1.96);
\path[draw=drawColor,line width= 0.4pt,line join=round,line cap=round,fill=fillColor] (330.45,451.86) circle (  1.96);
\path[draw=drawColor,line width= 0.4pt,line join=round,line cap=round,fill=fillColor] (330.71,451.95) circle (  1.96);
\path[draw=drawColor,line width= 0.4pt,line join=round,line cap=round,fill=fillColor] (330.97,452.05) circle (  1.96);
\path[draw=drawColor,line width= 0.4pt,line join=round,line cap=round,fill=fillColor] (331.23,452.14) circle (  1.96);
\path[draw=drawColor,line width= 0.4pt,line join=round,line cap=round,fill=fillColor] (331.49,452.24) circle (  1.96);
\path[draw=drawColor,line width= 0.4pt,line join=round,line cap=round,fill=fillColor] (331.75,452.32) circle (  1.96);
\path[draw=drawColor,line width= 0.4pt,line join=round,line cap=round,fill=fillColor] (332.01,452.42) circle (  1.96);
\path[draw=drawColor,line width= 0.4pt,line join=round,line cap=round,fill=fillColor] (332.27,452.52) circle (  1.96);
\path[draw=drawColor,line width= 0.4pt,line join=round,line cap=round,fill=fillColor] (332.53,452.61) circle (  1.96);
\path[draw=drawColor,line width= 0.4pt,line join=round,line cap=round,fill=fillColor] (332.79,452.71) circle (  1.96);
\path[draw=drawColor,line width= 0.4pt,line join=round,line cap=round,fill=fillColor] (333.05,452.79) circle (  1.96);
\path[draw=drawColor,line width= 0.4pt,line join=round,line cap=round,fill=fillColor] (333.31,452.89) circle (  1.96);
\path[draw=drawColor,line width= 0.4pt,line join=round,line cap=round,fill=fillColor] (333.57,452.98) circle (  1.96);
\path[draw=drawColor,line width= 0.4pt,line join=round,line cap=round,fill=fillColor] (333.83,453.07) circle (  1.96);
\path[draw=drawColor,line width= 0.4pt,line join=round,line cap=round,fill=fillColor] (334.09,453.17) circle (  1.96);
\path[draw=drawColor,line width= 0.4pt,line join=round,line cap=round,fill=fillColor] (334.35,453.26) circle (  1.96);
\path[draw=drawColor,line width= 0.4pt,line join=round,line cap=round,fill=fillColor] (334.61,453.35) circle (  1.96);
\path[draw=drawColor,line width= 0.4pt,line join=round,line cap=round,fill=fillColor] (334.87,453.44) circle (  1.96);
\path[draw=drawColor,line width= 0.4pt,line join=round,line cap=round,fill=fillColor] (335.13,453.53) circle (  1.96);
\path[draw=drawColor,line width= 0.4pt,line join=round,line cap=round,fill=fillColor] (335.39,453.62) circle (  1.96);
\path[draw=drawColor,line width= 0.4pt,line join=round,line cap=round,fill=fillColor] (335.65,453.71) circle (  1.96);
\path[draw=drawColor,line width= 0.4pt,line join=round,line cap=round,fill=fillColor] (335.91,453.80) circle (  1.96);
\path[draw=drawColor,line width= 0.4pt,line join=round,line cap=round,fill=fillColor] (336.17,453.89) circle (  1.96);
\path[draw=drawColor,line width= 0.4pt,line join=round,line cap=round,fill=fillColor] (336.43,453.97) circle (  1.96);
\path[draw=drawColor,line width= 0.4pt,line join=round,line cap=round,fill=fillColor] (336.69,454.07) circle (  1.96);
\path[draw=drawColor,line width= 0.4pt,line join=round,line cap=round,fill=fillColor] (336.95,454.16) circle (  1.96);
\path[draw=drawColor,line width= 0.4pt,line join=round,line cap=round,fill=fillColor] (337.21,454.24) circle (  1.96);
\path[draw=drawColor,line width= 0.4pt,line join=round,line cap=round,fill=fillColor] (337.47,454.34) circle (  1.96);
\path[draw=drawColor,line width= 0.4pt,line join=round,line cap=round,fill=fillColor] (337.73,454.42) circle (  1.96);
\path[draw=drawColor,line width= 0.4pt,line join=round,line cap=round,fill=fillColor] (337.99,454.51) circle (  1.96);
\path[draw=drawColor,line width= 0.4pt,line join=round,line cap=round,fill=fillColor] (338.25,454.59) circle (  1.96);
\path[draw=drawColor,line width= 0.4pt,line join=round,line cap=round,fill=fillColor] (338.51,454.69) circle (  1.96);
\path[draw=drawColor,line width= 0.4pt,line join=round,line cap=round,fill=fillColor] (338.77,454.77) circle (  1.96);
\path[draw=drawColor,line width= 0.4pt,line join=round,line cap=round,fill=fillColor] (339.03,454.86) circle (  1.96);
\path[draw=drawColor,line width= 0.4pt,line join=round,line cap=round,fill=fillColor] (339.29,454.94) circle (  1.96);
\path[draw=drawColor,line width= 0.4pt,line join=round,line cap=round,fill=fillColor] (339.55,455.03) circle (  1.96);
\path[draw=drawColor,line width= 0.4pt,line join=round,line cap=round,fill=fillColor] (339.81,455.11) circle (  1.96);
\path[draw=drawColor,line width= 0.4pt,line join=round,line cap=round,fill=fillColor] (340.07,455.20) circle (  1.96);
\path[draw=drawColor,line width= 0.4pt,line join=round,line cap=round,fill=fillColor] (340.33,455.28) circle (  1.96);
\path[draw=drawColor,line width= 0.4pt,line join=round,line cap=round,fill=fillColor] (340.59,455.37) circle (  1.96);
\path[draw=drawColor,line width= 0.4pt,line join=round,line cap=round,fill=fillColor] (340.85,455.45) circle (  1.96);
\path[draw=drawColor,line width= 0.4pt,line join=round,line cap=round,fill=fillColor] (341.11,455.54) circle (  1.96);
\path[draw=drawColor,line width= 0.4pt,line join=round,line cap=round,fill=fillColor] (341.37,455.62) circle (  1.96);
\path[draw=drawColor,line width= 0.4pt,line join=round,line cap=round,fill=fillColor] (341.63,455.70) circle (  1.96);
\path[draw=drawColor,line width= 0.4pt,line join=round,line cap=round,fill=fillColor] (341.89,455.78) circle (  1.96);
\path[draw=drawColor,line width= 0.4pt,line join=round,line cap=round,fill=fillColor] (342.15,455.87) circle (  1.96);
\path[draw=drawColor,line width= 0.4pt,line join=round,line cap=round,fill=fillColor] (342.41,455.95) circle (  1.96);
\path[draw=drawColor,line width= 0.4pt,line join=round,line cap=round,fill=fillColor] (342.67,456.04) circle (  1.96);
\path[draw=drawColor,line width= 0.4pt,line join=round,line cap=round,fill=fillColor] (342.92,456.11) circle (  1.96);
\path[draw=drawColor,line width= 0.4pt,line join=round,line cap=round,fill=fillColor] (343.18,456.20) circle (  1.96);
\path[draw=drawColor,line width= 0.4pt,line join=round,line cap=round,fill=fillColor] (343.44,456.28) circle (  1.96);
\path[draw=drawColor,line width= 0.4pt,line join=round,line cap=round,fill=fillColor] (343.70,456.36) circle (  1.96);
\path[draw=drawColor,line width= 0.4pt,line join=round,line cap=round,fill=fillColor] (343.96,456.44) circle (  1.96);
\path[draw=drawColor,line width= 0.4pt,line join=round,line cap=round,fill=fillColor] (344.22,456.53) circle (  1.96);
\path[draw=drawColor,line width= 0.4pt,line join=round,line cap=round,fill=fillColor] (344.48,456.60) circle (  1.96);
\path[draw=drawColor,line width= 0.4pt,line join=round,line cap=round,fill=fillColor] (344.74,456.69) circle (  1.96);
\path[draw=drawColor,line width= 0.4pt,line join=round,line cap=round,fill=fillColor] (345.00,456.76) circle (  1.96);
\path[draw=drawColor,line width= 0.4pt,line join=round,line cap=round,fill=fillColor] (345.26,456.85) circle (  1.96);
\path[draw=drawColor,line width= 0.4pt,line join=round,line cap=round,fill=fillColor] (345.52,456.92) circle (  1.96);
\path[draw=drawColor,line width= 0.4pt,line join=round,line cap=round,fill=fillColor] (345.78,457.01) circle (  1.96);
\path[draw=drawColor,line width= 0.4pt,line join=round,line cap=round,fill=fillColor] (346.04,457.08) circle (  1.96);
\path[draw=drawColor,line width= 0.4pt,line join=round,line cap=round,fill=fillColor] (346.30,457.17) circle (  1.96);
\path[draw=drawColor,line width= 0.4pt,line join=round,line cap=round,fill=fillColor] (346.56,457.24) circle (  1.96);
\path[draw=drawColor,line width= 0.4pt,line join=round,line cap=round,fill=fillColor] (346.82,457.32) circle (  1.96);
\path[draw=drawColor,line width= 0.4pt,line join=round,line cap=round,fill=fillColor] (347.08,457.40) circle (  1.96);
\path[draw=drawColor,line width= 0.4pt,line join=round,line cap=round,fill=fillColor] (347.34,457.48) circle (  1.96);
\path[draw=drawColor,line width= 0.4pt,line join=round,line cap=round,fill=fillColor] (347.60,457.55) circle (  1.96);
\path[draw=drawColor,line width= 0.4pt,line join=round,line cap=round,fill=fillColor] (347.86,457.64) circle (  1.96);
\path[draw=drawColor,line width= 0.4pt,line join=round,line cap=round,fill=fillColor] (348.12,457.71) circle (  1.96);
\path[draw=drawColor,line width= 0.4pt,line join=round,line cap=round,fill=fillColor] (348.38,457.78) circle (  1.96);
\path[draw=drawColor,line width= 0.4pt,line join=round,line cap=round,fill=fillColor] (348.64,457.86) circle (  1.96);
\path[draw=drawColor,line width= 0.4pt,line join=round,line cap=round,fill=fillColor] (348.90,457.93) circle (  1.96);
\path[draw=drawColor,line width= 0.4pt,line join=round,line cap=round,fill=fillColor] (349.16,458.01) circle (  1.96);
\path[draw=drawColor,line width= 0.4pt,line join=round,line cap=round,fill=fillColor] (349.42,458.09) circle (  1.96);
\path[draw=drawColor,line width= 0.4pt,line join=round,line cap=round,fill=fillColor] (349.68,458.17) circle (  1.96);
\path[draw=drawColor,line width= 0.4pt,line join=round,line cap=round,fill=fillColor] (349.94,458.24) circle (  1.96);
\path[draw=drawColor,line width= 0.4pt,line join=round,line cap=round,fill=fillColor] (350.20,458.32) circle (  1.96);
\path[draw=drawColor,line width= 0.4pt,line join=round,line cap=round,fill=fillColor] (350.46,458.39) circle (  1.96);
\path[draw=drawColor,line width= 0.4pt,line join=round,line cap=round,fill=fillColor] (350.72,458.47) circle (  1.96);
\path[draw=drawColor,line width= 0.4pt,line join=round,line cap=round,fill=fillColor] (350.98,458.54) circle (  1.96);
\path[draw=drawColor,line width= 0.4pt,line join=round,line cap=round,fill=fillColor] (351.24,458.62) circle (  1.96);
\path[draw=drawColor,line width= 0.4pt,line join=round,line cap=round,fill=fillColor] (351.50,458.69) circle (  1.96);
\path[draw=drawColor,line width= 0.4pt,line join=round,line cap=round,fill=fillColor] (351.76,458.76) circle (  1.96);
\path[draw=drawColor,line width= 0.4pt,line join=round,line cap=round,fill=fillColor] (352.02,458.84) circle (  1.96);
\path[draw=drawColor,line width= 0.4pt,line join=round,line cap=round,fill=fillColor] (352.28,458.90) circle (  1.96);
\path[draw=drawColor,line width= 0.4pt,line join=round,line cap=round,fill=fillColor] (352.54,458.98) circle (  1.96);
\path[draw=drawColor,line width= 0.4pt,line join=round,line cap=round,fill=fillColor] (352.80,459.05) circle (  1.96);
\path[draw=drawColor,line width= 0.4pt,line join=round,line cap=round,fill=fillColor] (353.06,459.13) circle (  1.96);
\path[draw=drawColor,line width= 0.4pt,line join=round,line cap=round,fill=fillColor] (353.32,459.20) circle (  1.96);
\path[draw=drawColor,line width= 0.4pt,line join=round,line cap=round,fill=fillColor] (353.58,459.26) circle (  1.96);
\path[draw=drawColor,line width= 0.4pt,line join=round,line cap=round,fill=fillColor] (353.84,459.34) circle (  1.96);
\path[draw=drawColor,line width= 0.4pt,line join=round,line cap=round,fill=fillColor] (354.10,459.41) circle (  1.96);
\path[draw=drawColor,line width= 0.4pt,line join=round,line cap=round,fill=fillColor] (354.36,459.49) circle (  1.96);
\path[draw=drawColor,line width= 0.4pt,line join=round,line cap=round,fill=fillColor] (354.62,459.55) circle (  1.96);
\path[draw=drawColor,line width= 0.4pt,line join=round,line cap=round,fill=fillColor] (354.88,459.63) circle (  1.96);
\path[draw=drawColor,line width= 0.4pt,line join=round,line cap=round,fill=fillColor] (355.14,459.69) circle (  1.96);
\path[draw=drawColor,line width= 0.4pt,line join=round,line cap=round,fill=fillColor] (355.40,459.76) circle (  1.96);
\path[draw=drawColor,line width= 0.4pt,line join=round,line cap=round,fill=fillColor] (355.66,459.84) circle (  1.96);
\path[draw=drawColor,line width= 0.4pt,line join=round,line cap=round,fill=fillColor] (355.92,459.90) circle (  1.96);
\path[draw=drawColor,line width= 0.4pt,line join=round,line cap=round,fill=fillColor] (356.18,459.98) circle (  1.96);
\path[draw=drawColor,line width= 0.4pt,line join=round,line cap=round,fill=fillColor] (356.44,460.04) circle (  1.96);
\path[draw=drawColor,line width= 0.4pt,line join=round,line cap=round,fill=fillColor] (356.70,460.12) circle (  1.96);
\path[draw=drawColor,line width= 0.4pt,line join=round,line cap=round,fill=fillColor] (356.96,460.18) circle (  1.96);
\path[draw=drawColor,line width= 0.4pt,line join=round,line cap=round,fill=fillColor] (357.22,460.25) circle (  1.96);
\path[draw=drawColor,line width= 0.4pt,line join=round,line cap=round,fill=fillColor] (357.48,460.32) circle (  1.96);
\path[draw=drawColor,line width= 0.4pt,line join=round,line cap=round,fill=fillColor] (357.74,460.38) circle (  1.96);
\path[draw=drawColor,line width= 0.4pt,line join=round,line cap=round,fill=fillColor] (358.00,460.46) circle (  1.96);
\path[draw=drawColor,line width= 0.4pt,line join=round,line cap=round,fill=fillColor] (358.26,460.52) circle (  1.96);
\path[draw=drawColor,line width= 0.4pt,line join=round,line cap=round,fill=fillColor] (358.52,460.59) circle (  1.96);
\path[draw=drawColor,line width= 0.4pt,line join=round,line cap=round,fill=fillColor] (358.78,460.66) circle (  1.96);
\path[draw=drawColor,line width= 0.4pt,line join=round,line cap=round,fill=fillColor] (359.04,460.72) circle (  1.96);
\path[draw=drawColor,line width= 0.4pt,line join=round,line cap=round,fill=fillColor] (359.30,460.79) circle (  1.96);
\path[draw=drawColor,line width= 0.4pt,line join=round,line cap=round,fill=fillColor] (359.55,460.86) circle (  1.96);
\path[draw=drawColor,line width= 0.4pt,line join=round,line cap=round,fill=fillColor] (359.81,460.92) circle (  1.96);
\path[draw=drawColor,line width= 0.4pt,line join=round,line cap=round,fill=fillColor] (360.07,460.99) circle (  1.96);
\path[draw=drawColor,line width= 0.4pt,line join=round,line cap=round,fill=fillColor] (360.33,461.05) circle (  1.96);
\path[draw=drawColor,line width= 0.4pt,line join=round,line cap=round,fill=fillColor] (360.59,461.13) circle (  1.96);
\path[draw=drawColor,line width= 0.4pt,line join=round,line cap=round,fill=fillColor] (360.85,461.19) circle (  1.96);
\path[draw=drawColor,line width= 0.4pt,line join=round,line cap=round,fill=fillColor] (361.11,461.25) circle (  1.96);
\path[draw=drawColor,line width= 0.4pt,line join=round,line cap=round,fill=fillColor] (361.37,461.32) circle (  1.96);
\path[draw=drawColor,line width= 0.4pt,line join=round,line cap=round,fill=fillColor] (361.63,461.38) circle (  1.96);
\path[draw=drawColor,line width= 0.4pt,line join=round,line cap=round,fill=fillColor] (361.89,461.45) circle (  1.96);
\path[draw=drawColor,line width= 0.4pt,line join=round,line cap=round,fill=fillColor] (362.15,461.51) circle (  1.96);
\path[draw=drawColor,line width= 0.4pt,line join=round,line cap=round,fill=fillColor] (362.41,461.57) circle (  1.96);
\path[draw=drawColor,line width= 0.4pt,line join=round,line cap=round,fill=fillColor] (362.67,461.64) circle (  1.96);
\path[draw=drawColor,line width= 0.4pt,line join=round,line cap=round,fill=fillColor] (362.93,461.70) circle (  1.96);
\path[draw=drawColor,line width= 0.4pt,line join=round,line cap=round,fill=fillColor] (363.19,461.77) circle (  1.96);
\path[draw=drawColor,line width= 0.4pt,line join=round,line cap=round,fill=fillColor] (363.45,461.83) circle (  1.96);
\path[draw=drawColor,line width= 0.4pt,line join=round,line cap=round,fill=fillColor] (363.71,461.89) circle (  1.96);
\path[draw=drawColor,line width= 0.4pt,line join=round,line cap=round,fill=fillColor] (363.97,461.96) circle (  1.96);
\path[draw=drawColor,line width= 0.4pt,line join=round,line cap=round,fill=fillColor] (364.23,462.02) circle (  1.96);
\path[draw=drawColor,line width= 0.4pt,line join=round,line cap=round,fill=fillColor] (364.49,462.08) circle (  1.96);
\path[draw=drawColor,line width= 0.4pt,line join=round,line cap=round,fill=fillColor] (364.75,462.15) circle (  1.96);
\path[draw=drawColor,line width= 0.4pt,line join=round,line cap=round,fill=fillColor] (365.01,462.20) circle (  1.96);
\path[draw=drawColor,line width= 0.4pt,line join=round,line cap=round,fill=fillColor] (365.27,462.27) circle (  1.96);
\path[draw=drawColor,line width= 0.4pt,line join=round,line cap=round,fill=fillColor] (365.53,462.33) circle (  1.96);
\path[draw=drawColor,line width= 0.4pt,line join=round,line cap=round,fill=fillColor] (365.79,462.39) circle (  1.96);
\path[draw=drawColor,line width= 0.4pt,line join=round,line cap=round,fill=fillColor] (366.05,462.46) circle (  1.96);
\path[draw=drawColor,line width= 0.4pt,line join=round,line cap=round,fill=fillColor] (366.31,462.51) circle (  1.96);
\path[draw=drawColor,line width= 0.4pt,line join=round,line cap=round,fill=fillColor] (366.57,462.57) circle (  1.96);
\path[draw=drawColor,line width= 0.4pt,line join=round,line cap=round,fill=fillColor] (366.83,462.64) circle (  1.96);
\path[draw=drawColor,line width= 0.4pt,line join=round,line cap=round,fill=fillColor] (367.09,462.69) circle (  1.96);
\path[draw=drawColor,line width= 0.4pt,line join=round,line cap=round,fill=fillColor] (367.35,462.76) circle (  1.96);
\path[draw=drawColor,line width= 0.4pt,line join=round,line cap=round,fill=fillColor] (367.61,462.82) circle (  1.96);
\path[draw=drawColor,line width= 0.4pt,line join=round,line cap=round,fill=fillColor] (367.87,462.87) circle (  1.96);
\path[draw=drawColor,line width= 0.4pt,line join=round,line cap=round,fill=fillColor] (368.13,462.94) circle (  1.96);
\path[draw=drawColor,line width= 0.4pt,line join=round,line cap=round,fill=fillColor] (368.39,462.99) circle (  1.96);
\path[draw=drawColor,line width= 0.4pt,line join=round,line cap=round,fill=fillColor] (368.65,463.05) circle (  1.96);
\path[draw=drawColor,line width= 0.4pt,line join=round,line cap=round,fill=fillColor] (368.91,463.12) circle (  1.96);
\path[draw=drawColor,line width= 0.4pt,line join=round,line cap=round,fill=fillColor] (369.17,463.17) circle (  1.96);
\path[draw=drawColor,line width= 0.4pt,line join=round,line cap=round,fill=fillColor] (369.43,463.23) circle (  1.96);
\path[draw=drawColor,line width= 0.4pt,line join=round,line cap=round,fill=fillColor] (369.69,463.29) circle (  1.96);
\path[draw=drawColor,line width= 0.4pt,line join=round,line cap=round,fill=fillColor] (369.95,463.35) circle (  1.96);
\path[draw=drawColor,line width= 0.4pt,line join=round,line cap=round,fill=fillColor] (370.21,463.40) circle (  1.96);
\path[draw=drawColor,line width= 0.4pt,line join=round,line cap=round,fill=fillColor] (370.47,463.46) circle (  1.96);
\path[draw=drawColor,line width= 0.4pt,line join=round,line cap=round,fill=fillColor] (370.73,463.52) circle (  1.96);
\path[draw=drawColor,line width= 0.4pt,line join=round,line cap=round,fill=fillColor] (370.99,463.57) circle (  1.96);
\path[draw=drawColor,line width= 0.4pt,line join=round,line cap=round,fill=fillColor] (371.25,463.64) circle (  1.96);
\path[draw=drawColor,line width= 0.4pt,line join=round,line cap=round,fill=fillColor] (371.51,463.69) circle (  1.96);
\path[draw=drawColor,line width= 0.4pt,line join=round,line cap=round,fill=fillColor] (371.77,463.75) circle (  1.96);
\path[draw=drawColor,line width= 0.4pt,line join=round,line cap=round,fill=fillColor] (372.03,463.81) circle (  1.96);
\path[draw=drawColor,line width= 0.4pt,line join=round,line cap=round,fill=fillColor] (372.29,463.86) circle (  1.96);
\path[draw=drawColor,line width= 0.4pt,line join=round,line cap=round,fill=fillColor] (372.55,463.92) circle (  1.96);
\path[draw=drawColor,line width= 0.4pt,line join=round,line cap=round,fill=fillColor] (372.81,463.98) circle (  1.96);
\path[draw=drawColor,line width= 0.4pt,line join=round,line cap=round,fill=fillColor] (373.07,464.03) circle (  1.96);
\path[draw=drawColor,line width= 0.4pt,line join=round,line cap=round,fill=fillColor] (373.33,464.09) circle (  1.96);
\path[draw=drawColor,line width= 0.4pt,line join=round,line cap=round,fill=fillColor] (373.59,464.14) circle (  1.96);
\path[draw=drawColor,line width= 0.4pt,line join=round,line cap=round,fill=fillColor] (373.85,464.19) circle (  1.96);
\path[draw=drawColor,line width= 0.4pt,line join=round,line cap=round,fill=fillColor] (374.11,464.26) circle (  1.96);
\path[draw=drawColor,line width= 0.4pt,line join=round,line cap=round,fill=fillColor] (374.37,464.31) circle (  1.96);
\path[draw=drawColor,line width= 0.4pt,line join=round,line cap=round,fill=fillColor] (374.63,464.36) circle (  1.96);
\path[draw=drawColor,line width= 0.4pt,line join=round,line cap=round,fill=fillColor] (374.89,464.42) circle (  1.96);
\path[draw=drawColor,line width= 0.4pt,line join=round,line cap=round,fill=fillColor] (375.15,464.47) circle (  1.96);
\path[draw=drawColor,line width= 0.4pt,line join=round,line cap=round,fill=fillColor] (375.41,464.52) circle (  1.96);
\path[draw=drawColor,line width= 0.4pt,line join=round,line cap=round,fill=fillColor] (375.67,464.58) circle (  1.96);
\path[draw=drawColor,line width= 0.4pt,line join=round,line cap=round,fill=fillColor] (375.93,464.63) circle (  1.96);
\path[draw=drawColor,line width= 0.4pt,line join=round,line cap=round,fill=fillColor] (376.18,464.68) circle (  1.96);
\path[draw=drawColor,line width= 0.4pt,line join=round,line cap=round,fill=fillColor] (376.44,464.74) circle (  1.96);
\path[draw=drawColor,line width= 0.4pt,line join=round,line cap=round,fill=fillColor] (376.70,464.79) circle (  1.96);
\path[draw=drawColor,line width= 0.4pt,line join=round,line cap=round,fill=fillColor] (376.96,464.84) circle (  1.96);
\path[draw=drawColor,line width= 0.4pt,line join=round,line cap=round,fill=fillColor] (377.22,464.89) circle (  1.96);
\path[draw=drawColor,line width= 0.4pt,line join=round,line cap=round,fill=fillColor] (377.48,464.95) circle (  1.96);
\path[draw=drawColor,line width= 0.4pt,line join=round,line cap=round,fill=fillColor] (377.74,465.00) circle (  1.96);
\path[draw=drawColor,line width= 0.4pt,line join=round,line cap=round,fill=fillColor] (378.00,465.05) circle (  1.96);
\path[draw=drawColor,line width= 0.4pt,line join=round,line cap=round,fill=fillColor] (378.26,465.11) circle (  1.96);
\path[draw=drawColor,line width= 0.4pt,line join=round,line cap=round,fill=fillColor] (378.52,465.16) circle (  1.96);
\path[draw=drawColor,line width= 0.4pt,line join=round,line cap=round,fill=fillColor] (378.78,465.21) circle (  1.96);
\path[draw=drawColor,line width= 0.4pt,line join=round,line cap=round,fill=fillColor] (379.04,465.27) circle (  1.96);
\path[draw=drawColor,line width= 0.4pt,line join=round,line cap=round,fill=fillColor] (379.30,465.32) circle (  1.96);
\path[draw=drawColor,line width= 0.4pt,line join=round,line cap=round,fill=fillColor] (379.56,465.36) circle (  1.96);
\path[draw=drawColor,line width= 0.4pt,line join=round,line cap=round,fill=fillColor] (379.82,465.42) circle (  1.96);
\path[draw=drawColor,line width= 0.4pt,line join=round,line cap=round,fill=fillColor] (380.08,465.47) circle (  1.96);
\path[draw=drawColor,line width= 0.4pt,line join=round,line cap=round,fill=fillColor] (380.34,465.52) circle (  1.96);
\path[draw=drawColor,line width= 0.4pt,line join=round,line cap=round,fill=fillColor] (380.60,465.58) circle (  1.96);
\path[draw=drawColor,line width= 0.4pt,line join=round,line cap=round,fill=fillColor] (380.86,465.62) circle (  1.96);
\path[draw=drawColor,line width= 0.4pt,line join=round,line cap=round,fill=fillColor] (381.12,465.67) circle (  1.96);
\path[draw=drawColor,line width= 0.4pt,line join=round,line cap=round,fill=fillColor] (381.38,465.72) circle (  1.96);
\path[draw=drawColor,line width= 0.4pt,line join=round,line cap=round,fill=fillColor] (381.64,465.77) circle (  1.96);
\path[draw=drawColor,line width= 0.4pt,line join=round,line cap=round,fill=fillColor] (381.90,465.82) circle (  1.96);
\path[draw=drawColor,line width= 0.4pt,line join=round,line cap=round,fill=fillColor] (382.16,465.87) circle (  1.96);
\path[draw=drawColor,line width= 0.4pt,line join=round,line cap=round,fill=fillColor] (382.42,465.92) circle (  1.96);
\path[draw=drawColor,line width= 0.4pt,line join=round,line cap=round,fill=fillColor] (382.68,465.97) circle (  1.96);
\path[draw=drawColor,line width= 0.4pt,line join=round,line cap=round,fill=fillColor] (382.94,466.02) circle (  1.96);
\path[draw=drawColor,line width= 0.4pt,line join=round,line cap=round,fill=fillColor] (383.20,466.07) circle (  1.96);
\path[draw=drawColor,line width= 0.4pt,line join=round,line cap=round,fill=fillColor] (383.46,466.12) circle (  1.96);
\path[draw=drawColor,line width= 0.4pt,line join=round,line cap=round,fill=fillColor] (383.72,466.16) circle (  1.96);
\path[draw=drawColor,line width= 0.4pt,line join=round,line cap=round,fill=fillColor] (383.98,466.21) circle (  1.96);
\path[draw=drawColor,line width= 0.4pt,line join=round,line cap=round,fill=fillColor] (384.24,466.26) circle (  1.96);
\path[draw=drawColor,line width= 0.4pt,line join=round,line cap=round,fill=fillColor] (384.50,466.31) circle (  1.96);
\path[draw=drawColor,line width= 0.4pt,line join=round,line cap=round,fill=fillColor] (384.76,466.35) circle (  1.96);
\path[draw=drawColor,line width= 0.4pt,line join=round,line cap=round,fill=fillColor] (385.02,466.41) circle (  1.96);
\path[draw=drawColor,line width= 0.4pt,line join=round,line cap=round,fill=fillColor] (385.28,466.45) circle (  1.96);
\path[draw=drawColor,line width= 0.4pt,line join=round,line cap=round,fill=fillColor] (385.54,466.50) circle (  1.96);
\path[draw=drawColor,line width= 0.4pt,line join=round,line cap=round,fill=fillColor] (385.80,466.54) circle (  1.96);
\path[draw=drawColor,line width= 0.4pt,line join=round,line cap=round,fill=fillColor] (386.06,466.60) circle (  1.96);
\path[draw=drawColor,line width= 0.4pt,line join=round,line cap=round,fill=fillColor] (386.32,466.64) circle (  1.96);
\path[draw=drawColor,line width= 0.4pt,line join=round,line cap=round,fill=fillColor] (386.58,466.68) circle (  1.96);
\path[draw=drawColor,line width= 0.4pt,line join=round,line cap=round,fill=fillColor] (386.84,466.74) circle (  1.96);
\path[draw=drawColor,line width= 0.4pt,line join=round,line cap=round,fill=fillColor] (387.10,466.78) circle (  1.96);
\path[draw=drawColor,line width= 0.4pt,line join=round,line cap=round,fill=fillColor] (387.36,466.83) circle (  1.96);
\path[draw=drawColor,line width= 0.4pt,line join=round,line cap=round,fill=fillColor] (387.62,466.87) circle (  1.96);
\path[draw=drawColor,line width= 0.4pt,line join=round,line cap=round,fill=fillColor] (387.88,466.92) circle (  1.96);
\path[draw=drawColor,line width= 0.4pt,line join=round,line cap=round,fill=fillColor] (388.14,466.96) circle (  1.96);
\path[draw=drawColor,line width= 0.4pt,line join=round,line cap=round,fill=fillColor] (388.40,467.01) circle (  1.96);
\path[draw=drawColor,line width= 0.4pt,line join=round,line cap=round,fill=fillColor] (388.66,467.06) circle (  1.96);
\path[draw=drawColor,line width= 0.4pt,line join=round,line cap=round,fill=fillColor] (388.92,467.10) circle (  1.96);
\path[draw=drawColor,line width= 0.4pt,line join=round,line cap=round,fill=fillColor] (389.18,467.15) circle (  1.96);
\path[draw=drawColor,line width= 0.4pt,line join=round,line cap=round,fill=fillColor] (389.44,467.19) circle (  1.96);
\path[draw=drawColor,line width= 0.4pt,line join=round,line cap=round,fill=fillColor] (389.70,467.24) circle (  1.96);
\path[draw=drawColor,line width= 0.4pt,line join=round,line cap=round,fill=fillColor] (389.96,467.28) circle (  1.96);
\path[draw=drawColor,line width= 0.4pt,line join=round,line cap=round,fill=fillColor] (390.22,467.32) circle (  1.96);
\path[draw=drawColor,line width= 0.4pt,line join=round,line cap=round,fill=fillColor] (390.48,467.37) circle (  1.96);
\path[draw=drawColor,line width= 0.4pt,line join=round,line cap=round,fill=fillColor] (390.74,467.42) circle (  1.96);
\path[draw=drawColor,line width= 0.4pt,line join=round,line cap=round,fill=fillColor] (391.00,467.46) circle (  1.96);
\path[draw=drawColor,line width= 0.4pt,line join=round,line cap=round,fill=fillColor] (391.26,467.50) circle (  1.96);
\path[draw=drawColor,line width= 0.4pt,line join=round,line cap=round,fill=fillColor] (391.52,467.54) circle (  1.96);
\path[draw=drawColor,line width= 0.4pt,line join=round,line cap=round,fill=fillColor] (391.78,467.59) circle (  1.96);
\path[draw=drawColor,line width= 0.4pt,line join=round,line cap=round,fill=fillColor] (392.04,467.63) circle (  1.96);
\path[draw=drawColor,line width= 0.4pt,line join=round,line cap=round,fill=fillColor] (392.30,467.67) circle (  1.96);
\path[draw=drawColor,line width= 0.4pt,line join=round,line cap=round,fill=fillColor] (392.56,467.72) circle (  1.96);
\path[draw=drawColor,line width= 0.4pt,line join=round,line cap=round,fill=fillColor] (392.81,467.76) circle (  1.96);
\path[draw=drawColor,line width= 0.4pt,line join=round,line cap=round,fill=fillColor] (393.07,467.80) circle (  1.96);
\path[draw=drawColor,line width= 0.4pt,line join=round,line cap=round,fill=fillColor] (393.33,467.84) circle (  1.96);
\path[draw=drawColor,line width= 0.4pt,line join=round,line cap=round,fill=fillColor] (393.59,467.89) circle (  1.96);
\path[draw=drawColor,line width= 0.4pt,line join=round,line cap=round,fill=fillColor] (393.85,467.93) circle (  1.96);
\path[draw=drawColor,line width= 0.4pt,line join=round,line cap=round,fill=fillColor] (394.11,467.97) circle (  1.96);
\path[draw=drawColor,line width= 0.4pt,line join=round,line cap=round,fill=fillColor] (394.37,468.01) circle (  1.96);
\path[draw=drawColor,line width= 0.4pt,line join=round,line cap=round,fill=fillColor] (394.63,468.06) circle (  1.96);
\path[draw=drawColor,line width= 0.4pt,line join=round,line cap=round,fill=fillColor] (394.89,468.10) circle (  1.96);
\path[draw=drawColor,line width= 0.4pt,line join=round,line cap=round,fill=fillColor] (395.15,468.14) circle (  1.96);
\path[draw=drawColor,line width= 0.4pt,line join=round,line cap=round,fill=fillColor] (395.41,468.18) circle (  1.96);
\path[draw=drawColor,line width= 0.4pt,line join=round,line cap=round,fill=fillColor] (395.67,468.23) circle (  1.96);
\path[draw=drawColor,line width= 0.4pt,line join=round,line cap=round,fill=fillColor] (395.93,468.27) circle (  1.96);
\path[draw=drawColor,line width= 0.4pt,line join=round,line cap=round,fill=fillColor] (396.19,468.31) circle (  1.96);
\path[draw=drawColor,line width= 0.4pt,line join=round,line cap=round,fill=fillColor] (396.45,468.34) circle (  1.96);
\path[draw=drawColor,line width= 0.4pt,line join=round,line cap=round,fill=fillColor] (396.71,468.39) circle (  1.96);
\path[draw=drawColor,line width= 0.4pt,line join=round,line cap=round,fill=fillColor] (396.97,468.43) circle (  1.96);
\path[draw=drawColor,line width= 0.4pt,line join=round,line cap=round,fill=fillColor] (397.23,468.47) circle (  1.96);
\path[draw=drawColor,line width= 0.4pt,line join=round,line cap=round,fill=fillColor] (397.49,468.51) circle (  1.96);
\path[draw=drawColor,line width= 0.4pt,line join=round,line cap=round,fill=fillColor] (397.75,468.55) circle (  1.96);
\path[draw=drawColor,line width= 0.4pt,line join=round,line cap=round,fill=fillColor] (398.01,468.59) circle (  1.96);
\path[draw=drawColor,line width= 0.4pt,line join=round,line cap=round,fill=fillColor] (398.27,468.63) circle (  1.96);
\path[draw=drawColor,line width= 0.4pt,line join=round,line cap=round,fill=fillColor] (398.53,468.67) circle (  1.96);
\path[draw=drawColor,line width= 0.4pt,line join=round,line cap=round,fill=fillColor] (398.79,468.70) circle (  1.96);
\path[draw=drawColor,line width= 0.4pt,line join=round,line cap=round,fill=fillColor] (399.05,468.75) circle (  1.96);
\path[draw=drawColor,line width= 0.4pt,line join=round,line cap=round,fill=fillColor] (399.31,468.79) circle (  1.96);
\path[draw=drawColor,line width= 0.4pt,line join=round,line cap=round,fill=fillColor] (399.57,468.83) circle (  1.96);
\path[draw=drawColor,line width= 0.4pt,line join=round,line cap=round,fill=fillColor] (399.83,468.86) circle (  1.96);
\path[draw=drawColor,line width= 0.4pt,line join=round,line cap=round,fill=fillColor] (400.09,468.91) circle (  1.96);
\path[draw=drawColor,line width= 0.4pt,line join=round,line cap=round,fill=fillColor] (400.35,468.95) circle (  1.96);
\path[draw=drawColor,line width= 0.4pt,line join=round,line cap=round,fill=fillColor] (400.61,468.98) circle (  1.96);
\path[draw=drawColor,line width= 0.4pt,line join=round,line cap=round,fill=fillColor] (400.87,469.02) circle (  1.96);
\path[draw=drawColor,line width= 0.4pt,line join=round,line cap=round,fill=fillColor] (401.13,469.06) circle (  1.96);
\path[draw=drawColor,line width= 0.4pt,line join=round,line cap=round,fill=fillColor] (401.39,469.10) circle (  1.96);
\path[draw=drawColor,line width= 0.4pt,line join=round,line cap=round,fill=fillColor] (401.65,469.14) circle (  1.96);
\path[draw=drawColor,line width= 0.4pt,line join=round,line cap=round,fill=fillColor] (401.91,469.17) circle (  1.96);
\path[draw=drawColor,line width= 0.4pt,line join=round,line cap=round,fill=fillColor] (402.17,469.22) circle (  1.96);
\path[draw=drawColor,line width= 0.4pt,line join=round,line cap=round,fill=fillColor] (402.43,469.25) circle (  1.96);
\path[draw=drawColor,line width= 0.4pt,line join=round,line cap=round,fill=fillColor] (402.69,469.29) circle (  1.96);
\path[draw=drawColor,line width= 0.4pt,line join=round,line cap=round,fill=fillColor] (402.95,469.32) circle (  1.96);
\path[draw=drawColor,line width= 0.4pt,line join=round,line cap=round,fill=fillColor] (403.21,469.36) circle (  1.96);
\path[draw=drawColor,line width= 0.4pt,line join=round,line cap=round,fill=fillColor] (403.47,469.40) circle (  1.96);
\path[draw=drawColor,line width= 0.4pt,line join=round,line cap=round,fill=fillColor] (403.73,469.44) circle (  1.96);
\path[draw=drawColor,line width= 0.4pt,line join=round,line cap=round,fill=fillColor] (403.99,469.47) circle (  1.96);
\path[draw=drawColor,line width= 0.4pt,line join=round,line cap=round,fill=fillColor] (404.25,469.51) circle (  1.96);
\path[draw=drawColor,line width= 0.4pt,line join=round,line cap=round,fill=fillColor] (404.51,469.55) circle (  1.96);
\path[draw=drawColor,line width= 0.4pt,line join=round,line cap=round,fill=fillColor] (404.77,469.59) circle (  1.96);
\path[draw=drawColor,line width= 0.4pt,line join=round,line cap=round,fill=fillColor] (405.03,469.62) circle (  1.96);
\path[draw=drawColor,line width= 0.4pt,line join=round,line cap=round,fill=fillColor] (405.29,469.65) circle (  1.96);
\path[draw=drawColor,line width= 0.4pt,line join=round,line cap=round,fill=fillColor] (405.55,469.69) circle (  1.96);
\path[draw=drawColor,line width= 0.4pt,line join=round,line cap=round,fill=fillColor] (405.81,469.73) circle (  1.96);
\path[draw=drawColor,line width= 0.4pt,line join=round,line cap=round,fill=fillColor] (406.07,469.77) circle (  1.96);
\path[draw=drawColor,line width= 0.4pt,line join=round,line cap=round,fill=fillColor] (406.33,469.80) circle (  1.96);
\path[draw=drawColor,line width= 0.4pt,line join=round,line cap=round,fill=fillColor] (406.59,469.83) circle (  1.96);
\path[draw=drawColor,line width= 0.4pt,line join=round,line cap=round,fill=fillColor] (406.85,469.88) circle (  1.96);
\path[draw=drawColor,line width= 0.4pt,line join=round,line cap=round,fill=fillColor] (407.11,469.91) circle (  1.96);
\path[draw=drawColor,line width= 0.4pt,line join=round,line cap=round,fill=fillColor] (407.37,469.94) circle (  1.96);
\path[draw=drawColor,line width= 0.4pt,line join=round,line cap=round,fill=fillColor] (407.63,469.98) circle (  1.96);
\path[draw=drawColor,line width= 0.4pt,line join=round,line cap=round,fill=fillColor] (407.89,470.01) circle (  1.96);
\path[draw=drawColor,line width= 0.4pt,line join=round,line cap=round,fill=fillColor] (408.15,470.05) circle (  1.96);
\path[draw=drawColor,line width= 0.4pt,line join=round,line cap=round,fill=fillColor] (408.41,470.08) circle (  1.96);
\path[draw=drawColor,line width= 0.4pt,line join=round,line cap=round,fill=fillColor] (408.67,470.12) circle (  1.96);
\path[draw=drawColor,line width= 0.4pt,line join=round,line cap=round,fill=fillColor] (408.93,470.15) circle (  1.96);
\path[draw=drawColor,line width= 0.4pt,line join=round,line cap=round,fill=fillColor] (409.18,470.18) circle (  1.96);
\path[draw=drawColor,line width= 0.4pt,line join=round,line cap=round,fill=fillColor] (409.44,470.22) circle (  1.96);
\path[draw=drawColor,line width= 0.4pt,line join=round,line cap=round,fill=fillColor] (409.70,470.25) circle (  1.96);
\path[draw=drawColor,line width= 0.4pt,line join=round,line cap=round,fill=fillColor] (409.96,470.29) circle (  1.96);
\path[draw=drawColor,line width= 0.4pt,line join=round,line cap=round,fill=fillColor] (410.22,470.32) circle (  1.96);
\path[draw=drawColor,line width= 0.4pt,line join=round,line cap=round,fill=fillColor] (410.48,470.35) circle (  1.96);
\path[draw=drawColor,line width= 0.4pt,line join=round,line cap=round,fill=fillColor] (410.74,470.39) circle (  1.96);
\path[draw=drawColor,line width= 0.4pt,line join=round,line cap=round,fill=fillColor] (411.00,470.42) circle (  1.96);
\path[draw=drawColor,line width= 0.4pt,line join=round,line cap=round,fill=fillColor] (411.26,470.45) circle (  1.96);
\path[draw=drawColor,line width= 0.4pt,line join=round,line cap=round,fill=fillColor] (411.52,470.49) circle (  1.96);
\path[draw=drawColor,line width= 0.4pt,line join=round,line cap=round,fill=fillColor] (411.78,470.52) circle (  1.96);
\path[draw=drawColor,line width= 0.4pt,line join=round,line cap=round,fill=fillColor] (412.04,470.56) circle (  1.96);
\path[draw=drawColor,line width= 0.4pt,line join=round,line cap=round,fill=fillColor] (412.30,470.59) circle (  1.96);
\path[draw=drawColor,line width= 0.4pt,line join=round,line cap=round,fill=fillColor] (412.56,470.62) circle (  1.96);
\path[draw=drawColor,line width= 0.4pt,line join=round,line cap=round,fill=fillColor] (412.82,470.65) circle (  1.96);
\path[draw=drawColor,line width= 0.4pt,line join=round,line cap=round,fill=fillColor] (413.08,470.68) circle (  1.96);
\path[draw=drawColor,line width= 0.4pt,line join=round,line cap=round,fill=fillColor] (413.34,470.72) circle (  1.96);
\path[draw=drawColor,line width= 0.4pt,line join=round,line cap=round,fill=fillColor] (413.60,470.75) circle (  1.96);
\path[draw=drawColor,line width= 0.4pt,line join=round,line cap=round,fill=fillColor] (413.86,470.78) circle (  1.96);
\path[draw=drawColor,line width= 0.4pt,line join=round,line cap=round,fill=fillColor] (414.12,470.81) circle (  1.96);
\path[draw=drawColor,line width= 0.4pt,line join=round,line cap=round,fill=fillColor] (414.38,470.84) circle (  1.96);
\path[draw=drawColor,line width= 0.4pt,line join=round,line cap=round,fill=fillColor] (414.64,470.88) circle (  1.96);
\path[draw=drawColor,line width= 0.4pt,line join=round,line cap=round,fill=fillColor] (414.90,470.91) circle (  1.96);
\path[draw=drawColor,line width= 0.4pt,line join=round,line cap=round,fill=fillColor] (415.16,470.94) circle (  1.96);
\path[draw=drawColor,line width= 0.4pt,line join=round,line cap=round,fill=fillColor] (415.42,470.97) circle (  1.96);
\path[draw=drawColor,line width= 0.4pt,line join=round,line cap=round,fill=fillColor] (415.68,471.00) circle (  1.96);
\path[draw=drawColor,line width= 0.4pt,line join=round,line cap=round,fill=fillColor] (415.94,471.04) circle (  1.96);
\path[draw=drawColor,line width= 0.4pt,line join=round,line cap=round,fill=fillColor] (416.20,471.07) circle (  1.96);
\path[draw=drawColor,line width= 0.4pt,line join=round,line cap=round,fill=fillColor] (416.46,471.10) circle (  1.96);
\path[draw=drawColor,line width= 0.4pt,line join=round,line cap=round,fill=fillColor] (416.72,471.13) circle (  1.96);
\path[draw=drawColor,line width= 0.4pt,line join=round,line cap=round,fill=fillColor] (416.98,471.16) circle (  1.96);
\path[draw=drawColor,line width= 0.4pt,line join=round,line cap=round,fill=fillColor] (417.24,471.20) circle (  1.96);
\path[draw=drawColor,line width= 0.4pt,line join=round,line cap=round,fill=fillColor] (417.50,471.23) circle (  1.96);
\path[draw=drawColor,line width= 0.4pt,line join=round,line cap=round,fill=fillColor] (417.76,471.25) circle (  1.96);
\path[draw=drawColor,line width= 0.4pt,line join=round,line cap=round,fill=fillColor] (418.02,471.28) circle (  1.96);
\path[draw=drawColor,line width= 0.4pt,line join=round,line cap=round,fill=fillColor] (418.28,471.31) circle (  1.96);
\path[draw=drawColor,line width= 0.4pt,line join=round,line cap=round,fill=fillColor] (418.54,471.34) circle (  1.96);
\path[draw=drawColor,line width= 0.4pt,line join=round,line cap=round,fill=fillColor] (418.80,471.38) circle (  1.96);
\path[draw=drawColor,line width= 0.4pt,line join=round,line cap=round,fill=fillColor] (419.06,471.41) circle (  1.96);
\path[draw=drawColor,line width= 0.4pt,line join=round,line cap=round,fill=fillColor] (419.32,471.43) circle (  1.96);
\path[draw=drawColor,line width= 0.4pt,line join=round,line cap=round,fill=fillColor] (419.58,471.46) circle (  1.96);
\path[draw=drawColor,line width= 0.4pt,line join=round,line cap=round,fill=fillColor] (419.84,471.49) circle (  1.96);
\path[draw=drawColor,line width= 0.4pt,line join=round,line cap=round,fill=fillColor] (420.10,471.53) circle (  1.96);
\path[draw=drawColor,line width= 0.4pt,line join=round,line cap=round,fill=fillColor] (420.36,471.56) circle (  1.96);
\path[draw=drawColor,line width= 0.4pt,line join=round,line cap=round,fill=fillColor] (420.62,471.58) circle (  1.96);
\path[draw=drawColor,line width= 0.4pt,line join=round,line cap=round,fill=fillColor] (420.88,471.61) circle (  1.96);
\path[draw=drawColor,line width= 0.4pt,line join=round,line cap=round,fill=fillColor] (421.14,471.64) circle (  1.96);
\path[draw=drawColor,line width= 0.4pt,line join=round,line cap=round,fill=fillColor] (421.40,471.66) circle (  1.96);
\path[draw=drawColor,line width= 0.4pt,line join=round,line cap=round,fill=fillColor] (421.66,471.70) circle (  1.96);
\path[draw=drawColor,line width= 0.4pt,line join=round,line cap=round,fill=fillColor] (421.92,471.73) circle (  1.96);
\path[draw=drawColor,line width= 0.4pt,line join=round,line cap=round,fill=fillColor] (422.18,471.76) circle (  1.96);
\path[draw=drawColor,line width= 0.4pt,line join=round,line cap=round,fill=fillColor] (422.44,471.78) circle (  1.96);
\path[draw=drawColor,line width= 0.4pt,line join=round,line cap=round,fill=fillColor] (422.70,471.81) circle (  1.96);
\path[draw=drawColor,line width= 0.4pt,line join=round,line cap=round,fill=fillColor] (422.96,471.84) circle (  1.96);
\path[draw=drawColor,line width= 0.4pt,line join=round,line cap=round,fill=fillColor] (423.22,471.87) circle (  1.96);
\path[draw=drawColor,line width= 0.4pt,line join=round,line cap=round,fill=fillColor] (423.48,471.90) circle (  1.96);
\path[draw=drawColor,line width= 0.4pt,line join=round,line cap=round,fill=fillColor] (423.74,471.93) circle (  1.96);
\path[draw=drawColor,line width= 0.4pt,line join=round,line cap=round,fill=fillColor] (424.00,471.95) circle (  1.96);
\path[draw=drawColor,line width= 0.4pt,line join=round,line cap=round,fill=fillColor] (424.26,471.98) circle (  1.96);
\path[draw=drawColor,line width= 0.4pt,line join=round,line cap=round,fill=fillColor] (424.52,472.00) circle (  1.96);
\path[draw=drawColor,line width= 0.4pt,line join=round,line cap=round,fill=fillColor] (424.78,472.04) circle (  1.96);
\path[draw=drawColor,line width= 0.4pt,line join=round,line cap=round,fill=fillColor] (425.04,472.07) circle (  1.96);
\path[draw=drawColor,line width= 0.4pt,line join=round,line cap=round,fill=fillColor] (425.30,472.09) circle (  1.96);
\path[draw=drawColor,line width= 0.4pt,line join=round,line cap=round,fill=fillColor] (425.56,472.12) circle (  1.96);
\path[draw=drawColor,line width= 0.4pt,line join=round,line cap=round,fill=fillColor] (425.81,472.14) circle (  1.96);
\path[draw=drawColor,line width= 0.4pt,line join=round,line cap=round,fill=fillColor] (426.07,472.17) circle (  1.96);
\path[draw=drawColor,line width= 0.4pt,line join=round,line cap=round,fill=fillColor] (426.33,472.20) circle (  1.96);
\path[draw=drawColor,line width= 0.4pt,line join=round,line cap=round,fill=fillColor] (426.59,472.23) circle (  1.96);
\path[draw=drawColor,line width= 0.4pt,line join=round,line cap=round,fill=fillColor] (426.85,472.25) circle (  1.96);
\path[draw=drawColor,line width= 0.4pt,line join=round,line cap=round,fill=fillColor] (427.11,472.28) circle (  1.96);
\path[draw=drawColor,line width= 0.4pt,line join=round,line cap=round,fill=fillColor] (427.37,472.30) circle (  1.96);
\path[draw=drawColor,line width= 0.4pt,line join=round,line cap=round,fill=fillColor] (427.63,472.33) circle (  1.96);
\path[draw=drawColor,line width= 0.4pt,line join=round,line cap=round,fill=fillColor] (427.89,472.36) circle (  1.96);
\path[draw=drawColor,line width= 0.4pt,line join=round,line cap=round,fill=fillColor] (428.15,472.39) circle (  1.96);
\path[draw=drawColor,line width= 0.4pt,line join=round,line cap=round,fill=fillColor] (428.41,472.41) circle (  1.96);
\path[draw=drawColor,line width= 0.4pt,line join=round,line cap=round,fill=fillColor] (428.67,472.44) circle (  1.96);
\path[draw=drawColor,line width= 0.4pt,line join=round,line cap=round,fill=fillColor] (428.93,472.46) circle (  1.96);
\path[draw=drawColor,line width= 0.4pt,line join=round,line cap=round,fill=fillColor] (429.19,472.49) circle (  1.96);
\path[draw=drawColor,line width= 0.4pt,line join=round,line cap=round,fill=fillColor] (429.45,472.52) circle (  1.96);
\path[draw=drawColor,line width= 0.4pt,line join=round,line cap=round,fill=fillColor] (429.71,472.55) circle (  1.96);
\path[draw=drawColor,line width= 0.4pt,line join=round,line cap=round,fill=fillColor] (429.97,472.57) circle (  1.96);
\path[draw=drawColor,line width= 0.4pt,line join=round,line cap=round,fill=fillColor] (430.23,472.59) circle (  1.96);
\path[draw=drawColor,line width= 0.4pt,line join=round,line cap=round,fill=fillColor] (430.49,472.62) circle (  1.96);
\path[draw=drawColor,line width= 0.4pt,line join=round,line cap=round,fill=fillColor] (430.75,472.64) circle (  1.96);
\path[draw=drawColor,line width= 0.4pt,line join=round,line cap=round,fill=fillColor] (431.01,472.67) circle (  1.96);
\path[draw=drawColor,line width= 0.4pt,line join=round,line cap=round,fill=fillColor] (431.27,472.70) circle (  1.96);
\path[draw=drawColor,line width= 0.4pt,line join=round,line cap=round,fill=fillColor] (431.53,472.72) circle (  1.96);
\path[draw=drawColor,line width= 0.4pt,line join=round,line cap=round,fill=fillColor] (431.79,472.75) circle (  1.96);
\path[draw=drawColor,line width= 0.4pt,line join=round,line cap=round,fill=fillColor] (432.05,472.77) circle (  1.96);
\path[draw=drawColor,line width= 0.4pt,line join=round,line cap=round,fill=fillColor] (432.31,472.79) circle (  1.96);
\path[draw=drawColor,line width= 0.4pt,line join=round,line cap=round,fill=fillColor] (432.57,472.82) circle (  1.96);
\path[draw=drawColor,line width= 0.4pt,line join=round,line cap=round,fill=fillColor] (432.83,472.85) circle (  1.96);
\path[draw=drawColor,line width= 0.4pt,line join=round,line cap=round,fill=fillColor] (433.09,472.87) circle (  1.96);
\path[draw=drawColor,line width= 0.4pt,line join=round,line cap=round,fill=fillColor] (433.35,472.90) circle (  1.96);
\path[draw=drawColor,line width= 0.4pt,line join=round,line cap=round,fill=fillColor] (433.61,472.92) circle (  1.96);
\path[draw=drawColor,line width= 0.4pt,line join=round,line cap=round,fill=fillColor] (433.87,472.94) circle (  1.96);
\path[draw=drawColor,line width= 0.4pt,line join=round,line cap=round,fill=fillColor] (434.13,472.97) circle (  1.96);
\path[draw=drawColor,line width= 0.4pt,line join=round,line cap=round,fill=fillColor] (434.39,472.99) circle (  1.96);
\path[draw=drawColor,line width= 0.4pt,line join=round,line cap=round,fill=fillColor] (434.65,473.02) circle (  1.96);
\path[draw=drawColor,line width= 0.4pt,line join=round,line cap=round,fill=fillColor] (434.91,473.04) circle (  1.96);
\path[draw=drawColor,line width= 0.4pt,line join=round,line cap=round,fill=fillColor] (435.17,473.07) circle (  1.96);
\path[draw=drawColor,line width= 0.4pt,line join=round,line cap=round,fill=fillColor] (435.43,473.09) circle (  1.96);
\path[draw=drawColor,line width= 0.4pt,line join=round,line cap=round,fill=fillColor] (435.69,473.11) circle (  1.96);
\path[draw=drawColor,line width= 0.4pt,line join=round,line cap=round,fill=fillColor] (435.95,473.13) circle (  1.96);
\path[draw=drawColor,line width= 0.4pt,line join=round,line cap=round,fill=fillColor] (436.21,473.15) circle (  1.96);
\path[draw=drawColor,line width= 0.4pt,line join=round,line cap=round,fill=fillColor] (436.47,473.19) circle (  1.96);
\path[draw=drawColor,line width= 0.4pt,line join=round,line cap=round,fill=fillColor] (436.73,473.21) circle (  1.96);
\path[draw=drawColor,line width= 0.4pt,line join=round,line cap=round,fill=fillColor] (436.99,473.23) circle (  1.96);
\path[draw=drawColor,line width= 0.4pt,line join=round,line cap=round,fill=fillColor] (437.25,473.25) circle (  1.96);
\path[draw=drawColor,line width= 0.4pt,line join=round,line cap=round,fill=fillColor] (437.51,473.27) circle (  1.96);
\path[draw=drawColor,line width= 0.4pt,line join=round,line cap=round,fill=fillColor] (437.77,473.30) circle (  1.96);
\path[draw=drawColor,line width= 0.4pt,line join=round,line cap=round,fill=fillColor] (438.03,473.32) circle (  1.96);
\path[draw=drawColor,line width= 0.4pt,line join=round,line cap=round,fill=fillColor] (438.29,473.35) circle (  1.96);
\path[draw=drawColor,line width= 0.4pt,line join=round,line cap=round,fill=fillColor] (438.55,473.37) circle (  1.96);
\path[draw=drawColor,line width= 0.4pt,line join=round,line cap=round,fill=fillColor] (438.81,473.39) circle (  1.96);
\path[draw=drawColor,line width= 0.4pt,line join=round,line cap=round,fill=fillColor] (439.07,473.41) circle (  1.96);
\path[draw=drawColor,line width= 0.4pt,line join=round,line cap=round,fill=fillColor] (439.33,473.43) circle (  1.96);
\path[draw=drawColor,line width= 0.4pt,line join=round,line cap=round,fill=fillColor] (439.59,473.45) circle (  1.96);
\path[draw=drawColor,line width= 0.4pt,line join=round,line cap=round,fill=fillColor] (439.85,473.48) circle (  1.96);
\path[draw=drawColor,line width= 0.4pt,line join=round,line cap=round,fill=fillColor] (440.11,473.51) circle (  1.96);
\path[draw=drawColor,line width= 0.4pt,line join=round,line cap=round,fill=fillColor] (440.37,473.53) circle (  1.96);
\path[draw=drawColor,line width= 0.4pt,line join=round,line cap=round,fill=fillColor] (440.63,473.55) circle (  1.96);
\path[draw=drawColor,line width= 0.4pt,line join=round,line cap=round,fill=fillColor] (440.89,473.57) circle (  1.96);
\path[draw=drawColor,line width= 0.4pt,line join=round,line cap=round,fill=fillColor] (441.15,473.59) circle (  1.96);
\path[draw=drawColor,line width= 0.4pt,line join=round,line cap=round,fill=fillColor] (441.41,473.61) circle (  1.96);
\path[draw=drawColor,line width= 0.4pt,line join=round,line cap=round,fill=fillColor] (441.67,473.63) circle (  1.96);
\path[draw=drawColor,line width= 0.4pt,line join=round,line cap=round,fill=fillColor] (441.93,473.65) circle (  1.96);
\path[draw=drawColor,line width= 0.4pt,line join=round,line cap=round,fill=fillColor] (442.19,473.68) circle (  1.96);
\path[draw=drawColor,line width= 0.4pt,line join=round,line cap=round,fill=fillColor] (442.44,473.70) circle (  1.96);
\path[draw=drawColor,line width= 0.4pt,line join=round,line cap=round,fill=fillColor] (442.70,473.72) circle (  1.96);
\path[draw=drawColor,line width= 0.4pt,line join=round,line cap=round,fill=fillColor] (442.96,473.74) circle (  1.96);
\path[draw=drawColor,line width= 0.4pt,line join=round,line cap=round,fill=fillColor] (443.22,473.76) circle (  1.96);
\path[draw=drawColor,line width= 0.4pt,line join=round,line cap=round,fill=fillColor] (443.48,473.78) circle (  1.96);
\path[draw=drawColor,line width= 0.4pt,line join=round,line cap=round,fill=fillColor] (443.74,473.80) circle (  1.96);
\path[draw=drawColor,line width= 0.4pt,line join=round,line cap=round,fill=fillColor] (444.00,473.82) circle (  1.96);
\path[draw=drawColor,line width= 0.4pt,line join=round,line cap=round,fill=fillColor] (444.26,473.85) circle (  1.96);
\path[draw=drawColor,line width= 0.4pt,line join=round,line cap=round,fill=fillColor] (444.52,473.87) circle (  1.96);
\path[draw=drawColor,line width= 0.4pt,line join=round,line cap=round,fill=fillColor] (444.78,473.89) circle (  1.96);
\path[draw=drawColor,line width= 0.4pt,line join=round,line cap=round,fill=fillColor] (445.04,473.91) circle (  1.96);
\path[draw=drawColor,line width= 0.4pt,line join=round,line cap=round,fill=fillColor] (445.30,473.93) circle (  1.96);
\path[draw=drawColor,line width= 0.4pt,line join=round,line cap=round,fill=fillColor] (445.56,473.95) circle (  1.96);
\path[draw=drawColor,line width= 0.4pt,line join=round,line cap=round,fill=fillColor] (445.82,473.97) circle (  1.96);
\path[draw=drawColor,line width= 0.4pt,line join=round,line cap=round,fill=fillColor] (446.08,473.99) circle (  1.96);
\path[draw=drawColor,line width= 0.4pt,line join=round,line cap=round,fill=fillColor] (446.34,474.02) circle (  1.96);
\path[draw=drawColor,line width= 0.4pt,line join=round,line cap=round,fill=fillColor] (446.60,474.03) circle (  1.96);
\path[draw=drawColor,line width= 0.4pt,line join=round,line cap=round,fill=fillColor] (446.86,474.05) circle (  1.96);
\path[draw=drawColor,line width= 0.4pt,line join=round,line cap=round,fill=fillColor] (447.12,474.07) circle (  1.96);
\path[draw=drawColor,line width= 0.4pt,line join=round,line cap=round,fill=fillColor] (447.38,474.09) circle (  1.96);
\path[draw=drawColor,line width= 0.4pt,line join=round,line cap=round,fill=fillColor] (447.64,474.11) circle (  1.96);
\path[draw=drawColor,line width= 0.4pt,line join=round,line cap=round,fill=fillColor] (447.90,474.13) circle (  1.96);
\path[draw=drawColor,line width= 0.4pt,line join=round,line cap=round,fill=fillColor] (448.16,474.15) circle (  1.96);
\path[draw=drawColor,line width= 0.4pt,line join=round,line cap=round,fill=fillColor] (448.42,474.18) circle (  1.96);
\path[draw=drawColor,line width= 0.4pt,line join=round,line cap=round,fill=fillColor] (448.68,474.19) circle (  1.96);
\path[draw=drawColor,line width= 0.4pt,line join=round,line cap=round,fill=fillColor] (448.94,474.21) circle (  1.96);
\path[draw=drawColor,line width= 0.4pt,line join=round,line cap=round,fill=fillColor] (449.20,474.23) circle (  1.96);
\path[draw=drawColor,line width= 0.4pt,line join=round,line cap=round,fill=fillColor] (449.46,474.25) circle (  1.96);
\path[draw=drawColor,line width= 0.4pt,line join=round,line cap=round,fill=fillColor] (449.72,474.27) circle (  1.96);
\path[draw=drawColor,line width= 0.4pt,line join=round,line cap=round,fill=fillColor] (449.98,474.29) circle (  1.96);
\path[draw=drawColor,line width= 0.4pt,line join=round,line cap=round,fill=fillColor] (450.24,474.30) circle (  1.96);
\path[draw=drawColor,line width= 0.4pt,line join=round,line cap=round,fill=fillColor] (450.50,474.32) circle (  1.96);
\path[draw=drawColor,line width= 0.4pt,line join=round,line cap=round,fill=fillColor] (450.76,474.35) circle (  1.96);
\path[draw=drawColor,line width= 0.4pt,line join=round,line cap=round,fill=fillColor] (451.02,474.37) circle (  1.96);
\path[draw=drawColor,line width= 0.4pt,line join=round,line cap=round,fill=fillColor] (451.28,474.39) circle (  1.96);
\path[draw=drawColor,line width= 0.4pt,line join=round,line cap=round,fill=fillColor] (451.54,474.40) circle (  1.96);
\path[draw=drawColor,line width= 0.4pt,line join=round,line cap=round,fill=fillColor] (451.80,474.42) circle (  1.96);
\path[draw=drawColor,line width= 0.4pt,line join=round,line cap=round,fill=fillColor] (452.06,474.44) circle (  1.96);
\path[draw=drawColor,line width= 0.4pt,line join=round,line cap=round,fill=fillColor] (452.32,474.46) circle (  1.96);
\path[draw=drawColor,line width= 0.4pt,line join=round,line cap=round,fill=fillColor] (452.58,474.47) circle (  1.96);
\path[draw=drawColor,line width= 0.4pt,line join=round,line cap=round,fill=fillColor] (452.84,474.49) circle (  1.96);
\path[draw=drawColor,line width= 0.4pt,line join=round,line cap=round,fill=fillColor] (453.10,474.52) circle (  1.96);
\path[draw=drawColor,line width= 0.4pt,line join=round,line cap=round,fill=fillColor] (453.36,474.54) circle (  1.96);
\path[draw=drawColor,line width= 0.4pt,line join=round,line cap=round,fill=fillColor] (453.62,474.55) circle (  1.96);
\path[draw=drawColor,line width= 0.4pt,line join=round,line cap=round,fill=fillColor] (453.88,474.57) circle (  1.96);
\path[draw=drawColor,line width= 0.4pt,line join=round,line cap=round,fill=fillColor] (454.14,474.59) circle (  1.96);
\path[draw=drawColor,line width= 0.4pt,line join=round,line cap=round,fill=fillColor] (454.40,474.60) circle (  1.96);
\path[draw=drawColor,line width= 0.4pt,line join=round,line cap=round,fill=fillColor] (454.66,474.62) circle (  1.96);
\path[draw=drawColor,line width= 0.4pt,line join=round,line cap=round,fill=fillColor] (454.92,474.64) circle (  1.96);
\path[draw=drawColor,line width= 0.4pt,line join=round,line cap=round,fill=fillColor] (455.18,474.65) circle (  1.96);
\path[draw=drawColor,line width= 0.4pt,line join=round,line cap=round,fill=fillColor] (455.44,474.68) circle (  1.96);
\path[draw=drawColor,line width= 0.4pt,line join=round,line cap=round,fill=fillColor] (455.70,474.70) circle (  1.96);
\path[draw=drawColor,line width= 0.4pt,line join=round,line cap=round,fill=fillColor] (455.96,474.71) circle (  1.96);
\path[draw=drawColor,line width= 0.4pt,line join=round,line cap=round,fill=fillColor] (456.22,474.73) circle (  1.96);
\path[draw=drawColor,line width= 0.4pt,line join=round,line cap=round,fill=fillColor] (456.48,474.75) circle (  1.96);
\path[draw=drawColor,line width= 0.4pt,line join=round,line cap=round,fill=fillColor] (456.74,474.76) circle (  1.96);
\path[draw=drawColor,line width= 0.4pt,line join=round,line cap=round,fill=fillColor] (457.00,474.78) circle (  1.96);
\path[draw=drawColor,line width= 0.4pt,line join=round,line cap=round,fill=fillColor] (457.26,474.80) circle (  1.96);
\path[draw=drawColor,line width= 0.4pt,line join=round,line cap=round,fill=fillColor] (457.52,474.81) circle (  1.96);
\path[draw=drawColor,line width= 0.4pt,line join=round,line cap=round,fill=fillColor] (457.78,474.84) circle (  1.96);
\path[draw=drawColor,line width= 0.4pt,line join=round,line cap=round,fill=fillColor] (458.04,474.86) circle (  1.96);
\path[draw=drawColor,line width= 0.4pt,line join=round,line cap=round,fill=fillColor] (458.30,474.87) circle (  1.96);
\path[draw=drawColor,line width= 0.4pt,line join=round,line cap=round,fill=fillColor] (458.56,474.89) circle (  1.96);
\path[draw=drawColor,line width= 0.4pt,line join=round,line cap=round,fill=fillColor] (458.82,474.90) circle (  1.96);
\path[draw=drawColor,line width= 0.4pt,line join=round,line cap=round,fill=fillColor] (459.07,474.92) circle (  1.96);
\path[draw=drawColor,line width= 0.4pt,line join=round,line cap=round,fill=fillColor] (459.33,474.94) circle (  1.96);
\path[draw=drawColor,line width= 0.4pt,line join=round,line cap=round,fill=fillColor] (459.59,474.95) circle (  1.96);
\path[draw=drawColor,line width= 0.4pt,line join=round,line cap=round,fill=fillColor] (459.85,474.97) circle (  1.96);
\path[draw=drawColor,line width= 0.4pt,line join=round,line cap=round,fill=fillColor] (460.11,474.98) circle (  1.96);
\path[draw=drawColor,line width= 0.4pt,line join=round,line cap=round,fill=fillColor] (460.37,475.01) circle (  1.96);
\path[draw=drawColor,line width= 0.4pt,line join=round,line cap=round,fill=fillColor] (460.63,475.02) circle (  1.96);
\path[draw=drawColor,line width= 0.4pt,line join=round,line cap=round,fill=fillColor] (460.89,475.04) circle (  1.96);
\path[draw=drawColor,line width= 0.4pt,line join=round,line cap=round,fill=fillColor] (461.15,475.05) circle (  1.96);
\path[draw=drawColor,line width= 0.4pt,line join=round,line cap=round,fill=fillColor] (461.41,475.07) circle (  1.96);
\path[draw=drawColor,line width= 0.4pt,line join=round,line cap=round,fill=fillColor] (461.67,475.08) circle (  1.96);
\path[draw=drawColor,line width= 0.4pt,line join=round,line cap=round,fill=fillColor] (461.93,475.10) circle (  1.96);
\path[draw=drawColor,line width= 0.4pt,line join=round,line cap=round,fill=fillColor] (462.19,475.12) circle (  1.96);
\path[draw=drawColor,line width= 0.4pt,line join=round,line cap=round,fill=fillColor] (462.45,475.13) circle (  1.96);
\path[draw=drawColor,line width= 0.4pt,line join=round,line cap=round,fill=fillColor] (462.71,475.15) circle (  1.96);
\path[draw=drawColor,line width= 0.4pt,line join=round,line cap=round,fill=fillColor] (462.97,475.17) circle (  1.96);
\path[draw=drawColor,line width= 0.4pt,line join=round,line cap=round,fill=fillColor] (463.23,475.19) circle (  1.96);
\path[draw=drawColor,line width= 0.4pt,line join=round,line cap=round,fill=fillColor] (463.49,475.20) circle (  1.96);
\path[draw=drawColor,line width= 0.4pt,line join=round,line cap=round,fill=fillColor] (463.75,475.22) circle (  1.96);
\path[draw=drawColor,line width= 0.4pt,line join=round,line cap=round,fill=fillColor] (464.01,475.23) circle (  1.96);
\path[draw=drawColor,line width= 0.4pt,line join=round,line cap=round,fill=fillColor] (464.27,475.24) circle (  1.96);
\path[draw=drawColor,line width= 0.4pt,line join=round,line cap=round,fill=fillColor] (464.53,475.26) circle (  1.96);
\path[draw=drawColor,line width= 0.4pt,line join=round,line cap=round,fill=fillColor] (464.79,475.27) circle (  1.96);
\path[draw=drawColor,line width= 0.4pt,line join=round,line cap=round,fill=fillColor] (465.05,475.29) circle (  1.96);
\path[draw=drawColor,line width= 0.4pt,line join=round,line cap=round,fill=fillColor] (465.31,475.30) circle (  1.96);
\path[draw=drawColor,line width= 0.4pt,line join=round,line cap=round,fill=fillColor] (465.57,475.33) circle (  1.96);
\path[draw=drawColor,line width= 0.4pt,line join=round,line cap=round,fill=fillColor] (465.83,475.34) circle (  1.96);
\path[draw=drawColor,line width= 0.4pt,line join=round,line cap=round,fill=fillColor] (466.09,475.36) circle (  1.96);
\path[draw=drawColor,line width= 0.4pt,line join=round,line cap=round,fill=fillColor] (466.35,475.37) circle (  1.96);
\path[draw=drawColor,line width= 0.4pt,line join=round,line cap=round,fill=fillColor] (466.61,475.38) circle (  1.96);
\path[draw=drawColor,line width= 0.4pt,line join=round,line cap=round,fill=fillColor] (466.87,475.40) circle (  1.96);
\path[draw=drawColor,line width= 0.4pt,line join=round,line cap=round,fill=fillColor] (467.13,475.41) circle (  1.96);
\path[draw=drawColor,line width= 0.4pt,line join=round,line cap=round,fill=fillColor] (467.39,475.43) circle (  1.96);
\path[draw=drawColor,line width= 0.4pt,line join=round,line cap=round,fill=fillColor] (467.65,475.44) circle (  1.96);
\path[draw=drawColor,line width= 0.4pt,line join=round,line cap=round,fill=fillColor] (467.91,475.45) circle (  1.96);
\path[draw=drawColor,line width= 0.4pt,line join=round,line cap=round,fill=fillColor] (468.17,475.47) circle (  1.96);
\path[draw=drawColor,line width= 0.4pt,line join=round,line cap=round,fill=fillColor] (468.43,475.48) circle (  1.96);
\path[draw=drawColor,line width= 0.4pt,line join=round,line cap=round,fill=fillColor] (468.69,475.51) circle (  1.96);
\path[draw=drawColor,line width= 0.4pt,line join=round,line cap=round,fill=fillColor] (468.95,475.52) circle (  1.96);
\path[draw=drawColor,line width= 0.4pt,line join=round,line cap=round,fill=fillColor] (469.21,475.53) circle (  1.96);
\path[draw=drawColor,line width= 0.4pt,line join=round,line cap=round,fill=fillColor] (469.47,475.55) circle (  1.96);
\path[draw=drawColor,line width= 0.4pt,line join=round,line cap=round,fill=fillColor] (469.73,475.56) circle (  1.96);
\path[draw=drawColor,line width= 0.4pt,line join=round,line cap=round,fill=fillColor] (469.99,475.57) circle (  1.96);
\path[draw=drawColor,line width= 0.4pt,line join=round,line cap=round,fill=fillColor] (470.25,475.59) circle (  1.96);
\path[draw=drawColor,line width= 0.4pt,line join=round,line cap=round,fill=fillColor] (470.51,475.60) circle (  1.96);
\path[draw=drawColor,line width= 0.4pt,line join=round,line cap=round,fill=fillColor] (470.77,475.61) circle (  1.96);
\path[draw=drawColor,line width= 0.4pt,line join=round,line cap=round,fill=fillColor] (471.03,475.63) circle (  1.96);
\path[draw=drawColor,line width= 0.4pt,line join=round,line cap=round,fill=fillColor] (471.29,475.64) circle (  1.96);
\path[draw=drawColor,line width= 0.4pt,line join=round,line cap=round,fill=fillColor] (471.55,475.66) circle (  1.96);
\path[draw=drawColor,line width= 0.4pt,line join=round,line cap=round,fill=fillColor] (471.81,475.68) circle (  1.96);
\path[draw=drawColor,line width= 0.4pt,line join=round,line cap=round,fill=fillColor] (472.07,475.69) circle (  1.96);
\path[draw=drawColor,line width= 0.4pt,line join=round,line cap=round,fill=fillColor] (472.33,475.70) circle (  1.96);
\path[draw=drawColor,line width= 0.4pt,line join=round,line cap=round,fill=fillColor] (472.59,475.72) circle (  1.96);
\path[draw=drawColor,line width= 0.4pt,line join=round,line cap=round,fill=fillColor] (472.85,475.73) circle (  1.96);
\path[draw=drawColor,line width= 0.4pt,line join=round,line cap=round,fill=fillColor] (473.11,475.74) circle (  1.96);
\path[draw=drawColor,line width= 0.4pt,line join=round,line cap=round,fill=fillColor] (473.37,475.76) circle (  1.96);
\path[draw=drawColor,line width= 0.4pt,line join=round,line cap=round,fill=fillColor] (473.63,475.77) circle (  1.96);
\path[draw=drawColor,line width= 0.4pt,line join=round,line cap=round,fill=fillColor] (473.89,475.78) circle (  1.96);
\path[draw=drawColor,line width= 0.4pt,line join=round,line cap=round,fill=fillColor] (474.15,475.79) circle (  1.96);
\path[draw=drawColor,line width= 0.4pt,line join=round,line cap=round,fill=fillColor] (474.41,475.81) circle (  1.96);
\path[draw=drawColor,line width= 0.4pt,line join=round,line cap=round,fill=fillColor] (474.67,475.83) circle (  1.96);
\path[draw=drawColor,line width= 0.4pt,line join=round,line cap=round,fill=fillColor] (474.93,475.84) circle (  1.96);
\path[draw=drawColor,line width= 0.4pt,line join=round,line cap=round,fill=fillColor] (475.19,475.85) circle (  1.96);
\path[draw=drawColor,line width= 0.4pt,line join=round,line cap=round,fill=fillColor] (475.45,475.87) circle (  1.96);
\path[draw=drawColor,line width= 0.4pt,line join=round,line cap=round,fill=fillColor] (475.70,475.88) circle (  1.96);
\path[draw=drawColor,line width= 0.4pt,line join=round,line cap=round,fill=fillColor] (475.96,475.89) circle (  1.96);
\path[draw=drawColor,line width= 0.4pt,line join=round,line cap=round,fill=fillColor] (476.22,475.90) circle (  1.96);
\path[draw=drawColor,line width= 0.4pt,line join=round,line cap=round,fill=fillColor] (476.48,475.92) circle (  1.96);
\path[draw=drawColor,line width= 0.4pt,line join=round,line cap=round,fill=fillColor] (476.74,475.93) circle (  1.96);
\path[draw=drawColor,line width= 0.4pt,line join=round,line cap=round,fill=fillColor] (477.00,475.94) circle (  1.96);
\path[draw=drawColor,line width= 0.4pt,line join=round,line cap=round,fill=fillColor] (477.26,475.95) circle (  1.96);
\path[draw=drawColor,line width= 0.4pt,line join=round,line cap=round,fill=fillColor] (477.52,475.97) circle (  1.96);
\path[draw=drawColor,line width= 0.4pt,line join=round,line cap=round,fill=fillColor] (477.78,475.98) circle (  1.96);
\path[draw=drawColor,line width= 0.4pt,line join=round,line cap=round,fill=fillColor] (478.04,476.00) circle (  1.96);
\path[draw=drawColor,line width= 0.4pt,line join=round,line cap=round,fill=fillColor] (478.30,476.01) circle (  1.96);
\path[draw=drawColor,line width= 0.4pt,line join=round,line cap=round,fill=fillColor] (478.56,476.02) circle (  1.96);
\path[draw=drawColor,line width= 0.4pt,line join=round,line cap=round,fill=fillColor] (478.82,476.03) circle (  1.96);
\path[draw=drawColor,line width= 0.4pt,line join=round,line cap=round,fill=fillColor] (479.08,476.05) circle (  1.96);
\path[draw=drawColor,line width= 0.4pt,line join=round,line cap=round,fill=fillColor] (479.34,476.06) circle (  1.96);
\path[draw=drawColor,line width= 0.4pt,line join=round,line cap=round,fill=fillColor] (479.60,476.07) circle (  1.96);
\end{scope}
\begin{scope}
\path[clip] (  0.00,  0.00) rectangle (505.89,505.89);
\definecolor{drawColor}{gray}{0.30}
\node[text=drawColor,anchor=base east,inner sep=0pt, outer sep=0pt, scale=  0.88] at ( 38.12, 50.34) {5632};
\node[text=drawColor,anchor=base east,inner sep=0pt, outer sep=0pt, scale=  0.88] at ( 38.12, 92.92) {5888};
\node[text=drawColor,anchor=base east,inner sep=0pt, outer sep=0pt, scale=  0.88] at ( 38.12,135.49) {6144};
\node[text=drawColor,anchor=base east,inner sep=0pt, outer sep=0pt, scale=  0.88] at ( 38.12,178.06) {6400};
\node[text=drawColor,anchor=base east,inner sep=0pt, outer sep=0pt, scale=  0.88] at ( 38.12,220.63) {6656};
\node[text=drawColor,anchor=base east,inner sep=0pt, outer sep=0pt, scale=  0.88] at ( 38.12,263.21) {6912};
\node[text=drawColor,anchor=base east,inner sep=0pt, outer sep=0pt, scale=  0.88] at ( 38.12,305.78) {7168};
\node[text=drawColor,anchor=base east,inner sep=0pt, outer sep=0pt, scale=  0.88] at ( 38.12,348.35) {7424};
\node[text=drawColor,anchor=base east,inner sep=0pt, outer sep=0pt, scale=  0.88] at ( 38.12,390.93) {7680};
\node[text=drawColor,anchor=base east,inner sep=0pt, outer sep=0pt, scale=  0.88] at ( 38.12,433.50) {7936};
\node[text=drawColor,anchor=base east,inner sep=0pt, outer sep=0pt, scale=  0.88] at ( 38.12,476.07) {8192};
\node[text=drawColor,anchor=base east,inner sep=0pt, outer sep=0pt, scale=  0.88] at ( 38.12,476.07) {8192};
\end{scope}
\begin{scope}
\path[clip] (  0.00,  0.00) rectangle (505.89,505.89);
\definecolor{drawColor}{gray}{0.20}
\path[draw=drawColor,line width= 0.6pt,line join=round] ( 40.32, 53.37) --
	( 43.07, 53.37);
\path[draw=drawColor,line width= 0.6pt,line join=round] ( 40.32, 95.95) --
	( 43.07, 95.95);
\path[draw=drawColor,line width= 0.6pt,line join=round] ( 40.32,138.52) --
	( 43.07,138.52);
\path[draw=drawColor,line width= 0.6pt,line join=round] ( 40.32,181.09) --
	( 43.07,181.09);
\path[draw=drawColor,line width= 0.6pt,line join=round] ( 40.32,223.66) --
	( 43.07,223.66);
\path[draw=drawColor,line width= 0.6pt,line join=round] ( 40.32,266.24) --
	( 43.07,266.24);
\path[draw=drawColor,line width= 0.6pt,line join=round] ( 40.32,308.81) --
	( 43.07,308.81);
\path[draw=drawColor,line width= 0.6pt,line join=round] ( 40.32,351.38) --
	( 43.07,351.38);
\path[draw=drawColor,line width= 0.6pt,line join=round] ( 40.32,393.96) --
	( 43.07,393.96);
\path[draw=drawColor,line width= 0.6pt,line join=round] ( 40.32,436.53) --
	( 43.07,436.53);
\path[draw=drawColor,line width= 0.6pt,line join=round] ( 40.32,479.10) --
	( 43.07,479.10);
\path[draw=drawColor,line width= 0.6pt,line join=round] ( 40.32,479.10) --
	( 43.07,479.10);
\end{scope}
\begin{scope}
\path[clip] (  0.00,  0.00) rectangle (505.89,505.89);
\definecolor{drawColor}{gray}{0.20}
\path[draw=drawColor,line width= 0.6pt,line join=round] ( 63.85, 29.34) --
	( 63.85, 32.09);
\path[draw=drawColor,line width= 0.6pt,line join=round] (115.82, 29.34) --
	(115.82, 32.09);
\path[draw=drawColor,line width= 0.6pt,line join=round] (167.79, 29.34) --
	(167.79, 32.09);
\path[draw=drawColor,line width= 0.6pt,line join=round] (219.76, 29.34) --
	(219.76, 32.09);
\path[draw=drawColor,line width= 0.6pt,line join=round] (271.73, 29.34) --
	(271.73, 32.09);
\path[draw=drawColor,line width= 0.6pt,line join=round] (323.70, 29.34) --
	(323.70, 32.09);
\path[draw=drawColor,line width= 0.6pt,line join=round] (375.67, 29.34) --
	(375.67, 32.09);
\path[draw=drawColor,line width= 0.6pt,line join=round] (427.63, 29.34) --
	(427.63, 32.09);
\path[draw=drawColor,line width= 0.6pt,line join=round] (479.60, 29.34) --
	(479.60, 32.09);
\end{scope}
\begin{scope}
\path[clip] (  0.00,  0.00) rectangle (505.89,505.89);
\definecolor{drawColor}{gray}{0.30}
\node[text=drawColor,anchor=base,inner sep=0pt, outer sep=0pt, scale=  0.88] at ( 63.85, 21.08) {2200};
\node[text=drawColor,anchor=base,inner sep=0pt, outer sep=0pt, scale=  0.88] at (115.82, 21.08) {2400};
\node[text=drawColor,anchor=base,inner sep=0pt, outer sep=0pt, scale=  0.88] at (167.79, 21.08) {2600};
\node[text=drawColor,anchor=base,inner sep=0pt, outer sep=0pt, scale=  0.88] at (219.76, 21.08) {2800};
\node[text=drawColor,anchor=base,inner sep=0pt, outer sep=0pt, scale=  0.88] at (271.73, 21.08) {3000};
\node[text=drawColor,anchor=base,inner sep=0pt, outer sep=0pt, scale=  0.88] at (323.70, 21.08) {3200};
\node[text=drawColor,anchor=base,inner sep=0pt, outer sep=0pt, scale=  0.88] at (375.67, 21.08) {3400};
\node[text=drawColor,anchor=base,inner sep=0pt, outer sep=0pt, scale=  0.88] at (427.63, 21.08) {3600};
\node[text=drawColor,anchor=base,inner sep=0pt, outer sep=0pt, scale=  0.88] at (479.60, 21.08) {3800};
\end{scope}
\begin{scope}
\path[clip] (  0.00,  0.00) rectangle (505.89,505.89);
\definecolor{drawColor}{RGB}{0,0,0}
\node[text=drawColor,anchor=base,inner sep=0pt, outer sep=0pt, scale=  1.10] at (271.73,  8.00) {Anzahl an Adressen im Pool (x)};
\end{scope}
\begin{scope}
\path[clip] (  0.00,  0.00) rectangle (505.89,505.89);
\definecolor{drawColor}{RGB}{0,0,0}
\node[text=drawColor,rotate= 90.00,anchor=base,inner sep=0pt, outer sep=0pt, scale=  1.10] at ( 13.08,266.24) {Erwartungswert der gefundenen Eviction Sets};
\end{scope}
\end{tikzpicture}

\caption{Beschreibt das Verhältnis der Poolgröße zu der Größe des Erwartungswertes der gefundenen Eviction Sets}
\end{scaletikzpicturetowidth}
\end{center}

TODO verschiedene reale bechmarks einfügen

\subsection{Eviction-Set Suchalgorithmus}

Im Folgenden soll der Algorithmus beschrieben werden, welcher die Adressen im Pool verschiedenen Eviction-Sets zuordnet.
Der Algorithmus ist in der Lage Adressen Cache-Sets zuzuordnen ohne näheres über die CPU (L3-Cache Größe usw.) und die Adressbits 12 bis 63 zu wissen.
Die Idee des Algorithmus ist nicht neu, sondern wurde etwa schon in \cite{DriveByPaper} oder \cite{PrimeAndAbort} in teilweise abgewandelter Form beschrieben.

Der Eviction-Set-Konstruktionsalgorithmus besteht aus 3 Hauptpahsen, der Expand, Contract und Collect-Phase. 
Zu Anfang wird zufällig eine Zeugenadresse aus dem Adresspool ausgewählt, für welche im Folgenden ein Eviction-Set gefunden werden soll.
Die Annahme in der Expand-Phase ist, dass eine bestimmte Teilmenge der Adressen aus dem Pool, genannt Candidate-Set, ein Eviction-Set für die Zeugenadresse bildet, sofern der Pool groß genug ist.
Um ein Candidate-Set zu testen wird zuerst auf die Zeugenadresse zugegriffen, um sicherzustellen, dass diese im Cache landet.
Danach wird auf alle Adressen aus dem Candidate-Set zugegriffen und abschließend die Zugriffszeit auf Zeugenadresse gemessen. 
Sofern das Candidate-Set ein Eviction-Set für die Zeugenadresse ist, werden die Daten der Zeugenadresse aus dem Cache verdrängt, welches bei einer erneuten Messung der Zeugenadresse mit einer erhöhten Zugriffszeit einhergeht.
Dieser Vorgang wird mehrmals wiederholt, um den Einfluss des Timer- und System-Rauschens zu vermindern.

In der Expand-Phase wird dem Addresspool iterativ eine zufällige Adresse entnommen und dem anfangs leeren Candidate-Set hinzugefügt (siehe auch Pseudocode \ref{alg:evictionSetExpand}). 
Nach jeder Iteration wird das Candidate-Set auf die eben beschriebene Weise getestet.
Falls das Candidate-Set ein Eviction-Set für die Zeugenadresse ist, wird zur nächsten Phase übergegangen, andernfalls die Iteration fortgesetzt.

\begin{algorithm}[h]
\DontPrintSemicolon
\caption{Psuedo-Code für Expand-Phase des Eviction-Set Algorithmus}
\label{alg:evictionSetExpand}

\Fn{$Expand(evictionSet, memoryBlocks)$}{
	\While{size(candidateSet) > 0}{
		witnesss = SelectRandomItem(candidateSet)\;
		\If{checkevict(evictionSet, witnesss)}{
			\Return witnesss
		}
		evictionSet.add(witnesss)\;
	}
	\Return failed;
}
\end{algorithm}

Im Allgemeinen beinhaltet das Candidate-Set nach der Expand-Phase mehrere hundert Einträge von denen eine Teilmenge der Größe 16 ebenfalls ein Eviction-Set für die Zeugenadresse bilden würde.
Diese überflüssigen Einträge würden den Prime and Probe Vorgang erheblich verlangsamen, weswegen in der Contract-Phase versucht wird das Candidate-Set auf die Größe 16 einzudampfen (siehe auch Pseudocode \ref{alg:evictionSetContract}).
Hierzu wird ein Element aus dem Candidate-Set entfernt und erneut getestet ob diese reduzierte Candidate-Set noch ein Eviction-Set für die Zeugenadresse ist.
Falls ja wird dieses Element wieder da Adresspool hinzugefügt, andernfalls verbleibt es im Candidate-Set da es  notwendiger Bestandteil des Eviction-Sets ist.
Dieser Vorgang wird für jedes Element im Candidate-Set einmal durchgeführt, sodass im fehlerfreien Fall schlussendlich 16 Elemente im Candidate-Set verbleiben.

\begin{algorithm}[h]
\DontPrintSemicolon
\caption{Psuedo-Code für Contract-Phase des Eviction-Set Algorithmus}
\label{alg:evictionSetContract}
\Fn{$Contract(evictionSet, memoryBlocks, witness)$}{
	\ForEach{candidate in evictionSet}{
		\If{checkevict(evictionSet, witness)}{
			mermoryBlocks.add(candidate)\;
			evictionSet.add(candidate)\;	
		}		
	}
}
\end{algorithm}

Wenn im Anschluss sofort wieder eine neue Zeugenadresse aus dem Pool gewählt würde, dann könnte eine Adresse gewählt werden, welche auf das selbe Cache-Set wie die vorherige Zeugenadresse abgebildet werden würde.
Deshalb folgt im Anschluss an eine erfolgreiche Contract-Phase die Collect-Phase.
In dieser werden alle Adressen aus dem Pool entfernt, welche ebenfalls von dem in der Contract-Phase gefundenen Eviction-Set aus dem Cache verdrängt werden (siehe auch Pseudocode \ref{alg:evictionSetCollect}).
Durch diesen Schritt wird also vermieden, dass die spätere Menge von Eviction-Set dahingehend überprüft werden müsste, ob Eviction-Sets paarweise das selbe zugrundeliegende Cache-Set besitzen.
Hierzu wird eine Adresse aus dem Pool durch einen Zugriff in den Cache geladen und anschließend auf alle Einträge im Eviction-Set zugegriffen.
Danach wird die Zugriffszeit auf die Adresse gemessen und bei einer erhöhten Zeit aus dem Pool entfernt, da dann die Adresse auf das selbe Cache-Set wie die Einträge des Eviction-Set bzw. die letzte Zeugenadresse abgebildet wird.

\begin{algorithm}[h]
\DontPrintSemicolon
\caption{Psuedo-Code für Collect-Phase des Eviction-Set Algorithmus}
\label{alg:evictionSetCollect}

\Fn{$Collect(evictionSet, memoryBlocks)$}{
	witnessSet = empty\;
	\ForEach{candidate in mermoryBlocks}{
		\If{checkevict(evictionSet, candidate)}{
			memoryBlocks.delete(candidate)\;
			witnessSet.add(candidate)\;
		}
	}
	\Return witnessSet;
}
\end{algorithm}

Zusammenfassend wird in der Expand-Phase das Candidate-Set soweit vergrößert bis es ein Eviction-Set bildet, dann in der Contract-Phase auf die Größe 16 verkleinert und anschließend in der Collect-Phase alle auf das selbe Cache-Set abbildende Adressen aus dem Pool entfernt (siehe auch Pseudocode \ref{alg:evictionSetOverview}).
Das gefundene Eviction-Set wird gespeichert und der Vorgang solange wiederholt bis die Anzahl der Pooladressen kleiner als 16 ist oder aufgrund von Fehlern in einer Phase mehrmals kein Eviction-Set gefunden wurde.

\begin{algorithm}[h]
\DontPrintSemicolon
\caption{Psuedo-Code für Eviction-Set Algorithmus}
\label{alg:evictionSetOverview}

\Fn{$EvictionSetFinder(memoryBlocks)$}{
    groups $\leftarrow$ empty\;
    \While{size(memoryBlocks > 0}{
        evictionSet $\leftarrow$ empty\;
		witness $\leftarrow$ expand(evictionSet, memoryBlocks)\;
		
		\If{witness != failed}{
    		contract(evictionSet, memoryBlocks, witness)\;
    		witnessSet $\leftarrow$ collect(evictionSet, memoryBlocks, witnessSet)\;
    		groups.add(union(evictionSet, witness, witnessSet))\;
		}
    }
}
\end{algorithm}

\subsection{Optimierung der Phasen}

\subsection{Details der realen Implementierung}

Wie weiter oben beschrieben werden insbesondere der Test ob ein Set ein Eviction-Set für bestimmte
Adresse darstellt mehrfach wiederholt um Fehler auszuschließen. Problematisch ist dies vor allem in der Expand-Phase, da das Candidate-Set eine größere von mehreren hundert Einträgen annimmt und nach jeder Iteration gegen die Zeugenadresse getestet wird. 
Eine einzige erhöhte Zugriffszeitmessung würde das Candidate-Set fälschlicherweise als Eviction-Set für die Zeugenadresse einordnen. 
Es wurde festgestellt, dass erhöhte Zugriffszeiten mehrmals hintereinander auftreten können, deshalb wird ein erhöhte Messung in der Expand-Phase 20 mal erneut überprüft. 
Dabei bricht die Überprüfung ab sobald eine der Messungen eine widersprüchliche Aussage zulässt.

Trotz der hohen Anzahl der Wiederholungen sind die Kosten hierfür gering, da im fehlerfreien Optimalfall nur zusätzlich 20 Prüfungen anfallen und in den meisten Fehlerfällen nur einzelne zusätzliche Überprüfungen durchgeführt werden. 
Dem gegenüber steht Vorteil nicht fälschlicherweise in die Contract-Phase zu wechseln und dort erst spät zu merken, dass das Candidate-Set kein Eviction-Set ist.
Wenn nun das Candidate-Set kein Eviction-Set für die Zeugenadresse ist, würde in der naiven Implementierung der Contract-Phase über alle Einträge des Candidate-Set iteriert werden, aber kein Einträgt je gelöscht.
Denn die Zugriffszeit auf die Zeugenadresse wird immer niedrig bleiben, was den Algorithmus zur fälschlichen Annahme verleitet, alle Einträge seien für das Eviction-Set notwendig.
Wir wissen jedoch, dass die Größe des Eviction-Set der Assoziativität des L3-Caches entsprechen muss und können die Contract-Phase abbrechen sobald mehr als der Assoziativität entsprechenden Anzahl von Einträgen des Candidate-Set als notwendig eingestuft wurden, da dann ein Fehler vorliegt.


\newtextend

TODO Algorithmus in Einzelheiten beschreiben
\todo[size=\footnotesize]{Beschreibe den Algorithmus bitte in einzelteilen, und gib auch den code dafür einzeln an. Der komplette Code gehört in den Anhang.}\todo[size=\footnotesize]{Nutze eine kleinere Schriftart für listings}


\section{Verdeckter Kanal}

Die maximale Sendegeschwindigkeit eines Kanals ist durch die Rate, mit welcher der Sender ein beliebiges Cache-Set primen kann, begrenzt.
Damit der Empfänger ein zufälliges Rauschen von einem Priming unterscheiden kann, sollte der Sender mehrere Einträge aus dem zu primenden Cache-Set verdrängen, wobei im Optimalfall die Anzahl der zugegriffenen Speicheradressen der Assoziativität des Caches entspricht.
Hiermit wird die Wahrscheinlichkeit erhöht, dass sich die vom Empfänger im Probe-Schritt gemessene Zugriffszeit signifikant von Fällen unterscheidet, in denen zufällig einzelne Einträge aus dem überwachten Cache-Set verdrängt werden. 
%Daraus folgernd nehmen wir an, dass der Sender in seiner Priming-Phase auf der Assoziativität entsprechend viele Speicheradressen zugreift.
Im Folgenden sollen verschiedene Methoden des Primens eines Cache-Sets verglichen werden, indem entweder die Anzahl der zugegriffen Speicheradressen oder die Zugriffsmethode verändert werden.
Wenn etwa die Zahl der zugegriffenen Speicheraddressen verringert wird, sind auf der einen Seite mehr Timeslots in einem Zeitabschnitt möglich, und die Chance sinkt, dass benachbarte Timeslots zusätzlich beeinflusst werden. Auf der anderen Seite sind die messbaren Ausschläge der Zugriffszeiten verringert, wodurch ein bewusst geprimtes Cache-Set schwieriger von einem Messrauschen oder von zufälligen Zugriffen unterschieden werden kann.
Sende- und Empfangsseite können durchaus abweichende Parameter verwenden, wenn wie etwa im vorliegenden Fall der Empfänger langsamer als der Sender arbeitet. Um die Timeslots anzugleichen, könnte der Empfänger die Dauer einer Priming-Operation durch die Senkung der Anzahl der zugegriffenen Speicheraddressen verringern und der Empfänger andersherum die Dauer für eine Priming-Operation erhöhen. 

\begin{algorithm}[h]
\DontPrintSemicolon
\caption{Psuedo-Code für Pointer-Chasing-Methode}
\label{alg:pointerChasing}

\Fn{$AccessTimeEvictionSet(pointerToAddress)$}{
    pointerToAddressFirst $\leftarrow$ pointerToAddress\;
    timestampBefore $\leftarrow$ getTimeStamp()\;
    \While{pointerToAddressFirst != pointerToAddress}{
        pointerToAddress $\leftarrow$ readValue(pointerToAddress)\;
    }
    \Return getTimeStamp() - timestampBefore
}

\end{algorithm}

Auf dem Testrechner benötigt ein in C geschriebenes Sendeprogramm für eine Million Prime-Vorgänge mit 16 Adressen und der Single-Pointer-Chasing-Methode (siehe Algorithmus \ref{alg:pointerChasing}) etwa 323 Millionen Taktzyklen.
Im Optimalfall kann im Timeslot $x$ ein erfolgter Prime-Vorgang als 1 und ein nicht erfolgter Prime-Vorgang als 0 interpretiert werden.
Bei einem typischen All-Core-Turbo-Takt von 3,4 Ghz des i7-4770 ergibt sich so eine maximale Senderate von TODO ~10,5 Mbit/s.
Diese Rate wird jedoch vom Empfänger beschränkt, welcher zusätzlich noch eine Zeitmessung durchführen muss. Der Worst-Case ist hier ein in Webassembly geschriebener Empfangsroutine, da dort eine Zeitmessung kostenintensiver ist. In Chromium 66 können eine Million Messungen eines Cache-Sets in etwa 200 ms durchgeführt werden. Im Mittel dauert eine Messung also 0,2 \textmu s, womit eine Empfangsrate von maximal ~5 MBit/s realisiert werden kann.

%Um die Performance zu erhöhen kann wie erwähnt die Anzahl der zugegriffen Speicheradressen reduziert werden. 
Im Folgenden soll die maximal mögliche Senderate unter optimalen Bedingungen ermittelt werden. 
Hierfür wird im Voraus ein Cache-Set ausgewählt, auf dem im Idle-Zustand des Systems ein geringes Rauschen herrscht.
Um die Synchronisation des Senders und Empfängers aufrechtzuerhalten, wird nach 10 gesendeten Bits ein Synchronisationsblock eingefügt, welcher durch $sb$-Prime-Vorgänge auf der Senderseite erzeugt wird. 
Eine 1 wird durch $s$-Prime-Vorgänge repräsentiert und eine 0 durch das Unterlassen der Prime-Vorgänge. 
Um die einzelnen Bits auseinanderzuhalten, wird zwischen jedem gesendeten Bit eine Pause von $p$-Taktzyklen eingelegt.

%Um einen Kanal zu Initialisieren 


%TODO: benchmark cache set finder
%entwickle bessere benchmark prozedur, messe zeit für contract jedes es und mittle dann


%Problem: v8 compiliert lazy, d.h. nur häufig verwendete

\newtext


\subsection{Angriffe auf RSA Key Generierung}

Details zu Implementierung in Mozilla NSS

Mozilla Network Security Services(NSS) ist ein Menge von Bibliotheken welche eine plattformübergreifende Entwicklung von sicheren Client- und Server-Anwendungen anstrebt. Dabei wird unterem anderem TLS oder S/MIME implementiert. Mozilla NSS wird etwa im Firefox-Browser und der Mail-Anwendung Thunderbird eingesetzt.
Der Quellcode ist unter der Mozilla Public License verfügbar und kann online etwa im Firefox-Repository \cite{MozillaDXR} eingesehen werden.

Im folgenden soll die Schlüsselerzuegung für das RSA-Verfahren in Mozilla NSS beschrieben werden.

Der Code zur Schlüsselerzeugung liegt im Unterordner lib/freebl. Sobald die Schlüsselparameter $p,q,n,d,e$ bestimmt wurden, werden diese in der Funktion RSA_PrivateKeyCheck auf Gültigkeit überprüft (siehe Pseudo-Code \ref{alg:RSA_PrivateKeyCheck}).

\begin{algorithm}[h]
\DontPrintSemicolon
\caption{Pseudo-Code für RSA_PrivateKeyCheck aus rsa.c}
\label{alg:RSA_PrivateKeyCheck}

\Fn{$RSA_PrivateKeyCheck(key)$}{
    assert(p $\neq$ q)\;
    assert(n == p * q)\;
    assert(gcd(e, p-1) == 1)\;
    assert(gcd(e, q-1) == 1)\;
    assert(d*e == 1 mod p-1)\;
    assert(d*e == 1 mod q-1)\;
    assert(d_p == d mod p-1)\;
    assert(d_q == d mod q-1)\;
    assert(q * q^-1 == 1 mod p)\;
}
\end{algorithm}

%Ausgehend von der Funktion RSA_NewKey in rsa.c

%\begin{algorithm}[h]
%\DontPrintSemicolon
%\caption{Pseudo-Code für RSA_NewKey aus rsa.c}
%\label{alg:mp_gcd}
%
%\Fn{$RSA_NewKey(keySizeInBits, e)$}{
%    p $rightarrow$ generate_prime(keySizeInBits)\;
%    q $rightarrow$ generate_prime(keySizeInBits)\;
%    d $rightarrow$ rsa_build_from_primes(p,q,e)\;
%}
%\end{algorithm}

Relvant für diese Arbeit sind im Wesentlichen Zeile 4 und 5 in denen die Teilerfremdheit von $e$ zu $p-1$ und $q-1$, d.h. $gcd(e,p-1) = 1$ und $gcd(e,q-1) = 1$ geprüft wird.
Aus Performancegründen wird der Exponent $e$, anders als ursprünglich im RSA-Algorithmus beschrieben, auf den Wert 65537 fixiert.
Interessant ist die Funktion $mp_gcd$ (Pseudocode siehe \ref{alg:mp_gcd}), welche den größten gemeinsamen Teiler nachdem binären Verfahren von Josef Strein \cite{} berechnet. Dieser Algorithmus verwendet zum berechnen des ggT ausschließlich Rechts-Shift-Operationen (Teilen durch 2) und Subtraktionen, wodurch dieser besonders für die in diesem Kontext verwendeten großen Zahlen interessant ist.
Die Zeilen 1 bis 6 der Funktion $mp_gcd$ können in diesem Fall ignoriert werden, da der Exponent $e$ wie oben beschrieben immer 65537 und damit ungerade ist. Bedeutender hingegen ist die $while$-Schleife in den Zeilen 11 bis 17, welche abhängig von den Eingaben Fallunterscheidungen durchführt. Das Ziel ist hier die Rechts-Shift-Operation (Zeile 13) von der Subtraktionsfunktion (Zeile 17) zu unterscheiden, um die Zustände während der Berechnung zu rekonstruieren.

\begin{algorithm}[h]
\DontPrintSemicolon
\caption{Psuedo-Code für mp_gcd nach Josef Stein}
\label{alg:mp_gcd}

\Fn{$mp_gcd(u,v)$}{
    k $\leftarrow$ 0\;
    \While{iseven(u) \& iseven(v)}{
        u $\leftarrow$ u/2\;
        v $\leftarrow$ v/2\;
        k++\;
    }

    \If{isodd(u)}{
        t $\leftarrow$ -v\;
    } \Else {
        t $\leftarrow$ v\;
    }

    \While{t $\neq$ 0}{
        \While{iseven(t)}{
            t $\leftarrow$ t/2\;
        }
        \If{t > 0} {
            u $\leftarrow$ t\;
        } else {
            v $\leftarrow$ -t\;
        }
        t $\leftarrow$ u - v\;
    }

    \Return u*2^k
}
\end{algorithm}

\newtextend

%Was wird angegriffen

%Verweis auf Paper Cache-Timing Attacks on RSA Key Generation