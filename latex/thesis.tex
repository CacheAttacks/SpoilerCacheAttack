% =========================================
% == Latex-Template for Theses
% == by René Schönfelder
% == schoenfr@isp.uni-luebeck.de
% =========================================
% 
% This latex project serves as a template for theses in the MINT section of the
% University of Lübeck. The first version was created during the authors
% master's and steadily evolved from there on.
% 
% The template may be used and distributed freely. In the case of questions or
% suggestions just send an email to schoenfr@isp.uni-luebeck.de. This document
% is in english so that international students may also profit from this
% template. However, the content is written in german because the guidelines
% expect german theses and most students are writing in german..

% The document class ``scrbook'' describes books:
\documentclass[
  a4paper,               % a4 paper
  twoside,               % two-sided
  headings=small,        % make headings smaller
  DIV=12,                % divide the paper in 12 parts
  BCOR=8mm,              % 8mm binding correction
  headinclude=true,      % use a header
  footinclude=false,     % do not use a footer
  numbers=noenddot,      % no dot after last number in headings
  11pt]{scrbook}         % font size 11pt

% Encodings
\usepackage[T1]{fontenc}
% Use this for UTF8 input in .tex-files
\usepackage[utf8]{inputenc}
% You can use english alternatively
\usepackage[ngerman]{babel}

% PDF properties
\usepackage{hyperref}
\hypersetup{
  pdfauthor={Moritz Krebbel},
  pdfsubject={Drive-by cache attacks},
  pdftitle={Ausnutzung spekulativer Ausführung durch drive-by Cache-Angriffe},
  pdfkeywords={chache attacks, drive-by attacks, speculative execution}
}

% Color for Corporate Design
\usepackage{xcolor}
\definecolor{oceangreen}{cmyk}{1,.0,.20,.78}

% If you want to color the headings, then you can use the following command. It
% makes most of your pages colored. Please notice, that printing colored pages
% may be a lot more expensive than printing gray scale.
%\addtokomafont{sectioning}{\rmfamily\color{oceangreen}}

% We use the font family ``Palatino'', which does not correspond to the
% guidelines but is still valid. You could also use \usepackage{times}.
\usepackage{palatino}
\fontfamily{ppl}
\selectfont

% General packages
\usepackage{blindtext}
\usepackage{graphicx}               % Include graphics
\usepackage{appendix}               % Use an appendix
\usepackage{textpos}                % Easy text positioning
\usepackage{typearea}               % Standard for defining geometry
\usepackage{tikz}                   % TikZ pictures
\usepackage{amsmath,amssymb,amsthm} % Math
\usepackage{scrhack}
\usepackage{listings}               % Listings
\usepackage[ruled,vlined,linesnumbered]{algorithm2e} % Algorithms
\usepackage{stmaryrd}               % Lightning arrow \lightning
\usepackage{url}                    % Formatting URLs
\usepackage{rotating}               % Rotation of figures
\usepackage{todonotes}              % Make TODO-notes
% This package is used for the following spacing options
\usepackage{setspace}
\usepackage[linguistics]{forest}
\usepackage{breqn}
\usepackage{color}
\usepackage{environ}
\usepackage{minted}

\usepackage[skip=10pt,font={small}]{caption}

%white on black
\usepackage{xcolor}
\usepackage{pagecolor}

%set size of algorithm env
\SetAlFnt{\small}

\makeatletter
\newsavebox{\measure@tikzpicture}
\NewEnviron{scaletikzpicturetowidth}[1]{%
  \def\tikz@width{#1}%
  \def\tikzscale{1}\begin{lrbox}{\measure@tikzpicture}%
  \BODY
  \end{lrbox}%
  \pgfmathparse{#1/\wd\measure@tikzpicture}%
  \edef\tikzscale{\pgfmathresult}%
  \BODY
}
\makeatother

\usepackage{chngcntr}               % Do not reset the counter for footnotes
\counterwithout{footnote}{chapter}  % on every new chapter

\usepackage{eurosym}
\usepackage{newunicodechar}
\newunicodechar{€}{\euro}

\usepackage{textcomp}

\renewcommand{\algorithmcfname}{Algorithmus}

\newcommand{\twodots}{\mathrel{{.}\,{.}}\nobreak}

% Listing style
\lstset{
  basicstyle={\ttfamily \footnotesize},
  language = Java,
  numbers=left, numberstyle=\tiny,
  frame=tb, framerule=1pt,
  xleftmargin=0.2cm, linewidth=1\textwidth,
  %backgroundcolor=\color{bg},
  keywordstyle=\bf\color{oceangreen},
  morecomment=[l][\color{blue}]{\/*, \*/},
  %emph={for,end,while,function,if,else},
  emphstyle={\color{blue}}
}

\definecolor{lightgray}{rgb}{.9,.9,.9}
\definecolor{darkgray}{rgb}{.4,.4,.4}
\definecolor{purple}{rgb}{0.65, 0.12, 0.82}

\lstdefinelanguage{JavaScript}{
  keywords={typeof, new, true, false, catch, function, return, null, catch, switch, var, if, in, while, do, else, case, break},
  keywordstyle=\color{blue}\bfseries,
  ndkeywords={class, export, boolean, throw, implements, import, this},
  ndkeywordstyle=\color{darkgray}\bfseries,
  identifierstyle=\color{black},
  sensitive=false,
  comment=[l]{//},
  morecomment=[s]{/*}{*/},
  commentstyle=\color{purple}\ttfamily,
  stringstyle=\color{red}\ttfamily,
  morestring=[b]',
  morestring=[b]"
}

\lstset{
   language=JavaScript,
   extendedchars=true,
   basicstyle=\footnotesize\ttfamily,
   showstringspaces=false,
   showspaces=false,
   numbers=left,
   numberstyle=\footnotesize,
   numbersep=9pt,
   tabsize=2,
   breaklines=true,
   showtabs=false,
   captionpos=b
}

\newif\ifonline
\IfFileExists{nored.tex}{\onlinefalse}{\onlinetrue}

\newcommand{\newtext}{\ifonline\color{red}\fi}
\newcommand{\newtextend}{\ifonline\color{black}\fi}

\ifonline\pagecolor{black}\fi
\ifonline\color{brown}\fi

% If you want to list your bib as a usual section listed in the table of
% contents, you can use the following commands.
%\makeatletter
%\renewcommand*\bib@heading{%
%  \chapter{\refname}
%  \@mkboth{\refname}{\refname}
%}
%\makeatother

% Prevent warnings ``hbox badness'' in the bibliography
\usepackage{etoolbox}
\apptocmd{\sloppy}{\hbadness 10000\relax}{}{}

% Prevent a lot of ``hbox badness'' warnings
\setlength{\emergencystretch}{1em}

% 1.5 line spacing (the guidelines say at most 1.2, but we decided to allow
% up to 1.5)
\setstretch{1.2}

% No indentation in the beginning of a paragraph
\setlength{\parindent}{0pt}

% The space between two paragraphs
%\setlength{\parskip}{1em}

% More space inside of tables
% \renewcommand*\arraystretch{1.5}

% Do not number subsubsections
\setcounter{secnumdepth}{2}
\setcounter{tocdepth}{2}

% Load macros and hyphenations
% This file contains various useful definitions
% You may also add your own macros here

% Theorem environments
\newtheorem{definition}{Definition}[chapter]
\newtheorem{proposition}[definition]{Proposition}
\newtheorem{theorem}[definition]{Theorem}
\newtheorem{corollary}[definition]{Corollary}
\newtheorem{conjecture}[definition]{Conjecture}
\newtheorem{lemma}[definition]{Lemma}
\renewenvironment{proof}{\textbf{Proof.}}{%
  \nopagebreak{\par\hfill\textbf{\qed}}\par%
}

% Math structures
\newcommand{\with}{{\ |\ }}
\newcommand{\set}[1]{\left\{#1\right\}}
\newcommand{\abs}[1]{\left|#1\right|}
\newcommand{\norm}[1]{\left\|#1\right\|}
\newcommand{\floor}[1]{\left\lfloor#1\right\rfloor}
\newcommand{\ceil}[1]{\left\lceil#1\right\rceil}
\newcommand{\id}[1]{\text{#1}}
\newcommand{\isdef}{:=}
\DeclareMathOperator*{\argmax}{arg\,max}
\DeclareMathOperator*{\argmin}{arg\,min}

% Math symbols
\newcommand{\E}{\mathbb E} % Expected value
\newcommand{\N}{\mathbb N} % Natural numbers
\newcommand{\Z}{\mathbb Z} % Integer numbers
\newcommand{\R}{\mathbb R} % Real numbers
\newcommand{\C}{\mathbb C} % Complex numbers
\newcommand{\Q}{\mathbb Q} % Rational numbers
\newcommand{\B}{\mathbb B} % Boolean values

\newcommand{\X}{\textbf{X\ }}
\newcommand{\U}{\textbf{\ U\ }}
\newcommand{\F}{\textbf{F\ }}
\newcommand{\G}{\textbf{G\ }}
\newcommand{\true}{\textbf{true}}
\newcommand{\false}{\textbf{false}}
% You can define hyphenations as follows

\hyphenation{Trans-for-ma-tion}


% The actual document starts here
\begin{document}
  % The pages in the front are numbered with roman numerals.
  \frontmatter
  % There are specific guidelines on how to write the title page. This is an
% example of how it could look like. Please notice, that even if your thesis is
% written in english, the title page has to be in german. If you don't speak
% german, ask your advisor for help.
\begin{titlepage}

% Use single-spaced line breaking just for the title page
{\setstretch{1.0}

% Additional top margin
\vspace*{1cm}

% Logo of the institute
\includegraphics[height=2.5cm]{pictures/its-logo.png}
%\includegraphics[height=2.5cm]{pictures/isp_de.png}
% If you are interested in other logos, ask your advisor

% Margin between logo and title
\vspace*{2.5cm}

% YOU HAVE TO WRITE YOUR TITLE IN BOTH LANGUAGES! German and english!

%Drive-By Cache Angriffe auf die RSA Schlüsselgenerierung ODER
%Cache Angriffe auf die RSA Schlüsselgenerierung im Browser ODER
%Browserbasierte Cache Angriffe auf die RSA Schlüsselgenerierung


% The german title of your thesis
\textbf{\LARGE{Implementierung und Untersuchung von browserbasierten Cache-Angriffen auf die RSA Schlüsselgenerierung oder Machbarkeitsstudie von browserbarsierten Cache-Angriffen auf die RSA Schlüsselgenerierung}} \vspace*{1em} \\
% The english title of your thesis
\textit{\LARGE{}}

% 2 lines of space
\vspace*{2em}

% Type of this thesis
\textbf{Masterarbeit}

% Spacing
\vspace*{1em}

% Your course of study
im Rahmen des Studiengangs \\ % within the study courses of
\textbf{Informatik} \\ % computer science or informatics (both are ok)
der Universität zu Lübeck % at the University of Lübeck

% Spacing
\vspace*{1.5em}

% Your name
vorgelegt von \\ % handed in by
\textbf{Moritz Krebbel}

% Spacing
\vspace*{1.5em}

% Name of your supervisor
ausgegeben und betreut von \\ % issued and supervised by
\textbf{Prof.\ Dr.\ Thomas Eisenbarth}

% Spacing
\vspace*{1.5em}

% Optional information about scientific assistance
mit Unterstützung von \\ % with the assistance of
Ida Bruhns

% Spacing
\vspace*{2em}

\vfill

% TODO: Remember to put the correct date here
Lübeck, den \today
}
\end{titlepage}

  \chapter*{Abstract}

%Einleitungssatz (zu thema hinführen):
%\todo{ja, du brauchst einen Einleitungssatz, aber ohne Meltdown/Spectre. Irgendwas in Richtung: Mikroarchitekturangriffe haben sich als sehr mächtig erwiesen, erfordern aber meist ausführung von code auf der Opfermaschine. Allerdings wurde durch PAPER gezeigt, dass auch aus Browser möglich. Solche Angriffe werden in Skriptsprachen usw. und sind besonders verherend, weil... . }
%Browserbasierte Seitenkanalangriffe auf die Mikroarchitektur der CPU sind besonders verheerend (falsches Adjektiv. Hier muss so etwas wie "einfach" oder "leicht machbar" stehen), da bereits der Besuch einer Webseite für das Opfer ausreicht, um einen Angriff erfolgreich zu starten. [Wichtig ist hier die Perspektive: Schreibst du aus Sicht des Opfers oder aus Sicht des Angreifers.]

Mikroarchitekturangriffe haben sich als sehr mächtig erwiesen, erfordern aber meist die Ausführung von einem nativen Code auf dem System des Opfers.
Im Jahr 2015 wurde gezeigt (von wem? OKSK 15?)  \cite{TheSpyInTheSandbox}, dass Browser-basierte Angriffe, bei denen bereits der Besuch einer Webseite durch das Opfer ausreicht, möglich sind.
%kurz näher beschreiben wie das in etwa funktionert:
So ein Angriff wird in Scriptsprachen wie Javascript implementiert und nutzt den geteilten L3-Cache der CPU aus, um Informationen über die Speicherzugriffe des Opferprogramms zu erhalten.
Der Angriff ist weder auf einen Exploit im Browser noch auf fahrlässiges Verhalten der Nutzer angewiesen.


%was passiert in dieser Arbeit/Was sind wesentliche Ergenisse?
%-Neue potenzielle gefährliche leakage in NSS und OpenPGP.js
%-versuchte Portierung eines nativen RSA-Schlüsselgenerierungsangriff auf OpenSSL zu Browser-basierten Angriff auf NSS
%-Probleme bei Portierung wie Bremsen, verminderte Angriffsgeschwindigkeit im Browser werden analysiert
%-Eviction-Set-Suchalgorithmus wird verbessert, neue Variante beschrieben und mit altem Algorithmus vergleichen

%\todo{wurde der Drive by angriff nachimplementiert und es wurde versucht, den nativen Angriff auf RSA-Primzahlgenerierung (QUELLE) in den Browser zu portieren. Zusätzlich wurden} 

In dieser Arbeit wird ein Browser-basierter Prime-and-Probe-Angriff implementiert und die Portierung eines nativen Angriffs auf die RSA-Primzahlgenerierung \cite{RSAKeyGeneration2} in OpenSSL hin zu einem Browser-basierten Angriff auf Mozilla NSS untersucht.
Zusätzlich werden neue Leakages in der RSA-Primzahlgenerierung von Mozilla NSS und OpenPGP.js beschrieben und analysiert.
Des Weiteren wird ein Verhalten von Intel-Prozessoren ausgenutzt, um einen neuen Algorithmus zum Finden von Eviction-Sets zu implementieren, der die in der Praxis wichtige Initalisierungsphase eines Angriffs beschleunigt. 

%(Hier fehlt ein 2. Satzteil wie etwa: "und damit den Zugriff auf die Opferdateien sicher ermöglicht.") 
  %% This declaration is essential to every thesis. You need to declare, that this
% work is done by yourself. It is necessary to write ``an Eides statt''.

\chapter*{Erklärung}
Ich versichere an Eides statt, die vorliegende Arbeit selbstständig und nur
unter Benutzung der angegebenen Hilfsmittel angefertigt zu haben.

\vspace{5em}

% You need to personally sign your thesis here. DO NOT FORGET THIS!!!

\rule{0.4\textwidth}{0.4pt}

Lübeck, \today

  %\chapter*{Danksagungen}

% Replace this blind text with your acknowledgements
%\blindtext
Als erstes möchte ich allen Personen danken die mich bei der Anfertigung dieser Arbeit unterstützt haben. Besonders hervorheben möchte ich:

 
   \begin{itemize}
    \item meine Betreuer für die freundliche Unterstützung und die vielen hilfreichen Tipps
    \item meine Familie für das ausgiebige Korrekturlesen meiner Arbeit und der Gabe nützlicher Anregungen sowie der immerwährenden Unterstützung während meines Studiums
    \item meine Kommilitonin für einerseits anregende Diskussionen und andererseits der visuellen Detektierung von Fehlern
    \item den Pool, dessen Rechner ein mächtiges Werkzeug für die Implementation ausgiebiger Benchmarks sind
   \end{itemize}
  \tableofcontents

  % In the main matter the pages are numbered with arabic numerals.
  \mainmatter
  %\chapter{Einführung}
\label{chapter:introduction}

%\todo{Sollen in die Einleitung Quellen?}

Seitenkanalangriffe sind ein mächtiges Werkzeug für die Kryptanalyse, da sie es erlauben es geheime Informationen wie den Schlüssel aus einem Gerät zu extrahieren.
Hierfür wird beispielsweise der Stromverbrauch oder die Berechnungsdauer verschiedener Ein- und Ausgaben gemessen.
Bei physikalischen Seitenkanalangriffen ist die Angreiferin gezwungen eine gewisse örtliche Nähe zum System besitzen.

Seitenkanalangriffen auf die Mikroarchitektur eines Prozessors sind von dieser Einschränkung befreit, da die
Ausführung von Code auf dem Zielsystem ausreicht.
Seit der ersten Beschreibung von Seitenkanalangriffen auf die Mikroarchitektur vor über zehn Jahren, sind diese zu einem ernst zu nehmenden Sicherheitsrisiko herangewachsen.

Auf der Mikroarchitekturebene konkurriert der Opferprozess mit dem Angriffsprozess um die Ressourcen des Prozessors, selbst wenn der Angriffsprozess mit keinen außerordentlichen Privilegien ausgestattet ist.
Aufgrund dieser Konkurrenz kann die Angreiferin Informationen über den Opferprozess gewinnen, indem sie durch Messungen Varianzen bei dem Zugriff auf die Ressourcen feststellt.

Eine Reihe von Ressourcen hat sich als problematisch erwiesen, etwa die Einheiten der Sprungvorhersage oder der Return-Stack-Buffer.
Das mit Abstand größte Problem stellen die vielfältigen Caches dar, welche eine feine Auflösung der Speicherzugriffe des Opferprozesses erlauben.
Diese ermöglichen es geheime Informationen wie kryptografische Schlüssel von RSA \cite{CacheBleedOpenSSLRSA} oder AES \cite{BernsteinAES} aus dem Opferprozess zu extrahieren.
Weitere Anwendungsmöglichkeiten sind Keylogger \cite{Keylogger} und Überwachung der Webseitenaufrufe oder Nutzeraktivität \cite{TheSpyInTheSandbox}.

Mit Ausnahme der wohl bekanntesten Vertreter in Form von Meltdown und Spectre ist wenig darüber bekannt ob Cache-Angriffe praktisch relevant sind oder wie häufig sie eingesetzt werden.
Mit verantwortlich sind die in der Literatur zum Großteil angenommen Angriffsszenario, welches die Ausführung von nativen Code auf dem Opfersystem voraussetzt.
Dieses Szenario hat seine Berechtigung unter anderem bei Angriffen auf virtuelle Maschinen die auf derselben Hardware laufen (Cross-VM-Attack).

Auch Mehrbenutzersystemen fallen unter dieses Szenario, sind aber eher in Unternehmen und Organisationen von Relevanz.
So ist es einem Studierenden der Universität zu Lübeck jederzeit möglich sich per SSH auf einen beliebigen Pool-Rechner einzuloggen, unabhängig davon ob der Rechner lokal in Benutzung ist.

In einem typischen Endbenutzerszenario ist es realitätsfern vorauszusetzen, dass vom Opfer nativer Angriffscode ausgeführt wird.
Denn den meisten Benutzern ist die einhergehende Gefahr der Ausführung von unbekannten nativen Code bewusst und andernfalls würden mögliche Cache-Angriffe das geringste Problem darstellen.

Daher gibt es Bestrebungen Cache-Angriffe auf Endbenutzergeräten zu ermöglichen \cite{TheSpyInTheSandbox,DriveByPaper,ASLROnTheLine}, in dem der Angriffscode in Websprachen wie Javascript übersetzt und vom Browser des Opfers ausgeführt wird.
Da der Browser faktisch bei jedem Webseitenbesuch fremden und unbekannten Code ausführt, sind Möglichkeiten für die Angreiferin in Javascript und Co. entsprechend restriktiv.
Daher sind Angriffe nur eingeschränkt oder aufwendiger auszuführen, wie sich im Verlaufe der Arbeit zeigen wird. 

In dieser Arbeit wird sich auf die folgenden Fragen konzentriert:

\begin{enumerate}
\item Die Arbeiten \cite{TheSpyInTheSandbox,DriveByPaper,ASLROnTheLine} haben demonstriert, dass Cache-Angriffe im Browser möglich sind. 
Allerdings sind diese vor den Gegenmaßnahmen der Browserhersteller gegen Meltdown und Spectre veröffentlicht worden.
Sind also im Oktober 2018 Cache-Angriffe aus dem Browser heraus möglich?

\item Welchen Einschränkungen unterliegen die Cache-Angriffe im Browser gegenüber Cache-Angriffen mit nativem Code?

\item Wie kann die Initialisierung des Angriffs, das heißt die notwendige Eviciton-Set-Suche beschleunigt werden?

\item Gibt es im Bereich der RSA-Schlüsselgenerierung Leakages die Informationen über den Schlüssel verraten und sind diese im Browser ausnutzbar?
\end{enumerate}

\section{Ergebnisse}

Für die Beantwortung der ersten Frage spielt der Browser eine wichtige Rolle, da für Cache-Angriffe hochauflösende Timer im Nanosekundenbereich benötigt werden.
In Chrome ist das Erzeugen eines solchen Timers durch einen Counter-Thread standardmäßig möglich, wobei dies in Firefox zurzeit ausschließlich in abgeänderten Einstellungen möglich ist.

Der angesprochene Counter-Thread ist während des gesamten Angriffs voll ausgelastet, belegt also einen ganzen virtuellen Kern.
Da Entwickler hochauflösende Timer nachfragen, wird Mozilla ebenso wie Google in Zukunft versuchen wieder hochauflösende Timer anbieten, wobei der zurzeit nötige Timer-Thread entfallen wird.
Dies wird passieren, sobald Maßnahmen gegen Meltdown und Spectre implementiert sind TODO cite.

\par\medskip

Beim Prime and Probe Angriff steht Browserimplementation einer nativen Implementation in nichts nach. Allerdings ist neben den Kosten für den Counter-Thread die zeitliche Auflösung des Angriffs um \todo{c code slot time messen} \% reduziert.

Der Opferprozess wird oft künstlich gebremst, da die Berechnungen auch beim Angriff mit nativen Code häufig zu schnell sind.
Auch im Browser lassen sich Methoden umsetzen, um den Opferprozess zu bremsen. \todo{bremsvergleich c vs wasm/js}
Die bevorzugte native Bremsmethode ist im Browser aufgrund der fehlenden clflush-Instruktion nicht umsetzbar.
Anhand eines Angriffs auf die RSA-Schlüsselgenerierung wird gezeigt, dass die Portierung eines nativ laufenden Angriffs aufgrund dieser Einschränkungen nicht immer möglich ist.

\par\medskip

Um einen realistischen Angriff zu erhalten, ist eine kurze Initialisierungsphase Voraussetzung.
Der in vielen Papern beschriebene Eviction-Set-Suchalgorithmus \cite{PrimeAndAbort, LiuPrimeAndProbe, DriveByPaper}, wurde in vielerlei Hinsicht optimiert und die Ergebnisse anschließend mit der ursprünglichen Variante verglichen. 
Des Weiteren wurde eine neue noch unveröffentlichte Technik im Kontext der Eviction-Set-Suche im Browser erprobt.
Diese nutzt spezifische Eigenschaften der Store-Queue in Intel Prozessoren aus.
Damit kann die Suche gegenüber dem optimierten Standardalgorithmus um \todo{Benchmarks für StoreFor vorantreiben} \% beschleunigen.

Häufig muss mehr als eine Cache-Line überwacht werden, weshalb es für die zeitliche Auflösung von Vorteil ist mehrere Angriffsinstanzen zu starten.
Diese müssen jedoch die Eviction-Set-Suche nacheinander durchführen, da sie sich ansonsten gegenseitig beeinflussen würden.
Somit würde sich der Performancevorteil in diesen Fällen amplifizieren.

\par\medskip

Es wurden sowohl die Schlüsselgenerierung von Mozilla NSS als auch von OpenPGP.js auf Leakages untersucht. 
Zum Einsatz kam dabei ein Werkzeug, welches automatisch potenzielle Leakages aufdeckt.
Sowohl in der Primzahlgenerierung von Mozilla NSS als auch OpenPGP.js wurden Leakages entdeckt, welche die verwendeten Primzahlen in einem Gleichungssystem beschreiben.
Es konnte nicht abschließend geklärt werden, ob diese Gleichungssysteme bei praktischen Instanzen wie RSA-2048 in annehmbarer Zeit gelöst werden können.

\todo{Soll in diesem Absatz klarer stehen, dass der Angriff nicht praktisch funktioniert?}
Des Weiteren wurde ein bekannter Cache-Angriff auf die RSA-Schlüsselgenerierung in OpenSSL nach Mozilla NSS portiert.
Der ursprüngliche Angriff setzte die Ausführung nativen Codes voraus, wobei in dieser Arbeit eine Umsetzung im Browser geprüft wurde.
Dabei konnten mehrere Probleme identifiziert werden, welche Portierungen von nativen Cache-Angriffen hin zu Implementierungen im Browser erschweren.

\section{Gliederung der Arbeit}

Diese Arbeit ist in sechs Hauptkapitel untergliedert. 
Nach der Einleitung werden im zweiten Kapitel die notwendigen technischen Grundlagen erläutert, die für das Verständnis der nachfolgenden Kapitel bedeutend sind.

Die Implementierung des Angriffs wird im dritten Kapitel erörtert. 
Im Zuge dessen wird die Initialisierungsphase des Angriffs optimiert und evaluiert, sowie ein verdeckter Kanal als Beispiel vorgebracht.

Das vierte Kapitel befasst sich mit der Analyse möglicher Angriffsziele innerhalb der RSA-Schüsselgenerierung von Mozilla NSS und OpenPGP.js.
Dabei werden Unterschiede zwischen beiden Implementierungen verglichen und die Übertragbarkeit eines nativen Angriffs aus OpenSSL diskutiert.

Im fünften Kapitel werden die Ursachen und Folgen der Ergebnisse erläutert, sowie mögliche Verbesserungen und Erweiterungen vorgestellt.

Im letzten Kapitel wird ein Fazit gezogen und ein Ausblick auf mögliche Änderungen der Thematik in der Zukunft gewährt.

\section{Forschungsstand/Verwandte Arbeiten}

Hier soll ein Überblick bereits getätigter Forschungsarbeiten im Rahmen der für die Arbeit relevanten Themenbereiche Cache-Angriffe, Drive-by-Attacks und Angriffe auf die RSA Schlüsselgenerierung gegeben werden. Zudem soll eine Abgrenzung gegenüber verwandten, aber für diese Arbeit nicht relevanten Themenbereiche stattfinden.

\subsection{Cache-Angriffe}

Cache-Angriffe sind kein neues Phänomen \cite{BernsteinAES, CacheAttacksCountermeasuresAESShamir}, sondern wurden schon im Jahr 2003 \cite{DESCacheAttack2003} erstmals beschrieben.
Die ersten Angriffe zielten auf den L1- und L2-Datencache \cite{CacheAttacksCountermeasuresAESShamir} ab, wobei die Angriffe mit der Zeit auf andere Caches, wie den L1-Instruktionscache \cite{NewResultsInstructionCacheAttacks} und den geteilten L3-Cache \cite{CacheAttacksCloud, LiuPrimeAndProbe} ausgeweitet wurden.
Des Weiteren wurden mit dem Return-Stack-Buffer \cite{Maisuradze2018ret2specSE} und der Sprungvorhersage \cite{BranchPredictionVulnerabilitiesOpenSSL, PredictingSecretKeysViaBranchPrediction, CovertChannelsThroughBranchPredictors} auch andere Cache-Typen angegriffen.
Einen guten Überblick bietet das Survey-Paper von Qian ge et al. \cite{SurveyTimingAttacksCountermeasures}, wobei dies Ende 2016 veröffentlichte Paper aktuelle Entwicklungen nicht berücksichtigen kann.

\subsection{Cache-Angriffe im Browser}

Alle oben zitierten Arbeiten setzen die Ausführung von nativen Code auf dem Opfersystem voraus. Im Jahr 2015 wurde die erste Arbeit veröffentlicht \cite{TheSpyInTheSandbox}, welche sich mit Cache-Angriffen im Browser beschäftigt.
Die Autoren konnten den Aufruf von verschiedenen bekannten Webseiten unterschiedlichen Cache-Zugriffsmustern zuordnen, um so das Surfverhalten von Nutzern zu überwachen.
Gras et al. \cite{ASLROnTheLine} konnten im Browser erfolgreich die Speicherverwürfelung ASLR aushebeln, wobei sich der Angriff aufgrund der niedrigen zeitlichen Auflösung nicht zur Schlüsselextraktion eignet.

Mit Rowhammer-Angriff \cite{Rowhammer} ist es möglich Bits im Speicher ohne Schreibzugriffe zu verändern, um beispielsweise Sicherheitsvorkehrungen zu umgehen. 
Dieser wurde von Gruss et al. erfolgreich in Javascript implementiert \cite{RowhammerJS}.

Code der im Hinblick darauf designt wurde unabhängig von den Eingaben dieselbe Laufzeit und denselben Codepfad zu besitzen, verhindert viele Leakages.
Im Paper \cite{DriveByPaper} wurde dargestellt, dass solcher Code durch die Ausführung im Browser angreifbar wird.
Durch die Freiheiten des Javascript-Compilers ist es möglich, dass je nach Eingabe unterschiedliche Codepfade ausführt werden und somit eine Leakage entsteht.

\subsection{Cache-Angriffe auf RSA}

Viele Paper haben gezeigt \cite{CacheBleedOpenSSLRSA, FlushReload, DriveByPaper}, dass sich der RSA-Schlüssel oder Teile davon mittels Cache-Angriffen extrahieren lassen. Das Angriffsziel waren dabei Komponenten der Ver- und Entschlüsselung beispielsweise die modulare Exponentiationsfunktion \cite{CacheBleedOpenSSLRSA, DriveByPaper, DriveByPaper}.

Weniger Aufmerksamkeit wurde der RSA-Schlüsselgenerierung geschenkt \cite{RSAKeyGeneration2}, da diese im Gegensatz zur Ver- und Entschlüsselung nur einmal ausgeführt wird und damit schwieriger anzugreifen ist.

%\cite{BSIRSAKeyGeneration}  

\subsection{Spekulative Ausführung, Enklaven und Hyperthreading}
\label{related_work}

Vorherige Angriffe hatten die Extraktion von spezifischen Informationen wie Schlüsseln zum Ziel, allerdings können mit Cache-Angriffen auch beliebige Speicherinhalte, sogar über Sicherheitsgrenzen hinweg, ausgelesen werden.
Die im Jahr 2018 veröffentlichen Angriffe Meltdown \cite{MeltdownPaper} und Spectre \cite{SpectrePaper} nutzen die spekulative Ausführung der CPU aus, um an Speicherinhalte zu gelangen, für die keine Zugriffsrechte vorliegen.
Meltdown greift dafür direkt auf die Speicherbereiche zu, wohingegen Spectre die Sprungvorhersage manipuliert, um den Zugriff privilegierter Prozesse auf bestimmte Speicherinhalte zu lenken.

Beide Varianten erzeugen invalide Speicherzugriffe während der spekulativen Ausführung, das heißt der nach außen sichtbare Zustand der CPU bleibt unverändert erhalten, jedoch ändert sich der interne Zustand wie etwa die Cache-Inhalte.
Um den Informationen über den veränderten internen Zustand der CPU zu erhalten, werden analog zu anderen Angriffen Differenzen in der Zugriffszeit auf verschiedene Cache-Einträge genutzt.

Heutzutage teilen sich eine Reihe von parallel ausgeführten Prozessen die Ressourcen des Systems.
Sofern eine sichere Abschottung der Prozesse gegeneinander nicht gegeben ist, kann dies Sicherheitsprobleme erzeugen.
Intel SGX erzeugt eine vertrauenswürdige Ausfürhungsumgebung, welche Speicherbereiche bereitstellt auf die selbst Prozesse mit erhöhten Rechten keinen Zugriff haben.
Wie \cite{CacheZoom,CacheAttacksIntelSGX} zeigen, sind diese sogenannten Enklaven ebenfalls über Caches angreifbar.

Sofern der Prozess der Angreiferin und des Opfers auf demselben physischen Kern laufen, ergeben sich weitergehende Angriffsmöglichkeiten.
So kann beispielsweise eine erhöhte räumliche Auflösung ermöglicht werden, welche über die übliche 64 Byte Cache-Line-Auflösung hinaus geht \cite{MemJam}.

Ein im Juli 2018 veröffentlichter Angriff \cite{TLBleed} nutzt konkurrierende Zugriffe auf den Translation lookaside buffer(TLB) aus, welcher zwischen zwei Hyperthreads eines selben physischen Kerns geteilt ist.
Um Angriffe auf geteilte Caches zu verhindern, kann der Cache partitioniert werden, wobei keine vergleichbare Technik für den TLB bekannt ist.
Im Zuge der Veröffentlichung wurde im Betriebssystem OpenBSD Hyperthreading auf allen Intel-Prozessoren standardmäßig deaktiviert \cite{OpenBSDHyperthreading}.

Im Browser kann nicht ohne Weiteres ein Kern festgelegt werden, auf dem der Angriffscode ausgeführt wird.
Als mögliche Lösung könnte die Angreiferin $n-1$ Instanzen ihres Angriffs starten, wobei $n$ der Anzahl der virtuellen Kerne entspricht.
Somit würde durchgehend eine Instanz auf demselben physischen Kern wie das Opferprogramm laufen.
  \chapter{Grundlagen}
\label{chapter:basics}

Im folgenden sollen Verfahren und Techniken erläutert werden, welche für das Verständnis der späteren Kapitel essenziell sind.

\section{Virtuelle Speicherverwaltung}

Virtuelle Speicherverwaltung stellt eine Abstraktion für die vorhandenen physikalischen Speichermedien wie etwa dem Hauptspeicher oder der Festplatte bereit.
Das Betriebssystem übersetzt virtuelle Adressen, welche von Prozessen genutzt werden, mit Hilfe der Hardware in physikalische Adressen. Jedem Prozess steht der gleiche virtuelle Adressraum zur Verfügung, wobei das Betriebssystem dafür Sorge trägt für jeden Prozess die richtige Zuordnung von virtueller zu physikalischer Adresse sicherzustellen.
Die Vorteile der virtuellen Speicherverwaltung sind erhöhte Sicherheit durch die Speicherisolierung aller Prozesse. So kann eine fehlerhafte Schreiboperation eines Prozesses keinen Fehler in anderen Prozessen verursachen, da gleiche virtuelle Adressen vom Betriebssystem auf unterschiedliche physikalische Adressen abgebildet werden. Des Weiteren kann ein Prozess mehr Hauptspeicher nutzen als physikalisch vorhanden ist, indem Daten vom Betriebssystem auf andere Speichermedien wie die Festplatte ausgelagert werden.

\section{Caches}

Die Geschwindigkeitsentwicklung des Hauptspeichers konnte in den letzten Jahren nicht mit der des Prozessors Schritt halten. Der Cache ist ein Vergleich zum Hauptspeicher kleinerer, aber schnellerer Pufferspeicher, welcher im aktuellen Kontext häufig benötigte Daten vorhält. Ohne Caches wäre ein Prozessor häufig gezwungen auf Daten des langsamen Hauptspeichers zu warten und würde in seiner Verarbeitungsgeschwindigkeit ausgebremst. Auch in anderen Ebenen sind Caches sinnvoll, wie etwa im Browser, wobei in dieser Arbeit vor allem die Caches im Prozessor relevant sind. Die Testrechner verwenden Intel-Prozessoren mit der Core-Architektur, weshalb diese in folgenden Erklärungen häufig als Beispiel dient.

\subsection{Cache-Organisation}

Ein CPU-Cache enthält mehrere Einträge, welche folgende Bestandteile besitzen:
\begin{enum}
\item Cache-Line: Die gecacheten Daten, wobei die Länge in der Core-Architektur etwa 64 Bytes beträgt.
\item Address-Tag: Die Adresse im Hauptspeicher von der die Daten in der Cache-Line stammen.
\item Flag-Bits: Etwa das "Dirty"-Bit welches anzeigt ob die Daten der Cache-Line noch mit denen im Hauptspeicher übereinstimmen.
\end{enum}


\subsection{Assoziativität}

Die Ersetzungsstrategie entscheidet in welchem Cache-Eintrag die zu chachenden Daten einer bestimmten Adresse aus dem Hauptspeicher landen. Sofern die Auswahl des Cache-Eintrags keinerlei Beschränkung unterliegt, wird von einem voll-assoziativen Cache gesprochen. Das andere Extrem wäre ein einfach-assoziativer Cache bzw. eine direkte Abbildung, wobei die Adresse des Hauptspeichers von der die Daten stammen den zu wählenden Cache-Eintrag eindeutig festlegt.

\subsection{Flush and Reload}

\subsection{Prime + Probe}

Ein Eviction-Set sei eine Menge, welche es vermag einen Cache-Eintrag aus einem Cache zu verdrängen. D.h. ein Eviction-Set welches einen Eintrag aus dem L3-Cache löscht, würde den gleichen Zweck wie der clflush-Assemblerbefehl im Flush + Reload Angriff erfüllen. Um einen Eintrag aus dem Cache zu verdrängen, müssen mehrere Adressen der Daten aus dem Eviction-Set von der CPU auf die gleiche Cache-Set wie der zu verdrängende Eintrag abgebildet werden, sodass die Größe eines Eviction-Sets mindestens die Assoziativität des Caches erreichen sollte.

Die Idee beim Prime + Probe Angriff besteht darin, in einer sich wiederholenden Abfolge zuerst den Cache zu Primen, dann das Opferprogramm Berechnungen durchführen zu lassen und anschließend zu Proben. In der Priming-Phase werden mittels der Eviction-Set gezielt Cache-Sets vollständig mit den Daten aus dem Eviction-Set belegt. In der anschließenden Berechnungsphase werden einige Einträge aus den geprimten Cache-Sets vom Opferprogramm verdrängt. Abschließend berechnet die Angreiferin die Summe der Zugriffszeiten auf alle Einträge in einem Eviction-Set. Sofern das Opferprogramm in dem zum Eviction-Set korrelierenden Cache-Set Einträge verdrängt hat, kann die Angreiferin eine Abweichung nach oben in ihrer Messung feststellen, da die verdrängten Einträge eine erhöhte Zugriffszeit beisteuern.

Die zur Durchführung essenziellen Eviction-Set sind nicht immer leicht zu finden, da manche Umgebungen nur eingeschränkte Adressräume zu Verfügung stellen. So sind beim später näher beleuchteten Web-Assembly maximal die untersten 12-Bit der Adressen mit den physikalischen Adressen identisch, welche die CPU nutzt um Adressen in Cache-Sets abzubilden.
In solchen Fällen müssen die Eviction-Sets in einem Trial-and-Error Verfahren ermittelt wie es der Algorithmus %\ref[alg:evictionSet}
beschreibt.

%Beschreibe Algorithmus
%Hierfür werden zuerst wiederholt Speicherblöcke angefordert, wobei solche in einer Menge gesammelt werden, welche 

\SetKwProg{Fn}{Function}{}{}

\begin{algorithm}[h]
\DontPrintSemicolon
\caption{Psuedo-Code für Eviction-Set Algorithmus}
\label{alg:evictionSet}

\Fn{$EvictionSetFinder(memoryBlocks)$}{
    groups \leftarrow empty\;
    \While{size(memoryBlocks > 0}{
        evictionSet $\leftarrow$ empty\;
		witness $\leftarrow$ expand(evictionSet, memoryBlocks)\;
		
		\If{witness != failed}{
    		contract(evictionSet, memoryBlocks, witness)\;
    		witnessSet $\leftarrow$ collect(evictionSet, memoryBlocks, witnessSet)\;
    		groups.add(union(evictionSet, witness, witnessSet))\;
		}
    }
}

\Fn{$Expand(evictionSet, memoryBlocks)$}{
	\While{size(candidateSet) > 0}{
		witnesss = SelectRandomItem(candidateSet)\;
		\If{checkevict(evictionSet, witnesss)}{
			\Return witnesss
		}
		evictionSet.add(witnesss)\;
	}
	\Return failed;
}

\Fn{$Contract(evictionSet, memoryBlocks, witness)$}{
	\ForEach{candidate in evictionSet}{
		\If{checkevict(evictionSet, witness)}{
			mermoryBlocks.add(candidate)\;
			evictionSet.add(candidate)\;	
		}		
	}
}

\Fn{$Collect(evictionSet, memoryBlocks)$}{
	witnessSet = empty\;
	\ForEach{candidate in mermoryBlocks}{
		\If{checkevict(evictionSet, candidate)}{
			memoryBlocks.delete(candidate)\;
			witnessSet.add(candidate)\;
		}
	}
	\Return witnessSet;
}

\end{algorithm}
  \chapter{Implementierung}
\label{chapter:preparation}

\SetKwProg{Fn}{Function}{}{}

Das folgende Kapitel beschreibt, mit Hilfe welcher Softwaretools der Cache-Angriff implementiert wird.

In dem in dieser Arbeit verwendeten praxisnahen Angriffsmodell reicht für den Start eines Angriffs der Besuch des Opfers auf einer vorher präparierten Website, auch Drive-By-Angriff genannt, aus. 
Dafür genügt bereits eine eingebundene JavaScript-Werbeanzeige, die von der Angreiferin kontrolliert wird. 
Gegenüber Angriffen die ein Ausführen von nativen Code auf dem Endgerät des Opfers verlangen, ist mit diesen Voraussetzungen ein deutlich größerer Angriffsvektor gegeben.
Da Aufgrund des Angriffsmodells nur Webtechnologien verfügbar sind, liegt komplette Angriffscode in JavaScript und Webassembly vor. 
Frühere Implementierungen von Cache-Angriffen im Browser \cite{TheSpyInTheSandbox} hatten noch keine Möglichkeit, Webassembly zu verwenden, weshalb deren kompletter Angriffscode in JavaScript geschrieben war. 
Webassembly ermöglicht hardwarenähere Programmierung und den Vorteil, dass der Code anders als in JavaScript nicht während der Laufzeit optimiert werden muss. 
Des Weiteren steht mit dem Emscripten-Compiler ein Tool bereit, welches die Übersetzung von C-Code in Webassembly anbietet. Somit kann ein bestehender Angriffscode in C, in diesem Fall von Mastik, als Grundlage verwendet werden, und eine komplette fehleranfällige Neuimplementierung in JavaScript entfällt.

\section{Timer in JavaScript}

Der hier ausgeführte Cache-Angriff benötigt wie in beschrieben präzise Timer, welche eine Auflösung von unter 30 ns bereitstellen sollten. Dennoch könnte die Suche nach \textit{Eviction-Sets} auch mit schlechteren Timerauflösungen bewerkstelligt werden, indem Operationen mehrfach ausgeführt werden und die Differenz der aufsummierten Zeiten zur Bewertung herangezogen wird.
Im \textit{Eviction-Set}-Algorithmus könnte etwa die Funktion $checkevict$ wie in Algorithmus \ref{alg:checkevict_low_resolution} angepasst werden, wobei der Parameter $repeatIterations$ abhängig von der Timerauflösung gewählt wird. 
Es besteht jedoch das Problem, schwache Aktivitäten im Cache-Set während des eigentlichen Angriffs aufzudecken, da im Worst-Case nur ein Eintrag aus dem beobachteten Cache-Set verdrängt wird und somit lediglich die Zugriffszeit zwischen einem Hit und einem Miss ausschlaggebend ist. 
In diesem Fall könnte die Dauer mehrerer Prime and Probe-Iterationen gesamtheitlich gemessen werden, und zwar unter der Vermutung, dass auf die für die Verdrängung verantwortliche Adresse über die Zeit mehrfach zugegriffen wird. %\todo{Ich verstehe den nächsten Satz nicht. Satz redesignt}
Die direkten Auswirkungen eines niedrig aufgelösten Timers sind also eine geringere zeitliche Auflösung von Cache-Aktivitäten oder die Nichtregestrierung von schwachen.

\begin{algorithm}[h]
\DontPrintSemicolon
\caption{Pseudo-Code für $checkevict$ im Fall von einer niedrig aufgelösten getTimestamp}
\label{alg:checkevict_low_resolution}

\Fn{$checkevict(possibleEvictionSet, witness)$}{
    timestampBefore <- getTimestamp()\;
    \For{i=1 to repeatIterations}{
        accessMemory(possibleEvictionSet)\;
        accessMemory(witness)\;
    }
	timestampAfter <- getTimestamp()\;
	\Return timestampAfter - timestampBefore > threshold;
}

\end{algorithm}

Der W3C hat die High-Resolution-Time-API spezifiziert, welche die Methode performance.now() beinhaltet, die einen aktuellen Timestamp zurückgibt. Im Firefox hatte die Methode in früheren Versionen eine hinreichend genaue Auflösung im Nanosekundenbereich, wobei in Reaktion auf die Sicherheitslücken Meltdown und Spectre die Auflösung schrittweise auf 2 ms im aktuellen Firefox 60 abgesenkt wurde. 
Auch in den Browsern Edge und Chrome wurden im Zuge der Veröffentlichung von Meltdown und Spectre die Auflösung von perfomance.now() verringert.
Darüber hinaus wird bei beiden Browsern auf den zurückgegebenen Timestamp ein Timerjitter addiert.

So bieten Edge und Chrome zurzeit (Stand Juni 2018) eine Auflösung von 20 \textmu s + 20 \textmu s Jitter respektive 100 \textmu s + 100 \textmu s Jitter.
Das Paper "Fantastic Timers and where to find them" \cite{FantasticTimers} beschreibt diverse andere Methoden, um mit Hilfe von JavaScript Timer zu generieren. 
Es Werden etwa CSS-Animationen und Nachrichtenkanäle als mögliche Zeitgeber untersucht.
Allerdings ist nur eine geeignete Methode dabei, da die Auflösung aller anderen mindestens im hohen einstelligen \textmu s Bereich liegt und somit der Parameter $repeatIterations$ auf Werte von etwa 1000 gesetzt werden müsste, um zuverlässig \textit{Eviction Sets} zu finden. 
Hierdurch würde die benötigte Ausführungszeit zum Finden der \textit{Eviction Sets} auf ein Maß ansteigen, welches nicht mehr zum angenommenen Angriffsmodell passen würde.

%\newtext

Als einziges angemessenes Zeitmessungswerkzeug verbleibt der SharedArrayBuffer aus Javascript. 
Die Ausführung von Javascript passiert in einem Thread, wobei die Möglichkeit besteht, sogenannte Webworker zu starten, welche Code aus einem Skript in einem eigenen Thread ausführen.
Der Speicherbereich eines Webworkers und des Mainthreads sind strikt getrennt, so dass Daten ursprünglich über Nachrichten ausgetauscht werden mussten. Hier setzt der SharedArrayBuffer an, welcher einen geteilten Speicherbereich zwischen Mainthread und Webworker definiert.

Gemäß Code-Listing \ref{alg_list:sharedArrayBufferWorkerMain} wird im Mainthread zuerst ein SharedArrayBuffer von 4 Bytes angelegt. Anschließend wird der als Zeitgeber fungierende Webworker gestartet und ihm eine Referenz auf den eben angelegten SharedArrayBuffer übersandt. 

\begin{figure}[h]
\begin{lstlisting}[caption=main.js: Code welcher den counterWorker für Zeitmessungen verwendet,label=alg_list:sharedArrayBufferWorkerMain]
var sharedArrayBuffer = new SharedArrayBuffer(4);
var counterWorker = new Webworker('counterWebworker.js');
counterWorker.postMessage(sharedArrayBuffer);
var sharedArrayBufferUin32Array = new Uint32Array(sharedArrayBuffer);

function measureTime(func){
    var t1 = Atomics.load(sharedArrayBufferUin32Array[0]);
    func();
    var t2 = Atomics.load(sharedArrayBufferUin32Array[0]);
    return t2 - t1;
}
\end{lstlisting}
\end{figure}

Die Zählvariable soll hier eine Größe von 32 Bit haben, weshalb abschließend ein Uint32 Array definiert wird, dessen Inhalt auf den SharedArrayBuffer referenziert. 
Ein Zählvariable kann nun durch Lesen des ersten Eintrags des Arrays erhalten werden. 
Problematisch ist jedoch, dass auf die Zählvariable sowohl lesend vom Mainthread als auch schreibend vom Webworker zugegriffen wird. 
Dadurch können die im Mainthread gelesenen Werte veraltet sein, da der SharedArrayBuffer noch nicht zwischen beiden Threads synchronisiert wurde. 
Abhilfe schafft hier die von Javascript bereitgestellte Atomics-Library, welche es ermöglicht, die Leseoperation atomar auszuführen.

Der Webworker iteriert nun in einem eigenen Thread die Zählvariable in einer Endlosschleife (siehe auch Pseudocode \ref{alg_list:sharedArrayBufferWorker}). 
Zuerst wird dazu dem Webworker via message die Referenz auf einen im Mainthread erstellten SharedArrayBuffer übergeben.
Anschließend wird im Webworker ein \textit{Uint32Array} angelegt, welches mit dem übergebenen SharedArrayBuffer verknüpft ist. 
Zum Schluss geht der Webworker in die Endlosschleife über, in welcher die Zählvariable \textit{sharedArray[0]} durchgehend iteriert.

\begin{figure}[h]
\begin{lstlisting}[caption=counterWebworker.js: Webworker welcher die Zählvariable in einer Endlosschleife iteriert,label=alg_list:sharedArrayBufferWorker]
self.addEventListener('message', (m) => {
  // Create an Int32Array on top of the shared memory array 
  const sharedArray = new Uint32Array(m.data);
  while{true}{
    sharedArray[0]++;
  }
});
\end{lstlisting}
\end{figure}

Das Iterieren einer Variable benötigt nur wenige Taktzyklen, weshalb der aktuelle Wert der Zählvariable als Zeitstempel interpretiert werden kann. 
Die Auflösung dieser Methode hängt also von der Geschwindigkeit der Iteration sowie der Speichersynchronisation des SharedArrayBuffers zwischen Mainthread und Webworker ab.

In Versuchen mit verschiedenen Webbrowsern und Hardwarekonfigurationen zeigte sich, dass die Auflösung mindestens im einstelligen Nanosekundenbreich liegt und somit ausreichend genau ist, um den Unterschied zwischen einem Cache-Miss und Hit festzustellen \ref{tbl:times_res}.

Um die Auflösung zu bestimmen wird die Javascriptfunktion $performance.now()$ zur Hilfe genommen \ref{alg_list:getResolutionNS}. Die Funktion wait_edge() ruft zuerst performance.now() für den Startwert auf und wartet anschließend in einer Endlosschleife bis performance.now() einen höheren Wert als den Startwert zurückgibt. 
Dieser höhere Wert wird ebenso wie der aktuelle Stand der Zählvariable gespeichert. 
Dann folgt ein erneuter Aufruf von wait_edge(), wobei bei zurückkehren wieder der letzte performance.now()-Wert sowie der Wert der Zählvariable gespeichert wird.
Für die performance.now()-Funktion ist die Auflösung bekannt, weshalb aus den Differenzen der performance.now()-Werte und der Werte der Zählvariable eine Auflösung für den Timer errechnet werden kann.
Wie oben beschrieben addiert Chrome auf den performance.now()-Wert einen Timerjitter, weswegen diese Prozedur 20000 mal durchgeführt wurde um valide Mittelwerte zu erhalten.

Die Zählvariable besitzt den Datentyp Uint32 und in Javascript ergibt eine Iteration des Wertes $2^{32}-1$ wieder 0. 
Daher ist es naheliegend den gesamten Wertebereich von 0 bis $2^{32}-1$ auszuschöpfen und bei Messwerten über $2^{31}$ anzunehmen, dass in diesem Zeitraum ein Overflow stattfand.
Beim Testen mit Firefox zeigte sich jedoch, dass die Iteration der Zählvariable ab dem Wert $2^{31}$ signifikant langsamer wird, wobei dieses Phänomen mit Chrome nicht zu beobachten war.
Als Workaround wurde eine Abfrage hinzugefügt, die bei einem überschreiten von $2^{31}$ die Zählvariable auf 0 zurücksetzt.
Die Ergebnisse zeigen, dass dieser Workaround die Auflösung in Chrome nur minimal verschlechtert, dafür aber in Firefox signifikant verbessert.
Ohne diese Änderungen sorgt der Timer in Firefox für Probleme in der Zeitmessung, da ein Zeitintervall beim überschreiten des Wertes $2^{31}$ wegen der verringerten Iterationsgeschwindigkeit als deutlich verkürzt wahrgenommen wird.

\label{tbl:times_res}
\begin{table}[h]
\caption{Zeitauflösung des SharedArrayBuffer-Zählers mit verschiedenen Browsern auf Ubuntu 16.04.5 LTS (GNU/Linux 4.4.0-131-generic x86_64) mit einem i7-4770. Wertebereich der Uint32 Zählvariable wird in der linken Spalte ausgeschöpft, in der rechten hingegen wird nur bis $2^{31}$ gezählt.}
\begin{tabular}{lllll}
                           & Zählen bis $2^{32}$ & Zählen bis $2^{31}-1$ &  &  \\[10pt]
Chromium 68.0.3440.106     & $\sim$2,7ns                      & $\sim$3ns                        &  &  \\
Google Chrome 69.0.3497.81 & $\sim$2,7ns                      & $\sim$3ns                        &  &  \\
Firefox 63.0b4             & $\sim$5,1ns                      & $\sim$2,2ns                      &  & 
\end{tabular}
\end{table}


\begin{figure}[h]
\begin{lstlisting}[caption=main.js: Code welcher die Timerauflösung bestimmt,label=alg_list:getResolutionNS]
var start = wait_edge();
var start_count = Atomics.load(Module['sharedArrayCounter'], 0);
var end = wait_edge();
var end_count = Atomics.load(Module['sharedArrayCounter'], 0);
nsPerTick += (end - start) * 10^6 / (end_count - start_count);

function wait_edge() {
  var next, last = performance.now();
  while ((next = performance.now()) == last) {}
  return next;
}
\end{lstlisting}
\end{figure}

Diese Methode geht allerdings mit dem Nachteil einher, dass der Webworker-Thread in der Messphase einen CPU-Kern komplett auslastet. Das heißt, dass im Angriffsszenario das Opferprogramm, der Javascript-Mainthread und der Webworker gleichzeitig rechnen, so dass mindestens 3 CPU Kerne benötigt werden. 
Sofern sich der Webworker einen physischen Kern mit einen anderen aktiven Prozess teilt, können die gemessen Zeiten einer stärkeren Volatilität durch die erhöhte Iterationsdauer unterliegen.
In der Konsequenz reduziert sich die Auflösung des Zeitgebers, wobei ein ausreichende Genauigkeit dennoch gegeben ist, da beide Prozesse in etwa die gleiche Rechenzeit zugesprochen bekommen.

Aufgrund der Option, den SharedArrayBuffer als Timer zweckzuentfremden, wurde dieser im Zuge der Veröffentlichung von Meltdown und Spectre in allen gängigen Webbrowsern deaktiviert \cite{FirefoxSharedArrayBuffer}. Jedoch planen die Hersteller, das Feature in Zukunft wieder zu aktivieren, sobald die Gefahr von Angriffen wie Meltdown und Spectre reduziert ist. 
Google ist der erste Hersteller, der in seinem Chrome-Browser mit Version 68 SharedArrayBuffer wieder aktiviert hat \cite{ChromeSharedArrayBufferAgain}. 
Als logische Konsequenz wurde angekündigt, in Zukunft wieder hochauflösende Timer bereitzustellen, da ein SharedArrayBuffer genau diese Möglichkeit schon jetzt bereitstellt \cite{ChromeHighResolutionTimerAgain}.

Aus diesen Gründen wird im Folgenden davon ausgegangen, dass das Opfer SharedArrayBuffer in seinem Webbrowser aktiviert hat.

%\todo{Hier steht ein h in eckigen Klammern. Listings funktionieren nicht so, das sind keine float-umgebungen. Darum bricht auch mitten im Listing die Seite um. Um es "hübsch" in den text einzufügen musst du das listing in eine figure packen. Ansonsten taucht es einfach immer genau da auf, wo du es hinsetzt.}

%moved from section grundlagen
\section{Eviction-Set Algorithmus in der Javascript-Umgebung}
%\section{Cache-Angriff in JavaScript und Webassembly}
Der wichtigste Teil für einen Prime-and-Probe-Angriff ist die Fähigkeit, zuverlässig Eviction-Sets zu finden. Wie im Grundlagenkapitel beschrieben, führt die CPU das Cache-Mapping anhand der physischen Adressen durch. Webassembly emuliert eine 32-Bit-Umgebung, welche die internen Adressen in virtuelle Adressen des Hostprozesses, hier des Browsers, übersetzt. 
Webassembly-Code verwendet nur Adressen der emulierten Umgebung und hat keinerlei Zugriff auf das Mapping zu den virtuellen Adressen. 

Somit sind die physischen Adressen in Webassembly durch gleich zwei Abstraktionsschichten geschützt. 
Jedoch lässt sich für das Finden der Eviction-Sets die Eigenschaft ausnutzen, dass im Betriebssystem 4-KiB-Pages existieren, sodass die letzten 12 Bits der virtuellen und physischen Adresse identisch sind. 
Des Weiteren alloziert %(??? erstellt?) hier der Fachausdruck
Webassembly 4KiB-große Blöcke, die zur Übereinstimmung der 12 letzten Bits der Webassembly-Adresse mit der virtuellen und physischen führen.

Um aus dieser Eigenschaft Kapital zu schlagen, wird ein Array mindestens entsprechend der Größe des L3-Caches in Webassembly angelegt. Im Folgenden soll der Intel i7-4770 mit 8 MiB großem L3-Cache erneut als Basis dienen. In diesem Array sind nun $x$ Blöcke der Größe 4 KiB, deren letzten 12 Adressbits mit der physischen Adresse übereinstimmen. Im sogenannten Adresspool seien nun die Adressen des Arrays, bei denen die letzten 12 Bits gleich sind, also insgesamt $x$ Stück.

Der i7-4770 besitzt 8192 Cache-Sets, die auf 4 Slices aufgeteilt sind, wobei für das Mapping der 2048 Cache-Sets innerhalb eines Slices nur die untersten 18 Bits der physischen Adresse relevant sind. Dabei bestimmen die Bits 6 bis 17 eindeutig das Cache-Set und die Bits 0 bis 5 das Offset innerhalb der Cache-Line.

In welchem der 4 Slices die Daten landen, wird anhand der Adressbits 18 bis 63 bestimmt.
Durch die Kenntnis der untersten 12 Bits der physischen Adresse sind gleichzeitig 6 Bits (6 bis 11) bekannt, welche für die Zuordnung zu den Cache-Sets verantwortlich sind.

Angenommen, im Pool sind ausschließlich Adressen, bei denen die letzten 12 Bits auf 0 gesetzt sind. 
Somit ist ein Abstand von $2^12$ für aufeinanderfolgende Adressen gegeben und jede Adresse lässt sich genau einem der 4-KiB-großen Blöcke zuordnen.
Dann kann erwartet werden, dass im Mittel jede 128. 
Adresse auf das gleiche Cache-Set gemappt wird. Es gibt 8192 Möglichkeiten, eine Adresse einem Cache-Set zuzuordnen, also 13 Bits an Unsicherheit.
Durch die Kenntnis der untersten 12 Bits der Adresse sind davon 6 Bits bekannt, welche für eine eindeutige Zuordnung sorgen. 
Es bleiben noch 7 Bits an Unsicherheit, die durch Kenntnis der restlichen Adressbits beseitigt werden könnten.

\subsection{Wahl der Adresspoolgröße}
\label{addressPoolSize}

Um die Anzahl der Blöcke beziehungsweise die Arraygröße in Webassembly sinnvoll zu bestimmen, kann zuerst die vereinfachte Annahme getroffen werden, dass die physischen Adressbits 12 bis 63 der 4-KiB-Blöcke zufällig gewählt sind. 
Die unbekannte Cache-Mapping-Funktion nimmt nun die zufälligen Bits 12 bis 63 und die auf 0 gesetzten Bits 0 bis 11 entgegen und gibt eines von 128 möglichen Cache-Sets zurück.
Ziel ist es, mit einer Poolgröße $x$ und einer hohen Wahrscheinlichkeit für jedes Cache-Sets 16 Zuordnungen beziehungsweise ein Eviction-Set zu finden.
Gesucht ist die Wahrscheinlichkeit, bei einer Poolgröße $x$ mindestens 16 Zuordnungen zu einem fixen Cache-Set $cs$ bei 128 Möglichkeiten zu finden.
Hierfür eignet sich das Urnenmodell für Ziehungen mit Zurücklegen ohne Berücksichtigung der Reihenfolge, wobei die $x$ Adressen im Pool die Kugeln und die Cache-Sets die Farben der Kugeln repräsentieren. Sei $P(count(cs)>=16)$ die Wahrscheinlichkeit dafür, dass in der gezogenen Folge mindestens 16 mal das Cache-Set $cs$ auftaucht, unter der Bedingung, dass die Poolgröße $\#add = x$.
Leichter ist es in diesem Fall, die Gegenwahrscheinlichkeit zu berechnen, die mit
\begin{align*}
P(count(cs)<16|\#add = x) &=
\left( \sum\limits_{i=0}^{15}P(count(cs)=i|\#add = x) \right) \\&=
\left( \sum\limits_{i=0}^{15} {x \choose i} \frac{127}{128}^{x-i} \cdot \frac{1}{128}^i  \right)
\end{align*}
beschrieben ist.

Die Wahrscheinlichkeit des $P(count(cs)>=16)$ ist etwa bedeutend im ersten Szenario:
Angenommen, die Angreiferin hat es auf eine bestimmte Adresse abgesehen und möchte ein korrelierendes Eviction-Set finden. 
Wie wahrscheinlich ist ein erfolgreicher Angriff beziehungsweise wie hoch ist die Wahrscheinlichkeit, ein korrelierendes Eviction-Set zu finden? 
Das Diagramm \ref{fig:combined_es_prob} zeigt die Erfolgswahrscheinlichkeiten für verschiedene $x$-Werte. 

%\todo{Die Linie muss dünner. Die Bildunterschrift muss die cache-sets und die assoziativität beeinhalten. Das Bild soll mit der Unterschrift verständlich sein, ohne den restlichen text lesen zu müssen. Schriftgröße für Bildunterschriften und beschriftungen gern kleiner.}

Im zweiten Szenario möchte die Angreiferin in einem komplexen Angriff eine Vielzahl von Adressen in unterschiedlichen Cache-Sets überwachen. 
Sie interessiert sich nun dafür, wie wahrscheinlich es ist, alle 8192 Cache-Sets zu finden und somit auch die für sie relevanten. 

Die Wahrscheinlichkeit, für alle 128 möglichen Cache-Sets jeweils 16 Zuordnungen und damit gleichbedeutend alle 8192 möglichen Eviction Sets konstruieren zu können, ist approximativ mit
\begin{align*}
P(count(cs)>=16|\#add = x)^{128} = (1-P(count(cs)<16|\#add = x))^{128}
\end{align*}
beschrieben.
Die Ereignisse sind nicht unabhängig, weshalb diese Vereinfachung zu Ungenauigkeiten führt.
Diese sind aber im zur Veranschaulichung gezeigten Bereich unter 0,3 Prozentpunkten und damit visuell nicht identifizierbar.

Die durchgezogene Linie im Diagramm \ref{fig:combined_es_prob} gibt die Erfolgswahrscheinlichkeiten für verschiedene $x$-Werte im zweiten Szenario an. 

%\todo{siehe oben. Du hast drei mal den gleichen graphen! vielleicht kann man die in ein Diagramm zeichnen? Wie viel Mehrwert bieten sie?}

\label{fig:combined_es_prob}
\begin{figure}[h]
\centering
\begin{scaletikzpicturetowidth}{\textwidth}
% Created by tikzDevice version 0.12 on 2018-10-03 05:17:48
% !TEX encoding = UTF-8 Unicode
\begin{tikzpicture}[x=1pt,y=1pt]
\definecolor{fillColor}{RGB}{255,255,255}
\path[use as bounding box,fill=fillColor,fill opacity=0.00] (0,0) rectangle (432.17,289.08);
\begin{scope}
\path[clip] (  0.00,  0.00) rectangle (432.17,289.08);
\definecolor{drawColor}{RGB}{255,255,255}
\definecolor{fillColor}{RGB}{255,255,255}

\path[draw=drawColor,line width= 0.6pt,line join=round,line cap=round,fill=fillColor] (  0.00,  0.00) rectangle (432.17,289.08);
\end{scope}
\begin{scope}
\path[clip] ( 39.22, 32.09) rectangle (426.67,283.58);
\definecolor{fillColor}{gray}{0.92}

\path[fill=fillColor] ( 39.22, 32.09) rectangle (426.67,283.58);
\definecolor{drawColor}{RGB}{255,255,255}

\path[draw=drawColor,line width= 0.3pt,line join=round] ( 39.22, 33.99) --
	(426.67, 33.99);

\path[draw=drawColor,line width= 0.3pt,line join=round] ( 39.22, 53.04) --
	(426.67, 53.04);

\path[draw=drawColor,line width= 0.3pt,line join=round] ( 39.22, 72.10) --
	(426.67, 72.10);

\path[draw=drawColor,line width= 0.3pt,line join=round] ( 39.22, 91.15) --
	(426.67, 91.15);

\path[draw=drawColor,line width= 0.3pt,line join=round] ( 39.22,110.20) --
	(426.67,110.20);

\path[draw=drawColor,line width= 0.3pt,line join=round] ( 39.22,129.25) --
	(426.67,129.25);

\path[draw=drawColor,line width= 0.3pt,line join=round] ( 39.22,148.31) --
	(426.67,148.31);

\path[draw=drawColor,line width= 0.3pt,line join=round] ( 39.22,167.36) --
	(426.67,167.36);

\path[draw=drawColor,line width= 0.3pt,line join=round] ( 39.22,186.41) --
	(426.67,186.41);

\path[draw=drawColor,line width= 0.3pt,line join=round] ( 39.22,205.46) --
	(426.67,205.46);

\path[draw=drawColor,line width= 0.3pt,line join=round] ( 39.22,224.52) --
	(426.67,224.52);

\path[draw=drawColor,line width= 0.3pt,line join=round] ( 39.22,243.57) --
	(426.67,243.57);

\path[draw=drawColor,line width= 0.3pt,line join=round] ( 39.22,262.62) --
	(426.67,262.62);

\path[draw=drawColor,line width= 0.3pt,line join=round] ( 39.22,281.67) --
	(426.67,281.67);

\path[draw=drawColor,line width= 0.3pt,line join=round] ( 51.11, 32.09) --
	( 51.11,283.58);

\path[draw=drawColor,line width= 0.3pt,line join=round] ( 80.94, 32.09) --
	( 80.94,283.58);

\path[draw=drawColor,line width= 0.3pt,line join=round] (140.60, 32.09) --
	(140.60,283.58);

\path[draw=drawColor,line width= 0.3pt,line join=round] (200.26, 32.09) --
	(200.26,283.58);

\path[draw=drawColor,line width= 0.3pt,line join=round] (259.91, 32.09) --
	(259.91,283.58);

\path[draw=drawColor,line width= 0.3pt,line join=round] (319.57, 32.09) --
	(319.57,283.58);

\path[draw=drawColor,line width= 0.3pt,line join=round] (379.23, 32.09) --
	(379.23,283.58);

\path[draw=drawColor,line width= 0.6pt,line join=round] ( 39.22, 43.52) --
	(426.67, 43.52);

\path[draw=drawColor,line width= 0.6pt,line join=round] ( 39.22, 62.57) --
	(426.67, 62.57);

\path[draw=drawColor,line width= 0.6pt,line join=round] ( 39.22, 81.62) --
	(426.67, 81.62);

\path[draw=drawColor,line width= 0.6pt,line join=round] ( 39.22,100.68) --
	(426.67,100.68);

\path[draw=drawColor,line width= 0.6pt,line join=round] ( 39.22,119.73) --
	(426.67,119.73);

\path[draw=drawColor,line width= 0.6pt,line join=round] ( 39.22,138.78) --
	(426.67,138.78);

\path[draw=drawColor,line width= 0.6pt,line join=round] ( 39.22,157.83) --
	(426.67,157.83);

\path[draw=drawColor,line width= 0.6pt,line join=round] ( 39.22,176.89) --
	(426.67,176.89);

\path[draw=drawColor,line width= 0.6pt,line join=round] ( 39.22,195.94) --
	(426.67,195.94);

\path[draw=drawColor,line width= 0.6pt,line join=round] ( 39.22,214.99) --
	(426.67,214.99);

\path[draw=drawColor,line width= 0.6pt,line join=round] ( 39.22,234.04) --
	(426.67,234.04);

\path[draw=drawColor,line width= 0.6pt,line join=round] ( 39.22,253.10) --
	(426.67,253.10);

\path[draw=drawColor,line width= 0.6pt,line join=round] ( 39.22,272.15) --
	(426.67,272.15);

\path[draw=drawColor,line width= 0.6pt,line join=round] (110.77, 32.09) --
	(110.77,283.58);

\path[draw=drawColor,line width= 0.6pt,line join=round] (170.43, 32.09) --
	(170.43,283.58);

\path[draw=drawColor,line width= 0.6pt,line join=round] (230.08, 32.09) --
	(230.08,283.58);

\path[draw=drawColor,line width= 0.6pt,line join=round] (289.74, 32.09) --
	(289.74,283.58);

\path[draw=drawColor,line width= 0.6pt,line join=round] (349.40, 32.09) --
	(349.40,283.58);

\path[draw=drawColor,line width= 0.6pt,line join=round] (409.06, 32.09) --
	(409.06,283.58);
\definecolor{drawColor}{RGB}{0,0,0}

\path[draw=drawColor,line width= 0.6pt,line join=round] (250.85, 43.93) --
	(250.97, 44.45) --
	(251.08, 44.98) --
	(251.20, 45.50) --
	(251.32, 46.02) --
	(251.44, 46.55) --
	(251.56, 47.07) --
	(251.68, 47.60) --
	(251.80, 48.12) --
	(251.92, 48.65) --
	(252.04, 49.17) --
	(252.16, 49.70) --
	(252.28, 50.22) --
	(252.40, 50.74) --
	(252.52, 51.27) --
	(252.64, 51.79) --
	(252.76, 52.31) --
	(252.87, 52.84) --
	(252.99, 53.36) --
	(253.11, 53.88) --
	(253.23, 54.41) --
	(253.35, 54.93) --
	(253.47, 55.45) --
	(253.59, 55.97) --
	(253.71, 56.50) --
	(253.83, 57.02) --
	(253.95, 57.54) --
	(254.07, 58.06) --
	(254.19, 58.58) --
	(254.31, 59.10) --
	(254.43, 59.62) --
	(254.55, 60.15) --
	(254.66, 60.67) --
	(254.78, 61.19) --
	(254.90, 61.71) --
	(255.02, 62.23) --
	(255.14, 62.74) --
	(255.26, 63.26) --
	(255.38, 63.78) --
	(255.50, 64.30) --
	(255.62, 64.82) --
	(255.74, 65.34) --
	(255.86, 65.85) --
	(255.98, 66.37) --
	(256.10, 66.89) --
	(256.22, 67.41) --
	(256.33, 67.92) --
	(256.45, 68.44) --
	(256.57, 68.95) --
	(256.69, 69.47) --
	(256.81, 69.98) --
	(256.93, 70.50) --
	(257.05, 71.01) --
	(257.17, 71.53) --
	(257.29, 72.04) --
	(257.41, 72.55) --
	(257.53, 73.07) --
	(257.65, 73.58) --
	(257.77, 74.09) --
	(257.89, 74.60) --
	(258.01, 75.12) --
	(258.12, 75.63) --
	(258.24, 76.14) --
	(258.36, 76.65) --
	(258.48, 77.16) --
	(258.60, 77.67) --
	(258.72, 78.17) --
	(258.84, 78.68) --
	(258.96, 79.19) --
	(259.08, 79.70) --
	(259.20, 80.20) --
	(259.32, 80.71) --
	(259.44, 81.22) --
	(259.56, 81.72) --
	(259.68, 82.23) --
	(259.80, 82.73) --
	(259.91, 83.24) --
	(260.03, 83.74) --
	(260.15, 84.24) --
	(260.27, 84.75) --
	(260.39, 85.25) --
	(260.51, 85.75) --
	(260.63, 86.25) --
	(260.75, 86.75) --
	(260.87, 87.25) --
	(260.99, 87.75) --
	(261.11, 88.25) --
	(261.23, 88.75) --
	(261.35, 89.25) --
	(261.47, 89.75) --
	(261.58, 90.24) --
	(261.70, 90.74) --
	(261.82, 91.23) --
	(261.94, 91.73) --
	(262.06, 92.23) --
	(262.18, 92.72) --
	(262.30, 93.21) --
	(262.42, 93.71) --
	(262.54, 94.20) --
	(262.66, 94.69) --
	(262.78, 95.18) --
	(262.90, 95.67) --
	(263.02, 96.16) --
	(263.14, 96.65) --
	(263.26, 97.14) --
	(263.37, 97.63) --
	(263.49, 98.12) --
	(263.61, 98.60) --
	(263.73, 99.09) --
	(263.85, 99.58) --
	(263.97,100.06) --
	(264.09,100.55) --
	(264.21,101.03) --
	(264.33,101.51) --
	(264.45,101.99) --
	(264.57,102.48) --
	(264.69,102.96) --
	(264.81,103.44) --
	(264.93,103.92) --
	(265.05,104.40) --
	(265.16,104.88) --
	(265.28,105.35) --
	(265.40,105.83) --
	(265.52,106.31) --
	(265.64,106.78) --
	(265.76,107.26) --
	(265.88,107.73) --
	(266.00,108.21) --
	(266.12,108.68) --
	(266.24,109.15) --
	(266.36,109.63) --
	(266.48,110.10) --
	(266.60,110.57) --
	(266.72,111.04) --
	(266.83,111.51) --
	(266.95,111.97) --
	(267.07,112.44) --
	(267.19,112.91) --
	(267.31,113.38) --
	(267.43,113.84) --
	(267.55,114.31) --
	(267.67,114.77) --
	(267.79,115.23) --
	(267.91,115.69) --
	(268.03,116.16) --
	(268.15,116.62) --
	(268.27,117.08) --
	(268.39,117.54) --
	(268.51,118.00) --
	(268.62,118.45) --
	(268.74,118.91) --
	(268.86,119.37) --
	(268.98,119.82) --
	(269.10,120.28) --
	(269.22,120.73) --
	(269.34,121.19) --
	(269.46,121.64) --
	(269.58,122.09) --
	(269.70,122.54) --
	(269.82,122.99) --
	(269.94,123.44) --
	(270.06,123.89) --
	(270.18,124.34) --
	(270.30,124.79) --
	(270.41,125.23) --
	(270.53,125.68) --
	(270.65,126.12) --
	(270.77,126.57) --
	(270.89,127.01) --
	(271.01,127.45) --
	(271.13,127.89) --
	(271.25,128.33) --
	(271.37,128.77) --
	(271.49,129.21) --
	(271.61,129.65) --
	(271.73,130.09) --
	(271.85,130.53) --
	(271.97,130.96) --
	(272.09,131.40) --
	(272.20,131.83) --
	(272.32,132.27) --
	(272.44,132.70) --
	(272.56,133.13) --
	(272.68,133.56) --
	(272.80,133.99) --
	(272.92,134.42) --
	(273.04,134.85) --
	(273.16,135.28) --
	(273.28,135.70) --
	(273.40,136.13) --
	(273.52,136.55) --
	(273.64,136.98) --
	(273.76,137.40) --
	(273.87,137.82) --
	(273.99,138.25) --
	(274.11,138.67) --
	(274.23,139.09) --
	(274.35,139.51) --
	(274.47,139.93) --
	(274.59,140.34) --
	(274.71,140.76) --
	(274.83,141.18) --
	(274.95,141.59) --
	(275.07,142.00) --
	(275.19,142.42) --
	(275.31,142.83) --
	(275.43,143.24) --
	(275.55,143.65) --
	(275.66,144.06) --
	(275.78,144.47) --
	(275.90,144.88) --
	(276.02,145.29) --
	(276.14,145.69) --
	(276.26,146.10) --
	(276.38,146.50) --
	(276.50,146.91) --
	(276.62,147.31) --
	(276.74,147.71) --
	(276.86,148.11) --
	(276.98,148.51) --
	(277.10,148.91) --
	(277.22,149.31) --
	(277.34,149.71) --
	(277.45,150.11) --
	(277.57,150.50) --
	(277.69,150.90) --
	(277.81,151.29) --
	(277.93,151.68) --
	(278.05,152.08) --
	(278.17,152.47) --
	(278.29,152.86) --
	(278.41,153.25) --
	(278.53,153.64) --
	(278.65,154.03) --
	(278.77,154.41) --
	(278.89,154.80) --
	(279.01,155.18) --
	(279.12,155.57) --
	(279.24,155.95) --
	(279.36,156.34) --
	(279.48,156.72) --
	(279.60,157.10) --
	(279.72,157.48) --
	(279.84,157.86) --
	(279.96,158.24) --
	(280.08,158.61) --
	(280.20,158.99) --
	(280.32,159.37) --
	(280.44,159.74) --
	(280.56,160.11) --
	(280.68,160.49) --
	(280.80,160.86) --
	(280.91,161.23) --
	(281.03,161.60) --
	(281.15,161.97) --
	(281.27,162.34) --
	(281.39,162.71) --
	(281.51,163.07) --
	(281.63,163.44) --
	(281.75,163.80) --
	(281.87,164.17) --
	(281.99,164.53) --
	(282.11,164.89) --
	(282.23,165.25) --
	(282.35,165.61) --
	(282.47,165.97) --
	(282.59,166.33) --
	(282.70,166.69) --
	(282.82,167.05) --
	(282.94,167.40) --
	(283.06,167.76) --
	(283.18,168.11) --
	(283.30,168.47) --
	(283.42,168.82) --
	(283.54,169.17) --
	(283.66,169.52) --
	(283.78,169.87) --
	(283.90,170.22) --
	(284.02,170.57) --
	(284.14,170.92) --
	(284.26,171.26) --
	(284.37,171.61) --
	(284.49,171.95) --
	(284.61,172.30) --
	(284.73,172.64) --
	(284.85,172.98) --
	(284.97,173.32) --
	(285.09,173.66) --
	(285.21,174.00) --
	(285.33,174.34) --
	(285.45,174.68) --
	(285.57,175.01) --
	(285.69,175.35) --
	(285.81,175.68) --
	(285.93,176.02) --
	(286.05,176.35) --
	(286.16,176.68) --
	(286.28,177.01) --
	(286.40,177.34) --
	(286.52,177.67) --
	(286.64,178.00) --
	(286.76,178.33) --
	(286.88,178.66) --
	(287.00,178.98) --
	(287.12,179.31) --
	(287.24,179.63) --
	(287.36,179.96) --
	(287.48,180.28) --
	(287.60,180.60) --
	(287.72,180.92) --
	(287.84,181.24) --
	(287.95,181.56) --
	(288.07,181.88) --
	(288.19,182.19) --
	(288.31,182.51) --
	(288.43,182.83) --
	(288.55,183.14) --
	(288.67,183.46) --
	(288.79,183.77) --
	(288.91,184.08) --
	(289.03,184.39) --
	(289.15,184.70) --
	(289.27,185.01) --
	(289.39,185.32) --
	(289.51,185.63) --
	(289.62,185.93) --
	(289.74,186.24) --
	(289.86,186.55) --
	(289.98,186.85) --
	(290.10,187.15) --
	(290.22,187.46) --
	(290.34,187.76) --
	(290.46,188.06) --
	(290.58,188.36) --
	(290.70,188.66) --
	(290.82,188.96) --
	(290.94,189.25) --
	(291.06,189.55) --
	(291.18,189.85) --
	(291.30,190.14) --
	(291.41,190.44) --
	(291.53,190.73) --
	(291.65,191.02) --
	(291.77,191.31) --
	(291.89,191.60) --
	(292.01,191.89) --
	(292.13,192.18) --
	(292.25,192.47) --
	(292.37,192.76) --
	(292.49,193.04) --
	(292.61,193.33) --
	(292.73,193.62) --
	(292.85,193.90) --
	(292.97,194.18) --
	(293.09,194.47) --
	(293.20,194.75) --
	(293.32,195.03) --
	(293.44,195.31) --
	(293.56,195.59) --
	(293.68,195.86) --
	(293.80,196.14) --
	(293.92,196.42) --
	(294.04,196.69) --
	(294.16,196.97) --
	(294.28,197.24) --
	(294.40,197.52) --
	(294.52,197.79) --
	(294.64,198.06) --
	(294.76,198.33) --
	(294.87,198.60) --
	(294.99,198.87) --
	(295.11,199.14) --
	(295.23,199.41) --
	(295.35,199.67) --
	(295.47,199.94) --
	(295.59,200.21) --
	(295.71,200.47) --
	(295.83,200.73) --
	(295.95,201.00) --
	(296.07,201.26) --
	(296.19,201.52) --
	(296.31,201.78) --
	(296.43,202.04) --
	(296.55,202.30) --
	(296.66,202.56) --
	(296.78,202.81) --
	(296.90,203.07) --
	(297.02,203.33) --
	(297.14,203.58) --
	(297.26,203.83) --
	(297.38,204.09) --
	(297.50,204.34) --
	(297.62,204.59) --
	(297.74,204.84) --
	(297.86,205.09) --
	(297.98,205.34) --
	(298.10,205.59) --
	(298.22,205.84) --
	(298.34,206.09) --
	(298.45,206.33) --
	(298.57,206.58) --
	(298.69,206.82) --
	(298.81,207.07) --
	(298.93,207.31) --
	(299.05,207.55) --
	(299.17,207.79) --
	(299.29,208.04) --
	(299.41,208.28) --
	(299.53,208.52) --
	(299.65,208.75) --
	(299.77,208.99) --
	(299.89,209.23) --
	(300.01,209.47) --
	(300.12,209.70) --
	(300.24,209.94) --
	(300.36,210.17) --
	(300.48,210.40) --
	(300.60,210.64) --
	(300.72,210.87) --
	(300.84,211.10) --
	(300.96,211.33) --
	(301.08,211.56) --
	(301.20,211.79) --
	(301.32,212.02) --
	(301.44,212.24) --
	(301.56,212.47) --
	(301.68,212.70) --
	(301.80,212.92) --
	(301.91,213.15) --
	(302.03,213.37) --
	(302.15,213.59) --
	(302.27,213.82) --
	(302.39,214.04) --
	(302.51,214.26) --
	(302.63,214.48) --
	(302.75,214.70) --
	(302.87,214.92) --
	(302.99,215.14) --
	(303.11,215.35) --
	(303.23,215.57) --
	(303.35,215.79) --
	(303.47,216.00) --
	(303.59,216.22) --
	(303.70,216.43) --
	(303.82,216.64) --
	(303.94,216.85) --
	(304.06,217.07) --
	(304.18,217.28) --
	(304.30,217.49) --
	(304.42,217.70) --
	(304.54,217.91) --
	(304.66,218.11) --
	(304.78,218.32) --
	(304.90,218.53) --
	(305.02,218.73) --
	(305.14,218.94) --
	(305.26,219.15) --
	(305.37,219.35) --
	(305.49,219.55) --
	(305.61,219.76) --
	(305.73,219.96) --
	(305.85,220.16) --
	(305.97,220.36) --
	(306.09,220.56) --
	(306.21,220.76) --
	(306.33,220.96) --
	(306.45,221.16) --
	(306.57,221.35) --
	(306.69,221.55) --
	(306.81,221.75) --
	(306.93,221.94) --
	(307.05,222.14) --
	(307.16,222.33) --
	(307.28,222.52) --
	(307.40,222.72) --
	(307.52,222.91) --
	(307.64,223.10) --
	(307.76,223.29) --
	(307.88,223.48) --
	(308.00,223.67) --
	(308.12,223.86) --
	(308.24,224.05) --
	(308.36,224.23) --
	(308.48,224.42) --
	(308.60,224.61) --
	(308.72,224.79) --
	(308.84,224.98) --
	(308.95,225.16) --
	(309.07,225.35) --
	(309.19,225.53) --
	(309.31,225.71) --
	(309.43,225.89) --
	(309.55,226.07) --
	(309.67,226.26) --
	(309.79,226.44) --
	(309.91,226.61) --
	(310.03,226.79) --
	(310.15,226.97) --
	(310.27,227.15) --
	(310.39,227.33) --
	(310.51,227.50) --
	(310.63,227.68) --
	(310.74,227.85) --
	(310.86,228.03) --
	(310.98,228.20) --
	(311.10,228.37) --
	(311.22,228.55) --
	(311.34,228.72) --
	(311.46,228.89) --
	(311.58,229.06) --
	(311.70,229.23) --
	(311.82,229.40) --
	(311.94,229.57) --
	(312.06,229.74) --
	(312.18,229.91) --
	(312.30,230.07) --
	(312.41,230.24) --
	(312.53,230.41) --
	(312.65,230.57) --
	(312.77,230.74) --
	(312.89,230.90) --
	(313.01,231.06) --
	(313.13,231.23) --
	(313.25,231.39) --
	(313.37,231.55) --
	(313.49,231.71) --
	(313.61,231.87) --
	(313.73,232.03) --
	(313.85,232.19) --
	(313.97,232.35) --
	(314.09,232.51) --
	(314.20,232.67) --
	(314.32,232.83) --
	(314.44,232.98) --
	(314.56,233.14) --
	(314.68,233.30) --
	(314.80,233.45) --
	(314.92,233.61) --
	(315.04,233.76) --
	(315.16,233.91) --
	(315.28,234.07) --
	(315.40,234.22) --
	(315.52,234.37) --
	(315.64,234.52) --
	(315.76,234.67) --
	(315.88,234.82) --
	(315.99,234.97) --
	(316.11,235.12) --
	(316.23,235.27) --
	(316.35,235.42) --
	(316.47,235.57) --
	(316.59,235.71) --
	(316.71,235.86) --
	(316.83,236.00) --
	(316.95,236.15) --
	(317.07,236.30) --
	(317.19,236.44) --
	(317.31,236.58) --
	(317.43,236.73) --
	(317.55,236.87) --
	(317.66,237.01) --
	(317.78,237.15) --
	(317.90,237.29) --
	(318.02,237.44) --
	(318.14,237.58) --
	(318.26,237.72) --
	(318.38,237.85) --
	(318.50,237.99) --
	(318.62,238.13) --
	(318.74,238.27) --
	(318.86,238.41) --
	(318.98,238.54) --
	(319.10,238.68) --
	(319.22,238.81) --
	(319.34,238.95) --
	(319.45,239.08) --
	(319.57,239.22) --
	(319.69,239.35) --
	(319.81,239.49) --
	(319.93,239.62) --
	(320.05,239.75) --
	(320.17,239.88) --
	(320.29,240.01) --
	(320.41,240.14) --
	(320.53,240.27) --
	(320.65,240.40) --
	(320.77,240.53) --
	(320.89,240.66) --
	(321.01,240.79) --
	(321.13,240.92) --
	(321.24,241.05) --
	(321.36,241.17) --
	(321.48,241.30) --
	(321.60,241.43) --
	(321.72,241.55) --
	(321.84,241.68) --
	(321.96,241.80) --
	(322.08,241.92) --
	(322.20,242.05) --
	(322.32,242.17) --
	(322.44,242.29) --
	(322.56,242.42) --
	(322.68,242.54) --
	(322.80,242.66) --
	(322.91,242.78) --
	(323.03,242.90) --
	(323.15,243.02) --
	(323.27,243.14) --
	(323.39,243.26) --
	(323.51,243.38) --
	(323.63,243.50) --
	(323.75,243.61) --
	(323.87,243.73) --
	(323.99,243.85) --
	(324.11,243.96) --
	(324.23,244.08) --
	(324.35,244.20) --
	(324.47,244.31) --
	(324.59,244.43) --
	(324.70,244.54) --
	(324.82,244.65) --
	(324.94,244.77) --
	(325.06,244.88) --
	(325.18,244.99) --
	(325.30,245.10) --
	(325.42,245.22) --
	(325.54,245.33) --
	(325.66,245.44) --
	(325.78,245.55) --
	(325.90,245.66) --
	(326.02,245.77) --
	(326.14,245.88) --
	(326.26,245.99) --
	(326.38,246.09) --
	(326.49,246.20) --
	(326.61,246.31) --
	(326.73,246.42) --
	(326.85,246.52) --
	(326.97,246.63) --
	(327.09,246.74) --
	(327.21,246.84) --
	(327.33,246.95) --
	(327.45,247.05) --
	(327.57,247.15) --
	(327.69,247.26) --
	(327.81,247.36) --
	(327.93,247.47) --
	(328.05,247.57) --
	(328.16,247.67) --
	(328.28,247.77) --
	(328.40,247.87) --
	(328.52,247.97) --
	(328.64,248.08) --
	(328.76,248.18) --
	(328.88,248.28) --
	(329.00,248.38) --
	(329.12,248.47) --
	(329.24,248.57) --
	(329.36,248.67) --
	(329.48,248.77) --
	(329.60,248.87) --
	(329.72,248.96) --
	(329.84,249.06) --
	(329.95,249.16) --
	(330.07,249.25) --
	(330.19,249.35) --
	(330.31,249.45) --
	(330.43,249.54) --
	(330.55,249.63) --
	(330.67,249.73) --
	(330.79,249.82) --
	(330.91,249.92) --
	(331.03,250.01) --
	(331.15,250.10) --
	(331.27,250.19) --
	(331.39,250.29) --
	(331.51,250.38) --
	(331.63,250.47) --
	(331.74,250.56) --
	(331.86,250.65) --
	(331.98,250.74) --
	(332.10,250.83) --
	(332.22,250.92) --
	(332.34,251.01) --
	(332.46,251.10) --
	(332.58,251.19) --
	(332.70,251.28) --
	(332.82,251.36) --
	(332.94,251.45) --
	(333.06,251.54) --
	(333.18,251.63) --
	(333.30,251.71) --
	(333.41,251.80) --
	(333.53,251.88) --
	(333.65,251.97) --
	(333.77,252.06) --
	(333.89,252.14) --
	(334.01,252.22) --
	(334.13,252.31) --
	(334.25,252.39) --
	(334.37,252.48) --
	(334.49,252.56) --
	(334.61,252.64) --
	(334.73,252.72) --
	(334.85,252.81) --
	(334.97,252.89) --
	(335.09,252.97) --
	(335.20,253.05) --
	(335.32,253.13) --
	(335.44,253.21) --
	(335.56,253.29) --
	(335.68,253.37) --
	(335.80,253.45) --
	(335.92,253.53) --
	(336.04,253.61) --
	(336.16,253.69) --
	(336.28,253.77) --
	(336.40,253.85) --
	(336.52,253.92) --
	(336.64,254.00) --
	(336.76,254.08) --
	(336.88,254.15) --
	(336.99,254.23) --
	(337.11,254.31) --
	(337.23,254.38) --
	(337.35,254.46) --
	(337.47,254.53) --
	(337.59,254.61) --
	(337.71,254.68) --
	(337.83,254.76) --
	(337.95,254.83) --
	(338.07,254.91) --
	(338.19,254.98) --
	(338.31,255.05) --
	(338.43,255.13) --
	(338.55,255.20) --
	(338.66,255.27) --
	(338.78,255.34) --
	(338.90,255.41) --
	(339.02,255.49) --
	(339.14,255.56) --
	(339.26,255.63) --
	(339.38,255.70) --
	(339.50,255.77) --
	(339.62,255.84) --
	(339.74,255.91) --
	(339.86,255.98) --
	(339.98,256.05) --
	(340.10,256.12) --
	(340.22,256.19) --
	(340.34,256.25) --
	(340.45,256.32) --
	(340.57,256.39) --
	(340.69,256.46) --
	(340.81,256.52) --
	(340.93,256.59) --
	(341.05,256.66) --
	(341.17,256.73) --
	(341.29,256.79) --
	(341.41,256.86) --
	(341.53,256.92) --
	(341.65,256.99) --
	(341.77,257.05) --
	(341.89,257.12) --
	(342.01,257.18) --
	(342.13,257.25) --
	(342.24,257.31) --
	(342.36,257.38) --
	(342.48,257.44) --
	(342.60,257.50) --
	(342.72,257.57) --
	(342.84,257.63) --
	(342.96,257.69) --
	(343.08,257.75) --
	(343.20,257.81) --
	(343.32,257.88) --
	(343.44,257.94) --
	(343.56,258.00) --
	(343.68,258.06) --
	(343.80,258.12) --
	(343.91,258.18) --
	(344.03,258.24) --
	(344.15,258.30) --
	(344.27,258.36) --
	(344.39,258.42) --
	(344.51,258.48) --
	(344.63,258.54) --
	(344.75,258.60) --
	(344.87,258.66) --
	(344.99,258.72) --
	(345.11,258.77) --
	(345.23,258.83) --
	(345.35,258.89) --
	(345.47,258.95) --
	(345.59,259.00) --
	(345.70,259.06) --
	(345.82,259.12) --
	(345.94,259.17) --
	(346.06,259.23) --
	(346.18,259.29) --
	(346.30,259.34) --
	(346.42,259.40) --
	(346.54,259.45) --
	(346.66,259.51) --
	(346.78,259.56) --
	(346.90,259.62) --
	(347.02,259.67) --
	(347.14,259.73) --
	(347.26,259.78) --
	(347.38,259.83) --
	(347.49,259.89) --
	(347.61,259.94) --
	(347.73,259.99) --
	(347.85,260.05) --
	(347.97,260.10) --
	(348.09,260.15) --
	(348.21,260.20) --
	(348.33,260.26) --
	(348.45,260.31) --
	(348.57,260.36) --
	(348.69,260.41) --
	(348.81,260.46) --
	(348.93,260.51) --
	(349.05,260.56) --
	(349.17,260.61) --
	(349.28,260.66) --
	(349.40,260.71) --
	(349.52,260.76) --
	(349.64,260.81) --
	(349.76,260.86) --
	(349.88,260.91) --
	(350.00,260.96) --
	(350.12,261.01) --
	(350.24,261.06) --
	(350.36,261.11) --
	(350.48,261.16) --
	(350.60,261.20) --
	(350.72,261.25) --
	(350.84,261.30) --
	(350.95,261.35) --
	(351.07,261.39) --
	(351.19,261.44) --
	(351.31,261.49) --
	(351.43,261.54) --
	(351.55,261.58) --
	(351.67,261.63) --
	(351.79,261.67) --
	(351.91,261.72) --
	(352.03,261.77) --
	(352.15,261.81) --
	(352.27,261.86) --
	(352.39,261.90) --
	(352.51,261.95) --
	(352.63,261.99) --
	(352.74,262.04) --
	(352.86,262.08) --
	(352.98,262.12) --
	(353.10,262.17) --
	(353.22,262.21) --
	(353.34,262.26) --
	(353.46,262.30) --
	(353.58,262.34) --
	(353.70,262.39) --
	(353.82,262.43) --
	(353.94,262.47) --
	(354.06,262.51) --
	(354.18,262.56) --
	(354.30,262.60) --
	(354.42,262.64) --
	(354.53,262.68) --
	(354.65,262.73) --
	(354.77,262.77) --
	(354.89,262.81) --
	(355.01,262.85) --
	(355.13,262.89) --
	(355.25,262.93) --
	(355.37,262.97) --
	(355.49,263.01) --
	(355.61,263.05) --
	(355.73,263.09) --
	(355.85,263.13) --
	(355.97,263.17) --
	(356.09,263.21) --
	(356.20,263.25) --
	(356.32,263.29) --
	(356.44,263.33) --
	(356.56,263.37) --
	(356.68,263.41) --
	(356.80,263.45) --
	(356.92,263.48) --
	(357.04,263.52) --
	(357.16,263.56) --
	(357.28,263.60) --
	(357.40,263.64) --
	(357.52,263.67) --
	(357.64,263.71) --
	(357.76,263.75) --
	(357.88,263.79) --
	(357.99,263.82) --
	(358.11,263.86) --
	(358.23,263.90) --
	(358.35,263.93) --
	(358.47,263.97) --
	(358.59,264.01) --
	(358.71,264.04) --
	(358.83,264.08) --
	(358.95,264.11) --
	(359.07,264.15) --
	(359.19,264.19) --
	(359.31,264.22) --
	(359.43,264.26) --
	(359.55,264.29) --
	(359.67,264.33) --
	(359.78,264.36) --
	(359.90,264.40) --
	(360.02,264.43) --
	(360.14,264.46) --
	(360.26,264.50) --
	(360.38,264.53) --
	(360.50,264.57) --
	(360.62,264.60) --
	(360.74,264.63) --
	(360.86,264.67) --
	(360.98,264.70) --
	(361.10,264.73) --
	(361.22,264.77) --
	(361.34,264.80) --
	(361.45,264.83) --
	(361.57,264.87) --
	(361.69,264.90) --
	(361.81,264.93) --
	(361.93,264.96) --
	(362.05,264.99) --
	(362.17,265.03) --
	(362.29,265.06) --
	(362.41,265.09) --
	(362.53,265.12) --
	(362.65,265.15) --
	(362.77,265.18) --
	(362.89,265.21) --
	(363.01,265.25) --
	(363.13,265.28) --
	(363.24,265.31) --
	(363.36,265.34) --
	(363.48,265.37) --
	(363.60,265.40) --
	(363.72,265.43) --
	(363.84,265.46) --
	(363.96,265.49) --
	(364.08,265.52) --
	(364.20,265.55) --
	(364.32,265.58) --
	(364.44,265.61) --
	(364.56,265.64) --
	(364.68,265.67) --
	(364.80,265.69) --
	(364.92,265.72) --
	(365.03,265.75) --
	(365.15,265.78) --
	(365.27,265.81) --
	(365.39,265.84) --
	(365.51,265.87) --
	(365.63,265.89) --
	(365.75,265.92) --
	(365.87,265.95) --
	(365.99,265.98) --
	(366.11,266.01) --
	(366.23,266.03) --
	(366.35,266.06) --
	(366.47,266.09) --
	(366.59,266.12) --
	(366.70,266.14) --
	(366.82,266.17) --
	(366.94,266.20) --
	(367.06,266.22) --
	(367.18,266.25) --
	(367.30,266.28) --
	(367.42,266.30) --
	(367.54,266.33) --
	(367.66,266.36) --
	(367.78,266.38) --
	(367.90,266.41) --
	(368.02,266.43) --
	(368.14,266.46) --
	(368.26,266.48) --
	(368.38,266.51) --
	(368.49,266.54) --
	(368.61,266.56) --
	(368.73,266.59) --
	(368.85,266.61) --
	(368.97,266.64) --
	(369.09,266.66) --
	(369.21,266.69) --
	(369.33,266.71) --
	(369.45,266.73) --
	(369.57,266.76) --
	(369.69,266.78) --
	(369.81,266.81) --
	(369.93,266.83) --
	(370.05,266.86) --
	(370.17,266.88) --
	(370.28,266.90) --
	(370.40,266.93) --
	(370.52,266.95) --
	(370.64,266.97) --
	(370.76,267.00) --
	(370.88,267.02) --
	(371.00,267.04) --
	(371.12,267.07) --
	(371.24,267.09) --
	(371.36,267.11) --
	(371.48,267.14) --
	(371.60,267.16) --
	(371.72,267.18) --
	(371.84,267.20) --
	(371.95,267.23) --
	(372.07,267.25) --
	(372.19,267.27) --
	(372.31,267.29) --
	(372.43,267.31) --
	(372.55,267.34) --
	(372.67,267.36) --
	(372.79,267.38) --
	(372.91,267.40) --
	(373.03,267.42) --
	(373.15,267.44) --
	(373.27,267.47) --
	(373.39,267.49) --
	(373.51,267.51) --
	(373.63,267.53) --
	(373.74,267.55) --
	(373.86,267.57) --
	(373.98,267.59) --
	(374.10,267.61) --
	(374.22,267.63) --
	(374.34,267.65) --
	(374.46,267.67) --
	(374.58,267.69) --
	(374.70,267.71) --
	(374.82,267.74) --
	(374.94,267.76) --
	(375.06,267.78) --
	(375.18,267.80) --
	(375.30,267.82) --
	(375.42,267.83) --
	(375.53,267.85) --
	(375.65,267.87) --
	(375.77,267.89) --
	(375.89,267.91) --
	(376.01,267.93) --
	(376.13,267.95) --
	(376.25,267.97) --
	(376.37,267.99) --
	(376.49,268.01) --
	(376.61,268.03) --
	(376.73,268.05) --
	(376.85,268.06) --
	(376.97,268.08) --
	(377.09,268.10) --
	(377.20,268.12) --
	(377.32,268.14) --
	(377.44,268.16) --
	(377.56,268.18) --
	(377.68,268.19) --
	(377.80,268.21) --
	(377.92,268.23) --
	(378.04,268.25) --
	(378.16,268.27) --
	(378.28,268.28) --
	(378.40,268.30) --
	(378.52,268.32) --
	(378.64,268.34) --
	(378.76,268.35) --
	(378.88,268.37) --
	(378.99,268.39) --
	(379.11,268.41) --
	(379.23,268.42) --
	(379.35,268.44) --
	(379.47,268.46) --
	(379.59,268.47) --
	(379.71,268.49) --
	(379.83,268.51) --
	(379.95,268.52) --
	(380.07,268.54) --
	(380.19,268.56) --
	(380.31,268.57) --
	(380.43,268.59) --
	(380.55,268.61) --
	(380.67,268.62) --
	(380.78,268.64) --
	(380.90,268.65) --
	(381.02,268.67) --
	(381.14,268.69) --
	(381.26,268.70) --
	(381.38,268.72) --
	(381.50,268.73) --
	(381.62,268.75) --
	(381.74,268.76) --
	(381.86,268.78) --
	(381.98,268.80) --
	(382.10,268.81) --
	(382.22,268.83) --
	(382.34,268.84) --
	(382.46,268.86) --
	(382.57,268.87) --
	(382.69,268.89) --
	(382.81,268.90) --
	(382.93,268.92) --
	(383.05,268.93) --
	(383.17,268.95) --
	(383.29,268.96) --
	(383.41,268.98) --
	(383.53,268.99) --
	(383.65,269.01) --
	(383.77,269.02) --
	(383.89,269.03) --
	(384.01,269.05) --
	(384.13,269.06) --
	(384.24,269.08) --
	(384.36,269.09) --
	(384.48,269.11) --
	(384.60,269.12) --
	(384.72,269.13) --
	(384.84,269.15) --
	(384.96,269.16) --
	(385.08,269.17) --
	(385.20,269.19) --
	(385.32,269.20) --
	(385.44,269.22) --
	(385.56,269.23) --
	(385.68,269.24) --
	(385.80,269.26) --
	(385.92,269.27) --
	(386.03,269.28) --
	(386.15,269.30) --
	(386.27,269.31) --
	(386.39,269.32) --
	(386.51,269.34) --
	(386.63,269.35) --
	(386.75,269.36) --
	(386.87,269.37) --
	(386.99,269.39) --
	(387.11,269.40) --
	(387.23,269.41) --
	(387.35,269.42) --
	(387.47,269.44) --
	(387.59,269.45) --
	(387.71,269.46) --
	(387.82,269.48) --
	(387.94,269.49) --
	(388.06,269.50) --
	(388.18,269.51) --
	(388.30,269.52) --
	(388.42,269.54) --
	(388.54,269.55) --
	(388.66,269.56) --
	(388.78,269.57) --
	(388.90,269.58) --
	(389.02,269.60) --
	(389.14,269.61) --
	(389.26,269.62) --
	(389.38,269.63) --
	(389.49,269.64) --
	(389.61,269.65) --
	(389.73,269.67) --
	(389.85,269.68) --
	(389.97,269.69) --
	(390.09,269.70) --
	(390.21,269.71) --
	(390.33,269.72) --
	(390.45,269.73) --
	(390.57,269.75) --
	(390.69,269.76) --
	(390.81,269.77) --
	(390.93,269.78) --
	(391.05,269.79) --
	(391.17,269.80) --
	(391.28,269.81) --
	(391.40,269.82) --
	(391.52,269.83) --
	(391.64,269.84) --
	(391.76,269.85) --
	(391.88,269.87) --
	(392.00,269.88) --
	(392.12,269.89) --
	(392.24,269.90) --
	(392.36,269.91) --
	(392.48,269.92) --
	(392.60,269.93) --
	(392.72,269.94) --
	(392.84,269.95) --
	(392.96,269.96) --
	(393.07,269.97) --
	(393.19,269.98) --
	(393.31,269.99) --
	(393.43,270.00) --
	(393.55,270.01) --
	(393.67,270.02) --
	(393.79,270.03) --
	(393.91,270.04) --
	(394.03,270.05) --
	(394.15,270.06) --
	(394.27,270.07) --
	(394.39,270.08) --
	(394.51,270.09) --
	(394.63,270.10) --
	(394.74,270.11) --
	(394.86,270.12) --
	(394.98,270.13) --
	(395.10,270.14) --
	(395.22,270.15) --
	(395.34,270.15) --
	(395.46,270.16) --
	(395.58,270.17) --
	(395.70,270.18) --
	(395.82,270.19) --
	(395.94,270.20) --
	(396.06,270.21) --
	(396.18,270.22) --
	(396.30,270.23) --
	(396.42,270.24) --
	(396.53,270.25) --
	(396.65,270.26) --
	(396.77,270.26) --
	(396.89,270.27) --
	(397.01,270.28) --
	(397.13,270.29) --
	(397.25,270.30) --
	(397.37,270.31) --
	(397.49,270.32) --
	(397.61,270.32) --
	(397.73,270.33) --
	(397.85,270.34) --
	(397.97,270.35) --
	(398.09,270.36) --
	(398.21,270.37) --
	(398.32,270.38) --
	(398.44,270.38) --
	(398.56,270.39) --
	(398.68,270.40) --
	(398.80,270.41) --
	(398.92,270.42) --
	(399.04,270.42) --
	(399.16,270.43) --
	(399.28,270.44) --
	(399.40,270.45) --
	(399.52,270.46) --
	(399.64,270.46) --
	(399.76,270.47) --
	(399.88,270.48) --
	(399.99,270.49) --
	(400.11,270.50) --
	(400.23,270.50) --
	(400.35,270.51) --
	(400.47,270.52) --
	(400.59,270.53) --
	(400.71,270.53) --
	(400.83,270.54) --
	(400.95,270.55) --
	(401.07,270.56) --
	(401.19,270.56) --
	(401.31,270.57) --
	(401.43,270.58) --
	(401.55,270.59) --
	(401.67,270.59) --
	(401.78,270.60) --
	(401.90,270.61) --
	(402.02,270.62) --
	(402.14,270.62) --
	(402.26,270.63) --
	(402.38,270.64) --
	(402.50,270.64) --
	(402.62,270.65) --
	(402.74,270.66) --
	(402.86,270.67) --
	(402.98,270.67) --
	(403.10,270.68) --
	(403.22,270.69) --
	(403.34,270.69) --
	(403.46,270.70) --
	(403.57,270.71) --
	(403.69,270.71) --
	(403.81,270.72) --
	(403.93,270.73) --
	(404.05,270.73) --
	(404.17,270.74) --
	(404.29,270.75) --
	(404.41,270.75) --
	(404.53,270.76) --
	(404.65,270.77) --
	(404.77,270.77) --
	(404.89,270.78) --
	(405.01,270.79) --
	(405.13,270.79) --
	(405.24,270.80) --
	(405.36,270.81) --
	(405.48,270.81) --
	(405.60,270.82) --
	(405.72,270.82) --
	(405.84,270.83) --
	(405.96,270.84) --
	(406.08,270.84) --
	(406.20,270.85) --
	(406.32,270.86) --
	(406.44,270.86) --
	(406.56,270.87) --
	(406.68,270.87) --
	(406.80,270.88) --
	(406.92,270.89) --
	(407.03,270.89) --
	(407.15,270.90) --
	(407.27,270.90) --
	(407.39,270.91) --
	(407.51,270.92) --
	(407.63,270.92) --
	(407.75,270.93) --
	(407.87,270.93) --
	(407.99,270.94) --
	(408.11,270.94) --
	(408.23,270.95) --
	(408.35,270.96) --
	(408.47,270.96) --
	(408.59,270.97) --
	(408.71,270.97) --
	(408.82,270.98) --
	(408.94,270.98) --
	(409.06,270.99);

\path[draw=drawColor,line width= 0.6pt,dash pattern=on 2pt off 2pt ,line join=round] ( 56.83, 56.49) --
	( 56.95, 56.78) --
	( 57.07, 57.08) --
	( 57.19, 57.38) --
	( 57.31, 57.67) --
	( 57.43, 57.97) --
	( 57.55, 58.27) --
	( 57.67, 58.56) --
	( 57.79, 58.86) --
	( 57.91, 59.16) --
	( 58.03, 59.45) --
	( 58.15, 59.75) --
	( 58.27, 60.05) --
	( 58.38, 60.34) --
	( 58.50, 60.64) --
	( 58.62, 60.93) --
	( 58.74, 61.23) --
	( 58.86, 61.53) --
	( 58.98, 61.82) --
	( 59.10, 62.12) --
	( 59.22, 62.42) --
	( 59.34, 62.71) --
	( 59.46, 63.01) --
	( 59.58, 63.31) --
	( 59.70, 63.60) --
	( 59.82, 63.90) --
	( 59.94, 64.19) --
	( 60.06, 64.49) --
	( 60.17, 64.79) --
	( 60.29, 65.08) --
	( 60.41, 65.38) --
	( 60.53, 65.67) --
	( 60.65, 65.97) --
	( 60.77, 66.27) --
	( 60.89, 66.56) --
	( 61.01, 66.86) --
	( 61.13, 67.15) --
	( 61.25, 67.45) --
	( 61.37, 67.75) --
	( 61.49, 68.04) --
	( 61.61, 68.34) --
	( 61.73, 68.63) --
	( 61.85, 68.93) --
	( 61.96, 69.22) --
	( 62.08, 69.52) --
	( 62.20, 69.81) --
	( 62.32, 70.11) --
	( 62.44, 70.41) --
	( 62.56, 70.70) --
	( 62.68, 71.00) --
	( 62.80, 71.29) --
	( 62.92, 71.59) --
	( 63.04, 71.88) --
	( 63.16, 72.18) --
	( 63.28, 72.47) --
	( 63.40, 72.77) --
	( 63.52, 73.06) --
	( 63.63, 73.36) --
	( 63.75, 73.65) --
	( 63.87, 73.94) --
	( 63.99, 74.24) --
	( 64.11, 74.53) --
	( 64.23, 74.83) --
	( 64.35, 75.12) --
	( 64.47, 75.42) --
	( 64.59, 75.71) --
	( 64.71, 76.00) --
	( 64.83, 76.30) --
	( 64.95, 76.59) --
	( 65.07, 76.89) --
	( 65.19, 77.18) --
	( 65.31, 77.47) --
	( 65.42, 77.77) --
	( 65.54, 78.06) --
	( 65.66, 78.36) --
	( 65.78, 78.65) --
	( 65.90, 78.94) --
	( 66.02, 79.24) --
	( 66.14, 79.53) --
	( 66.26, 79.82) --
	( 66.38, 80.11) --
	( 66.50, 80.41) --
	( 66.62, 80.70) --
	( 66.74, 80.99) --
	( 66.86, 81.29) --
	( 66.98, 81.58) --
	( 67.10, 81.87) --
	( 67.21, 82.16) --
	( 67.33, 82.46) --
	( 67.45, 82.75) --
	( 67.57, 83.04) --
	( 67.69, 83.33) --
	( 67.81, 83.62) --
	( 67.93, 83.92) --
	( 68.05, 84.21) --
	( 68.17, 84.50) --
	( 68.29, 84.79) --
	( 68.41, 85.08) --
	( 68.53, 85.37) --
	( 68.65, 85.66) --
	( 68.77, 85.96) --
	( 68.88, 86.25) --
	( 69.00, 86.54) --
	( 69.12, 86.83) --
	( 69.24, 87.12) --
	( 69.36, 87.41) --
	( 69.48, 87.70) --
	( 69.60, 87.99) --
	( 69.72, 88.28) --
	( 69.84, 88.57) --
	( 69.96, 88.86) --
	( 70.08, 89.15) --
	( 70.20, 89.44) --
	( 70.32, 89.73) --
	( 70.44, 90.02) --
	( 70.56, 90.31) --
	( 70.67, 90.60) --
	( 70.79, 90.89) --
	( 70.91, 91.18) --
	( 71.03, 91.47) --
	( 71.15, 91.75) --
	( 71.27, 92.04) --
	( 71.39, 92.33) --
	( 71.51, 92.62) --
	( 71.63, 92.91) --
	( 71.75, 93.20) --
	( 71.87, 93.48) --
	( 71.99, 93.77) --
	( 72.11, 94.06) --
	( 72.23, 94.35) --
	( 72.35, 94.64) --
	( 72.46, 94.92) --
	( 72.58, 95.21) --
	( 72.70, 95.50) --
	( 72.82, 95.78) --
	( 72.94, 96.07) --
	( 73.06, 96.36) --
	( 73.18, 96.64) --
	( 73.30, 96.93) --
	( 73.42, 97.22) --
	( 73.54, 97.50) --
	( 73.66, 97.79) --
	( 73.78, 98.07) --
	( 73.90, 98.36) --
	( 74.02, 98.65) --
	( 74.13, 98.93) --
	( 74.25, 99.22) --
	( 74.37, 99.50) --
	( 74.49, 99.79) --
	( 74.61,100.07) --
	( 74.73,100.36) --
	( 74.85,100.64) --
	( 74.97,100.92) --
	( 75.09,101.21) --
	( 75.21,101.49) --
	( 75.33,101.78) --
	( 75.45,102.06) --
	( 75.57,102.34) --
	( 75.69,102.63) --
	( 75.81,102.91) --
	( 75.92,103.19) --
	( 76.04,103.48) --
	( 76.16,103.76) --
	( 76.28,104.04) --
	( 76.40,104.32) --
	( 76.52,104.61) --
	( 76.64,104.89) --
	( 76.76,105.17) --
	( 76.88,105.45) --
	( 77.00,105.73) --
	( 77.12,106.01) --
	( 77.24,106.30) --
	( 77.36,106.58) --
	( 77.48,106.86) --
	( 77.60,107.14) --
	( 77.71,107.42) --
	( 77.83,107.70) --
	( 77.95,107.98) --
	( 78.07,108.26) --
	( 78.19,108.54) --
	( 78.31,108.82) --
	( 78.43,109.10) --
	( 78.55,109.38) --
	( 78.67,109.66) --
	( 78.79,109.94) --
	( 78.91,110.21) --
	( 79.03,110.49) --
	( 79.15,110.77) --
	( 79.27,111.05) --
	( 79.38,111.33) --
	( 79.50,111.61) --
	( 79.62,111.88) --
	( 79.74,112.16) --
	( 79.86,112.44) --
	( 79.98,112.72) --
	( 80.10,112.99) --
	( 80.22,113.27) --
	( 80.34,113.55) --
	( 80.46,113.82) --
	( 80.58,114.10) --
	( 80.70,114.37) --
	( 80.82,114.65) --
	( 80.94,114.93) --
	( 81.06,115.20) --
	( 81.17,115.48) --
	( 81.29,115.75) --
	( 81.41,116.03) --
	( 81.53,116.30) --
	( 81.65,116.58) --
	( 81.77,116.85) --
	( 81.89,117.12) --
	( 82.01,117.40) --
	( 82.13,117.67) --
	( 82.25,117.94) --
	( 82.37,118.22) --
	( 82.49,118.49) --
	( 82.61,118.76) --
	( 82.73,119.04) --
	( 82.85,119.31) --
	( 82.96,119.58) --
	( 83.08,119.85) --
	( 83.20,120.12) --
	( 83.32,120.40) --
	( 83.44,120.67) --
	( 83.56,120.94) --
	( 83.68,121.21) --
	( 83.80,121.48) --
	( 83.92,121.75) --
	( 84.04,122.02) --
	( 84.16,122.29) --
	( 84.28,122.56) --
	( 84.40,122.83) --
	( 84.52,123.10) --
	( 84.63,123.37) --
	( 84.75,123.64) --
	( 84.87,123.91) --
	( 84.99,124.17) --
	( 85.11,124.44) --
	( 85.23,124.71) --
	( 85.35,124.98) --
	( 85.47,125.25) --
	( 85.59,125.51) --
	( 85.71,125.78) --
	( 85.83,126.05) --
	( 85.95,126.32) --
	( 86.07,126.58) --
	( 86.19,126.85) --
	( 86.31,127.11) --
	( 86.42,127.38) --
	( 86.54,127.65) --
	( 86.66,127.91) --
	( 86.78,128.18) --
	( 86.90,128.44) --
	( 87.02,128.71) --
	( 87.14,128.97) --
	( 87.26,129.24) --
	( 87.38,129.50) --
	( 87.50,129.76) --
	( 87.62,130.03) --
	( 87.74,130.29) --
	( 87.86,130.55) --
	( 87.98,130.82) --
	( 88.10,131.08) --
	( 88.21,131.34) --
	( 88.33,131.60) --
	( 88.45,131.87) --
	( 88.57,132.13) --
	( 88.69,132.39) --
	( 88.81,132.65) --
	( 88.93,132.91) --
	( 89.05,133.17) --
	( 89.17,133.43) --
	( 89.29,133.69) --
	( 89.41,133.95) --
	( 89.53,134.21) --
	( 89.65,134.47) --
	( 89.77,134.73) --
	( 89.89,134.99) --
	( 90.00,135.25) --
	( 90.12,135.51) --
	( 90.24,135.77) --
	( 90.36,136.02) --
	( 90.48,136.28) --
	( 90.60,136.54) --
	( 90.72,136.80) --
	( 90.84,137.05) --
	( 90.96,137.31) --
	( 91.08,137.57) --
	( 91.20,137.82) --
	( 91.32,138.08) --
	( 91.44,138.34) --
	( 91.56,138.59) --
	( 91.67,138.85) --
	( 91.79,139.10) --
	( 91.91,139.36) --
	( 92.03,139.61) --
	( 92.15,139.87) --
	( 92.27,140.12) --
	( 92.39,140.38) --
	( 92.51,140.63) --
	( 92.63,140.88) --
	( 92.75,141.14) --
	( 92.87,141.39) --
	( 92.99,141.64) --
	( 93.11,141.89) --
	( 93.23,142.15) --
	( 93.35,142.40) --
	( 93.46,142.65) --
	( 93.58,142.90) --
	( 93.70,143.15) --
	( 93.82,143.40) --
	( 93.94,143.65) --
	( 94.06,143.90) --
	( 94.18,144.15) --
	( 94.30,144.40) --
	( 94.42,144.65) --
	( 94.54,144.90) --
	( 94.66,145.15) --
	( 94.78,145.40) --
	( 94.90,145.65) --
	( 95.02,145.90) --
	( 95.14,146.14) --
	( 95.25,146.39) --
	( 95.37,146.64) --
	( 95.49,146.89) --
	( 95.61,147.13) --
	( 95.73,147.38) --
	( 95.85,147.63) --
	( 95.97,147.87) --
	( 96.09,148.12) --
	( 96.21,148.36) --
	( 96.33,148.61) --
	( 96.45,148.85) --
	( 96.57,149.10) --
	( 96.69,149.34) --
	( 96.81,149.59) --
	( 96.92,149.83) --
	( 97.04,150.07) --
	( 97.16,150.32) --
	( 97.28,150.56) --
	( 97.40,150.80) --
	( 97.52,151.05) --
	( 97.64,151.29) --
	( 97.76,151.53) --
	( 97.88,151.77) --
	( 98.00,152.01) --
	( 98.12,152.26) --
	( 98.24,152.50) --
	( 98.36,152.74) --
	( 98.48,152.98) --
	( 98.60,153.22) --
	( 98.71,153.46) --
	( 98.83,153.70) --
	( 98.95,153.94) --
	( 99.07,154.17) --
	( 99.19,154.41) --
	( 99.31,154.65) --
	( 99.43,154.89) --
	( 99.55,155.13) --
	( 99.67,155.37) --
	( 99.79,155.60) --
	( 99.91,155.84) --
	(100.03,156.08) --
	(100.15,156.31) --
	(100.27,156.55) --
	(100.39,156.79) --
	(100.50,157.02) --
	(100.62,157.26) --
	(100.74,157.49) --
	(100.86,157.73) --
	(100.98,157.96) --
	(101.10,158.20) --
	(101.22,158.43) --
	(101.34,158.66) --
	(101.46,158.90) --
	(101.58,159.13) --
	(101.70,159.36) --
	(101.82,159.60) --
	(101.94,159.83) --
	(102.06,160.06) --
	(102.17,160.29) --
	(102.29,160.52) --
	(102.41,160.75) --
	(102.53,160.98) --
	(102.65,161.21) --
	(102.77,161.45) --
	(102.89,161.68) --
	(103.01,161.90) --
	(103.13,162.13) --
	(103.25,162.36) --
	(103.37,162.59) --
	(103.49,162.82) --
	(103.61,163.05) --
	(103.73,163.28) --
	(103.85,163.50) --
	(103.96,163.73) --
	(104.08,163.96) --
	(104.20,164.19) --
	(104.32,164.41) --
	(104.44,164.64) --
	(104.56,164.86) --
	(104.68,165.09) --
	(104.80,165.32) --
	(104.92,165.54) --
	(105.04,165.77) --
	(105.16,165.99) --
	(105.28,166.21) --
	(105.40,166.44) --
	(105.52,166.66) --
	(105.64,166.89) --
	(105.75,167.11) --
	(105.87,167.33) --
	(105.99,167.55) --
	(106.11,167.78) --
	(106.23,168.00) --
	(106.35,168.22) --
	(106.47,168.44) --
	(106.59,168.66) --
	(106.71,168.88) --
	(106.83,169.10) --
	(106.95,169.32) --
	(107.07,169.54) --
	(107.19,169.76) --
	(107.31,169.98) --
	(107.42,170.20) --
	(107.54,170.42) --
	(107.66,170.64) --
	(107.78,170.85) --
	(107.90,171.07) --
	(108.02,171.29) --
	(108.14,171.51) --
	(108.26,171.72) --
	(108.38,171.94) --
	(108.50,172.16) --
	(108.62,172.37) --
	(108.74,172.59) --
	(108.86,172.80) --
	(108.98,173.02) --
	(109.10,173.23) --
	(109.21,173.45) --
	(109.33,173.66) --
	(109.45,173.88) --
	(109.57,174.09) --
	(109.69,174.30) --
	(109.81,174.52) --
	(109.93,174.73) --
	(110.05,174.94) --
	(110.17,175.15) --
	(110.29,175.36) --
	(110.41,175.58) --
	(110.53,175.79) --
	(110.65,176.00) --
	(110.77,176.21) --
	(110.89,176.42) --
	(111.00,176.63) --
	(111.12,176.84) --
	(111.24,177.05) --
	(111.36,177.26) --
	(111.48,177.47) --
	(111.60,177.67) --
	(111.72,177.88) --
	(111.84,178.09) --
	(111.96,178.30) --
	(112.08,178.51) --
	(112.20,178.71) --
	(112.32,178.92) --
	(112.44,179.13) --
	(112.56,179.33) --
	(112.67,179.54) --
	(112.79,179.74) --
	(112.91,179.95) --
	(113.03,180.15) --
	(113.15,180.36) --
	(113.27,180.56) --
	(113.39,180.77) --
	(113.51,180.97) --
	(113.63,181.17) --
	(113.75,181.38) --
	(113.87,181.58) --
	(113.99,181.78) --
	(114.11,181.98) --
	(114.23,182.19) --
	(114.35,182.39) --
	(114.46,182.59) --
	(114.58,182.79) --
	(114.70,182.99) --
	(114.82,183.19) --
	(114.94,183.39) --
	(115.06,183.59) --
	(115.18,183.79) --
	(115.30,183.99) --
	(115.42,184.19) --
	(115.54,184.39) --
	(115.66,184.58) --
	(115.78,184.78) --
	(115.90,184.98) --
	(116.02,185.18) --
	(116.14,185.37) --
	(116.25,185.57) --
	(116.37,185.77) --
	(116.49,185.96) --
	(116.61,186.16) --
	(116.73,186.36) --
	(116.85,186.55) --
	(116.97,186.75) --
	(117.09,186.94) --
	(117.21,187.13) --
	(117.33,187.33) --
	(117.45,187.52) --
	(117.57,187.72) --
	(117.69,187.91) --
	(117.81,188.10) --
	(117.92,188.29) --
	(118.04,188.49) --
	(118.16,188.68) --
	(118.28,188.87) --
	(118.40,189.06) --
	(118.52,189.25) --
	(118.64,189.44) --
	(118.76,189.63) --
	(118.88,189.82) --
	(119.00,190.01) --
	(119.12,190.20) --
	(119.24,190.39) --
	(119.36,190.58) --
	(119.48,190.77) --
	(119.60,190.96) --
	(119.71,191.14) --
	(119.83,191.33) --
	(119.95,191.52) --
	(120.07,191.71) --
	(120.19,191.89) --
	(120.31,192.08) --
	(120.43,192.27) --
	(120.55,192.45) --
	(120.67,192.64) --
	(120.79,192.82) --
	(120.91,193.01) --
	(121.03,193.19) --
	(121.15,193.38) --
	(121.27,193.56) --
	(121.39,193.74) --
	(121.50,193.93) --
	(121.62,194.11) --
	(121.74,194.29) --
	(121.86,194.48) --
	(121.98,194.66) --
	(122.10,194.84) --
	(122.22,195.02) --
	(122.34,195.20) --
	(122.46,195.38) --
	(122.58,195.57) --
	(122.70,195.75) --
	(122.82,195.93) --
	(122.94,196.11) --
	(123.06,196.28) --
	(123.17,196.46) --
	(123.29,196.64) --
	(123.41,196.82) --
	(123.53,197.00) --
	(123.65,197.18) --
	(123.77,197.36) --
	(123.89,197.53) --
	(124.01,197.71) --
	(124.13,197.89) --
	(124.25,198.06) --
	(124.37,198.24) --
	(124.49,198.42) --
	(124.61,198.59) --
	(124.73,198.77) --
	(124.85,198.94) --
	(124.96,199.12) --
	(125.08,199.29) --
	(125.20,199.46) --
	(125.32,199.64) --
	(125.44,199.81) --
	(125.56,199.99) --
	(125.68,200.16) --
	(125.80,200.33) --
	(125.92,200.50) --
	(126.04,200.68) --
	(126.16,200.85) --
	(126.28,201.02) --
	(126.40,201.19) --
	(126.52,201.36) --
	(126.64,201.53) --
	(126.75,201.70) --
	(126.87,201.87) --
	(126.99,202.04) --
	(127.11,202.21) --
	(127.23,202.38) --
	(127.35,202.55) --
	(127.47,202.72) --
	(127.59,202.88) --
	(127.71,203.05) --
	(127.83,203.22) --
	(127.95,203.39) --
	(128.07,203.55) --
	(128.19,203.72) --
	(128.31,203.89) --
	(128.43,204.05) --
	(128.54,204.22) --
	(128.66,204.38) --
	(128.78,204.55) --
	(128.90,204.71) --
	(129.02,204.88) --
	(129.14,205.04) --
	(129.26,205.21) --
	(129.38,205.37) --
	(129.50,205.53) --
	(129.62,205.70) --
	(129.74,205.86) --
	(129.86,206.02) --
	(129.98,206.19) --
	(130.10,206.35) --
	(130.21,206.51) --
	(130.33,206.67) --
	(130.45,206.83) --
	(130.57,206.99) --
	(130.69,207.15) --
	(130.81,207.31) --
	(130.93,207.47) --
	(131.05,207.63) --
	(131.17,207.79) --
	(131.29,207.95) --
	(131.41,208.11) --
	(131.53,208.27) --
	(131.65,208.43) --
	(131.77,208.58) --
	(131.89,208.74) --
	(132.00,208.90) --
	(132.12,209.06) --
	(132.24,209.21) --
	(132.36,209.37) --
	(132.48,209.52) --
	(132.60,209.68) --
	(132.72,209.84) --
	(132.84,209.99) --
	(132.96,210.15) --
	(133.08,210.30) --
	(133.20,210.46) --
	(133.32,210.61) --
	(133.44,210.76) --
	(133.56,210.92) --
	(133.68,211.07) --
	(133.79,211.22) --
	(133.91,211.37) --
	(134.03,211.53) --
	(134.15,211.68) --
	(134.27,211.83) --
	(134.39,211.98) --
	(134.51,212.13) --
	(134.63,212.28) --
	(134.75,212.44) --
	(134.87,212.59) --
	(134.99,212.74) --
	(135.11,212.89) --
	(135.23,213.04) --
	(135.35,213.18) --
	(135.46,213.33) --
	(135.58,213.48) --
	(135.70,213.63) --
	(135.82,213.78) --
	(135.94,213.93) --
	(136.06,214.07) --
	(136.18,214.22) --
	(136.30,214.37) --
	(136.42,214.51) --
	(136.54,214.66) --
	(136.66,214.81) --
	(136.78,214.95) --
	(136.90,215.10) --
	(137.02,215.24) --
	(137.14,215.39) --
	(137.25,215.53) --
	(137.37,215.68) --
	(137.49,215.82) --
	(137.61,215.97) --
	(137.73,216.11) --
	(137.85,216.25) --
	(137.97,216.39) --
	(138.09,216.54) --
	(138.21,216.68) --
	(138.33,216.82) --
	(138.45,216.96) --
	(138.57,217.11) --
	(138.69,217.25) --
	(138.81,217.39) --
	(138.93,217.53) --
	(139.04,217.67) --
	(139.16,217.81) --
	(139.28,217.95) --
	(139.40,218.09) --
	(139.52,218.23) --
	(139.64,218.37) --
	(139.76,218.51) --
	(139.88,218.64) --
	(140.00,218.78) --
	(140.12,218.92) --
	(140.24,219.06) --
	(140.36,219.20) --
	(140.48,219.33) --
	(140.60,219.47) --
	(140.71,219.61) --
	(140.83,219.74) --
	(140.95,219.88) --
	(141.07,220.01) --
	(141.19,220.15) --
	(141.31,220.29) --
	(141.43,220.42) --
	(141.55,220.56) --
	(141.67,220.69) --
	(141.79,220.82) --
	(141.91,220.96) --
	(142.03,221.09) --
	(142.15,221.22) --
	(142.27,221.36) --
	(142.39,221.49) --
	(142.50,221.62) --
	(142.62,221.76) --
	(142.74,221.89) --
	(142.86,222.02) --
	(142.98,222.15) --
	(143.10,222.28) --
	(143.22,222.41) --
	(143.34,222.54) --
	(143.46,222.67) --
	(143.58,222.80) --
	(143.70,222.93) --
	(143.82,223.06) --
	(143.94,223.19) --
	(144.06,223.32) --
	(144.18,223.45) --
	(144.29,223.58) --
	(144.41,223.71) --
	(144.53,223.83) --
	(144.65,223.96) --
	(144.77,224.09) --
	(144.89,224.22) --
	(145.01,224.34) --
	(145.13,224.47) --
	(145.25,224.60) --
	(145.37,224.72) --
	(145.49,224.85) --
	(145.61,224.97) --
	(145.73,225.10) --
	(145.85,225.22) --
	(145.96,225.35) --
	(146.08,225.47) --
	(146.20,225.60) --
	(146.32,225.72) --
	(146.44,225.85) --
	(146.56,225.97) --
	(146.68,226.09) --
	(146.80,226.22) --
	(146.92,226.34) --
	(147.04,226.46) --
	(147.16,226.58) --
	(147.28,226.70) --
	(147.40,226.83) --
	(147.52,226.95) --
	(147.64,227.07) --
	(147.75,227.19) --
	(147.87,227.31) --
	(147.99,227.43) --
	(148.11,227.55) --
	(148.23,227.67) --
	(148.35,227.79) --
	(148.47,227.91) --
	(148.59,228.03) --
	(148.71,228.15) --
	(148.83,228.27) --
	(148.95,228.38) --
	(149.07,228.50) --
	(149.19,228.62) --
	(149.31,228.74) --
	(149.43,228.86) --
	(149.54,228.97) --
	(149.66,229.09) --
	(149.78,229.21) --
	(149.90,229.32) --
	(150.02,229.44) --
	(150.14,229.55) --
	(150.26,229.67) --
	(150.38,229.78) --
	(150.50,229.90) --
	(150.62,230.01) --
	(150.74,230.13) --
	(150.86,230.24) --
	(150.98,230.36) --
	(151.10,230.47) --
	(151.21,230.58) --
	(151.33,230.70) --
	(151.45,230.81) --
	(151.57,230.92) --
	(151.69,231.04) --
	(151.81,231.15) --
	(151.93,231.26) --
	(152.05,231.37) --
	(152.17,231.48) --
	(152.29,231.60) --
	(152.41,231.71) --
	(152.53,231.82) --
	(152.65,231.93) --
	(152.77,232.04) --
	(152.89,232.15) --
	(153.00,232.26) --
	(153.12,232.37) --
	(153.24,232.48) --
	(153.36,232.59) --
	(153.48,232.70) --
	(153.60,232.80) --
	(153.72,232.91) --
	(153.84,233.02) --
	(153.96,233.13) --
	(154.08,233.24) --
	(154.20,233.34) --
	(154.32,233.45) --
	(154.44,233.56) --
	(154.56,233.66) --
	(154.68,233.77) --
	(154.79,233.88) --
	(154.91,233.98) --
	(155.03,234.09) --
	(155.15,234.19) --
	(155.27,234.30) --
	(155.39,234.40) --
	(155.51,234.51) --
	(155.63,234.61) --
	(155.75,234.72) --
	(155.87,234.82) --
	(155.99,234.93) --
	(156.11,235.03) --
	(156.23,235.13) --
	(156.35,235.24) --
	(156.46,235.34) --
	(156.58,235.44) --
	(156.70,235.54) --
	(156.82,235.65) --
	(156.94,235.75) --
	(157.06,235.85) --
	(157.18,235.95) --
	(157.30,236.05) --
	(157.42,236.15) --
	(157.54,236.25) --
	(157.66,236.36) --
	(157.78,236.46) --
	(157.90,236.56) --
	(158.02,236.66) --
	(158.14,236.76) --
	(158.25,236.86) --
	(158.37,236.95) --
	(158.49,237.05) --
	(158.61,237.15) --
	(158.73,237.25) --
	(158.85,237.35) --
	(158.97,237.45) --
	(159.09,237.54) --
	(159.21,237.64) --
	(159.33,237.74) --
	(159.45,237.84) --
	(159.57,237.93) --
	(159.69,238.03) --
	(159.81,238.13) --
	(159.93,238.22) --
	(160.04,238.32) --
	(160.16,238.42) --
	(160.28,238.51) --
	(160.40,238.61) --
	(160.52,238.70) --
	(160.64,238.80) --
	(160.76,238.89) --
	(160.88,238.99) --
	(161.00,239.08) --
	(161.12,239.17) --
	(161.24,239.27) --
	(161.36,239.36) --
	(161.48,239.45) --
	(161.60,239.55) --
	(161.71,239.64) --
	(161.83,239.73) --
	(161.95,239.83) --
	(162.07,239.92) --
	(162.19,240.01) --
	(162.31,240.10) --
	(162.43,240.19) --
	(162.55,240.28) --
	(162.67,240.38) --
	(162.79,240.47) --
	(162.91,240.56) --
	(163.03,240.65) --
	(163.15,240.74) --
	(163.27,240.83) --
	(163.39,240.92) --
	(163.50,241.01) --
	(163.62,241.10) --
	(163.74,241.19) --
	(163.86,241.28) --
	(163.98,241.36) --
	(164.10,241.45) --
	(164.22,241.54) --
	(164.34,241.63) --
	(164.46,241.72) --
	(164.58,241.81) --
	(164.70,241.89) --
	(164.82,241.98) --
	(164.94,242.07) --
	(165.06,242.15) --
	(165.18,242.24) --
	(165.29,242.33) --
	(165.41,242.41) --
	(165.53,242.50) --
	(165.65,242.59) --
	(165.77,242.67) --
	(165.89,242.76) --
	(166.01,242.84) --
	(166.13,242.93) --
	(166.25,243.01) --
	(166.37,243.10) --
	(166.49,243.18) --
	(166.61,243.27) --
	(166.73,243.35) --
	(166.85,243.43) --
	(166.97,243.52) --
	(167.08,243.60) --
	(167.20,243.68) --
	(167.32,243.77) --
	(167.44,243.85) --
	(167.56,243.93) --
	(167.68,244.01) --
	(167.80,244.10) --
	(167.92,244.18) --
	(168.04,244.26) --
	(168.16,244.34) --
	(168.28,244.42) --
	(168.40,244.50) --
	(168.52,244.59) --
	(168.64,244.67) --
	(168.75,244.75) --
	(168.87,244.83) --
	(168.99,244.91) --
	(169.11,244.99) --
	(169.23,245.07) --
	(169.35,245.15) --
	(169.47,245.23) --
	(169.59,245.31) --
	(169.71,245.38) --
	(169.83,245.46) --
	(169.95,245.54) --
	(170.07,245.62) --
	(170.19,245.70) --
	(170.31,245.78) --
	(170.43,245.85) --
	(170.54,245.93) --
	(170.66,246.01) --
	(170.78,246.09) --
	(170.90,246.16) --
	(171.02,246.24) --
	(171.14,246.32) --
	(171.26,246.39) --
	(171.38,246.47) --
	(171.50,246.55) --
	(171.62,246.62) --
	(171.74,246.70) --
	(171.86,246.77) --
	(171.98,246.85) --
	(172.10,246.92) --
	(172.22,247.00) --
	(172.33,247.07) --
	(172.45,247.15) --
	(172.57,247.22) --
	(172.69,247.30) --
	(172.81,247.37) --
	(172.93,247.44) --
	(173.05,247.52) --
	(173.17,247.59) --
	(173.29,247.67) --
	(173.41,247.74) --
	(173.53,247.81) --
	(173.65,247.88) --
	(173.77,247.96) --
	(173.89,248.03) --
	(174.00,248.10) --
	(174.12,248.17) --
	(174.24,248.24) --
	(174.36,248.32) --
	(174.48,248.39) --
	(174.60,248.46) --
	(174.72,248.53) --
	(174.84,248.60) --
	(174.96,248.67) --
	(175.08,248.74) --
	(175.20,248.81) --
	(175.32,248.88) --
	(175.44,248.95) --
	(175.56,249.02) --
	(175.68,249.09) --
	(175.79,249.16) --
	(175.91,249.23) --
	(176.03,249.30) --
	(176.15,249.37) --
	(176.27,249.44) --
	(176.39,249.51) --
	(176.51,249.57) --
	(176.63,249.64) --
	(176.75,249.71) --
	(176.87,249.78) --
	(176.99,249.85) --
	(177.11,249.91) --
	(177.23,249.98) --
	(177.35,250.05) --
	(177.47,250.11) --
	(177.58,250.18) --
	(177.70,250.25) --
	(177.82,250.31) --
	(177.94,250.38) --
	(178.06,250.45) --
	(178.18,250.51) --
	(178.30,250.58) --
	(178.42,250.64) --
	(178.54,250.71) --
	(178.66,250.78) --
	(178.78,250.84) --
	(178.90,250.91) --
	(179.02,250.97) --
	(179.14,251.03) --
	(179.25,251.10) --
	(179.37,251.16) --
	(179.49,251.23) --
	(179.61,251.29) --
	(179.73,251.35) --
	(179.85,251.42) --
	(179.97,251.48) --
	(180.09,251.55) --
	(180.21,251.61) --
	(180.33,251.67) --
	(180.45,251.73) --
	(180.57,251.80) --
	(180.69,251.86) --
	(180.81,251.92) --
	(180.93,251.98) --
	(181.04,252.04) --
	(181.16,252.11) --
	(181.28,252.17) --
	(181.40,252.23) --
	(181.52,252.29) --
	(181.64,252.35) --
	(181.76,252.41) --
	(181.88,252.47) --
	(182.00,252.53) --
	(182.12,252.59) --
	(182.24,252.66) --
	(182.36,252.72) --
	(182.48,252.78) --
	(182.60,252.83) --
	(182.72,252.89) --
	(182.83,252.95) --
	(182.95,253.01) --
	(183.07,253.07) --
	(183.19,253.13) --
	(183.31,253.19) --
	(183.43,253.25) --
	(183.55,253.31) --
	(183.67,253.37) --
	(183.79,253.42) --
	(183.91,253.48) --
	(184.03,253.54) --
	(184.15,253.60) --
	(184.27,253.66) --
	(184.39,253.71) --
	(184.50,253.77) --
	(184.62,253.83) --
	(184.74,253.88) --
	(184.86,253.94) --
	(184.98,254.00) --
	(185.10,254.05) --
	(185.22,254.11) --
	(185.34,254.17) --
	(185.46,254.22) --
	(185.58,254.28) --
	(185.70,254.33) --
	(185.82,254.39) --
	(185.94,254.45) --
	(186.06,254.50) --
	(186.18,254.56) --
	(186.29,254.61) --
	(186.41,254.67) --
	(186.53,254.72) --
	(186.65,254.78) --
	(186.77,254.83) --
	(186.89,254.88) --
	(187.01,254.94) --
	(187.13,254.99) --
	(187.25,255.05) --
	(187.37,255.10) --
	(187.49,255.15) --
	(187.61,255.21) --
	(187.73,255.26) --
	(187.85,255.31) --
	(187.97,255.37) --
	(188.08,255.42) --
	(188.20,255.47) --
	(188.32,255.52) --
	(188.44,255.58) --
	(188.56,255.63) --
	(188.68,255.68) --
	(188.80,255.73) --
	(188.92,255.78) --
	(189.04,255.84) --
	(189.16,255.89) --
	(189.28,255.94) --
	(189.40,255.99) --
	(189.52,256.04) --
	(189.64,256.09) --
	(189.75,256.14) --
	(189.87,256.19) --
	(189.99,256.24) --
	(190.11,256.29) --
	(190.23,256.34) --
	(190.35,256.39) --
	(190.47,256.44) --
	(190.59,256.49) --
	(190.71,256.54) --
	(190.83,256.59) --
	(190.95,256.64) --
	(191.07,256.69) --
	(191.19,256.74) --
	(191.31,256.79) --
	(191.43,256.84) --
	(191.54,256.89) --
	(191.66,256.94) --
	(191.78,256.98) --
	(191.90,257.03) --
	(192.02,257.08) --
	(192.14,257.13) --
	(192.26,257.18) --
	(192.38,257.22) --
	(192.50,257.27) --
	(192.62,257.32) --
	(192.74,257.37) --
	(192.86,257.41) --
	(192.98,257.46) --
	(193.10,257.51) --
	(193.22,257.56) --
	(193.33,257.60) --
	(193.45,257.65) --
	(193.57,257.70) --
	(193.69,257.74) --
	(193.81,257.79) --
	(193.93,257.83) --
	(194.05,257.88) --
	(194.17,257.93) --
	(194.29,257.97) --
	(194.41,258.02) --
	(194.53,258.06) --
	(194.65,258.11) --
	(194.77,258.15) --
	(194.89,258.20) --
	(195.00,258.24) --
	(195.12,258.29) --
	(195.24,258.33) --
	(195.36,258.38) --
	(195.48,258.42) --
	(195.60,258.47) --
	(195.72,258.51) --
	(195.84,258.55) --
	(195.96,258.60) --
	(196.08,258.64) --
	(196.20,258.68) --
	(196.32,258.73) --
	(196.44,258.77) --
	(196.56,258.82) --
	(196.68,258.86) --
	(196.79,258.90) --
	(196.91,258.94) --
	(197.03,258.99) --
	(197.15,259.03) --
	(197.27,259.07) --
	(197.39,259.11) --
	(197.51,259.16) --
	(197.63,259.20) --
	(197.75,259.24) --
	(197.87,259.28) --
	(197.99,259.32) --
	(198.11,259.37) --
	(198.23,259.41) --
	(198.35,259.45) --
	(198.47,259.49) --
	(198.58,259.53) --
	(198.70,259.57) --
	(198.82,259.61) --
	(198.94,259.65) --
	(199.06,259.70) --
	(199.18,259.74) --
	(199.30,259.78) --
	(199.42,259.82) --
	(199.54,259.86) --
	(199.66,259.90) --
	(199.78,259.94) --
	(199.90,259.98) --
	(200.02,260.02) --
	(200.14,260.06) --
	(200.26,260.10) --
	(200.37,260.14) --
	(200.49,260.18) --
	(200.61,260.21) --
	(200.73,260.25) --
	(200.85,260.29) --
	(200.97,260.33) --
	(201.09,260.37) --
	(201.21,260.41) --
	(201.33,260.45) --
	(201.45,260.49) --
	(201.57,260.52) --
	(201.69,260.56) --
	(201.81,260.60) --
	(201.93,260.64) --
	(202.04,260.68) --
	(202.16,260.71) --
	(202.28,260.75) --
	(202.40,260.79) --
	(202.52,260.83) --
	(202.64,260.86) --
	(202.76,260.90) --
	(202.88,260.94) --
	(203.00,260.98) --
	(203.12,261.01) --
	(203.24,261.05) --
	(203.36,261.09) --
	(203.48,261.12) --
	(203.60,261.16) --
	(203.72,261.19) --
	(203.83,261.23) --
	(203.95,261.27) --
	(204.07,261.30) --
	(204.19,261.34) --
	(204.31,261.37) --
	(204.43,261.41) --
	(204.55,261.45) --
	(204.67,261.48) --
	(204.79,261.52) --
	(204.91,261.55) --
	(205.03,261.59) --
	(205.15,261.62) --
	(205.27,261.66) --
	(205.39,261.69) --
	(205.51,261.73) --
	(205.62,261.76) --
	(205.74,261.80) --
	(205.86,261.83) --
	(205.98,261.87) --
	(206.10,261.90) --
	(206.22,261.93) --
	(206.34,261.97) --
	(206.46,262.00) --
	(206.58,262.04) --
	(206.70,262.07) --
	(206.82,262.10) --
	(206.94,262.14) --
	(207.06,262.17) --
	(207.18,262.20) --
	(207.29,262.24) --
	(207.41,262.27) --
	(207.53,262.30) --
	(207.65,262.34) --
	(207.77,262.37) --
	(207.89,262.40) --
	(208.01,262.43) --
	(208.13,262.47) --
	(208.25,262.50) --
	(208.37,262.53) --
	(208.49,262.56) --
	(208.61,262.60) --
	(208.73,262.63) --
	(208.85,262.66) --
	(208.97,262.69) --
	(209.08,262.72) --
	(209.20,262.76) --
	(209.32,262.79) --
	(209.44,262.82) --
	(209.56,262.85) --
	(209.68,262.88) --
	(209.80,262.91) --
	(209.92,262.94) --
	(210.04,262.97) --
	(210.16,263.01) --
	(210.28,263.04) --
	(210.40,263.07) --
	(210.52,263.10) --
	(210.64,263.13) --
	(210.76,263.16) --
	(210.87,263.19) --
	(210.99,263.22) --
	(211.11,263.25) --
	(211.23,263.28) --
	(211.35,263.31) --
	(211.47,263.34) --
	(211.59,263.37) --
	(211.71,263.40) --
	(211.83,263.43) --
	(211.95,263.46) --
	(212.07,263.49) --
	(212.19,263.52) --
	(212.31,263.55) --
	(212.43,263.58) --
	(212.54,263.60) --
	(212.66,263.63) --
	(212.78,263.66) --
	(212.90,263.69) --
	(213.02,263.72) --
	(213.14,263.75) --
	(213.26,263.78) --
	(213.38,263.81) --
	(213.50,263.83) --
	(213.62,263.86) --
	(213.74,263.89) --
	(213.86,263.92) --
	(213.98,263.95) --
	(214.10,263.97) --
	(214.22,264.00) --
	(214.33,264.03) --
	(214.45,264.06) --
	(214.57,264.09) --
	(214.69,264.11) --
	(214.81,264.14) --
	(214.93,264.17) --
	(215.05,264.20) --
	(215.17,264.22) --
	(215.29,264.25) --
	(215.41,264.28) --
	(215.53,264.30) --
	(215.65,264.33) --
	(215.77,264.36) --
	(215.89,264.38) --
	(216.01,264.41) --
	(216.12,264.44) --
	(216.24,264.46) --
	(216.36,264.49) --
	(216.48,264.52) --
	(216.60,264.54) --
	(216.72,264.57) --
	(216.84,264.59) --
	(216.96,264.62) --
	(217.08,264.65) --
	(217.20,264.67) --
	(217.32,264.70) --
	(217.44,264.72) --
	(217.56,264.75) --
	(217.68,264.77) --
	(217.79,264.80) --
	(217.91,264.83) --
	(218.03,264.85) --
	(218.15,264.88) --
	(218.27,264.90) --
	(218.39,264.93) --
	(218.51,264.95) --
	(218.63,264.98) --
	(218.75,265.00) --
	(218.87,265.02) --
	(218.99,265.05) --
	(219.11,265.07) --
	(219.23,265.10) --
	(219.35,265.12) --
	(219.47,265.15) --
	(219.58,265.17) --
	(219.70,265.20) --
	(219.82,265.22) --
	(219.94,265.24) --
	(220.06,265.27) --
	(220.18,265.29) --
	(220.30,265.32) --
	(220.42,265.34) --
	(220.54,265.36) --
	(220.66,265.39) --
	(220.78,265.41) --
	(220.90,265.43) --
	(221.02,265.46) --
	(221.14,265.48) --
	(221.26,265.50) --
	(221.37,265.53) --
	(221.49,265.55) --
	(221.61,265.57) --
	(221.73,265.59) --
	(221.85,265.62) --
	(221.97,265.64) --
	(222.09,265.66) --
	(222.21,265.69) --
	(222.33,265.71) --
	(222.45,265.73) --
	(222.57,265.75) --
	(222.69,265.78) --
	(222.81,265.80) --
	(222.93,265.82) --
	(223.04,265.84) --
	(223.16,265.86) --
	(223.28,265.89) --
	(223.40,265.91) --
	(223.52,265.93) --
	(223.64,265.95) --
	(223.76,265.97) --
	(223.88,265.99) --
	(224.00,266.02) --
	(224.12,266.04) --
	(224.24,266.06) --
	(224.36,266.08) --
	(224.48,266.10) --
	(224.60,266.12) --
	(224.72,266.14) --
	(224.83,266.17) --
	(224.95,266.19) --
	(225.07,266.21) --
	(225.19,266.23) --
	(225.31,266.25) --
	(225.43,266.27) --
	(225.55,266.29) --
	(225.67,266.31) --
	(225.79,266.33) --
	(225.91,266.35) --
	(226.03,266.37) --
	(226.15,266.39) --
	(226.27,266.41) --
	(226.39,266.43) --
	(226.51,266.45) --
	(226.62,266.47) --
	(226.74,266.49) --
	(226.86,266.51) --
	(226.98,266.53) --
	(227.10,266.55) --
	(227.22,266.57) --
	(227.34,266.59) --
	(227.46,266.61) --
	(227.58,266.63) --
	(227.70,266.65) --
	(227.82,266.67) --
	(227.94,266.69) --
	(228.06,266.71) --
	(228.18,266.73) --
	(228.29,266.75) --
	(228.41,266.77) --
	(228.53,266.79) --
	(228.65,266.80) --
	(228.77,266.82) --
	(228.89,266.84) --
	(229.01,266.86) --
	(229.13,266.88) --
	(229.25,266.90) --
	(229.37,266.92) --
	(229.49,266.94) --
	(229.61,266.95) --
	(229.73,266.97) --
	(229.85,266.99) --
	(229.97,267.01) --
	(230.08,267.03) --
	(230.20,267.05) --
	(230.32,267.06) --
	(230.44,267.08) --
	(230.56,267.10) --
	(230.68,267.12) --
	(230.80,267.14) --
	(230.92,267.15) --
	(231.04,267.17) --
	(231.16,267.19) --
	(231.28,267.21) --
	(231.40,267.23) --
	(231.52,267.24) --
	(231.64,267.26) --
	(231.76,267.28) --
	(231.87,267.30) --
	(231.99,267.31) --
	(232.11,267.33) --
	(232.23,267.35) --
	(232.35,267.36) --
	(232.47,267.38) --
	(232.59,267.40) --
	(232.71,267.42) --
	(232.83,267.43) --
	(232.95,267.45) --
	(233.07,267.47) --
	(233.19,267.48) --
	(233.31,267.50) --
	(233.43,267.52) --
	(233.54,267.53) --
	(233.66,267.55) --
	(233.78,267.57) --
	(233.90,267.58) --
	(234.02,267.60) --
	(234.14,267.62) --
	(234.26,267.63) --
	(234.38,267.65) --
	(234.50,267.66) --
	(234.62,267.68) --
	(234.74,267.70) --
	(234.86,267.71) --
	(234.98,267.73) --
	(235.10,267.74) --
	(235.22,267.76) --
	(235.33,267.78) --
	(235.45,267.79) --
	(235.57,267.81) --
	(235.69,267.82) --
	(235.81,267.84) --
	(235.93,267.85) --
	(236.05,267.87) --
	(236.17,267.89) --
	(236.29,267.90) --
	(236.41,267.92) --
	(236.53,267.93) --
	(236.65,267.95) --
	(236.77,267.96) --
	(236.89,267.98) --
	(237.01,267.99) --
	(237.12,268.01) --
	(237.24,268.02) --
	(237.36,268.04) --
	(237.48,268.05) --
	(237.60,268.07) --
	(237.72,268.08) --
	(237.84,268.10) --
	(237.96,268.11) --
	(238.08,268.13) --
	(238.20,268.14) --
	(238.32,268.16) --
	(238.44,268.17) --
	(238.56,268.18) --
	(238.68,268.20) --
	(238.80,268.21) --
	(238.91,268.23) --
	(239.03,268.24) --
	(239.15,268.26) --
	(239.27,268.27) --
	(239.39,268.28) --
	(239.51,268.30) --
	(239.63,268.31) --
	(239.75,268.33) --
	(239.87,268.34) --
	(239.99,268.35) --
	(240.11,268.37) --
	(240.23,268.38) --
	(240.35,268.40) --
	(240.47,268.41) --
	(240.58,268.42) --
	(240.70,268.44) --
	(240.82,268.45) --
	(240.94,268.46) --
	(241.06,268.48) --
	(241.18,268.49) --
	(241.30,268.50) --
	(241.42,268.52) --
	(241.54,268.53) --
	(241.66,268.54) --
	(241.78,268.56) --
	(241.90,268.57) --
	(242.02,268.58) --
	(242.14,268.60) --
	(242.26,268.61) --
	(242.37,268.62) --
	(242.49,268.63) --
	(242.61,268.65) --
	(242.73,268.66) --
	(242.85,268.67) --
	(242.97,268.69) --
	(243.09,268.70) --
	(243.21,268.71) --
	(243.33,268.72) --
	(243.45,268.74) --
	(243.57,268.75) --
	(243.69,268.76) --
	(243.81,268.77) --
	(243.93,268.79) --
	(244.05,268.80) --
	(244.16,268.81) --
	(244.28,268.82) --
	(244.40,268.84) --
	(244.52,268.85) --
	(244.64,268.86) --
	(244.76,268.87) --
	(244.88,268.88) --
	(245.00,268.90) --
	(245.12,268.91) --
	(245.24,268.92) --
	(245.36,268.93) --
	(245.48,268.94) --
	(245.60,268.96) --
	(245.72,268.97) --
	(245.83,268.98) --
	(245.95,268.99) --
	(246.07,269.00) --
	(246.19,269.01) --
	(246.31,269.03) --
	(246.43,269.04) --
	(246.55,269.05) --
	(246.67,269.06) --
	(246.79,269.07) --
	(246.91,269.08) --
	(247.03,269.09) --
	(247.15,269.11) --
	(247.27,269.12) --
	(247.39,269.13) --
	(247.51,269.14) --
	(247.62,269.15) --
	(247.74,269.16) --
	(247.86,269.17) --
	(247.98,269.18) --
	(248.10,269.20) --
	(248.22,269.21) --
	(248.34,269.22) --
	(248.46,269.23) --
	(248.58,269.24) --
	(248.70,269.25) --
	(248.82,269.26) --
	(248.94,269.27) --
	(249.06,269.28) --
	(249.18,269.29) --
	(249.30,269.30) --
	(249.41,269.31) --
	(249.53,269.32) --
	(249.65,269.34) --
	(249.77,269.35) --
	(249.89,269.36) --
	(250.01,269.37) --
	(250.13,269.38) --
	(250.25,269.39) --
	(250.37,269.40) --
	(250.49,269.41) --
	(250.61,269.42) --
	(250.73,269.43) --
	(250.85,269.44) --
	(250.97,269.45) --
	(251.08,269.46) --
	(251.20,269.47) --
	(251.32,269.48) --
	(251.44,269.49) --
	(251.56,269.50) --
	(251.68,269.51) --
	(251.80,269.52) --
	(251.92,269.53) --
	(252.04,269.54) --
	(252.16,269.55) --
	(252.28,269.56) --
	(252.40,269.57) --
	(252.52,269.58) --
	(252.64,269.59) --
	(252.76,269.60) --
	(252.87,269.61) --
	(252.99,269.62) --
	(253.11,269.63) --
	(253.23,269.63) --
	(253.35,269.64) --
	(253.47,269.65) --
	(253.59,269.66) --
	(253.71,269.67) --
	(253.83,269.68) --
	(253.95,269.69) --
	(254.07,269.70) --
	(254.19,269.71) --
	(254.31,269.72) --
	(254.43,269.73) --
	(254.55,269.74) --
	(254.66,269.75) --
	(254.78,269.75) --
	(254.90,269.76) --
	(255.02,269.77) --
	(255.14,269.78) --
	(255.26,269.79) --
	(255.38,269.80) --
	(255.50,269.81) --
	(255.62,269.82) --
	(255.74,269.83) --
	(255.86,269.83) --
	(255.98,269.84) --
	(256.10,269.85) --
	(256.22,269.86) --
	(256.33,269.87) --
	(256.45,269.88) --
	(256.57,269.89) --
	(256.69,269.90) --
	(256.81,269.90) --
	(256.93,269.91) --
	(257.05,269.92) --
	(257.17,269.93) --
	(257.29,269.94) --
	(257.41,269.95) --
	(257.53,269.95) --
	(257.65,269.96) --
	(257.77,269.97) --
	(257.89,269.98) --
	(258.01,269.99) --
	(258.12,270.00) --
	(258.24,270.00) --
	(258.36,270.01) --
	(258.48,270.02) --
	(258.60,270.03) --
	(258.72,270.04) --
	(258.84,270.04) --
	(258.96,270.05) --
	(259.08,270.06) --
	(259.20,270.07) --
	(259.32,270.08) --
	(259.44,270.08) --
	(259.56,270.09) --
	(259.68,270.10) --
	(259.80,270.11) --
	(259.91,270.12) --
	(260.03,270.12) --
	(260.15,270.13) --
	(260.27,270.14) --
	(260.39,270.15) --
	(260.51,270.15) --
	(260.63,270.16) --
	(260.75,270.17) --
	(260.87,270.18) --
	(260.99,270.18) --
	(261.11,270.19) --
	(261.23,270.20) --
	(261.35,270.21) --
	(261.47,270.21) --
	(261.58,270.22) --
	(261.70,270.23) --
	(261.82,270.24) --
	(261.94,270.24) --
	(262.06,270.25) --
	(262.18,270.26) --
	(262.30,270.27) --
	(262.42,270.27) --
	(262.54,270.28) --
	(262.66,270.29) --
	(262.78,270.29) --
	(262.90,270.30) --
	(263.02,270.31) --
	(263.14,270.32) --
	(263.26,270.32) --
	(263.37,270.33) --
	(263.49,270.34) --
	(263.61,270.34) --
	(263.73,270.35) --
	(263.85,270.36) --
	(263.97,270.36) --
	(264.09,270.37) --
	(264.21,270.38) --
	(264.33,270.38) --
	(264.45,270.39) --
	(264.57,270.40) --
	(264.69,270.41) --
	(264.81,270.41) --
	(264.93,270.42) --
	(265.05,270.42) --
	(265.16,270.43) --
	(265.28,270.44) --
	(265.40,270.44) --
	(265.52,270.45) --
	(265.64,270.46) --
	(265.76,270.46) --
	(265.88,270.47) --
	(266.00,270.48) --
	(266.12,270.48) --
	(266.24,270.49) --
	(266.36,270.50) --
	(266.48,270.50) --
	(266.60,270.51) --
	(266.72,270.52) --
	(266.83,270.52) --
	(266.95,270.53) --
	(267.07,270.53) --
	(267.19,270.54) --
	(267.31,270.55) --
	(267.43,270.55) --
	(267.55,270.56) --
	(267.67,270.57) --
	(267.79,270.57) --
	(267.91,270.58) --
	(268.03,270.58) --
	(268.15,270.59) --
	(268.27,270.60) --
	(268.39,270.60) --
	(268.51,270.61) --
	(268.62,270.61) --
	(268.74,270.62) --
	(268.86,270.63) --
	(268.98,270.63) --
	(269.10,270.64) --
	(269.22,270.64) --
	(269.34,270.65) --
	(269.46,270.66) --
	(269.58,270.66) --
	(269.70,270.67) --
	(269.82,270.67) --
	(269.94,270.68) --
	(270.06,270.68) --
	(270.18,270.69) --
	(270.30,270.70) --
	(270.41,270.70) --
	(270.53,270.71) --
	(270.65,270.71) --
	(270.77,270.72) --
	(270.89,270.72) --
	(271.01,270.73) --
	(271.13,270.73) --
	(271.25,270.74) --
	(271.37,270.75) --
	(271.49,270.75) --
	(271.61,270.76) --
	(271.73,270.76) --
	(271.85,270.77) --
	(271.97,270.77) --
	(272.09,270.78) --
	(272.20,270.78) --
	(272.32,270.79) --
	(272.44,270.79) --
	(272.56,270.80) --
	(272.68,270.80) --
	(272.80,270.81) --
	(272.92,270.82) --
	(273.04,270.82) --
	(273.16,270.83) --
	(273.28,270.83) --
	(273.40,270.84) --
	(273.52,270.84) --
	(273.64,270.85) --
	(273.76,270.85) --
	(273.87,270.86) --
	(273.99,270.86) --
	(274.11,270.87) --
	(274.23,270.87) --
	(274.35,270.88) --
	(274.47,270.88) --
	(274.59,270.89) --
	(274.71,270.89) --
	(274.83,270.90) --
	(274.95,270.90) --
	(275.07,270.91) --
	(275.19,270.91) --
	(275.31,270.92) --
	(275.43,270.92) --
	(275.55,270.93) --
	(275.66,270.93) --
	(275.78,270.94) --
	(275.90,270.94) --
	(276.02,270.95) --
	(276.14,270.95) --
	(276.26,270.95) --
	(276.38,270.96) --
	(276.50,270.96) --
	(276.62,270.97) --
	(276.74,270.97) --
	(276.86,270.98) --
	(276.98,270.98) --
	(277.10,270.99) --
	(277.22,270.99) --
	(277.34,271.00) --
	(277.45,271.00) --
	(277.57,271.01) --
	(277.69,271.01) --
	(277.81,271.01) --
	(277.93,271.02) --
	(278.05,271.02) --
	(278.17,271.03) --
	(278.29,271.03) --
	(278.41,271.04) --
	(278.53,271.04) --
	(278.65,271.05) --
	(278.77,271.05) --
	(278.89,271.05) --
	(279.01,271.06) --
	(279.12,271.06) --
	(279.24,271.07) --
	(279.36,271.07) --
	(279.48,271.08) --
	(279.60,271.08) --
	(279.72,271.08) --
	(279.84,271.09) --
	(279.96,271.09) --
	(280.08,271.10) --
	(280.20,271.10) --
	(280.32,271.11) --
	(280.44,271.11) --
	(280.56,271.11) --
	(280.68,271.12) --
	(280.80,271.12) --
	(280.91,271.13) --
	(281.03,271.13) --
	(281.15,271.13) --
	(281.27,271.14) --
	(281.39,271.14) --
	(281.51,271.15) --
	(281.63,271.15) --
	(281.75,271.15) --
	(281.87,271.16) --
	(281.99,271.16) --
	(282.11,271.17) --
	(282.23,271.17) --
	(282.35,271.17) --
	(282.47,271.18) --
	(282.59,271.18) --
	(282.70,271.19) --
	(282.82,271.19) --
	(282.94,271.19) --
	(283.06,271.20) --
	(283.18,271.20) --
	(283.30,271.20) --
	(283.42,271.21) --
	(283.54,271.21) --
	(283.66,271.22) --
	(283.78,271.22) --
	(283.90,271.22) --
	(284.02,271.23) --
	(284.14,271.23) --
	(284.26,271.23) --
	(284.37,271.24) --
	(284.49,271.24) --
	(284.61,271.24) --
	(284.73,271.25) --
	(284.85,271.25) --
	(284.97,271.26) --
	(285.09,271.26) --
	(285.21,271.26) --
	(285.33,271.27) --
	(285.45,271.27) --
	(285.57,271.27) --
	(285.69,271.28) --
	(285.81,271.28) --
	(285.93,271.28) --
	(286.05,271.29) --
	(286.16,271.29) --
	(286.28,271.29) --
	(286.40,271.30) --
	(286.52,271.30) --
	(286.64,271.30) --
	(286.76,271.31) --
	(286.88,271.31) --
	(287.00,271.31) --
	(287.12,271.32) --
	(287.24,271.32) --
	(287.36,271.32) --
	(287.48,271.33) --
	(287.60,271.33) --
	(287.72,271.33) --
	(287.84,271.34) --
	(287.95,271.34) --
	(288.07,271.34) --
	(288.19,271.35) --
	(288.31,271.35) --
	(288.43,271.35) --
	(288.55,271.36) --
	(288.67,271.36) --
	(288.79,271.36) --
	(288.91,271.37) --
	(289.03,271.37) --
	(289.15,271.37) --
	(289.27,271.38) --
	(289.39,271.38) --
	(289.51,271.38) --
	(289.62,271.39) --
	(289.74,271.39);
\definecolor{fillColor}{RGB}{0,0,0}

\path[draw=drawColor,line width= 0.4pt,line join=round,line cap=round,fill=fillColor] ( 56.83, 63.02) circle (  1.96);

\path[draw=drawColor,line width= 0.4pt,line join=round,line cap=round,fill=fillColor] ( 87.38,125.38) circle (  1.96);

\path[draw=drawColor,line width= 0.4pt,line join=round,line cap=round,fill=fillColor] (117.92,173.91) circle (  1.96);

\path[draw=drawColor,line width= 0.4pt,line join=round,line cap=round,fill=fillColor] (148.47,234.64) circle (  1.96);

\path[draw=drawColor,line width= 0.4pt,line join=round,line cap=round,fill=fillColor] (179.02,249.08) circle (  1.96);

\path[draw=drawColor,line width= 0.4pt,line join=round,line cap=round,fill=fillColor] (209.56,261.13) circle (  1.96);

\path[draw=drawColor,line width= 0.4pt,line join=round,line cap=round,fill=fillColor] (240.11,270.51) circle (  1.96);

\path[draw=drawColor,line width= 0.4pt,line join=round,line cap=round,fill=fillColor] (270.65,272.15) circle (  1.96);

\path[draw=drawColor,line width= 0.4pt,line join=round,line cap=round,fill=fillColor] (301.20,271.40) circle (  1.96);

%\path[draw=drawColor,line width= 0.4pt,line join=round,line cap=round,fill=fillColor] (362.29,272.15) circle (  1.96);
\end{scope}
\begin{scope}
\path[clip] (  0.00,  0.00) rectangle (432.17,289.08);
\definecolor{drawColor}{gray}{0.30}

\node[text=drawColor,anchor=base east,inner sep=0pt, outer sep=0pt, scale=  0.88] at ( 34.27, 40.49) {40};

\node[text=drawColor,anchor=base east,inner sep=0pt, outer sep=0pt, scale=  0.88] at ( 34.27, 59.54) {45};

\node[text=drawColor,anchor=base east,inner sep=0pt, outer sep=0pt, scale=  0.88] at ( 34.27, 78.59) {50};

\node[text=drawColor,anchor=base east,inner sep=0pt, outer sep=0pt, scale=  0.88] at ( 34.27, 97.64) {55};

\node[text=drawColor,anchor=base east,inner sep=0pt, outer sep=0pt, scale=  0.88] at ( 34.27,116.70) {60};

\node[text=drawColor,anchor=base east,inner sep=0pt, outer sep=0pt, scale=  0.88] at ( 34.27,135.75) {65};

\node[text=drawColor,anchor=base east,inner sep=0pt, outer sep=0pt, scale=  0.88] at ( 34.27,154.80) {70};

\node[text=drawColor,anchor=base east,inner sep=0pt, outer sep=0pt, scale=  0.88] at ( 34.27,173.86) {75};

\node[text=drawColor,anchor=base east,inner sep=0pt, outer sep=0pt, scale=  0.88] at ( 34.27,192.91) {80};

\node[text=drawColor,anchor=base east,inner sep=0pt, outer sep=0pt, scale=  0.88] at ( 34.27,211.96) {85};

\node[text=drawColor,anchor=base east,inner sep=0pt, outer sep=0pt, scale=  0.88] at ( 34.27,231.01) {90};

\node[text=drawColor,anchor=base east,inner sep=0pt, outer sep=0pt, scale=  0.88] at ( 34.27,250.07) {95};

\node[text=drawColor,anchor=base east,inner sep=0pt, outer sep=0pt, scale=  0.88] at ( 34.27,269.12) {100};
\end{scope}
\begin{scope}
\path[clip] (  0.00,  0.00) rectangle (432.17,289.08);
\definecolor{drawColor}{gray}{0.20}

\path[draw=drawColor,line width= 0.6pt,line join=round] ( 36.47, 43.52) --
	( 39.22, 43.52);

\path[draw=drawColor,line width= 0.6pt,line join=round] ( 36.47, 62.57) --
	( 39.22, 62.57);

\path[draw=drawColor,line width= 0.6pt,line join=round] ( 36.47, 81.62) --
	( 39.22, 81.62);

\path[draw=drawColor,line width= 0.6pt,line join=round] ( 36.47,100.68) --
	( 39.22,100.68);

\path[draw=drawColor,line width= 0.6pt,line join=round] ( 36.47,119.73) --
	( 39.22,119.73);

\path[draw=drawColor,line width= 0.6pt,line join=round] ( 36.47,138.78) --
	( 39.22,138.78);

\path[draw=drawColor,line width= 0.6pt,line join=round] ( 36.47,157.83) --
	( 39.22,157.83);

\path[draw=drawColor,line width= 0.6pt,line join=round] ( 36.47,176.89) --
	( 39.22,176.89);

\path[draw=drawColor,line width= 0.6pt,line join=round] ( 36.47,195.94) --
	( 39.22,195.94);

\path[draw=drawColor,line width= 0.6pt,line join=round] ( 36.47,214.99) --
	( 39.22,214.99);

\path[draw=drawColor,line width= 0.6pt,line join=round] ( 36.47,234.04) --
	( 39.22,234.04);

\path[draw=drawColor,line width= 0.6pt,line join=round] ( 36.47,253.10) --
	( 39.22,253.10);

\path[draw=drawColor,line width= 0.6pt,line join=round] ( 36.47,272.15) --
	( 39.22,272.15);
\end{scope}
\begin{scope}
\path[clip] (  0.00,  0.00) rectangle (432.17,289.08);
\definecolor{drawColor}{gray}{0.20}

\path[draw=drawColor,line width= 0.6pt,line join=round] (110.77, 29.34) --
	(110.77, 32.09);

\path[draw=drawColor,line width= 0.6pt,line join=round] (170.43, 29.34) --
	(170.43, 32.09);

\path[draw=drawColor,line width= 0.6pt,line join=round] (230.08, 29.34) --
	(230.08, 32.09);

\path[draw=drawColor,line width= 0.6pt,line join=round] (289.74, 29.34) --
	(289.74, 32.09);

\path[draw=drawColor,line width= 0.6pt,line join=round] (349.40, 29.34) --
	(349.40, 32.09);

\path[draw=drawColor,line width= 0.6pt,line join=round] (409.06, 29.34) --
	(409.06, 32.09);
\end{scope}
\begin{scope}
\path[clip] (  0.00,  0.00) rectangle (432.17,289.08);
\definecolor{drawColor}{gray}{0.30}

\node[text=drawColor,anchor=base,inner sep=0pt, outer sep=0pt, scale=  0.88] at (110.77, 21.08) {2500};

\node[text=drawColor,anchor=base,inner sep=0pt, outer sep=0pt, scale=  0.88] at (170.43, 21.08) {3000};

\node[text=drawColor,anchor=base,inner sep=0pt, outer sep=0pt, scale=  0.88] at (230.08, 21.08) {3500};

\node[text=drawColor,anchor=base,inner sep=0pt, outer sep=0pt, scale=  0.88] at (289.74, 21.08) {4000};

\node[text=drawColor,anchor=base,inner sep=0pt, outer sep=0pt, scale=  0.88] at (349.40, 21.08) {4500};

\node[text=drawColor,anchor=base,inner sep=0pt, outer sep=0pt, scale=  0.88] at (409.06, 21.08) {5000};
\end{scope}
\begin{scope}
\path[clip] (  0.00,  0.00) rectangle (432.17,289.08);
\definecolor{drawColor}{RGB}{0,0,0}

\node[text=drawColor,anchor=base,inner sep=0pt, outer sep=0pt, scale=  1.10] at (232.95,  8.00) {Anzahl an Adressen im Pool (x)};
\end{scope}
\begin{scope}
\path[clip] (  0.00,  0.00) rectangle (432.17,289.08);
\definecolor{drawColor}{RGB}{0,0,0}

\node[text=drawColor,rotate= 90.00,anchor=base,inner sep=0pt, outer sep=0pt, scale=  1] at ( 13.08,157.83) {Wahrscheinlichkeit (\%)};
\end{scope}
\begin{scope}
\path[clip] (  0.00,  0.00) rectangle (432.17,289.08);
\definecolor{fillColor}{RGB}{255,255,255}

\path[fill=fillColor] (262.57, 35.28) rectangle (426.50, 79.19);
\end{scope}
\begin{scope}
\path[clip] (  0.00,  0.00) rectangle (432.17,289.08);
\definecolor{drawColor}{RGB}{255,255,255}
\definecolor{fillColor}{gray}{0.95}

\path[draw=drawColor,line width= 0.6pt,line join=round,line cap=round,fill=fillColor] (268.26, 55.43) rectangle (282.71, 69.88);
\end{scope}
\begin{scope}
\path[clip] (  0.00,  0.00) rectangle (432.17,289.08);
\definecolor{drawColor}{RGB}{0,0,0}

\path[draw=drawColor,line width= 0.6pt,line join=round] (269.70, 62.66) -- (281.27, 62.66);
\end{scope}
\begin{scope}
\path[clip] (  0.00,  0.00) rectangle (432.17,289.08);
\definecolor{drawColor}{RGB}{255,255,255}
\definecolor{fillColor}{gray}{0.95}

\path[draw=drawColor,line width= 0.6pt,line join=round,line cap=round,fill=fillColor] (268.26, 40.97) rectangle (282.71, 55.43);
\end{scope}
\begin{scope}
\path[clip] (  0.00,  0.00) rectangle (432.17,289.08);
\definecolor{drawColor}{RGB}{0,0,0}

\path[draw=drawColor,line width= 0.6pt,dash pattern=on 2pt off 2pt ,line join=round] (269.70, 48.20) -- (281.27, 48.20);
\end{scope}
\begin{scope}
\path[clip] (  0.00,  0.00) rectangle (432.17,289.08);
\definecolor{drawColor}{RGB}{0,0,0}

\node[text=drawColor,anchor=base west,inner sep=0pt, outer sep=0pt, scale=  0.85] at (284.52, 59.63) {WS alle Eviction-Sets zu finden};
\end{scope}
\begin{scope}
\path[clip] (  0.00,  0.00) rectangle (432.17,289.08);
\definecolor{drawColor}{RGB}{0,0,0}

\node[text=drawColor,anchor=base west,inner sep=0pt, outer sep=0pt, scale=  0.85] at (284.52, 45.17) {WS ein fixes Eviction-Set zu finden};
\end{scope}
\end{tikzpicture}

\end{scaletikzpicturetowidth}
\caption{Die gestrichelte Kurve (Y-Achse) veranschaulicht die Wahrscheinlichkeit, ein fixes Eviction-Set mit einer bestimmten Anzahl an Pooladressen (X-Achse) bei einer Assoziativität von 16 zu finden. Die durchgezogene Linie stellt die Wahrscheinlichkeit dar, alle 8192 Eviction-Sets zu finden.}
\end{figure}

Häufig wird in Benchmarks angegeben, wie viele Eviction-Sets überhaupt gefunden wurden.
Diese Fragestellung ist identisch mit dem ersten Szenario.
Denn wenn ein fixes Eviction-Set mit einer Wahrscheinlichkeit von $x$ Prozent gefunden wird und die Zuordnung wie hier zufällig ist, dann werden insgesamt im Mittel $x/100 \cdot 128$ Eviction-Sets gefunden.

TODO In Abb. 3.1 Kurve mit realen Daten einfügen
TODO Benchmark Zeit einfügen, beschreiben das teilweise hohe Aussschläge von 5+ Min

%\todo{ja, das finde ich auch. dringend.}

\subsection{Eviction-Set Suchalgorithmus}
\label{evictionSetSearchAlgo}

Im Folgenden soll der Algorithmus beschrieben werden, der die Adressen im Pool verschiedenen Eviction-Sets zuordnet.
Dieser Algorithmus ist in der Lage, Adressen Cache-Sets zuzuordnen, ohne Näheres über die CPU (L3-Cache Größe usw.) und die Adressbits 12 bis 63 zu wissen.
Der Algorithmus basiert auf den von \cite{DriveByPaper} und \cite{PrimeAndAbort} beschriebenen Algorithmen zum finden von Eviction-Sets, wobei einige Optimierungen und Modifikationen implementiert und getestet wurden.

Der Eviction-Set-Konstruktionsalgorithmus besteht aus drei Hauptphasen, der Expand-, Contract- und Collect-Phase. 
Zu Anfang wird zufällig eine Zeugenadresse aus dem Adresspool ausgewählt, für welche danach ein Eviction-Set gefunden werden soll.
Die Annahme in der Expand-Phase ist, dass eine bestimmte Teilmenge der Adressen aus dem Pool, genannt Candidate-Set, ein Eviction-Set für die Zeugenadresse bildet, sofern der Pool groß genug ist. 
Um ein Candidate-Set zu testen, wird zuerst auf die Zeugenadresse zugegriffen, um sicherzustellen, dass diese im Cache landet.
Danach wird auf alle Adressen aus dem Candidate-Set zugegriffen und abschließend die Zugriffszeit auf die Zeugenadresse gemessen. 
Sofern das Candidate-Set ein Eviction-Set für die Zeugenadresse ist, werden die Daten der Zeugenadresse aus dem Cache verdrängt, wobei dieses Vorgehen bei einer erneuten Messung der Zeugenadresse zu einer erhöhten Zugriffszeit führt.
Dieser Vorgang wird mehrmals wiederholt, um den Einfluss des Timer- und System-Rauschens zu vermindern.

In der Expand-Phase wird dem Addresspool iterativ eine zufällige Adresse entnommen und dem anfangs leeren Candidate-Set hinzugefügt (siehe auch Pseudocode \ref{alg:evictionSetExpand}), wobei die Zeugenadresse niemals Teil des Candidate-Set wird.
Nach jeder Iteration wird das Candidate-Set auf die eben beschriebene Weise getestet.
Falls das Candidate-Set ein Eviction-Set für die Zeugenadresse ist, wird zur nächsten Phase übergegangen, andernfalls wird die Iteration fortgesetzt.

\begin{algorithm}[h]
\DontPrintSemicolon
\caption{Pseudo-Code für Expand-Phase des Eviction-Set Algorithmus}
\label{alg:evictionSetExpand}

\Fn{$Expand(evictionSet, memoryBlocks)$}{
	\While{size(candidateSet) > 0}{
		witnesss = SelectRandomItem(candidateSet)\;
		\If{checkevict(evictionSet, witnesss)}{
			\Return witnesss
		}
		evictionSet.add(witnesss)\;
	}
	\Return failed;
}
\end{algorithm}

Im Allgemeinen beinhaltet das Candidate-Set nach der Expand-Phase mehrere hundert Einträge, von denen eine Teilmenge der Größe 16 bereits ein Eviction-Set für die Zeugenadresse bilden würde. 
Der überwiegende Teil der Einträge gehört nicht zum gleichen Cache-Set wie die Zeugenadresse.
Diese überflüssigen Einträge würden den Prime-and-Probe- Vorgang erheblich verlangsamen. Deshalb wird in der Contract-Phase versucht, das Candidate-Set auf die Größe 16 einzudampfen (siehe auch Pseudocode \ref{alg:evictionSetContract}).
Hierzu wird ein Element aus dem Candidate-Set entfernt und dann erneut getestet, ob dieses reduzierte Candidate-Set noch ein Eviction-Set für die Zeugenadresse ist.
Falls ja, wird dieses Element wieder dem Adresspool hinzugefügt, andernfalls verbleibt es im Candidate-Set, da es notwendiger Bestandteil des Eviction-Sets ist.
Dieser Vorgang wird für jedes Element im Candidate-Set einmal durchgeführt, so dass im fehlerfreien Fall schlussendlich 16 Elemente im Candidate-Set verbleiben.
Durch Messrauschen kann auch hier wieder ein Element fälschlicherweise als relevant für das Eviction-Set eingestuft werden, weshalb die Contract-Phase dreimal wiederholt wird.
Die zweite und dritte Wiederholung sind weit weniger kostenintensiv, da das Candidate-Set nach einer Contract-Phase bereits mehrere 100 Einträge bereinigt wurde.

\begin{algorithm}[h]
%\algsetup{linenosize=\small}
\DontPrintSemicolon
\caption{Pseudo-Code für Contract-Phase des Eviction-Set Algorithmus}
\label{alg:evictionSetContract}
\Fn{$Contract(evictionSet, memoryBlocks, witness)$}{
	\ForEach{candidate in evictionSet}{
		\If{checkevict(evictionSet, witness)}{
			mermoryBlocks.add(candidate)\;
			evictionSet.add(candidate)\;	
		}		
	}
}
\end{algorithm}

Wenn im Anschluss sofort wieder eine neue Zeugenadresse aus dem Pool gewählt wird, könnte eine Adresse gewählt werden, welche auf dasselbe Cache-Set wie die vorherige Zeugenadresse abgebildet wird.
Deshalb folgt im Anschluss an eine erfolgreiche Contract-Phase die Collect-Phase.
In dieser werden alle Adressen aus dem Pool entfernt, welche ebenfalls von dem in der Contract-Phase gefundenen Eviction-Set aus dem Cache verdrängt wurden (siehe auch Pseudocode \ref{alg:evictionSetCollect}).
Durch diesen Schritt wird also vermieden, dass die spätere Menge von Eviction-Sets dahingehend überprüft werden müsste, ob Eviction-Sets paarweise dasselbe zugrundeliegende Cache-Set besitzen. 
Zudem beschleunigt die Collect-Phase die nächsten Iterationen, da weniger Adressen im Pool zur Verfügung stehen.
Hierzu wird eine Adresse aus dem Pool durch einen Zugriff in den Cache geladen und anschließend auf alle Einträge im Eviction-Set zugegriffen.
Daraufhin wird die Zugriffszeit auf die Adresse gemessen und bei einer erhöhten Zeit aus dem Pool entfernt, da dann die Adresse auf dasselbe Cache-Set wie die Einträge des Eviction-Set beziehungsweise die letzte Zeugenadresse abgebildet wird.

\begin{algorithm}[h]
\DontPrintSemicolon
\caption{Pseudo-Code für Collect-Phase des Eviction-Set Algorithmus}
\label{alg:evictionSetCollect}

\Fn{$Collect(evictionSet, memoryBlocks)$}{
	witnessSet = empty\;
	\ForEach{candidate in mermoryBlocks}{
		\If{checkevict(evictionSet, candidate)}{
			memoryBlocks.delete(candidate)\;
			witnessSet.add(candidate)\;
		}
	}
	\Return witnessSet;
}
\end{algorithm}

Zusammenfassend wird in der Expand-Phase das Candidate-Set soweit vergrößert, bis es ein Eviction-Set bildet, danach wird es in der Contract-Phase auf die Größe 16 verkleinert und anschließend in der Collect-Phase werden alle auf dasselbe Cache-Set abgebildeten Adressen aus dem Pool entfernt (siehe auch Pseudocode \ref{alg:evictionSetOverview}).
Das gefundene Eviction-Set wird gespeichert und der Vorgang solange wiederholt, bis die Anzahl der Pooladressen kleiner als 16 ist oder aufgrund von Fehlern in einer Phase mehrmals kein Eviction-Set gefunden wurde.

\begin{algorithm}[h]
\DontPrintSemicolon
\caption{Pseudo-Code für Eviction-Set Algorithmus}
\label{alg:evictionSetOverview}

\Fn{$EvictionSetFinder(memoryBlocks)$}{
    groups $\leftarrow$ empty\;
    \While{size(memoryBlocks > 0}{
        evictionSet $\leftarrow$ empty\;
		witness $\leftarrow$ expand(evictionSet, memoryBlocks)\;
		
		\If{witness != failed}{
    		contract(evictionSet, memoryBlocks, witness)\;
    		witnessSet $\leftarrow$ collect(evictionSet, memoryBlocks, witnessSet)\;
    		groups.add(union(evictionSet, witness, witnessSet))\;
		}
    }
}
\end{algorithm}

\subsection{Optimierung der Phasen}

Ein Eviction-Set muss mindestens die Größe der Cache-Assoziativität besitzen, weshalb es naheliegend ist, in der Expand-Phase nicht mit einem leeren Candidate-Set sondern mit einem der Größe der Assoziativität zu starten und so in jeder Expand-Phase Überprüfungen in der Größenordnung Assoziativität minus 1 einzusparen.
Außerdem ist es nicht optimal, bei jeder Iteration dem Candidate-Set nur eine Adresse hinzuzufügen, da insbesondere bei einem kleinen Candidate-Set eine geringe Wahrscheinlichkeit besteht, dass eine zusätzliche Adresse dieses zu einem Eviction-Set werden lässt.
Die Idee ist folglich, bei einem noch kleinen Candidate-Set in jeder Iteration möglichst viele Adressen aus dem Pool hinzuzufügen und mit zunehmender Größe des Candidate-Sets die Anzahl der in jeder Iteration hinzukommenden Adressen zu verringern.

Contract-Phase löschen meherer Einträge problematisch da viele Fehler

%Da zufällig eine Adresse aus dem Pool gezogen wird und die Aufteilung der Adressen im Pool auf die Cache-Sets als uniform angenommen werden kann (siehe Abschnitt \ref{addressPoolSize}), stellt sich die Frage, welches die optimale Anzahl an Adressen ist, die bei der ersten Iteration der Expand-Phase hinzugefügt werden sollten.

\todo{hattest du nicht auch getestet, in der contract-phase aussortierte Adressen direkt wieder dem CS zuzufügen?}

\subsection{Details der realen Implementierung}

Wie weiter oben beschrieben, wird insbesondere der Test, ob ein Set ein Eviction-Set für eine bestimmte Adresse darstellt, mehrfach wiederholt, um Fehler auszuschließen. 
Problematisch ist dies vor allem in der Expand-Phase, da das Candidate-Set eine Größe von mehreren hundert Einträgen annimmt und nach jeder Iteration gegen die Zeugenadresse getestet wird. 
Eine einzige erhöhte Zugriffszeitmessung würde das Candidate-Set fälschlicherweise als Eviction-Set für die Zeugenadresse einordnen. 
Es wurde festgestellt, dass erhöhte Zugriffszeiten mehrmals hintereinander auftreten können. Deshalb wird eine erhöhte Messung in der Expand-Phase 20 mal erneut überprüft. 
Die Überprüfung ab sobald eine der Messungen eine widersprüchliche Aussage zulässt.

Trotz der hohen Anzahl von 20 Wiederholungen sind die Kosten hierfür gering, da im fehlerfreien Optimalfall nur zusätzlich 20 Prüfungen anfallen und in den meisten Fehlerfällen nur einzelne zusätzliche Überprüfungen durchgeführt werden. 
Demgegenüber steht der Vorteil, nicht fälschlicherweise in die Contract-Phase zu wechseln und dort erst spät zu merken, dass das Candidate-Set kein Eviction-Set ist.

Wenn nun das Candidate-Set kein Eviction-Set für die Zeugenadresse ist, würde in der naiven Implementierung der Contract-Phase über alle Einträge des Candidate-Sets iteriert werden, aber kein Eintrag gelöscht.
Denn die Zugriffszeit auf die Zeugenadresse wird immer niedrig bleiben, was den Algorithmus zur fälschlichen Annahme verleitet, alle Einträge seien für das Eviction-Set notwendig.
Wir wissen jedoch, dass die Größe des Eviction-Sets der Assoziativität des L3-Caches entsprechen muss und können die Contract-Phase abbrechen, sobald eine mehr als der Assoziativität entsprechende Anzahl von Einträgen des Candidate-Sets als notwendig eingestuft wurden, da dann ein Fehler vorliegt.

Angenommen es soll ein Prime-and-Probe-Operation ausgeführt und auf alle Einträge des Eviction-Sets wird mittels einer For-Schleife einmal zugegriffen.
\todo{pointer chasing erklären}

\section{Verbesserte Eviction-Set Suche mittels Store-to-load-forwarding}

Die Forscher TODO sind auf ein Verhalten bei Intel-Prozessoren gestoßen, das das Auffinden von Adressen ermöglicht, deren 20 letzten physischen Bits gleich sind, ohne auf Techniken wie Huge-Pages zugreifen zu müssen.
Solche Adressen seien nachfolgend \textit{colliding addresses} genannt.
Die Idee hierbei ist, den Store-Buffer in einer Schleife mit einer Vielzahl von Schreibbefehlen zu fluten und direkt danach eine Zeitmessung für das Lesen einer Adresse $x$ auszuführen.

Wenn einer der zuletzt hinzugefügten Befehle auf eine Adresse schreibt, deren letzte 20 physischen Adressbits identisch mit denen der Read-Adresse $x$ sind, dann lässt sich ein Ausschlag bei der Leseoperation für $x$ feststellen.

Pseudocode \ref{alg:storeForward} beschreibt das Suchverfahren, um colliding addresses zu finden.
Zuerst wird ein großer Speicherbereich alloziert %(???) Fachausdruck für Speicher reservieren
, der ein Vielfaches der Page-Größe von 4 KiB hat.
Wie im Abschnitt \ref{evictionSetSearchAlgo} wird hier mittels 4-KiB-Speicherblöcken gesucht, da so die letzten 12 Bits der Adresse mit Sicherheit identisch sind.
Dies beschleunigt die Suche um den Faktor 4096, da dann im Mittel etwa eine von $2^8$ statt eine von $2^{20}$ Adressen eine colliding address ist.
Im Pseudocode werden colliding addresses zu $evictionBuffer[0]$ gesucht.
Hierzu werden iterativ (Zeile 3) Speicherblöcke aus dem Pool auf diese Eigenschaft hin untersucht.
Wie oben beschrieben muss der Store-Buffer mit Schreibbefehlen geflutet werden, wobei $WINDOW$ deren Anzahl angibt.
Zeile 6 und 7 sorgen für die Schreibbefehle auf die Speicherblöcke $p + WINDOW - 1$ bis $p$ des Pools, sodass der Schreibzugriff auf den aktuell zu testenden Block $p$ als letztes erfolgt.
Die Zeilen 8 bis 10 messen die Zeit für einen Lesezugriff auf die Adresse $evictionBuffer[0]$ beziehungsweise Speicherblock 0.

Der Testvorgang für einen Speicherblock $p$ wird mehrfach wiederholt (Zeile 5) und die Zugriffszeit auf $evictionBuffer[0]$ gemittelt (Zeile 11), um Ausreißer bei der Zugriffszeit durch Timer- oder Systemrauschen auszuschließen. 

Grafik \ref{fig:colliding_addresses_js_measurement} zeigt die typische Verteilung der Zugriffszeiten auf $evictionBuffer[0]$, stellt also $measurementBuffer$ visuell dar.






\begin{algorithm}[h]
\DontPrintSemicolon
\caption{Pseudo-C-Code für das Finden von colliding addresses}
\label{alg:storeForward}

\Fn{$FindCollidingAddresses()$}{
    evictionBuffer = calloc(1, PAGE_SIZE * PAGE_COUNT)\;
    
    \For{p from WINDOW to PAGE_COUNT-1}{
        total = 0\;
        \ForEach{r in $[0 \twodots Rounds]$}{
            \For{i from WINDOW-1 to 0}{
                evictionBuffer[(p-i) * PAGE_SIZE] = 0\;
            }
            timeStamp = rdtscp()\;
            read evictionBuffer[0]\;
            total += rdtscp() - timeStamp\;
        }
        measurementBuffer[p] = total / ROUNDS\;
    }
}
\end{algorithm}

Tests haben ergeben, dass mit $WINDOW=64$ colliding addresses zuverlässig durch Peaks bei der Zeitmessung der Leseoperation für $x$ hervorstechen.
Ein höherer Wert für $WINDOW$ bringt keine Vorteile bei der Identifizierung der colliding addressses, erhöht allerdings die Laufzeit der Suche.
Andersherum verringern kleinere Werte für $WINDOW$ die Suchlaufzeit, sorgen aber auch für kleinere Peaks der colliding addresses bei der Zeitmessung, sodass eine Identifikation schwieriger wird.
Da die Identifizierung unter Javascript bereits gegenüber C erschwert ist, existiert kein Spielraum, um den Wert von $WINDOW$ abzusenken. 

TODO Benchmarks einfügen, gegen alte Variante vergleichen




%\newtextend



\label{fig:colliding_addresses_js_measurement}
\begin{figure}[h]
\centering
\begin{scaletikzpicturetowidth}{\textwidth}
% Created by tikzDevice version 0.12 on 2018-09-13 19:52:01
% !TEX encoding = UTF-8 Unicode
\begin{tikzpicture}[scale=\tikzscale,y=21pt]
\definecolor{fillColor}{RGB}{255,255,255}
\path[use as bounding box,fill=fillColor,fill opacity=0.00] (0,0) rectangle (505.89,505.89);
\begin{scope}
\path[clip] ( 49.20, 61.20) rectangle (480.69,456.69);
\definecolor{drawColor}{RGB}{0,0,0}
\definecolor{fillColor}{RGB}{0,0,0}

\path[draw=drawColor,line width= 0.4pt,line join=round,line cap=round,fill=fillColor] ( 65.18,152.14) circle (  1.50);

\path[draw=drawColor,line width= 0.4pt,line join=round,line cap=round,fill=fillColor] ( 67.08,136.88) circle (  1.50);

\path[draw=drawColor,line width= 0.4pt,line join=round,line cap=round,fill=fillColor] ( 68.99,167.40) circle (  1.50);

\path[draw=drawColor,line width= 0.4pt,line join=round,line cap=round,fill=fillColor] ( 70.89,136.88) circle (  1.50);

\path[draw=drawColor,line width= 0.4pt,line join=round,line cap=round,fill=fillColor] ( 72.79,121.62) circle (  1.50);

\path[draw=drawColor,line width= 0.4pt,line join=round,line cap=round,fill=fillColor] ( 74.69,136.88) circle (  1.50);

\path[draw=drawColor,line width= 0.4pt,line join=round,line cap=round,fill=fillColor] ( 76.60,106.36) circle (  1.50);

\path[draw=drawColor,line width= 0.4pt,line join=round,line cap=round,fill=fillColor] ( 78.50,106.36) circle (  1.50);

\path[draw=drawColor,line width= 0.4pt,line join=round,line cap=round,fill=fillColor] ( 80.40,121.62) circle (  1.50);

\path[draw=drawColor,line width= 0.4pt,line join=round,line cap=round,fill=fillColor] ( 82.30,136.88) circle (  1.50);

\path[draw=drawColor,line width= 0.4pt,line join=round,line cap=round,fill=fillColor] ( 84.21,411.53) circle (  1.50);

\path[draw=drawColor,line width= 0.4pt,line join=round,line cap=round,fill=fillColor] ( 86.11,411.53) circle (  1.50);

\path[draw=drawColor,line width= 0.4pt,line join=round,line cap=round,fill=fillColor] ( 88.01,365.75) circle (  1.50);

\path[draw=drawColor,line width= 0.4pt,line join=round,line cap=round,fill=fillColor] ( 89.91,381.01) circle (  1.50);

\path[draw=drawColor,line width= 0.4pt,line join=round,line cap=round,fill=fillColor] ( 91.82,319.98) circle (  1.50);

\path[draw=drawColor,line width= 0.4pt,line join=round,line cap=round,fill=fillColor] ( 93.72,350.49) circle (  1.50);

\path[draw=drawColor,line width= 0.4pt,line join=round,line cap=round,fill=fillColor] ( 95.62,365.75) circle (  1.50);

\path[draw=drawColor,line width= 0.4pt,line join=round,line cap=round,fill=fillColor] ( 97.52,335.24) circle (  1.50);

\path[draw=drawColor,line width= 0.4pt,line join=round,line cap=round,fill=fillColor] ( 99.43,289.46) circle (  1.50);

\path[draw=drawColor,line width= 0.4pt,line join=round,line cap=round,fill=fillColor] (101.33,289.46) circle (  1.50);

\path[draw=drawColor,line width= 0.4pt,line join=round,line cap=round,fill=fillColor] (103.23,228.43) circle (  1.50);

\path[draw=drawColor,line width= 0.4pt,line join=round,line cap=round,fill=fillColor] (105.13,213.17) circle (  1.50);

\path[draw=drawColor,line width= 0.4pt,line join=round,line cap=round,fill=fillColor] (107.04,197.91) circle (  1.50);

\path[draw=drawColor,line width= 0.4pt,line join=round,line cap=round,fill=fillColor] (108.94,197.91) circle (  1.50);

\path[draw=drawColor,line width= 0.4pt,line join=round,line cap=round,fill=fillColor] (110.84,182.65) circle (  1.50);

\path[draw=drawColor,line width= 0.4pt,line join=round,line cap=round,fill=fillColor] (112.74,167.40) circle (  1.50);

\path[draw=drawColor,line width= 0.4pt,line join=round,line cap=round,fill=fillColor] (114.65,136.88) circle (  1.50);

\path[draw=drawColor,line width= 0.4pt,line join=round,line cap=round,fill=fillColor] (116.55,136.88) circle (  1.50);

\path[draw=drawColor,line width= 0.4pt,line join=round,line cap=round,fill=fillColor] (118.45,136.88) circle (  1.50);

\path[draw=drawColor,line width= 0.4pt,line join=round,line cap=round,fill=fillColor] (120.35,167.40) circle (  1.50);

\path[draw=drawColor,line width= 0.4pt,line join=round,line cap=round,fill=fillColor] (122.26,152.14) circle (  1.50);

\path[draw=drawColor,line width= 0.4pt,line join=round,line cap=round,fill=fillColor] (124.16,121.62) circle (  1.50);

\path[draw=drawColor,line width= 0.4pt,line join=round,line cap=round,fill=fillColor] (126.06, 91.11) circle (  1.50);

\path[draw=drawColor,line width= 0.4pt,line join=round,line cap=round,fill=fillColor] (127.96,106.36) circle (  1.50);

\path[draw=drawColor,line width= 0.4pt,line join=round,line cap=round,fill=fillColor] (129.87,121.62) circle (  1.50);

\path[draw=drawColor,line width= 0.4pt,line join=round,line cap=round,fill=fillColor] (131.77,121.62) circle (  1.50);

\path[draw=drawColor,line width= 0.4pt,line join=round,line cap=round,fill=fillColor] (133.67,106.36) circle (  1.50);

\path[draw=drawColor,line width= 0.4pt,line join=round,line cap=round,fill=fillColor] (135.57,167.40) circle (  1.50);

\path[draw=drawColor,line width= 0.4pt,line join=round,line cap=round,fill=fillColor] (137.48,152.14) circle (  1.50);

\path[draw=drawColor,line width= 0.4pt,line join=round,line cap=round,fill=fillColor] (139.38,121.62) circle (  1.50);

\path[draw=drawColor,line width= 0.4pt,line join=round,line cap=round,fill=fillColor] (141.28,136.88) circle (  1.50);

\path[draw=drawColor,line width= 0.4pt,line join=round,line cap=round,fill=fillColor] (143.18,136.88) circle (  1.50);

\path[draw=drawColor,line width= 0.4pt,line join=round,line cap=round,fill=fillColor] (145.09,121.62) circle (  1.50);

\path[draw=drawColor,line width= 0.4pt,line join=round,line cap=round,fill=fillColor] (146.99,106.36) circle (  1.50);

\path[draw=drawColor,line width= 0.4pt,line join=round,line cap=round,fill=fillColor] (148.89, 91.11) circle (  1.50);

\path[draw=drawColor,line width= 0.4pt,line join=round,line cap=round,fill=fillColor] (150.79,121.62) circle (  1.50);

\path[draw=drawColor,line width= 0.4pt,line join=round,line cap=round,fill=fillColor] (152.70,182.65) circle (  1.50);

\path[draw=drawColor,line width= 0.4pt,line join=round,line cap=round,fill=fillColor] (154.60,182.65) circle (  1.50);

\path[draw=drawColor,line width= 0.4pt,line join=round,line cap=round,fill=fillColor] (156.50,182.65) circle (  1.50);

\path[draw=drawColor,line width= 0.4pt,line join=round,line cap=round,fill=fillColor] (158.40,152.14) circle (  1.50);

\path[draw=drawColor,line width= 0.4pt,line join=round,line cap=round,fill=fillColor] (160.31,182.65) circle (  1.50);

\path[draw=drawColor,line width= 0.4pt,line join=round,line cap=round,fill=fillColor] (162.21,167.40) circle (  1.50);

\path[draw=drawColor,line width= 0.4pt,line join=round,line cap=round,fill=fillColor] (164.11,167.40) circle (  1.50);

\path[draw=drawColor,line width= 0.4pt,line join=round,line cap=round,fill=fillColor] (166.01,167.40) circle (  1.50);

\path[draw=drawColor,line width= 0.4pt,line join=round,line cap=round,fill=fillColor] (167.92,136.88) circle (  1.50);

\path[draw=drawColor,line width= 0.4pt,line join=round,line cap=round,fill=fillColor] (169.82,121.62) circle (  1.50);

\path[draw=drawColor,line width= 0.4pt,line join=round,line cap=round,fill=fillColor] (171.72,152.14) circle (  1.50);

\path[draw=drawColor,line width= 0.4pt,line join=round,line cap=round,fill=fillColor] (173.62,152.14) circle (  1.50);

\path[draw=drawColor,line width= 0.4pt,line join=round,line cap=round,fill=fillColor] (175.53,136.88) circle (  1.50);

\path[draw=drawColor,line width= 0.4pt,line join=round,line cap=round,fill=fillColor] (177.43,182.65) circle (  1.50);

\path[draw=drawColor,line width= 0.4pt,line join=round,line cap=round,fill=fillColor] (179.33,167.40) circle (  1.50);

\path[draw=drawColor,line width= 0.4pt,line join=round,line cap=round,fill=fillColor] (181.23,167.40) circle (  1.50);

\path[draw=drawColor,line width= 0.4pt,line join=round,line cap=round,fill=fillColor] (183.14,121.62) circle (  1.50);

\path[draw=drawColor,line width= 0.4pt,line join=round,line cap=round,fill=fillColor] (185.04,121.62) circle (  1.50);

\path[draw=drawColor,line width= 0.4pt,line join=round,line cap=round,fill=fillColor] (186.94,136.88) circle (  1.50);

\path[draw=drawColor,line width= 0.4pt,line join=round,line cap=round,fill=fillColor] (188.84,136.88) circle (  1.50);

\path[draw=drawColor,line width= 0.4pt,line join=round,line cap=round,fill=fillColor] (190.75,182.65) circle (  1.50);

\path[draw=drawColor,line width= 0.4pt,line join=round,line cap=round,fill=fillColor] (192.65,182.65) circle (  1.50);

\path[draw=drawColor,line width= 0.4pt,line join=round,line cap=round,fill=fillColor] (194.55,152.14) circle (  1.50);

\path[draw=drawColor,line width= 0.4pt,line join=round,line cap=round,fill=fillColor] (196.45,121.62) circle (  1.50);

\path[draw=drawColor,line width= 0.4pt,line join=round,line cap=round,fill=fillColor] (198.36,136.88) circle (  1.50);

\path[draw=drawColor,line width= 0.4pt,line join=round,line cap=round,fill=fillColor] (200.26,121.62) circle (  1.50);

\path[draw=drawColor,line width= 0.4pt,line join=round,line cap=round,fill=fillColor] (202.16, 75.85) circle (  1.50);

\path[draw=drawColor,line width= 0.4pt,line join=round,line cap=round,fill=fillColor] (204.06, 91.11) circle (  1.50);

\path[draw=drawColor,line width= 0.4pt,line join=round,line cap=round,fill=fillColor] (205.97,106.36) circle (  1.50);

\path[draw=drawColor,line width= 0.4pt,line join=round,line cap=round,fill=fillColor] (207.87, 91.11) circle (  1.50);

\path[draw=drawColor,line width= 0.4pt,line join=round,line cap=round,fill=fillColor] (209.77,152.14) circle (  1.50);

\path[draw=drawColor,line width= 0.4pt,line join=round,line cap=round,fill=fillColor] (211.67,152.14) circle (  1.50);

\path[draw=drawColor,line width= 0.4pt,line join=round,line cap=round,fill=fillColor] (213.58,152.14) circle (  1.50);

\path[draw=drawColor,line width= 0.4pt,line join=round,line cap=round,fill=fillColor] (215.48,152.14) circle (  1.50);

\path[draw=drawColor,line width= 0.4pt,line join=round,line cap=round,fill=fillColor] (217.38,121.62) circle (  1.50);

\path[draw=drawColor,line width= 0.4pt,line join=round,line cap=round,fill=fillColor] (219.28,152.14) circle (  1.50);

\path[draw=drawColor,line width= 0.4pt,line join=round,line cap=round,fill=fillColor] (221.19,396.27) circle (  1.50);

\path[draw=drawColor,line width= 0.4pt,line join=round,line cap=round,fill=fillColor] (223.09,365.75) circle (  1.50);

\path[draw=drawColor,line width= 0.4pt,line join=round,line cap=round,fill=fillColor] (224.99,365.75) circle (  1.50);

\path[draw=drawColor,line width= 0.4pt,line join=round,line cap=round,fill=fillColor] (226.89,396.27) circle (  1.50);

\path[draw=drawColor,line width= 0.4pt,line join=round,line cap=round,fill=fillColor] (228.80,350.49) circle (  1.50);

\path[draw=drawColor,line width= 0.4pt,line join=round,line cap=round,fill=fillColor] (230.70,350.49) circle (  1.50);

\path[draw=drawColor,line width= 0.4pt,line join=round,line cap=round,fill=fillColor] (232.60,396.27) circle (  1.50);

\path[draw=drawColor,line width= 0.4pt,line join=round,line cap=round,fill=fillColor] (234.50,335.24) circle (  1.50);

\path[draw=drawColor,line width= 0.4pt,line join=round,line cap=round,fill=fillColor] (236.41,335.24) circle (  1.50);

\path[draw=drawColor,line width= 0.4pt,line join=round,line cap=round,fill=fillColor] (238.31,274.20) circle (  1.50);

\path[draw=drawColor,line width= 0.4pt,line join=round,line cap=round,fill=fillColor] (240.21,197.91) circle (  1.50);

\path[draw=drawColor,line width= 0.4pt,line join=round,line cap=round,fill=fillColor] (242.11,243.69) circle (  1.50);

\path[draw=drawColor,line width= 0.4pt,line join=round,line cap=round,fill=fillColor] (244.02,228.43) circle (  1.50);

\path[draw=drawColor,line width= 0.4pt,line join=round,line cap=round,fill=fillColor] (245.92,228.43) circle (  1.50);

\path[draw=drawColor,line width= 0.4pt,line join=round,line cap=round,fill=fillColor] (247.82,228.43) circle (  1.50);

\path[draw=drawColor,line width= 0.4pt,line join=round,line cap=round,fill=fillColor] (249.72,167.40) circle (  1.50);

\path[draw=drawColor,line width= 0.4pt,line join=round,line cap=round,fill=fillColor] (251.63,182.65) circle (  1.50);

\path[draw=drawColor,line width= 0.4pt,line join=round,line cap=round,fill=fillColor] (253.53,182.65) circle (  1.50);

\path[draw=drawColor,line width= 0.4pt,line join=round,line cap=round,fill=fillColor] (255.43,136.88) circle (  1.50);

\path[draw=drawColor,line width= 0.4pt,line join=round,line cap=round,fill=fillColor] (257.33,136.88) circle (  1.50);

\path[draw=drawColor,line width= 0.4pt,line join=round,line cap=round,fill=fillColor] (259.24,167.40) circle (  1.50);

\path[draw=drawColor,line width= 0.4pt,line join=round,line cap=round,fill=fillColor] (261.14,136.88) circle (  1.50);

\path[draw=drawColor,line width= 0.4pt,line join=round,line cap=round,fill=fillColor] (263.04,228.43) circle (  1.50);

\path[draw=drawColor,line width= 0.4pt,line join=round,line cap=round,fill=fillColor] (264.94,136.88) circle (  1.50);

\path[draw=drawColor,line width= 0.4pt,line join=round,line cap=round,fill=fillColor] (266.85,152.14) circle (  1.50);

\path[draw=drawColor,line width= 0.4pt,line join=round,line cap=round,fill=fillColor] (268.75, 91.11) circle (  1.50);

\path[draw=drawColor,line width= 0.4pt,line join=round,line cap=round,fill=fillColor] (270.65,106.36) circle (  1.50);

\path[draw=drawColor,line width= 0.4pt,line join=round,line cap=round,fill=fillColor] (272.56,136.88) circle (  1.50);

\path[draw=drawColor,line width= 0.4pt,line join=round,line cap=round,fill=fillColor] (274.46,136.88) circle (  1.50);

\path[draw=drawColor,line width= 0.4pt,line join=round,line cap=round,fill=fillColor] (276.36,121.62) circle (  1.50);

\path[draw=drawColor,line width= 0.4pt,line join=round,line cap=round,fill=fillColor] (278.26,136.88) circle (  1.50);

\path[draw=drawColor,line width= 0.4pt,line join=round,line cap=round,fill=fillColor] (280.17,152.14) circle (  1.50);

\path[draw=drawColor,line width= 0.4pt,line join=round,line cap=round,fill=fillColor] (282.07,152.14) circle (  1.50);

\path[draw=drawColor,line width= 0.4pt,line join=round,line cap=round,fill=fillColor] (283.97,136.88) circle (  1.50);

\path[draw=drawColor,line width= 0.4pt,line join=round,line cap=round,fill=fillColor] (285.87,152.14) circle (  1.50);

\path[draw=drawColor,line width= 0.4pt,line join=round,line cap=round,fill=fillColor] (287.78,152.14) circle (  1.50);

\path[draw=drawColor,line width= 0.4pt,line join=round,line cap=round,fill=fillColor] (289.68,152.14) circle (  1.50);

\path[draw=drawColor,line width= 0.4pt,line join=round,line cap=round,fill=fillColor] (291.58,121.62) circle (  1.50);

\path[draw=drawColor,line width= 0.4pt,line join=round,line cap=round,fill=fillColor] (293.48,121.62) circle (  1.50);

\path[draw=drawColor,line width= 0.4pt,line join=round,line cap=round,fill=fillColor] (295.39, 91.11) circle (  1.50);

\path[draw=drawColor,line width= 0.4pt,line join=round,line cap=round,fill=fillColor] (297.29,106.36) circle (  1.50);

\path[draw=drawColor,line width= 0.4pt,line join=round,line cap=round,fill=fillColor] (299.19,152.14) circle (  1.50);

\path[draw=drawColor,line width= 0.4pt,line join=round,line cap=round,fill=fillColor] (301.09,106.36) circle (  1.50);

\path[draw=drawColor,line width= 0.4pt,line join=round,line cap=round,fill=fillColor] (303.00,136.88) circle (  1.50);

\path[draw=drawColor,line width= 0.4pt,line join=round,line cap=round,fill=fillColor] (304.90,136.88) circle (  1.50);

\path[draw=drawColor,line width= 0.4pt,line join=round,line cap=round,fill=fillColor] (306.80, 91.11) circle (  1.50);

\path[draw=drawColor,line width= 0.4pt,line join=round,line cap=round,fill=fillColor] (308.70,136.88) circle (  1.50);

\path[draw=drawColor,line width= 0.4pt,line join=round,line cap=round,fill=fillColor] (310.61,121.62) circle (  1.50);

\path[draw=drawColor,line width= 0.4pt,line join=round,line cap=round,fill=fillColor] (312.51,121.62) circle (  1.50);

\path[draw=drawColor,line width= 0.4pt,line join=round,line cap=round,fill=fillColor] (314.41,106.36) circle (  1.50);

\path[draw=drawColor,line width= 0.4pt,line join=round,line cap=round,fill=fillColor] (316.31,136.88) circle (  1.50);

\path[draw=drawColor,line width= 0.4pt,line join=round,line cap=round,fill=fillColor] (318.22,152.14) circle (  1.50);

\path[draw=drawColor,line width= 0.4pt,line join=round,line cap=round,fill=fillColor] (320.12,167.40) circle (  1.50);

\path[draw=drawColor,line width= 0.4pt,line join=round,line cap=round,fill=fillColor] (322.02,136.88) circle (  1.50);

\path[draw=drawColor,line width= 0.4pt,line join=round,line cap=round,fill=fillColor] (323.92,167.40) circle (  1.50);

\path[draw=drawColor,line width= 0.4pt,line join=round,line cap=round,fill=fillColor] (325.83,136.88) circle (  1.50);

\path[draw=drawColor,line width= 0.4pt,line join=round,line cap=round,fill=fillColor] (327.73,136.88) circle (  1.50);

\path[draw=drawColor,line width= 0.4pt,line join=round,line cap=round,fill=fillColor] (329.63,167.40) circle (  1.50);

\path[draw=drawColor,line width= 0.4pt,line join=round,line cap=round,fill=fillColor] (331.53,152.14) circle (  1.50);

\path[draw=drawColor,line width= 0.4pt,line join=round,line cap=round,fill=fillColor] (333.44,167.40) circle (  1.50);

\path[draw=drawColor,line width= 0.4pt,line join=round,line cap=round,fill=fillColor] (335.34,182.65) circle (  1.50);

\path[draw=drawColor,line width= 0.4pt,line join=round,line cap=round,fill=fillColor] (337.24,167.40) circle (  1.50);

\path[draw=drawColor,line width= 0.4pt,line join=round,line cap=round,fill=fillColor] (339.14,136.88) circle (  1.50);

\path[draw=drawColor,line width= 0.4pt,line join=round,line cap=round,fill=fillColor] (341.05, 91.11) circle (  1.50);

\path[draw=drawColor,line width= 0.4pt,line join=round,line cap=round,fill=fillColor] (342.95,106.36) circle (  1.50);

\path[draw=drawColor,line width= 0.4pt,line join=round,line cap=round,fill=fillColor] (344.85,121.62) circle (  1.50);

\path[draw=drawColor,line width= 0.4pt,line join=round,line cap=round,fill=fillColor] (346.75,152.14) circle (  1.50);

\path[draw=drawColor,line width= 0.4pt,line join=round,line cap=round,fill=fillColor] (348.66,152.14) circle (  1.50);

\path[draw=drawColor,line width= 0.4pt,line join=round,line cap=round,fill=fillColor] (350.56,197.91) circle (  1.50);

\path[draw=drawColor,line width= 0.4pt,line join=round,line cap=round,fill=fillColor] (352.46,182.65) circle (  1.50);

\path[draw=drawColor,line width= 0.4pt,line join=round,line cap=round,fill=fillColor] (354.36,182.65) circle (  1.50);

\path[draw=drawColor,line width= 0.4pt,line join=round,line cap=round,fill=fillColor] (356.27,197.91) circle (  1.50);

\path[draw=drawColor,line width= 0.4pt,line join=round,line cap=round,fill=fillColor] (358.17,197.91) circle (  1.50);

\path[draw=drawColor,line width= 0.4pt,line join=round,line cap=round,fill=fillColor] (360.07,167.40) circle (  1.50);

\path[draw=drawColor,line width= 0.4pt,line join=round,line cap=round,fill=fillColor] (361.97,197.91) circle (  1.50);

\path[draw=drawColor,line width= 0.4pt,line join=round,line cap=round,fill=fillColor] (363.88,152.14) circle (  1.50);

\path[draw=drawColor,line width= 0.4pt,line join=round,line cap=round,fill=fillColor] (365.78,197.91) circle (  1.50);

\path[draw=drawColor,line width= 0.4pt,line join=round,line cap=round,fill=fillColor] (367.68,197.91) circle (  1.50);

\path[draw=drawColor,line width= 0.4pt,line join=round,line cap=round,fill=fillColor] (369.58,152.14) circle (  1.50);

\path[draw=drawColor,line width= 0.4pt,line join=round,line cap=round,fill=fillColor] (371.49,197.91) circle (  1.50);

\path[draw=drawColor,line width= 0.4pt,line join=round,line cap=round,fill=fillColor] (373.39,197.91) circle (  1.50);

\path[draw=drawColor,line width= 0.4pt,line join=round,line cap=round,fill=fillColor] (375.29,182.65) circle (  1.50);

\path[draw=drawColor,line width= 0.4pt,line join=round,line cap=round,fill=fillColor] (377.19,182.65) circle (  1.50);

\path[draw=drawColor,line width= 0.4pt,line join=round,line cap=round,fill=fillColor] (379.10,167.40) circle (  1.50);

\path[draw=drawColor,line width= 0.4pt,line join=round,line cap=round,fill=fillColor] (381.00,136.88) circle (  1.50);

\path[draw=drawColor,line width= 0.4pt,line join=round,line cap=round,fill=fillColor] (382.90,136.88) circle (  1.50);

\path[draw=drawColor,line width= 0.4pt,line join=round,line cap=round,fill=fillColor] (384.80,152.14) circle (  1.50);

\path[draw=drawColor,line width= 0.4pt,line join=round,line cap=round,fill=fillColor] (386.71,167.40) circle (  1.50);

\path[draw=drawColor,line width= 0.4pt,line join=round,line cap=round,fill=fillColor] (388.61,167.40) circle (  1.50);

\path[draw=drawColor,line width= 0.4pt,line join=round,line cap=round,fill=fillColor] (390.51,167.40) circle (  1.50);

\path[draw=drawColor,line width= 0.4pt,line join=round,line cap=round,fill=fillColor] (392.41,136.88) circle (  1.50);

\path[draw=drawColor,line width= 0.4pt,line join=round,line cap=round,fill=fillColor] (394.32,167.40) circle (  1.50);

\path[draw=drawColor,line width= 0.4pt,line join=round,line cap=round,fill=fillColor] (396.22,136.88) circle (  1.50);

\path[draw=drawColor,line width= 0.4pt,line join=round,line cap=round,fill=fillColor] (398.12,136.88) circle (  1.50);

\path[draw=drawColor,line width= 0.4pt,line join=round,line cap=round,fill=fillColor] (400.02,182.65) circle (  1.50);

\path[draw=drawColor,line width= 0.4pt,line join=round,line cap=round,fill=fillColor] (401.93,182.65) circle (  1.50);

\path[draw=drawColor,line width= 0.4pt,line join=round,line cap=round,fill=fillColor] (403.83,197.91) circle (  1.50);

\path[draw=drawColor,line width= 0.4pt,line join=round,line cap=round,fill=fillColor] (405.73,197.91) circle (  1.50);

\path[draw=drawColor,line width= 0.4pt,line join=round,line cap=round,fill=fillColor] (407.63,182.65) circle (  1.50);

\path[draw=drawColor,line width= 0.4pt,line join=round,line cap=round,fill=fillColor] (409.54,197.91) circle (  1.50);

\path[draw=drawColor,line width= 0.4pt,line join=round,line cap=round,fill=fillColor] (411.44,197.91) circle (  1.50);

\path[draw=drawColor,line width= 0.4pt,line join=round,line cap=round,fill=fillColor] (413.34,167.40) circle (  1.50);

\path[draw=drawColor,line width= 0.4pt,line join=round,line cap=round,fill=fillColor] (415.24,121.62) circle (  1.50);

\path[draw=drawColor,line width= 0.4pt,line join=round,line cap=round,fill=fillColor] (417.15,121.62) circle (  1.50);

\path[draw=drawColor,line width= 0.4pt,line join=round,line cap=round,fill=fillColor] (419.05,182.65) circle (  1.50);

\path[draw=drawColor,line width= 0.4pt,line join=round,line cap=round,fill=fillColor] (420.95,167.40) circle (  1.50);

\path[draw=drawColor,line width= 0.4pt,line join=round,line cap=round,fill=fillColor] (422.85,136.88) circle (  1.50);

\path[draw=drawColor,line width= 0.4pt,line join=round,line cap=round,fill=fillColor] (424.76,106.36) circle (  1.50);

\path[draw=drawColor,line width= 0.4pt,line join=round,line cap=round,fill=fillColor] (426.66,381.01) circle (  1.50);

\path[draw=drawColor,line width= 0.4pt,line join=round,line cap=round,fill=fillColor] (428.56,396.27) circle (  1.50);

\path[draw=drawColor,line width= 0.4pt,line join=round,line cap=round,fill=fillColor] (430.46,365.75) circle (  1.50);

\path[draw=drawColor,line width= 0.4pt,line join=round,line cap=round,fill=fillColor] (432.37,396.27) circle (  1.50);

\path[draw=drawColor,line width= 0.4pt,line join=round,line cap=round,fill=fillColor] (434.27,365.75) circle (  1.50);

\path[draw=drawColor,line width= 0.4pt,line join=round,line cap=round,fill=fillColor] (436.17,396.27) circle (  1.50);

\path[draw=drawColor,line width= 0.4pt,line join=round,line cap=round,fill=fillColor] (438.07,396.27) circle (  1.50);

\path[draw=drawColor,line width= 0.4pt,line join=round,line cap=round,fill=fillColor] (439.98,442.04) circle (  1.50);

\path[draw=drawColor,line width= 0.4pt,line join=round,line cap=round,fill=fillColor] (441.88,335.24) circle (  1.50);

\path[draw=drawColor,line width= 0.4pt,line join=round,line cap=round,fill=fillColor] (443.78,274.20) circle (  1.50);

\path[draw=drawColor,line width= 0.4pt,line join=round,line cap=round,fill=fillColor] (445.68,289.46) circle (  1.50);

\path[draw=drawColor,line width= 0.4pt,line join=round,line cap=round,fill=fillColor] (447.59,258.95) circle (  1.50);

\path[draw=drawColor,line width= 0.4pt,line join=round,line cap=round,fill=fillColor] (449.49,213.17) circle (  1.50);

\path[draw=drawColor,line width= 0.4pt,line join=round,line cap=round,fill=fillColor] (451.39,213.17) circle (  1.50);

\path[draw=drawColor,line width= 0.4pt,line join=round,line cap=round,fill=fillColor] (453.29,182.65) circle (  1.50);

\path[draw=drawColor,line width= 0.4pt,line join=round,line cap=round,fill=fillColor] (455.20,167.40) circle (  1.50);

\path[draw=drawColor,line width= 0.4pt,line join=round,line cap=round,fill=fillColor] (457.10,182.65) circle (  1.50);

\path[draw=drawColor,line width= 0.4pt,line join=round,line cap=round,fill=fillColor] (459.00,167.40) circle (  1.50);

\path[draw=drawColor,line width= 0.4pt,line join=round,line cap=round,fill=fillColor] (460.90,152.14) circle (  1.50);

\path[draw=drawColor,line width= 0.4pt,line join=round,line cap=round,fill=fillColor] (462.81,167.40) circle (  1.50);

\path[draw=drawColor,line width= 0.4pt,line join=round,line cap=round,fill=fillColor] (464.71,167.40) circle (  1.50);
\end{scope}
\begin{scope}
\path[clip] (  0.00,  0.00) rectangle (505.89,505.89);
\definecolor{drawColor}{RGB}{0,0,0}

\path[draw=drawColor,line width= 0.4pt,line join=round,line cap=round] (112.74, 61.20) -- (398.12, 61.20);

\path[draw=drawColor,line width= 0.4pt,line join=round,line cap=round] (112.74, 61.20) -- (112.74, 55.20);

\path[draw=drawColor,line width= 0.4pt,line join=round,line cap=round] (207.87, 61.20) -- (207.87, 55.20);

\path[draw=drawColor,line width= 0.4pt,line join=round,line cap=round] (303.00, 61.20) -- (303.00, 55.20);

\path[draw=drawColor,line width= 0.4pt,line join=round,line cap=round] (398.12, 61.20) -- (398.12, 55.20);

\node[text=drawColor,anchor=base,inner sep=0pt, outer sep=0pt, scale=  1.00] at (112.74, 39.60) {3000};

\node[text=drawColor,anchor=base,inner sep=0pt, outer sep=0pt, scale=  1.00] at (207.87, 39.60) {3050};

\node[text=drawColor,anchor=base,inner sep=0pt, outer sep=0pt, scale=  1.00] at (303.00, 39.60) {3100};

\node[text=drawColor,anchor=base,inner sep=0pt, outer sep=0pt, scale=  1.00] at (398.12, 39.60) {3150};

\path[draw=drawColor,line width= 0.4pt,line join=round,line cap=round] ( 49.20,136.88) -- ( 49.20,442.04);

\path[draw=drawColor,line width= 0.4pt,line join=round,line cap=round] ( 49.20,136.88) -- ( 43.20,136.88);

\path[draw=drawColor,line width= 0.4pt,line join=round,line cap=round] ( 49.20,213.17) -- ( 43.20,213.17);

\path[draw=drawColor,line width= 0.4pt,line join=round,line cap=round] ( 49.20,289.46) -- ( 43.20,289.46);

\path[draw=drawColor,line width= 0.4pt,line join=round,line cap=round] ( 49.20,365.75) -- ( 43.20,365.75);

\path[draw=drawColor,line width= 0.4pt,line join=round,line cap=round] ( 49.20,442.04) -- ( 43.20,442.04);

\node[text=drawColor,rotate= 90.00,anchor=base,inner sep=0pt, outer sep=0pt, scale=  1.00] at ( 34.80,136.88) {30};

\node[text=drawColor,rotate= 90.00,anchor=base,inner sep=0pt, outer sep=0pt, scale=  1.00] at ( 34.80,213.17) {35};

\node[text=drawColor,rotate= 90.00,anchor=base,inner sep=0pt, outer sep=0pt, scale=  1.00] at ( 34.80,289.46) {40};

\node[text=drawColor,rotate= 90.00,anchor=base,inner sep=0pt, outer sep=0pt, scale=  1.00] at ( 34.80,365.75) {45};

\node[text=drawColor,rotate= 90.00,anchor=base,inner sep=0pt, outer sep=0pt, scale=  1.00] at ( 34.80,442.04) {50};

\path[draw=drawColor,line width= 0.4pt,line join=round,line cap=round] ( 49.20, 61.20) --
	(480.69, 61.20) --
	(480.69,456.69) --
	( 49.20,456.69) --
	( 49.20, 61.20);
\end{scope}
\begin{scope}
\path[clip] (  0.00,  0.00) rectangle (505.89,505.89);
\definecolor{drawColor}{RGB}{0,0,0}

\node[text=drawColor,anchor=base,inner sep=0pt, outer sep=0pt, scale=  1.00] at (264.94, 15.60) {4KiB block index};

\node[text=drawColor,rotate= 90.00,anchor=base,inner sep=0pt, outer sep=0pt, scale=  1.00] at ( 10.80,258.94) {SharedArrayBuffer timer value};
\end{scope}
\begin{scope}
\path[clip] ( 49.20, 61.20) rectangle (480.69,456.69);
\definecolor{drawColor}{RGB}{0,0,0}

\path[draw=drawColor,line width= 0.4pt,line join=round,line cap=round] ( 84.21,411.53) circle (  5.25);

\path[draw=drawColor,line width= 0.4pt,line join=round,line cap=round] (221.19,396.27) circle (  5.25);

\path[draw=drawColor,line width= 0.4pt,line join=round,line cap=round] (426.66,381.01) circle (  5.25);
\end{scope}
\end{tikzpicture}

\end{scaletikzpicturetowidth}
\caption{Erkennung von colliding addresses unter Javascript. Punkte geben die Latenz für eine Leseoperation auf eine fixe Adresse an, nachdem Schreiboperationen auf jeden der 64 Blöcke im aktuellen Fenster durchgeführt wurden. Die drei dickeren Punkte stellen vom Algorithmus identifizierte colliding addresses dar. Es wurde verifiziert, dass diese drei Adressen Teil eines späteren Eviction-Set geworden sind und somit die Identifikation seitens des Algorithmus korrekt war.}
\end{figure}


%TODO Algorithmus in Einzelheiten beschreiben
%\todo[size=\footnotesize]{Beschreibe den Algorithmus bitte in einzelteilen, und gib auch den code dafür einzeln an. Der komplette Code gehört in den Anhang.}
%\todo[size=\footnotesize]{Nutze eine kleinere Schriftart für listings}


\section{Verdeckter Kanal}

Die maximale Sendegeschwindigkeit eines Kanals ist durch die Rate, mit welcher der Sender ein beliebiges Cache-Set primen kann, begrenzt.
Damit der Empfänger ein zufälliges Rauschen von einem Priming unterscheiden kann, sollte der Sender mehrere Einträge aus dem zu primenden Cache-Set verdrängen, wobei im Optimalfall die Anzahl der zugegriffenen Speicheradressen der Assoziativität des Caches entspricht.
Hiermit wird die Wahrscheinlichkeit erhöht, dass sich die vom Empfänger im Probe-Schritt gemessene Zugriffszeit signifikant von Fällen unterscheidet, in denen zufällig einzelne Einträge aus dem überwachten Cache-Set verdrängt werden. 
%Daraus folgernd nehmen wir an, dass der Sender in seiner Priming-Phase auf der Assoziativität entsprechend viele Speicheradressen zugreift.
Im Folgenden sollen verschiedene Methoden des Primens eines Cache-Sets verglichen werden, indem entweder die Anzahl der zugegriffen Speicheradressen oder die Zugriffsmethode verändert werden.
Wenn etwa die Zahl der zugegriffenen Speicheraddressen verringert wird, sind auf der einen Seite mehr Timeslots in einem Zeitabschnitt möglich, und die Chance sinkt, dass benachbarte Timeslots zusätzlich beeinflusst werden. Auf der anderen Seite sind die messbaren Ausschläge der Zugriffszeiten verringert, wodurch ein bewusst geprimtes Cache-Set schwieriger von einem Messrauschen oder von zufälligen Zugriffen unterschieden werden kann.
Sende- und Empfangsseite können durchaus abweichende Parameter verwenden, wenn wie etwa im vorliegenden Fall der Empfänger langsamer als der Sender arbeitet. Um die Timeslots anzugleichen, könnte der Empfänger die Dauer einer Priming-Operation durch die Senkung der Anzahl der zugegriffenen Speicheraddressen verringern und der Empfänger andersherum die Dauer für eine Priming-Operation erhöhen. 

\begin{algorithm}[h]
\DontPrintSemicolon
\caption{Pseudo-Code für Pointer-Chasing-Methode}
\label{alg:pointerChasing}

\Fn{$AccessTimeEvictionSet(pointerToAddress)$}{
    pointerToAddressFirst $\leftarrow$ pointerToAddress\;
    timestampBefore $\leftarrow$ getTimeStamp()\;
    \While{pointerToAddressFirst != pointerToAddress}{
        pointerToAddress $\leftarrow$ readValue(pointerToAddress)\;
    }
    \Return getTimeStamp() - timestampBefore
}

\end{algorithm}

Auf dem Testrechner benötigt ein in C geschriebenes Sendeprogramm für eine Million Prime-Vorgänge mit 16 Adressen und der Single-Pointer-Chasing-Methode (siehe Algorithmus \ref{alg:pointerChasing}) etwa 323 Millionen Taktzyklen.
Im Optimalfall kann im Timeslot $x$ ein durch den Sender erfolgter Prime-Vorgang als 1 und ein nicht erfolgter Prime-Vorgang  als 0 interpretiert werden.
%\todo{was ist ein erfolgreicher/nicht erfolgreicher prime-vorgang?}
%satz durch ergänzung vermutlich besser zu verstehen
Bei einem typischen All-Core-Turbo-Takt von 3,4 Ghz des i7-4770 ergibt sich so eine maximale Senderate von TODO ~10,5 Mbit/s.
Diese Rate wird jedoch vom Empfänger beschränkt, welcher zusätzlich noch eine Zeitmessung durchführen muss. Der Worst-Case ist hier eine in Webassembly geschriebene Empfangsroutine, da dort eine Zeitmessung kostenintensiver ist. In Chromium 66 können eine Million Messungen eines Cache-Sets in etwa 200 ms durchgeführt werden.
Im Mittel dauert eine Messung also 0,2 \textmu s, womit eine Empfangsrate von maximal ~5 MBit/s realisiert werden kann.

%Um die Performance zu erhöhen kann wie erwähnt die Anzahl der zugegriffen Speicheradressen reduziert werden. 
Im Folgenden soll die maximal mögliche Senderate unter optimalen Bedingungen ermittelt werden. 
Hierfür wird im Voraus ein Cache-Set ausgewählt, auf dem im Idle-Zustand des Systems ein geringes Rauschen herrscht.
Um die Synchronisation des Senders und Empfängers aufrechtzuerhalten, wird nach 10 gesendeten Bits ein Synchronisationsblock eingefügt, welcher durch $sb$-Prime-Vorgänge auf der Senderseite erzeugt wird. 
Eine 1 wird durch $s$-Prime-Vorgänge repräsentiert und eine 0 durch das Unterlassen der Prime-Vorgänge. 
Um die einzelnen Bits auseinanderzuhalten, wird zwischen jedem gesendeten Bit eine Pause von $p$-Taktzyklen eingelegt. 
Grafik \ref{fig:covert_channel} zeigt die Übertragung eines Bitstrings von einem C-Programm zum Javascript/Webassembly-Empfänger.
Zwischen dem Senden eines Bits wurde eine Pause von $p=1500$ Taktzyklen eingelegt.
Die Größe des Synchronisationsblocks ist auf $sb=10$ gesetzt.
Mit den in der Grafik verwendeten Parametern ergibt sich eine ungefähre Netto-Datenrate von 4,6 KB/s.

\captionsetup[figure]{skip=-15pt}
\label{fig:covert_channel}
\begin{figure}[h]
\centering
\begin{scaletikzpicturetowidth}{\textwidth}
% Created by tikzDevice version 0.12 on 2018-09-19 19:48:18
% !TEX encoding = UTF-8 Unicode
\begin{tikzpicture}[scale=\tikzscale,y=18pt]
\definecolor{fillColor}{RGB}{255,255,255}
\path[use as bounding box,fill=fillColor,fill opacity=0.00] (0,0) rectangle (505.89,505.89);
\begin{scope}
\path[clip] ( 75.60,315.34) rectangle (490.29,478.29);
\definecolor{drawColor}{RGB}{0,0,0}

\path[draw=drawColor,line width= 0.4pt,line join=round,line cap=round] ( 90.96,321.38) --
	( 92.06,321.38) --
	( 93.15,321.38) --
	( 94.25,321.38) --
	( 95.35,321.38) --
	( 96.44,321.38) --
	( 97.54,321.38) --
	( 98.64,321.38) --
	( 99.74,321.38) --
	(100.83,321.38) --
	(101.93,321.38) --
	(103.03,321.38) --
	(104.12,321.38) --
	(105.22,321.38) --
	(106.32,321.38) --
	(107.41,321.38) --
	(108.51,321.38) --
	(109.61,321.38) --
	(110.71,321.38) --
	(111.80,321.38) --
	(112.90,380.83) --
	(114.00,428.26) --
	(115.09,454.54) --
	(116.19,472.25) --
	(117.29,403.69) --
	(118.39,431.11) --
	(119.48,429.97) --
	(120.58,442.54) --
	(121.68,465.97) --
	(122.77,439.11) --
	(123.87,472.25) --
	(124.97,472.25) --
	(126.06,472.25) --
	(127.16,471.68) --
	(128.26,429.97) --
	(129.36,373.97) --
	(130.45,321.38) --
	(131.55,321.38) --
	(132.65,321.38) --
	(133.74,321.38) --
	(134.84,321.38) --
	(135.94,321.38) --
	(137.04,321.38) --
	(138.13,321.38) --
	(139.23,321.38) --
	(140.33,321.38) --
	(141.42,321.38) --
	(142.52,321.38) --
	(143.62,321.38) --
	(144.72,321.38) --
	(145.81,321.38) --
	(146.91,321.38) --
	(148.01,321.38) --
	(149.10,321.38) --
	(150.20,321.38) --
	(151.30,321.38) --
	(152.39,321.38) --
	(153.49,321.38) --
	(154.59,321.38) --
	(155.69,321.38) --
	(156.78,321.38) --
	(157.88,321.38) --
	(158.98,343.69) --
	(160.07,394.54) --
	(161.17,439.68) --
	(162.27,457.40) --
	(163.37,356.83) --
	(164.46,321.38) --
	(165.56,321.38) --
	(166.66,321.38) --
	(167.75,321.38) --
	(168.85,321.38) --
	(169.95,321.38) --
	(171.04,321.38) --
	(172.14,321.38) --
	(173.24,321.38) --
	(174.34,321.38) --
	(175.43,321.38) --
	(176.53,321.38) --
	(177.63,321.38) --
	(178.72,321.38) --
	(179.82,321.38) --
	(180.92,321.38) --
	(182.02,321.38) --
	(183.11,321.38) --
	(184.21,321.38) --
	(185.31,321.38) --
	(186.40,321.38) --
	(187.50,321.38) --
	(188.60,321.38) --
	(189.69,321.38) --
	(190.79,321.38) --
	(191.89,321.38) --
	(192.99,321.38) --
	(194.08,321.38) --
	(195.18,321.38) --
	(196.28,321.38) --
	(197.37,321.38) --
	(198.47,321.38) --
	(199.57,321.38) --
	(200.67,321.38) --
	(201.76,321.38) --
	(202.86,321.38) --
	(203.96,321.38) --
	(205.05,321.38) --
	(206.15,389.97) --
	(207.25,464.26) --
	(208.34,457.40) --
	(209.44,393.40) --
	(210.54,327.12) --
	(211.64,321.38) --
	(212.73,321.38) --
	(213.83,321.38) --
	(214.93,321.38) --
	(216.02,321.38) --
	(217.12,379.69) --
	(218.22,427.68) --
	(219.32,445.97) --
	(220.41,418.54) --
	(221.51,380.26) --
	(222.61,321.38) --
	(223.70,321.38) --
	(224.80,321.38) --
	(225.90,321.38) --
	(226.99,351.69) --
	(228.09,414.54) --
	(229.19,423.11) --
	(230.29,472.25) --
	(231.38,402.54) --
	(232.48,321.38) --
	(233.58,321.38) --
	(234.67,321.38) --
	(235.77,321.38) --
	(236.87,321.38) --
	(237.97,321.38) --
	(239.06,321.38) --
	(240.16,321.38) --
	(241.26,321.38) --
	(242.35,321.38) --
	(243.45,321.38) --
	(244.55,321.38) --
	(245.64,321.38) --
	(246.74,321.38) --
	(247.84,321.38) --
	(248.94,321.38) --
	(250.03,321.38) --
	(251.13,321.38) --
	(252.23,388.26) --
	(253.32,424.26) --
	(254.42,463.11) --
	(255.52,457.40) --
	(256.62,452.26) --
	(257.71,439.68) --
	(258.81,439.68) --
	(259.91,455.68) --
	(261.00,439.11) --
	(262.10,455.68) --
	(263.20,439.68) --
	(264.29,456.83) --
	(265.39,416.83) --
	(266.49,432.83) --
	(267.59,432.83) --
	(268.68,435.11) --
	(269.78,351.69) --
	(270.88,321.38) --
	(271.97,321.38) --
	(273.07,321.38) --
	(274.17,321.38) --
	(275.27,321.38) --
	(276.36,321.38) --
	(277.46,321.38) --
	(278.56,321.38) --
	(279.65,321.38) --
	(280.75,321.38) --
	(281.85,321.38) --
	(282.95,321.38) --
	(284.04,321.38) --
	(285.14,321.38) --
	(286.24,321.38) --
	(287.33,321.38) --
	(288.43,321.38) --
	(289.53,321.38) --
	(290.62,321.38) --
	(291.72,321.38) --
	(292.82,321.38) --
	(293.92,321.38) --
	(295.01,321.38) --
	(296.11,321.38) --
	(297.21,321.38) --
	(298.30,321.38) --
	(299.40,321.38) --
	(300.50,321.38) --
	(301.60,321.38) --
	(302.69,321.38) --
	(303.79,321.38) --
	(304.89,321.38) --
	(305.98,388.83) --
	(307.08,436.83) --
	(308.18,431.11) --
	(309.27,440.26) --
	(310.37,345.97) --
	(311.47,321.38) --
	(312.57,321.38) --
	(313.66,321.38) --
	(314.76,321.38) --
	(315.86,321.38) --
	(316.95,321.38) --
	(318.05,321.38) --
	(319.15,321.38) --
	(320.25,321.38) --
	(321.34,321.38) --
	(322.44,321.38) --
	(323.54,321.38) --
	(324.63,321.38) --
	(325.73,321.38) --
	(326.83,321.38) --
	(327.92,321.38) --
	(329.02,321.38) --
	(330.12,321.38) --
	(331.22,321.38) --
	(332.31,321.38) --
	(333.41,321.38) --
	(334.51,321.38) --
	(335.60,321.38) --
	(336.70,321.38) --
	(337.80,321.38) --
	(338.90,321.38) --
	(339.99,321.38) --
	(341.09,321.38) --
	(342.19,321.38) --
	(343.28,321.38) --
	(344.38,321.38) --
	(345.48,321.38) --
	(346.57,321.38) --
	(347.67,321.38) --
	(348.77,321.38) --
	(349.87,321.38) --
	(350.96,321.38) --
	(352.06,321.38) --
	(353.16,321.38) --
	(354.25,321.38) --
	(355.35,321.38) --
	(356.45,321.38) --
	(357.55,321.38) --
	(358.64,321.38) --
	(359.74,321.38) --
	(360.84,361.40) --
	(361.93,456.26) --
	(363.03,430.54) --
	(364.13,467.11) --
	(365.22,373.97) --
	(366.32,365.40) --
	(367.42,321.38) --
	(368.52,321.38) --
	(369.61,321.38) --
	(370.71,321.38) --
	(371.81,388.83) --
	(372.90,396.26) --
	(374.00,472.25) --
	(375.10,416.26) --
	(376.20,395.69) --
	(377.29,321.38) --
	(378.39,321.38) --
	(379.49,321.38) --
	(380.58,321.38) --
	(381.68,321.38) --
	(382.78,321.38) --
	(383.87,401.40) --
	(384.97,413.97) --
	(386.07,401.97) --
	(387.17,453.40) --
	(388.26,346.54) --
	(389.36,321.38) --
	(390.46,321.38) --
	(391.55,321.38) --
	(392.65,321.38) --
	(393.75,321.38) --
	(394.85,321.38) --
	(395.94,321.38) --
	(397.04,321.38) --
	(398.14,321.38) --
	(399.23,321.38) --
	(400.33,376.83) --
	(401.43,367.69) --
	(402.52,348.83) --
	(403.62,401.40) --
	(404.72,421.97) --
	(405.82,424.26) --
	(406.91,472.25) --
	(408.01,464.26) --
	(409.11,433.40) --
	(410.20,452.26) --
	(411.30,472.25) --
	(412.40,472.25) --
	(413.50,446.54) --
	(414.59,449.97) --
	(415.69,464.83) --
	(416.79,421.40) --
	(417.88,424.26) --
	(418.98,447.11) --
	(420.08,436.83) --
	(421.18,384.26) --
	(422.27,321.38) --
	(423.37,321.38) --
	(424.47,321.38) --
	(425.56,321.38) --
	(426.66,321.38) --
	(427.76,321.38) --
	(428.85,321.38) --
	(429.95,321.38) --
	(431.05,321.38) --
	(432.15,321.38) --
	(433.24,321.38) --
	(434.34,321.38) --
	(435.44,321.38) --
	(436.53,321.38) --
	(437.63,321.38) --
	(438.73,321.38) --
	(439.83,321.38) --
	(440.92,321.38) --
	(442.02,321.38) --
	(443.12,321.38) --
	(444.21,321.38) --
	(445.31,321.38) --
	(446.41,321.38) --
	(447.50,321.38) --
	(448.60,321.38) --
	(449.70,321.38) --
	(450.80,321.38) --
	(451.89,321.38) --
	(452.99,321.38) --
	(454.09,321.38) --
	(455.18,321.38) --
	(456.28,321.38) --
	(457.38,402.54) --
	(458.48,444.26) --
	(459.57,447.11) --
	(460.67,434.54) --
	(461.77,366.54) --
	(462.86,321.38) --
	(463.96,321.38) --
	(465.06,321.38) --
	(466.15,321.38) --
	(467.25,321.38) --
	(468.35,321.38) --
	(469.45,321.38) --
	(470.54,321.38) --
	(471.64,321.38) --
	(472.74,321.38) --
	(473.83,321.38) --
	(474.93,321.38);
\end{scope}
\begin{scope}
\path[clip] (  0.00,  0.00) rectangle (505.89,505.89);
\definecolor{drawColor}{RGB}{0,0,0}

\path[draw=drawColor,line width= 0.4pt,line join=round,line cap=round] ( 89.86,315.34) -- (473.83,315.34);

\path[draw=drawColor,line width= 0.4pt,line join=round,line cap=round] ( 89.86,315.34) -- ( 89.86,309.34);

\path[draw=drawColor,line width= 0.4pt,line join=round,line cap=round] (144.72,315.34) -- (144.72,309.34);

\path[draw=drawColor,line width= 0.4pt,line join=round,line cap=round] (199.57,315.34) -- (199.57,309.34);

\path[draw=drawColor,line width= 0.4pt,line join=round,line cap=round] (254.42,315.34) -- (254.42,309.34);

\path[draw=drawColor,line width= 0.4pt,line join=round,line cap=round] (309.27,315.34) -- (309.27,309.34);

\path[draw=drawColor,line width= 0.4pt,line join=round,line cap=round] (364.13,315.34) -- (364.13,309.34);

\path[draw=drawColor,line width= 0.4pt,line join=round,line cap=round] (418.98,315.34) -- (418.98,309.34);

\path[draw=drawColor,line width= 0.4pt,line join=round,line cap=round] (473.83,315.34) -- (473.83,309.34);

\node[text=drawColor,anchor=base,inner sep=0pt, outer sep=0pt, scale=  1.00] at ( 89.86,293.75) {0};

\node[text=drawColor,anchor=base,inner sep=0pt, outer sep=0pt, scale=  1.00] at (144.72,293.75) {50};

\node[text=drawColor,anchor=base,inner sep=0pt, outer sep=0pt, scale=  1.00] at (199.57,293.75) {100};

\node[text=drawColor,anchor=base,inner sep=0pt, outer sep=0pt, scale=  1.00] at (254.42,293.75) {150};

\node[text=drawColor,anchor=base,inner sep=0pt, outer sep=0pt, scale=  1.00] at (309.27,293.75) {200};

\node[text=drawColor,anchor=base,inner sep=0pt, outer sep=0pt, scale=  1.00] at (364.13,293.75) {250};

\node[text=drawColor,anchor=base,inner sep=0pt, outer sep=0pt, scale=  1.00] at (418.98,293.75) {300};

\node[text=drawColor,anchor=base,inner sep=0pt, outer sep=0pt, scale=  1.00] at (473.83,293.75) {350};

\path[draw=drawColor,line width= 0.4pt,line join=round,line cap=round] ( 75.60,315.34) --
	(490.29,315.34) --
	(490.29,478.29) --
	( 75.60,478.29) --
	( 75.60,315.34);
\end{scope}
\begin{scope}
\path[clip] ( 25.20,264.94) rectangle (504.69,504.69);
\definecolor{drawColor}{RGB}{0,0,0}

\node[text=drawColor,anchor=base,inner sep=0pt, outer sep=0pt, scale=  1.00] at (282.94,269.75) {Samples};

\node[text=drawColor,rotate= 90.00,anchor=base,inner sep=0pt, outer sep=0pt, scale=  1.00] at ( 37.20,396.82) {Zugrifsfzeit};
\end{scope}
\begin{scope}
\path[clip] (  0.00,  0.00) rectangle (505.89,505.89);
\definecolor{drawColor}{RGB}{0,0,0}

\path[draw=drawColor,line width= 0.4pt,line join=round,line cap=round] ( 75.60,315.34) -- ( 75.60,472.25);

\path[draw=drawColor,line width= 0.4pt,line join=round,line cap=round] ( 75.60,357.97) -- ( 69.60,357.97);

\path[draw=drawColor,line width= 0.4pt,line join=round,line cap=round] ( 75.60,415.11) -- ( 69.60,415.11);

\path[draw=drawColor,line width= 0.4pt,line join=round,line cap=round] ( 75.60,472.25) -- ( 69.60,472.25);

\node[text=drawColor,rotate= 90.00,anchor=base,inner sep=0pt, outer sep=0pt, scale=  1.00] at ( 61.20,357.97) {200};

\node[text=drawColor,rotate= 90.00,anchor=base,inner sep=0pt, outer sep=0pt, scale=  1.00] at ( 61.20,415.11) {300};

\node[text=drawColor,rotate= 90.00,anchor=base,inner sep=0pt, outer sep=0pt, scale=  1.00] at ( 61.20,472.25) {400};
\end{scope}
\begin{scope}
\path[clip] ( 75.60,315.34) rectangle (490.29,478.29);
\definecolor{drawColor}{RGB}{190,190,190}

\path[draw=drawColor,line width= 0.8pt,line join=round,line cap=round] ( 90.96,399.07) --
	( 92.06,399.07) --
	( 93.15,399.07) --
	( 94.25,399.07) --
	( 95.35,399.07) --
	( 96.44,399.07) --
	( 97.54,399.07) --
	( 98.64,399.07) --
	( 99.74,399.07) --
	(100.83,399.07) --
	(101.93,399.07) --
	(103.03,399.07) --
	(104.12,399.07) --
	(105.22,399.07) --
	(106.32,399.07) --
	(107.41,399.07) --
	(108.51,399.07) --
	(109.61,399.07) --
	(110.71,399.07) --
	(111.80,399.07) --
	(112.90,399.07) --
	(114.00,399.07) --
	(115.09,399.07) --
	(116.19,399.07) --
	(117.29,399.07) --
	(118.39,399.07) --
	(119.48,399.07) --
	(120.58,399.07) --
	(121.68,399.07) --
	(122.77,399.07) --
	(123.87,399.07) --
	(124.97,399.07) --
	(126.06,399.07) --
	(127.16,399.07) --
	(128.26,399.07) --
	(129.36,399.07) --
	(130.45,399.07) --
	(131.55,399.07) --
	(132.65,399.07) --
	(133.74,399.07) --
	(134.84,399.07) --
	(135.94,399.07) --
	(137.04,399.07) --
	(138.13,399.07) --
	(139.23,399.07) --
	(140.33,399.07) --
	(141.42,399.07) --
	(142.52,399.07) --
	(143.62,399.07) --
	(144.72,399.07) --
	(145.81,399.07) --
	(146.91,399.07) --
	(148.01,399.07) --
	(149.10,399.07) --
	(150.20,399.07) --
	(151.30,399.07) --
	(152.39,399.07) --
	(153.49,399.07) --
	(154.59,399.07) --
	(155.69,399.07) --
	(156.78,399.07) --
	(157.88,399.07) --
	(158.98,399.07) --
	(160.07,399.07) --
	(161.17,399.07) --
	(162.27,399.07) --
	(163.37,399.07) --
	(164.46,399.07) --
	(165.56,399.07) --
	(166.66,399.07) --
	(167.75,399.07) --
	(168.85,399.07) --
	(169.95,399.07) --
	(171.04,399.07) --
	(172.14,399.07) --
	(173.24,399.07) --
	(174.34,399.07) --
	(175.43,399.07) --
	(176.53,399.07) --
	(177.63,399.07) --
	(178.72,399.07) --
	(179.82,399.07) --
	(180.92,399.07) --
	(182.02,399.07) --
	(183.11,399.07) --
	(184.21,399.07) --
	(185.31,399.07) --
	(186.40,399.07) --
	(187.50,399.07) --
	(188.60,399.07) --
	(189.69,399.07) --
	(190.79,399.07) --
	(191.89,399.07) --
	(192.99,399.07) --
	(194.08,399.07) --
	(195.18,399.07) --
	(196.28,399.07) --
	(197.37,399.07) --
	(198.47,399.07) --
	(199.57,399.07) --
	(200.67,399.07) --
	(201.76,399.07) --
	(202.86,399.07) --
	(203.96,399.07) --
	(205.05,399.07) --
	(206.15,399.07) --
	(207.25,399.07) --
	(208.34,399.07) --
	(209.44,399.07) --
	(210.54,399.07) --
	(211.64,399.07) --
	(212.73,399.07) --
	(213.83,399.07) --
	(214.93,399.07) --
	(216.02,399.07) --
	(217.12,399.07) --
	(218.22,399.07) --
	(219.32,399.07) --
	(220.41,399.07) --
	(221.51,399.07) --
	(222.61,399.07) --
	(223.70,399.07) --
	(224.80,399.07) --
	(225.90,399.07) --
	(226.99,399.07) --
	(228.09,399.07) --
	(229.19,399.07) --
	(230.29,399.07) --
	(231.38,399.07) --
	(232.48,399.07) --
	(233.58,399.07) --
	(234.67,399.07) --
	(235.77,399.07) --
	(236.87,399.07) --
	(237.97,399.07) --
	(239.06,399.07) --
	(240.16,399.07) --
	(241.26,399.07) --
	(242.35,399.07) --
	(243.45,399.07) --
	(244.55,399.07) --
	(245.64,399.07) --
	(246.74,399.07) --
	(247.84,399.07) --
	(248.94,399.07) --
	(250.03,399.07) --
	(251.13,399.07) --
	(252.23,399.07) --
	(253.32,399.07) --
	(254.42,399.07) --
	(255.52,399.07) --
	(256.62,399.07) --
	(257.71,399.07) --
	(258.81,399.07) --
	(259.91,399.07) --
	(261.00,399.07) --
	(262.10,399.07) --
	(263.20,399.07) --
	(264.29,399.07) --
	(265.39,399.07) --
	(266.49,399.07) --
	(267.59,399.07) --
	(268.68,399.07) --
	(269.78,399.07) --
	(270.88,399.07) --
	(271.97,399.07) --
	(273.07,399.07) --
	(274.17,399.07) --
	(275.27,399.07) --
	(276.36,399.07) --
	(277.46,399.07) --
	(278.56,399.07) --
	(279.65,399.07) --
	(280.75,399.07) --
	(281.85,399.07) --
	(282.95,399.07) --
	(284.04,399.07) --
	(285.14,399.07) --
	(286.24,399.07) --
	(287.33,399.07) --
	(288.43,399.07) --
	(289.53,399.07) --
	(290.62,399.07) --
	(291.72,399.07) --
	(292.82,399.07) --
	(293.92,399.07) --
	(295.01,399.07) --
	(296.11,399.07) --
	(297.21,399.07) --
	(298.30,399.07) --
	(299.40,399.07) --
	(300.50,399.07) --
	(301.60,399.07) --
	(302.69,399.07) --
	(303.79,399.07) --
	(304.89,399.07) --
	(305.98,399.07) --
	(307.08,399.07) --
	(308.18,399.07) --
	(309.27,399.07) --
	(310.37,399.07) --
	(311.47,399.07) --
	(312.57,399.07) --
	(313.66,399.07) --
	(314.76,399.07) --
	(315.86,399.07) --
	(316.95,399.07) --
	(318.05,399.07) --
	(319.15,399.07) --
	(320.25,399.07) --
	(321.34,399.07) --
	(322.44,399.07) --
	(323.54,399.07) --
	(324.63,399.07) --
	(325.73,399.07) --
	(326.83,399.07) --
	(327.92,399.07) --
	(329.02,399.07) --
	(330.12,399.07) --
	(331.22,399.07) --
	(332.31,399.07) --
	(333.41,399.07) --
	(334.51,399.07) --
	(335.60,399.07) --
	(336.70,399.07) --
	(337.80,399.07) --
	(338.90,399.07) --
	(339.99,399.07) --
	(341.09,399.07) --
	(342.19,399.07) --
	(343.28,399.07) --
	(344.38,399.07) --
	(345.48,399.07) --
	(346.57,399.07) --
	(347.67,399.07) --
	(348.77,399.07) --
	(349.87,399.07) --
	(350.96,399.07) --
	(352.06,399.07) --
	(353.16,399.07) --
	(354.25,399.07) --
	(355.35,399.07) --
	(356.45,399.07) --
	(357.55,399.07) --
	(358.64,399.07) --
	(359.74,399.07) --
	(360.84,399.07) --
	(361.93,399.07) --
	(363.03,399.07) --
	(364.13,399.07) --
	(365.22,399.07) --
	(366.32,399.07) --
	(367.42,399.07) --
	(368.52,399.07) --
	(369.61,399.07) --
	(370.71,399.07) --
	(371.81,399.07) --
	(372.90,399.07) --
	(374.00,399.07) --
	(375.10,399.07) --
	(376.20,399.07) --
	(377.29,399.07) --
	(378.39,399.07) --
	(379.49,399.07) --
	(380.58,399.07) --
	(381.68,399.07) --
	(382.78,399.07) --
	(383.87,399.07) --
	(384.97,399.07) --
	(386.07,399.07) --
	(387.17,399.07) --
	(388.26,399.07) --
	(389.36,399.07) --
	(390.46,399.07) --
	(391.55,399.07) --
	(392.65,399.07) --
	(393.75,399.07) --
	(394.85,399.07) --
	(395.94,399.07) --
	(397.04,399.07) --
	(398.14,399.07) --
	(399.23,399.07) --
	(400.33,399.07) --
	(401.43,399.07) --
	(402.52,399.07) --
	(403.62,399.07) --
	(404.72,399.07) --
	(405.82,399.07) --
	(406.91,399.07) --
	(408.01,399.07) --
	(409.11,399.07) --
	(410.20,399.07) --
	(411.30,399.07) --
	(412.40,399.07) --
	(413.50,399.07) --
	(414.59,399.07) --
	(415.69,399.07) --
	(416.79,399.07) --
	(417.88,399.07) --
	(418.98,399.07) --
	(420.08,399.07) --
	(421.18,399.07) --
	(422.27,399.07) --
	(423.37,399.07) --
	(424.47,399.07) --
	(425.56,399.07) --
	(426.66,399.07) --
	(427.76,399.07) --
	(428.85,399.07) --
	(429.95,399.07) --
	(431.05,399.07) --
	(432.15,399.07) --
	(433.24,399.07) --
	(434.34,399.07) --
	(435.44,399.07) --
	(436.53,399.07) --
	(437.63,399.07) --
	(438.73,399.07) --
	(439.83,399.07) --
	(440.92,399.07) --
	(442.02,399.07) --
	(443.12,399.07) --
	(444.21,399.07) --
	(445.31,399.07) --
	(446.41,399.07) --
	(447.50,399.07) --
	(448.60,399.07) --
	(449.70,399.07) --
	(450.80,399.07) --
	(451.89,399.07) --
	(452.99,399.07) --
	(454.09,399.07) --
	(455.18,399.07) --
	(456.28,399.07) --
	(457.38,399.07) --
	(458.48,399.07) --
	(459.57,399.07) --
	(460.67,399.07) --
	(461.77,399.07) --
	(462.86,399.07) --
	(463.96,399.07) --
	(465.06,399.07) --
	(466.15,399.07) --
	(467.25,399.07) --
	(468.35,399.07) --
	(469.45,399.07) --
	(470.54,399.07) --
	(471.64,399.07) --
	(472.74,399.07) --
	(473.83,399.07) --
	(474.93,399.07);
\end{scope}
\begin{scope}
\path[clip] ( 75.60, 75.60) rectangle (490.29,238.54);
\definecolor{drawColor}{RGB}{0,0,0}

\path[draw=drawColor,line width= 0.8pt,line join=round,line cap=round] ( 90.96, 81.63) --
	( 90.96, 81.63) --
	( 92.06, 81.63) --
	( 92.06, 81.63) --
	( 93.15, 81.63) --
	( 93.15, 81.63) --
	( 94.25, 81.63) --
	( 94.25, 81.63) --
	( 95.35, 81.63) --
	( 95.35, 81.63) --
	( 96.44, 81.63) --
	( 96.44, 81.63) --
	( 97.54, 81.63) --
	( 97.54, 81.63) --
	( 98.64, 81.63) --
	( 98.64, 81.63) --
	( 99.74, 81.63) --
	( 99.74, 81.63) --
	(100.83, 81.63) --
	(100.83, 81.63) --
	(101.93, 81.63) --
	(101.93, 81.63) --
	(103.03, 81.63) --
	(103.03, 81.63) --
	(104.12, 81.63) --
	(104.12, 81.63) --
	(105.22, 81.63) --
	(105.22, 81.63) --
	(106.32, 81.63) --
	(106.32, 81.63) --
	(107.41, 81.63) --
	(107.41, 81.63) --
	(108.51, 81.63) --
	(108.51, 81.63) --
	(109.61, 81.63) --
	(109.61, 81.63) --
	(110.71, 81.63) --
	(110.71, 81.63) --
	(111.80, 81.63) --
	(111.80, 81.63) --
	(112.90, 81.63) --
	(112.90,232.51) --
	(114.00,232.51) --
	(114.00,232.51) --
	(115.09,232.51) --
	(115.09,232.51) --
	(116.19,232.51) --
	(116.19,232.51) --
	(117.29,232.51) --
	(117.29,232.51) --
	(118.39,232.51) --
	(118.39,232.51) --
	(119.48,232.51) --
	(119.48,232.51) --
	(120.58,232.51) --
	(120.58,232.51) --
	(121.68,232.51) --
	(121.68,232.51) --
	(122.77,232.51) --
	(122.77,232.51) --
	(123.87,232.51) --
	(123.87,232.51) --
	(124.97,232.51) --
	(124.97,232.51) --
	(126.06,232.51) --
	(126.06,232.51) --
	(127.16,232.51) --
	(127.16,232.51) --
	(128.26,232.51) --
	(128.26, 81.63) --
	(129.36, 81.63) --
	(129.36, 81.63) --
	(130.45, 81.63) --
	(130.45, 81.63) --
	(131.55, 81.63) --
	(131.55, 81.63) --
	(132.65, 81.63) --
	(132.65, 81.63) --
	(133.74, 81.63) --
	(133.74, 81.63) --
	(134.84, 81.63) --
	(134.84, 81.63) --
	(135.94, 81.63) --
	(135.94, 81.63) --
	(137.04, 81.63) --
	(137.04, 81.63) --
	(138.13, 81.63) --
	(138.13, 81.63) --
	(139.23, 81.63) --
	(139.23, 81.63) --
	(140.33, 81.63) --
	(140.33, 81.63) --
	(141.42, 81.63) --
	(141.42, 81.63) --
	(142.52, 81.63) --
	(142.52, 81.63) --
	(143.62, 81.63) --
	(143.62, 81.63) --
	(144.72, 81.63) --
	(144.72, 81.63) --
	(145.81, 81.63) --
	(145.81, 81.63) --
	(146.91, 81.63) --
	(146.91, 81.63) --
	(148.01, 81.63) --
	(148.01, 81.63) --
	(149.10, 81.63) --
	(149.10, 81.63) --
	(150.20, 81.63) --
	(150.20, 81.63) --
	(151.30, 81.63) --
	(151.30, 81.63) --
	(152.39, 81.63) --
	(152.39, 81.63) --
	(153.49, 81.63) --
	(153.49, 81.63) --
	(154.59, 81.63) --
	(154.59, 81.63) --
	(155.69, 81.63) --
	(155.69, 81.63) --
	(156.78, 81.63) --
	(156.78, 81.63) --
	(157.88, 81.63) --
	(157.88, 81.63) --
	(158.98, 81.63) --
	(158.98, 81.63) --
	(160.07, 81.63) --
	(160.07,232.51) --
	(161.17,232.51) --
	(161.17,232.51) --
	(162.27,232.51) --
	(162.27, 81.63) --
	(163.37, 81.63) --
	(163.37, 81.63) --
	(164.46, 81.63) --
	(164.46, 81.63) --
	(165.56, 81.63) --
	(165.56, 81.63) --
	(166.66, 81.63) --
	(166.66, 81.63) --
	(167.75, 81.63) --
	(167.75, 81.63) --
	(168.85, 81.63) --
	(168.85, 81.63) --
	(169.95, 81.63) --
	(169.95, 81.63) --
	(171.04, 81.63) --
	(171.04, 81.63) --
	(172.14, 81.63) --
	(172.14, 81.63) --
	(173.24, 81.63) --
	(173.24, 81.63) --
	(174.34, 81.63) --
	(174.34, 81.63) --
	(175.43, 81.63) --
	(175.43, 81.63) --
	(176.53, 81.63) --
	(176.53, 81.63) --
	(177.63, 81.63) --
	(177.63, 81.63) --
	(178.72, 81.63) --
	(178.72, 81.63) --
	(179.82, 81.63) --
	(179.82, 81.63) --
	(180.92, 81.63) --
	(180.92, 81.63) --
	(182.02, 81.63) --
	(182.02, 81.63) --
	(183.11, 81.63) --
	(183.11, 81.63) --
	(184.21, 81.63) --
	(184.21, 81.63) --
	(185.31, 81.63) --
	(185.31, 81.63) --
	(186.40, 81.63) --
	(186.40, 81.63) --
	(187.50, 81.63) --
	(187.50, 81.63) --
	(188.60, 81.63) --
	(188.60, 81.63) --
	(189.69, 81.63) --
	(189.69, 81.63) --
	(190.79, 81.63) --
	(190.79, 81.63) --
	(191.89, 81.63) --
	(191.89, 81.63) --
	(192.99, 81.63) --
	(192.99, 81.63) --
	(194.08, 81.63) --
	(194.08, 81.63) --
	(195.18, 81.63) --
	(195.18, 81.63) --
	(196.28, 81.63) --
	(196.28, 81.63) --
	(197.37, 81.63) --
	(197.37, 81.63) --
	(198.47, 81.63) --
	(198.47, 81.63) --
	(199.57, 81.63) --
	(199.57, 81.63) --
	(200.67, 81.63) --
	(200.67, 81.63) --
	(201.76, 81.63) --
	(201.76, 81.63) --
	(202.86, 81.63) --
	(202.86, 81.63) --
	(203.96, 81.63) --
	(203.96, 81.63) --
	(205.05, 81.63) --
	(205.05, 81.63) --
	(206.15, 81.63) --
	(206.15,232.51) --
	(207.25,232.51) --
	(207.25,232.51) --
	(208.34,232.51) --
	(208.34, 81.63) --
	(209.44, 81.63) --
	(209.44, 81.63) --
	(210.54, 81.63) --
	(210.54, 81.63) --
	(211.64, 81.63) --
	(211.64, 81.63) --
	(212.73, 81.63) --
	(212.73, 81.63) --
	(213.83, 81.63) --
	(213.83, 81.63) --
	(214.93, 81.63) --
	(214.93, 81.63) --
	(216.02, 81.63) --
	(216.02, 81.63) --
	(217.12, 81.63) --
	(217.12,232.51) --
	(218.22,232.51) --
	(218.22,232.51) --
	(219.32,232.51) --
	(219.32,232.51) --
	(220.41,232.51) --
	(220.41, 81.63) --
	(221.51, 81.63) --
	(221.51, 81.63) --
	(222.61, 81.63) --
	(222.61, 81.63) --
	(223.70, 81.63) --
	(223.70, 81.63) --
	(224.80, 81.63) --
	(224.80, 81.63) --
	(225.90, 81.63) --
	(225.90, 81.63) --
	(226.99, 81.63) --
	(226.99,232.51) --
	(228.09,232.51) --
	(228.09,232.51) --
	(229.19,232.51) --
	(229.19,232.51) --
	(230.29,232.51) --
	(230.29,232.51) --
	(231.38,232.51) --
	(231.38, 81.63) --
	(232.48, 81.63) --
	(232.48, 81.63) --
	(233.58, 81.63) --
	(233.58, 81.63) --
	(234.67, 81.63) --
	(234.67, 81.63) --
	(235.77, 81.63) --
	(235.77, 81.63) --
	(236.87, 81.63) --
	(236.87, 81.63) --
	(237.97, 81.63) --
	(237.97, 81.63) --
	(239.06, 81.63) --
	(239.06, 81.63) --
	(240.16, 81.63) --
	(240.16, 81.63) --
	(241.26, 81.63) --
	(241.26, 81.63) --
	(242.35, 81.63) --
	(242.35, 81.63) --
	(243.45, 81.63) --
	(243.45, 81.63) --
	(244.55, 81.63) --
	(244.55, 81.63) --
	(245.64, 81.63) --
	(245.64, 81.63) --
	(246.74, 81.63) --
	(246.74, 81.63) --
	(247.84, 81.63) --
	(247.84, 81.63) --
	(248.94, 81.63) --
	(248.94, 81.63) --
	(250.03, 81.63) --
	(250.03, 81.63) --
	(251.13, 81.63) --
	(251.13, 81.63) --
	(252.23, 81.63) --
	(252.23,232.51) --
	(253.32,232.51) --
	(253.32,232.51) --
	(254.42,232.51) --
	(254.42,232.51) --
	(255.52,232.51) --
	(255.52,232.51) --
	(256.62,232.51) --
	(256.62,232.51) --
	(257.71,232.51) --
	(257.71,232.51) --
	(258.81,232.51) --
	(258.81,232.51) --
	(259.91,232.51) --
	(259.91,232.51) --
	(261.00,232.51) --
	(261.00,232.51) --
	(262.10,232.51) --
	(262.10,232.51) --
	(263.20,232.51) --
	(263.20,232.51) --
	(264.29,232.51) --
	(264.29,232.51) --
	(265.39,232.51) --
	(265.39,232.51) --
	(266.49,232.51) --
	(266.49,232.51) --
	(267.59,232.51) --
	(267.59,232.51) --
	(268.68,232.51) --
	(268.68, 81.63) --
	(269.78, 81.63) --
	(269.78, 81.63) --
	(270.88, 81.63) --
	(270.88, 81.63) --
	(271.97, 81.63) --
	(271.97, 81.63) --
	(273.07, 81.63) --
	(273.07, 81.63) --
	(274.17, 81.63) --
	(274.17, 81.63) --
	(275.27, 81.63) --
	(275.27, 81.63) --
	(276.36, 81.63) --
	(276.36, 81.63) --
	(277.46, 81.63) --
	(277.46, 81.63) --
	(278.56, 81.63) --
	(278.56, 81.63) --
	(279.65, 81.63) --
	(279.65, 81.63) --
	(280.75, 81.63) --
	(280.75, 81.63) --
	(281.85, 81.63) --
	(281.85, 81.63) --
	(282.95, 81.63) --
	(282.95, 81.63) --
	(284.04, 81.63) --
	(284.04, 81.63) --
	(285.14, 81.63) --
	(285.14, 81.63) --
	(286.24, 81.63) --
	(286.24, 81.63) --
	(287.33, 81.63) --
	(287.33, 81.63) --
	(288.43, 81.63) --
	(288.43, 81.63) --
	(289.53, 81.63) --
	(289.53, 81.63) --
	(290.62, 81.63) --
	(290.62, 81.63) --
	(291.72, 81.63) --
	(291.72, 81.63) --
	(292.82, 81.63) --
	(292.82, 81.63) --
	(293.92, 81.63) --
	(293.92, 81.63) --
	(295.01, 81.63) --
	(295.01, 81.63) --
	(296.11, 81.63) --
	(296.11, 81.63) --
	(297.21, 81.63) --
	(297.21, 81.63) --
	(298.30, 81.63) --
	(298.30, 81.63) --
	(299.40, 81.63) --
	(299.40, 81.63) --
	(300.50, 81.63) --
	(300.50, 81.63) --
	(301.60, 81.63) --
	(301.60, 81.63) --
	(302.69, 81.63) --
	(302.69, 81.63) --
	(303.79, 81.63) --
	(303.79, 81.63) --
	(304.89, 81.63) --
	(304.89, 81.63) --
	(305.98, 81.63) --
	(305.98,232.51) --
	(307.08,232.51) --
	(307.08,232.51) --
	(308.18,232.51) --
	(308.18,232.51) --
	(309.27,232.51) --
	(309.27, 81.63) --
	(310.37, 81.63) --
	(310.37, 81.63) --
	(311.47, 81.63) --
	(311.47, 81.63) --
	(312.57, 81.63) --
	(312.57, 81.63) --
	(313.66, 81.63) --
	(313.66, 81.63) --
	(314.76, 81.63) --
	(314.76, 81.63) --
	(315.86, 81.63) --
	(315.86, 81.63) --
	(316.95, 81.63) --
	(316.95, 81.63) --
	(318.05, 81.63) --
	(318.05, 81.63) --
	(319.15, 81.63) --
	(319.15, 81.63) --
	(320.25, 81.63) --
	(320.25, 81.63) --
	(321.34, 81.63) --
	(321.34, 81.63) --
	(322.44, 81.63) --
	(322.44, 81.63) --
	(323.54, 81.63) --
	(323.54, 81.63) --
	(324.63, 81.63) --
	(324.63, 81.63) --
	(325.73, 81.63) --
	(325.73, 81.63) --
	(326.83, 81.63) --
	(326.83, 81.63) --
	(327.92, 81.63) --
	(327.92, 81.63) --
	(329.02, 81.63) --
	(329.02, 81.63) --
	(330.12, 81.63) --
	(330.12, 81.63) --
	(331.22, 81.63) --
	(331.22, 81.63) --
	(332.31, 81.63) --
	(332.31, 81.63) --
	(333.41, 81.63) --
	(333.41, 81.63) --
	(334.51, 81.63) --
	(334.51, 81.63) --
	(335.60, 81.63) --
	(335.60, 81.63) --
	(336.70, 81.63) --
	(336.70, 81.63) --
	(337.80, 81.63) --
	(337.80, 81.63) --
	(338.90, 81.63) --
	(338.90, 81.63) --
	(339.99, 81.63) --
	(339.99, 81.63) --
	(341.09, 81.63) --
	(341.09, 81.63) --
	(342.19, 81.63) --
	(342.19, 81.63) --
	(343.28, 81.63) --
	(343.28, 81.63) --
	(344.38, 81.63) --
	(344.38, 81.63) --
	(345.48, 81.63) --
	(345.48, 81.63) --
	(346.57, 81.63) --
	(346.57, 81.63) --
	(347.67, 81.63) --
	(347.67, 81.63) --
	(348.77, 81.63) --
	(348.77, 81.63) --
	(349.87, 81.63) --
	(349.87, 81.63) --
	(350.96, 81.63) --
	(350.96, 81.63) --
	(352.06, 81.63) --
	(352.06, 81.63) --
	(353.16, 81.63) --
	(353.16, 81.63) --
	(354.25, 81.63) --
	(354.25, 81.63) --
	(355.35, 81.63) --
	(355.35, 81.63) --
	(356.45, 81.63) --
	(356.45, 81.63) --
	(357.55, 81.63) --
	(357.55, 81.63) --
	(358.64, 81.63) --
	(358.64, 81.63) --
	(359.74, 81.63) --
	(359.74, 81.63) --
	(360.84, 81.63) --
	(360.84,232.51) --
	(361.93,232.51) --
	(361.93,232.51) --
	(363.03,232.51) --
	(363.03,232.51) --
	(364.13,232.51) --
	(364.13, 81.63) --
	(365.22, 81.63) --
	(365.22, 81.63) --
	(366.32, 81.63) --
	(366.32, 81.63) --
	(367.42, 81.63) --
	(367.42, 81.63) --
	(368.52, 81.63) --
	(368.52, 81.63) --
	(369.61, 81.63) --
	(369.61, 81.63) --
	(370.71, 81.63) --
	(370.71, 81.63) --
	(371.81, 81.63) --
	(371.81, 81.63) --
	(372.90, 81.63) --
	(372.90,232.51) --
	(374.00,232.51) --
	(374.00,232.51) --
	(375.10,232.51) --
	(375.10, 81.63) --
	(376.20, 81.63) --
	(376.20, 81.63) --
	(377.29, 81.63) --
	(377.29, 81.63) --
	(378.39, 81.63) --
	(378.39, 81.63) --
	(379.49, 81.63) --
	(379.49, 81.63) --
	(380.58, 81.63) --
	(380.58, 81.63) --
	(381.68, 81.63) --
	(381.68, 81.63) --
	(382.78, 81.63) --
	(382.78,232.51) --
	(383.87,232.51) --
	(383.87,232.51) --
	(384.97,232.51) --
	(384.97,232.51) --
	(386.07,232.51) --
	(386.07,232.51) --
	(387.17,232.51) --
	(387.17, 81.63) --
	(388.26, 81.63) --
	(388.26, 81.63) --
	(389.36, 81.63) --
	(389.36, 81.63) --
	(390.46, 81.63) --
	(390.46, 81.63) --
	(391.55, 81.63) --
	(391.55, 81.63) --
	(392.65, 81.63) --
	(392.65, 81.63) --
	(393.75, 81.63) --
	(393.75, 81.63) --
	(394.85, 81.63) --
	(394.85, 81.63) --
	(395.94, 81.63) --
	(395.94, 81.63) --
	(397.04, 81.63) --
	(397.04, 81.63) --
	(398.14, 81.63) --
	(398.14, 81.63) --
	(399.23, 81.63) --
	(399.23, 81.63) --
	(400.33, 81.63) --
	(400.33, 81.63) --
	(401.43, 81.63) --
	(401.43, 81.63) --
	(402.52, 81.63) --
	(402.52,232.51) --
	(403.62,232.51) --
	(403.62,232.51) --
	(404.72,232.51) --
	(404.72,232.51) --
	(405.82,232.51) --
	(405.82,232.51) --
	(406.91,232.51) --
	(406.91,232.51) --
	(408.01,232.51) --
	(408.01,232.51) --
	(409.11,232.51) --
	(409.11,232.51) --
	(410.20,232.51) --
	(410.20,232.51) --
	(411.30,232.51) --
	(411.30,232.51) --
	(412.40,232.51) --
	(412.40,232.51) --
	(413.50,232.51) --
	(413.50,232.51) --
	(414.59,232.51) --
	(414.59,232.51) --
	(415.69,232.51) --
	(415.69,232.51) --
	(416.79,232.51) --
	(416.79,232.51) --
	(417.88,232.51) --
	(417.88,232.51) --
	(418.98,232.51) --
	(418.98,232.51) --
	(420.08,232.51) --
	(420.08, 81.63) --
	(421.18, 81.63) --
	(421.18, 81.63) --
	(422.27, 81.63) --
	(422.27, 81.63) --
	(423.37, 81.63) --
	(423.37, 81.63) --
	(424.47, 81.63) --
	(424.47, 81.63) --
	(425.56, 81.63) --
	(425.56, 81.63) --
	(426.66, 81.63) --
	(426.66, 81.63) --
	(427.76, 81.63) --
	(427.76, 81.63) --
	(428.85, 81.63) --
	(428.85, 81.63) --
	(429.95, 81.63) --
	(429.95, 81.63) --
	(431.05, 81.63) --
	(431.05, 81.63) --
	(432.15, 81.63) --
	(432.15, 81.63) --
	(433.24, 81.63) --
	(433.24, 81.63) --
	(434.34, 81.63) --
	(434.34, 81.63) --
	(435.44, 81.63) --
	(435.44, 81.63) --
	(436.53, 81.63) --
	(436.53, 81.63) --
	(437.63, 81.63) --
	(437.63, 81.63) --
	(438.73, 81.63) --
	(438.73, 81.63) --
	(439.83, 81.63) --
	(439.83, 81.63) --
	(440.92, 81.63) --
	(440.92, 81.63) --
	(442.02, 81.63) --
	(442.02, 81.63) --
	(443.12, 81.63) --
	(443.12, 81.63) --
	(444.21, 81.63) --
	(444.21, 81.63) --
	(445.31, 81.63) --
	(445.31, 81.63) --
	(446.41, 81.63) --
	(446.41, 81.63) --
	(447.50, 81.63) --
	(447.50, 81.63) --
	(448.60, 81.63) --
	(448.60, 81.63) --
	(449.70, 81.63) --
	(449.70, 81.63) --
	(450.80, 81.63) --
	(450.80, 81.63) --
	(451.89, 81.63) --
	(451.89, 81.63) --
	(452.99, 81.63) --
	(452.99, 81.63) --
	(454.09, 81.63) --
	(454.09, 81.63) --
	(455.18, 81.63) --
	(455.18, 81.63) --
	(456.28, 81.63) --
	(456.28,232.51) --
	(457.38,232.51) --
	(457.38,232.51) --
	(458.48,232.51) --
	(458.48,232.51) --
	(459.57,232.51) --
	(459.57,232.51) --
	(460.67,232.51) --
	(460.67, 81.63) --
	(461.77, 81.63) --
	(461.77, 81.63) --
	(462.86, 81.63) --
	(462.86, 81.63) --
	(463.96, 81.63) --
	(463.96, 81.63) --
	(465.06, 81.63) --
	(465.06, 81.63) --
	(466.15, 81.63) --
	(466.15, 81.63) --
	(467.25, 81.63) --
	(467.25, 81.63) --
	(468.35, 81.63) --
	(468.35, 81.63) --
	(469.45, 81.63) --
	(469.45, 81.63) --
	(470.54, 81.63) --
	(470.54, 81.63) --
	(471.64, 81.63) --
	(471.64, 81.63) --
	(472.74, 81.63) --
	(472.74, 81.63) --
	(473.83, 81.63) --
	(473.83, 81.63) --
	(474.93, 81.63);
\end{scope}
\begin{scope}
\path[clip] (  0.00,  0.00) rectangle (505.89,505.89);
\definecolor{drawColor}{RGB}{0,0,0}

\path[draw=drawColor,line width= 0.4pt,line join=round,line cap=round] ( 89.86, 75.60) -- (473.83, 75.60);

\path[draw=drawColor,line width= 0.4pt,line join=round,line cap=round] ( 89.86, 75.60) -- ( 89.86, 69.60);

\path[draw=drawColor,line width= 0.4pt,line join=round,line cap=round] (144.72, 75.60) -- (144.72, 69.60);

\path[draw=drawColor,line width= 0.4pt,line join=round,line cap=round] (199.57, 75.60) -- (199.57, 69.60);

\path[draw=drawColor,line width= 0.4pt,line join=round,line cap=round] (254.42, 75.60) -- (254.42, 69.60);

\path[draw=drawColor,line width= 0.4pt,line join=round,line cap=round] (309.27, 75.60) -- (309.27, 69.60);

\path[draw=drawColor,line width= 0.4pt,line join=round,line cap=round] (364.13, 75.60) -- (364.13, 69.60);

\path[draw=drawColor,line width= 0.4pt,line join=round,line cap=round] (418.98, 75.60) -- (418.98, 69.60);

\path[draw=drawColor,line width= 0.4pt,line join=round,line cap=round] (473.83, 75.60) -- (473.83, 69.60);

\node[text=drawColor,anchor=base,inner sep=0pt, outer sep=0pt, scale=  1.00] at ( 89.86, 54.00) {0};

\node[text=drawColor,anchor=base,inner sep=0pt, outer sep=0pt, scale=  1.00] at (144.72, 54.00) {50};

\node[text=drawColor,anchor=base,inner sep=0pt, outer sep=0pt, scale=  1.00] at (199.57, 54.00) {100};

\node[text=drawColor,anchor=base,inner sep=0pt, outer sep=0pt, scale=  1.00] at (254.42, 54.00) {150};

\node[text=drawColor,anchor=base,inner sep=0pt, outer sep=0pt, scale=  1.00] at (309.27, 54.00) {200};

\node[text=drawColor,anchor=base,inner sep=0pt, outer sep=0pt, scale=  1.00] at (364.13, 54.00) {250};

\node[text=drawColor,anchor=base,inner sep=0pt, outer sep=0pt, scale=  1.00] at (418.98, 54.00) {300};

\node[text=drawColor,anchor=base,inner sep=0pt, outer sep=0pt, scale=  1.00] at (473.83, 54.00) {350};

\path[draw=drawColor,line width= 0.4pt,line join=round,line cap=round] ( 75.60, 75.60) --
	(490.29, 75.60) --
	(490.29,238.54) --
	( 75.60,238.54) --
	( 75.60, 75.60);
\end{scope}
\begin{scope}
\path[clip] ( 25.20, 25.20) rectangle (504.69,264.94);
\definecolor{drawColor}{RGB}{0,0,0}

\node[text=drawColor,rotate= 90.00,anchor=base,inner sep=0pt, outer sep=0pt, scale=  1.00] at ( 37.20,157.07) {Bitwert};
\end{scope}
\begin{scope}
\path[clip] (  0.00,  0.00) rectangle (505.89,505.89);
\definecolor{drawColor}{RGB}{0,0,0}

\path[draw=drawColor,line width= 0.4pt,line join=round,line cap=round] ( 75.60, 81.63) -- ( 75.60,232.51);

\path[draw=drawColor,line width= 0.4pt,line join=round,line cap=round] ( 75.60, 81.63) -- ( 69.60, 81.63);

\path[draw=drawColor,line width= 0.4pt,line join=round,line cap=round] ( 75.60,232.51) -- ( 69.60,232.51);

\node[text=drawColor,rotate= 90.00,anchor=base,inner sep=0pt, outer sep=0pt, scale=  1.00] at ( 61.20, 81.63) {0};

\node[text=drawColor,rotate= 90.00,anchor=base,inner sep=0pt, outer sep=0pt, scale=  1.00] at ( 61.20,232.51) {1};
\end{scope}
\end{tikzpicture}

\end{scaletikzpicturetowidth}
\caption{Verdeckter Kanal mit einem Javascript/Webassembly-Empfänger und einem C-Programm als Sender. Dargestellt ist das zweimalige Empfangen des Bitstrings "0010001110" im Bereich von 38 bis 147 und von 165 bis 282. Der obere Plot spiegelt die Zeitmessung mittels Prime-and-Probe wieder, wobei die graue durchgezogene Linie $2 \cdot Median(y)$ ist. Zugriffswerte oberhalb der Linie werden als 1 und unterhalb als 0 interpretiert. Der untere Plot zeigt dies daraus erkannte Muster. Größe Blöcke von Einsen wie etwa im Bereich 21 bis 37 dienen der Synchronisation zwischen Sender und Empfänger. Die Nullen im Bitstring werden über den Abstand zwischen Einsen beziehungsweise dem Synchronisationsblock ermittelt.}
\end{figure}
\captionsetup[figure]{skip=10pt}

%Um einen Kanal zu Initialisieren 


%TODO: benchmark cache set finder
%entwickle bessere benchmark prozedur, messe zeit für contract jedes es und mittle dann


%Problem: v8 compiliert lazy, d.h. nur häufig verwendete

%Was wird angegriffen

%Verweis auf Paper Cache-Timing Attacks on RSA Key Generation
  \chapter{Angriffe}
\label{chapter:results}

\section{RSA Key Generierung}

Frühere Arbeiten zu Cache-Angriffen hatten es vor allem auf die Verschlüsselung- und Enschlüsselungsroutionen der RSA-Implementierungen abgesehen \cite{}, weshalb diese Routinen besonders gehärtet und geprüft sind.
Ein andereren Einstiegspunkt für die Schlüsselextraktion bieteten die Routienen zur Schlüsselserzeugung, welche erst in den letzten Monat in Arbeiten näher untersucht wurden.

\subsection{Primzahlgenerierung}

Um einen neuen Schlüssel zu erzeugen werden zuerst zwei Primzahlen erzeugt. Im Folgenden soll die Primzahlgenerierung von Mozilla NSS näher beleuchtet werden.

\begin{algorithm}[h]
\DontPrintSemicolon
\caption{Pseudo-Code für die Primzahlgenerierung in Mozilla NSS}
\label{alg:primeGenerationNSS}

\Fn{$AccessTimeEvictionSet(pointerToAddress)$}{
    pointerToAddressFirst $\leftarrow$ pointerToAddress\;
    timestampBefore $\leftarrow$ getTimeStamp()\;
    \While{pointerToAddressFirst != pointerToAddress}{
        pointerToAddress $\leftarrow$ readValue(pointerToAddress)\;
    }
    \Return getTimeStamp() - timestampBefore
}

\end{algorithm}


\section{Angriffe auf RSA Key Generierung}

Details zu Implementierung in Mozilla NSS

Mozilla Network Security Services(NSS) ist ein Menge von Bibliotheken, welche eine plattformübergreifende Entwicklung von sicheren Client- und Server-Anwendungen anstrebt. Dabei wird unter anderem TLS oder S/MIME implementiert. Mozilla NSS wird etwa im Firefox-Browser und der Mail-Anwendung Thunderbird eingesetzt.
Der Quellcode ist unter der Mozilla Public License verfügbar und kann online etwa im Firefox-Repository \cite{MozillaDXR} eingesehen werden.

Im folgenden soll die Schlüsselerzeugung für das RSA-Verfahren in Mozilla NSS beschrieben werden.

Der Code zur Schlüsselerzeugung liegt im Unterordner lib/freebl. Sobald die Schlüsselparameter $p,q,n,d,e$ bestimmt wurden, werden diese in der Funktion RSA_PrivateKeyCheck auf Gültigkeit überprüft (siehe Pseudo-Code \ref{alg:RSA_PrivateKeyCheck}).

\begin{algorithm}[h]
\DontPrintSemicolon
\caption{Pseudo-Code für RSA_PrivateKeyCheck aus rsa.c}
\label{alg:RSA_PrivateKeyCheck}

\Fn{$RSA_PrivateKeyCheck(key)$}{
    assert(p $\neq$ q)\;
    assert(n == p * q)\;
    assert(gcd(e, p-1) == 1)\;
    assert(gcd(e, q-1) == 1)\;
    assert(d*e == 1 mod p-1)\;
    assert(d*e == 1 mod q-1)\;
    assert(d_p == d mod p-1)\;
    assert(d_q == d mod q-1)\;
    assert(q * q^-1 == 1 mod p)\;
}
\end{algorithm}

%Ausgehend von der Funktion RSA_NewKey in rsa.c

%\begin{algorithm}[h]
%\DontPrintSemicolon
%\caption{Pseudo-Code für RSA_NewKey aus rsa.c}
%\label{alg:mp_gcd}
%
%\Fn{$RSA_NewKey(keySizeInBits, e)$}{
%    p $rightarrow$ generate_prime(keySizeInBits)\;
%    q $rightarrow$ generate_prime(keySizeInBits)\;
%    d $rightarrow$ rsa_build_from_primes(p,q,e)\;
%}
%\end{algorithm}

Relvant für diese Arbeit sind im Wesentlichen die Zeilen 4 und 5, in denen die Teilerfremdheit von $e$ zu $p-1$ und $q-1$, d.h. $gcd(e,p-1) = 1$ und $gcd(e,q-1) = 1$ geprüft wird.
Aus Performancegründen wird der Exponent $e$, anders als ursprünglich im RSA-Algorithmus beschrieben, auf den Wert 65537 fixiert.
Interessant ist die Funktion $mp_gcd$ (Pseudocode siehe \ref{alg:mp_gcd}), welche den größten gemeinsamen Teiler nach dem binären Verfahren von Josef Strein \cite{} berechnet. Dieser Algorithmus verwendet zum Berechnen des ggT ausschließlich Rechts-Shift-Operationen (Teilen durch 2) und Subtraktionen, wodurch dieser besonders für die in diesem Kontext verwendeten großen Zahlen interessant ist.
Die Zeilen 1 bis 6 der Funktion $mp_gcd$ können in diesem Fall ignoriert werden, da der Exponent $e$ wie oben beschrieben immer 65537 und damit ungerade ist. Bedeutender hingegen ist die $while$-Schleife in den Zeilen 11 bis 17, welche abhängig von den Eingaben Fallunterscheidungen durchführt. Das Ziel ist hier, die Rechts-Shift-Operation (Zeile 13) von der Subtraktionsfunktion (Zeile 17) zu unterscheiden, um die Zustände während der Berechnung zu rekonstruieren. \todo{hier fehlt der verweis auf das paper}

\begin{algorithm}[h]
\DontPrintSemicolon
\caption{Pseudo-Code für mp_gcd nach Josef Stein}
\label{alg:mp_gcd}

\Fn{$mp_gcd(u,v)$}{
    k $\leftarrow$ 0\;
    \While{iseven(u) \& iseven(v)}{
        u $\leftarrow$ u/2\;
        v $\leftarrow$ v/2\;
        k++\;
    }

    \If{isodd(u)}{
        t $\leftarrow$ -v\;
    } \Else {
        t $\leftarrow$ v\;
    }

    \While{t $\neq$ 0}{
        \While{iseven(t)}{
            t $\leftarrow$ t/2\;
        }
        \If{t > 0} {
            u $\leftarrow$ t\;
        } else {
            v $\leftarrow$ -t\;
        }
        t $\leftarrow$ u - v\;
    }

    \Return u*2^k
}
\end{algorithm}
  \chapter{Diskussion}
\label{chapter:discussion}



\section{Geschwindigkeit von Eviction-Set Suche}
Wie in Abschnitt \ref{} beschrieben, unterliegen die Laufzeiten der Eviction sind


\section{Erweiterung auf mehrere Threads}
Da präzise Timer zurzeit mittels eines Shared-Array-Buffer erzeugt werden müssen, benötigt die Angreiferin mindestens drei virtuelle Kerne um ein Angriff auszuführen. Einen für die Iteration der Timer-Variable, einen für den Prime-and-Probe Angriffscode und einen auf dem das Opferprogramm läuft.

Theoretisch reichen deshalb bereits Prozessoren mit zwei physischen und vier virtuellen Kernen wie sie noch vielfach von Intel im Umlauf sind. Bei zwei physischen Kernen wird der Timer-Thread jedoch zeitweise exklusiv auf einen physischen Kern rechnen und sich zeitweise einen Kern mit einem der anderen Threads teilen.
Wenn einer anderer Thread auf dem gleichen Kern rechnet halbiert sich in etwa die Iterationsgeschwindigkeit der Timervariable, sodass die Werte des Timer starken Schwankungen unterliegen.
Problematisch ist dies etwa in der Expand-Phase, da nicht erkannt wird, dass eine Candiadte-Set bereits ein Eviction-Set für die Zeugenadresse, da die Werte des Timers das Gegenteil suggerieren.

Somit ist es essenziell erkannt werden wann der Timer-Thread ein Kern für sich alleine beansprucht.
Es könnte zur Laufzeit wiederholt nach einer festen Zeitspanne die Timergenauigkeit überprüft werden.
Der Wert der Funktion $performance.now()$ als Referenz heranzuziehen, wie im Kaptiel 2 geschehen, ist aufgrund der geringen Auflösung (Chrome etwa 0,1ms) zu langsam.
TODO auf nuc testen.
Als Alternative kann etwa die Laufzeit einer Prime-and-Probe Operation auf einem wenig aktiven Cache-Set als Referenzwert genommen werden.
Hierbei ist zu beachten das Prüfung im Thread des Angriffscodes läuft und dieser sich eventuell auch einen physischen Kern teilen muss.

Ein weiteres Problem ist die Wahl der Zeitspanne, da eine kleine viel Overhead erzeugen würde und eine große zu langsam reagiert, sodass in der Zwischenzeit bereits Timer-Werte falsch interpretiert wurden.

Wie erwähnt steht der Thread des Angriffscodes vor dem gleichen Problem, sodass sich die Dauer eine Prime-and-Probe Iteration im schlimmsten Fall verdoppeln kann.
Folglich sollte ein Angriff auch mit halbierter Geschwindigkeit der Prime-and-Probe Operation erfolgreich sein, da ansonsten Cache-Aktivitäten in bestimmten Zeitabschnitten verloren gehen.

Andersherum wird es im Laufe des Angriffs passieren, dass sich Opferprogramm und Timer einen physischen Kern teilen.
Sofern das Problem der halbierten Timerauflösung gelöst wird, wird die Auflösung des Prime-and-Probe-Angriffs effektiv verdoppelt, da auch das Opferprogramm mit halber Geschwindigkeit läuft.
Da die Zuordnung der Prozesse zu den virtuellen Kernen im Browser nicht beeinflussbar ist und somit aus Sicht der Angreiferin willkürlich erfolgt, werden nur zufällige Zeitabschnitte besser aufgelöst.
Allerdings sollten hier auch die Kosten gegengerechnet werden, die für die Erkennung der veränderten Timerauflösung entstehen.

Aufgrund der beschrieben Probleme und den Auswirkungen der Abmilderungen ebendieser ist es besser auf mindestens drei physische Kerne zurückgreifen zu können.
Sobald die Browserhersteller die Auflösung von $performance.now()$ wieder in den Nanosekundenbereich hieven, wären auch zwei physische Kerne ausreichend. 

Wenn das angegriffene Endgerät $n$ virtuelle Kerne besitzt können also $n-3$ für die Verlangsamung des Opferprogramms eingesetzt werden.

Ideen: 
Ist die Eviction-Set-Search schnell genug für praktische Angriffe?\\
Probleme mit verwendeten Timer erörtern (sprunghaft, anpassung der erwarteten werte während laufzeit fehlerhaft)\\


Vergleiche Bremsverhalten mit clflush in native c code\\
Schätze ab wann ein Angriff auf mp_gcd möglich wäre (evaluerien mit delay parameter)\\
  %\chapter{Schlussbetrachtung}
\label{chapter:conclusions}

In diesem Kapitel werden die Ergebnisse dieser Arbeit zusammengefasst und die gewonnenen Erkenntnisse bewertet.
Abschließend wird ein Ausblick in die Zukunft der Cache-Angriffe im Allgemeinen sowie Browserangriffe im Speziellen gewährt.

\section{Zusammenfassung}
\todo{dieser komplette Absatz (section) hat hier nichts zu suchen. Er ist schön und passt gut zu "Gliederung dier Arbeit" im ersten Kapitel.}
Diese Arbeit hat aktuelle Fragestellungen bezüglich Cache-Angriffen untersucht, wobei sich auf praxisnahe Angriffe aus dem Browser heraus fokussiert wurde.
Im Hinblick auf die Praxisrelevanz wurde eine neuartige Eviction-Set-Suche evaluiert, welche die Initialisierungsphase eines Angriffs beschleunigt.
Des Weiteren wurden Leakages in der RSA-Schlüsselgenerierung von Mozilla NSS und OpenPGP.js untersucht sowie Probleme bei der Portierung nativer Angriffen erörtert.

\par \medskip                         

Nach der Einleitung wurde der aktuelle Forschungsstand zum Thema in Abschnitt \ref{related_work} dargelegt, um eine Einordnung der Arbeit in die Forschung zu ermöglichen.

\par \medskip                     

In Kapitel \ref{chapter:basics} wurden die nötigen technische Grundlagen zum Verständnis der Arbeit vermittelt.
So wurde die in der Arbeit verwendete Angriffstechnik Prime and Probe vorgestellt und erklärt warum der Angriff in aktuellen Prozessoren funktioniert.
Des Weiteren wurden die notwendigen Voraussetzungen erläutert, wie beispielsweise die Verfügbarkeit hochpräziser Timer.
Außerdem wurden die vorteilhaften Merkmale eines Cache-Angriffs aus dem Browser beschrieben.
Abschließend wurde die Speicher-Disambiguierung erklärt, dessen Verständnis für die beschleunigte Eviction-Set-Suche elementar ist.

\par \medskip                     

Mit der realen Implementierung des Angriffs in Javascript beziehungsweise Webassembly beschäftigte sich Kapitel \ref{chapter:preparation}.
Es wurde beschrieben wie die Voraussetzungen für einen Angriff im Webkontext trotz begrenzter Möglichkeiten geschaffen werden können.
Der Umsetzung des Eviction-Set-Suchalgorithmus im Webkontext wurde beschrieben und dessen Leistung theoretisch analysiert.
Zudem wurde die Performance des Eviction-Set-Suchalgorithmus sowie diverse Optimierungen desselben in der Praxis evaluiert.
Ein spezifisches Verhalten von Intel-Prozessoren im Bezug auf die Speicher-Disambiguierung wurde erklärt.
Erstmals wurde beschrieben wie damit ein verbesserter Eviction-Set-Suchalgorthmius im Webkontext umsetzbar ist und wie dieser im Vergleich zur Standardversion performt.
Als Beispiel wurde ein verdeckter Kanal vom Browser zu einem nativ laufenden Programm aufgebaut.

\par \medskip                      

Im \ref{chapter:results}. Kapitel wurden verschiedene Leakages in der RSA-Schlüsselgenerierung von Mozilla NSS und OpenPGP.js analysiert.
Dabei wurde erörtert wie die Ergebnisse einer automatisierten Leakage-Erkennung miteinbezogen werden können.
Bei der Leakage in der RSA-Primzahlgenerierung von Mozilla NSS und OpenPGP.js ist offen geblieben, ob sich diese eignet Teile der Primzahlen effizient zu rekonstruieren.
Darüber hinaus wurde die Portierung eines nativen Angriffs auf OpenSSL, hin zu einem Angriff im Browser auf Mozilla NSS analysiert.
Dabei auftretende Fallstricke, wie beispielsweise der fehlende clflush-Befehl im Webkontext, wurden herausgearbeitet und mögliche Lösungen diskutiert.

\par \medskip                       

Die Ergebnisse der Arbeit wurden im \ref{chapter:discussion}. Kapitel besprochen.
So wurde beschrieben welche zusätzlichen Auswirkungen die Hardware und Software des Opfers auf die Erfolgswahrscheinlichkeit eines Angriffs hat.
Gegenmaßnahmen wurden erläutert und die in diesem Kontext wichtige Rolle der Browserhersteller genannt.
Des Weiteren wurden Vorteile der RSA-Schlüsselgenerierung in OpenPGP.js gegenüber Mozilla NSS dargestellt.
Am Schluss des Kapitels wurde die Umsetzung der populären Meltdown und Spectre Angriffe im Webkontext diskutiert.

\section{Bewertung}

Die Ergebnisse zeigen, dass L3-Cache-Angriffe im Browser möglich sind. \todo{das war aber schon durch das Paper vorher klar}
Dabei ist jedoch zum einen die Ausführung des Angriffs langsamer und zum anderen sind die Möglichkeiten zum Ausbremsen des Opferprozesses beschränkt.

Um dem zu begegnen, können mehrere Angriffsinstanzen gestartet werden.
Zum einen kann die Überwachung aufgeteilt werden, wobei die Synchronisation mit den Zeitstempeln möglich ist.
Zum anderen sind mehrere Bremsinstanzen für eine bessere Performancereduktion denkbar.
Jedoch steigt die Systemauslastung an und damit die Gefahr, dass der Angriff vom Opfer entdeckt wird. 

Weiterhin erhöht sich die Initialisierungsdauer linear mit der Anzahl der Angriffsinstanzen.
Diesem wurde mit der Einführung einer neuen schnelleren Variante des Eviction-Set-Suchalgorithmus entgegengewirkt.

Der Angriff nutzt spezifische Eigenschaften der Intel-Architektur aus und ist beispielsweise nicht auf Prozessoren des Konkurrenten AMD lauffähig. \todo{Wiederhole hier kurz, welche.}
Im Hinblick auf die Ersetzung des Desktop-Computers oder Notebooks durch Smartphone und Tablet, ist eine nähere Untersuchung der Mikroarchitekturen dieser Geräte interessant. \todo{das muss in den Ausblick}

Ein großer Vorteil des Angriffs ist, dass er wenig Spuren nach der Ausführung hinterlässt.
Trotz der Möglichkeit des Angriffs aus dem Browser heraus, ist dieser in der Praxis 
aufwendig.
Denn er ist speziell auf den Code einer Softwareversion abgestimmt und diese muss zum Zeitpunkt des Angriffs laufen und die gewünschten Berechnungen ausführen. \todo{Hinzufügen: dies ist auch eine Schwäche im urspünglichen paper ZITAT; wo etwa XY Verschlüsselungen/Entschlüsselungen über einen zeit raum von etwa XY laufen müssen.}

In diese Kategorie fällt auch die Leakage mit den Unklarheiten im Bezug auf die praktische Anwendung bei der RSA-Primzahlgenerierung, wobei praktische Szenarien denkbar sind. \todo{den Satz}

So werden an der Universität zu Lübeck zum Beginn des Wintersemsters in einem Einführungskurs die E-Mail-Zertifikate von den neuen Studenten lokal erzeugt.
Ein Mitarbeiter könnte versuchen die Unterseite des ITSC, welche die Zertifikatsgenerierung beschreibt, zu manipulieren.
Da der Zeitraum indem die Zertifikatsgenerierung stattfindet, bekannt sowie eingrenzt ist und es viele potenzielle Opfer gibt, kann der Angriff sich hier auszahlen.


%wichtigste ergebnisse:
%schnellere eviction-set suche und standard suche evaluiert
%potenzielle leakage bei rsa-primzahlgenerierung gefunden in nss und openpgpjs
%angriffsportierung von c nach browser nicht immer möglich probleme bremsen bzw angriff an sich langsamer

\section{Ausblick}

Als mögliche Lösung könnte die Angreiferin $n-1$ Instanzen ihres Angriffs starten, wobei $n$ der Anzahl der virtuellen Kerne entspricht.
Somit würde durchgehend eine Instanz auf demselben physischen Kern wie das Opferprogramm laufen.

Die Veröffentlichung von Meltdown und Spectre Anfang 2018 hat ein neues Bewusstsein für Relevanz von Cache-Angriffen geschaffen.
Im Zuge dessen haben die Browserhersteller alle öffentlich bekannten Optionen für hochauflösende Timer unterbunden.
Google ist der erste Hersteller, welcher seit Chrome Version 68 in der Standardeinstellung wieder hochauflösende Timer erlaubt.
Dieser Schritt wird von Google mit der Einführung von Page-Isolation-Technik \cite{ChromeSiteIsolation} begründet, die eine Aufspaltung von verschiedenen Webseiten in unterschiedliche Prozesse sicherstellt und somit hinreichend vor den Gefahren von Meltdown und Spectre schützen soll.
Gegen Angriffe auf den geteilten L3-Cache, welche in dieser Arbeit diskutiert wurden, hilft diese Technik nicht.
Somit werden L3-Cache-Angriffe wieder ermöglicht und es liegt in der Hand der Softwareentwicklern ihre Implementierungen gegen diese Angriffe zu härten.

Der geteilte und inklusive L3-Cache mit LRU oder einer vergleichbaren Eviciton-Policy wird in Intel-Prozessoren als das eigentliche Problem noch Jahre Bestand haben.
Zurzeit nutzen im Desktopbereich nur die High-End-Prozessoren in Form Von Skylake-X einen nicht inklusiven L3-Cache \cite{SkylakeXL3Cache}.
Im Mainstream werden nicht exklusive L3-Caches eventuell mit der neuen Architektur Ice Lake eingeführt, welche frühestens Ende 2019 erscheinen wird \cite{IceLakeReleaseDate}.
Dies legt zumindest die angekündigte Verdoppelung des L2-Caches pro Kern bei gleichbleibender L3-Cache Größe nahe.
Die Prozessoren des Konkurrenten AMD sind aufgrund der exklusiven Caches \cite{CacheRyzen} vor diesen Angriffen nicht betroffen.

Daher muss in Zukunft die Gefahr von Cache-Angriffen im Browser stärker berücksichtigt und mit Gegenmaßnahmen begegnet werden, da Websprachen der Angreiferin die komplette Kontrolle über den Code überlassen.








%Denn bei einem inclusive ausgeführten Design würde die effektive L3-Cache Größe durch den verdoppelten L2-Cache Weiter sinken.

  
  % The appendix chapters are labeled using the latin alphabet.
  %\appendix
  %\chapter{Some Appendix}

% Add your appendices like this
\Blindtext
  
  % You will usually want to avoid a notation section. If you really need it,
  % then you should think about placing it in the appendix. It is a far better
  % style to develop the required notation thoroughly within your thesis (and
  % use the notation section in the appendix just for a quick reference).
  % \chapter{Notation}

The following notations are used throughout this thesis
continously, they are mostly adapted from the works
of \cite{Dijkstra1959} and \cite{Karp1972}.

\begin{itemize}
  \item Total functions from a domain $X$ to a codomain $Y$ are denoted
  $f:\ X \to Y$, while partial functions are denoted by $g:\ X \leadsto Y$.
  \item The image of a function is denoted
  $f(X) = \set{y \in Y \with \exists\ x \in X:\ f(x) = y} \subseteq Y$.
  \item \ldots
\end{itemize}


  % The back contains the bibliography and maybe an overview over tables and
  % figures.
  \backmatter

  % The guidelines say, you should use 'cell'. Using cell in latex is quite
  % buggy, so you might want to use 'apalike' instead. Cell is based on apalike
  % and very similar. You may also try out amsalpha, alpha and plain.
  \bibliographystyle{plain}
  
  % The following includes the bibliography
  \bibliography{ref}

  % This is a list of your TODOs at the end of the document
  % Remember: You may need to compile your document twice to see everything
  %\listoftodos

\end{document}
