Fragen:
introduction:
\todo{das hier finde ich albern. Du kannst dir gern den Absatz umdefinieren, sodass du bei jedem Absatz einen medskip hast. Aber hier plötzlich damit anfangen finde ich komisch.}
\todo{Sollte nur hier als Stilmittel verwendet werden, damit die Absätze den obigen Fragen besser zugeordnet werden können.}


basics
Hier sollte mehr stehen als ein Satz, ergibt sich vielleicht später. Hier gehört die Erklärung des Themas hin sowie eine Bestimmung der Rewlevanz des Themas für die Forschung.} 



results:
\todo{schreib bitte in die Bildunterschrift mit rein, was man hier sieht. Das gilt für alle Abbildungen, wo du das noch nicht tust}




Prüfen: Initialisierungsdauer linear mit der Anzahl der Angriffsinstanzen

%(fixen)
Aufgrund des Fehlens eines Beweises für die NP-Schwere oder eines effizienten Lösungsalgorithmus' ist dem Autor die Komplexität dieses Problems bewusst.

schluss:
Daher muss in Zukunft die Gefahr von Cache-Angriffen im Browser stärker berücksichtigt und mit Gegenmaßnahmen begegnet werden, da Websprachen der Angreiferin die komplette Kontrolle über den Code überlassen. (Der letzte Satz muss konkreter über die von dir angesprochenen Gegenmaßnahmen Auskunft geben.)


Masse erzeugen:
Verwandete Arbeiten mehr über RSA-Angriff schreiben siehe TODO
Analyse der Mod-Funktion (optional)
Analyse der Miller-Rabin-Funktion (optional)

Kapitel Implementation:
Benchmark zu Eviction-Set Suche vervollständigen
Probleme mit Timer evaluieren
Vervollständigung von Optimierung der Phasen (Contract init value)
Vervollständigung von Details der realen Implment
Vervollständigung von StoreForward

Kapitel Identifikation von Angriffszielen:
Beschreibung von Jan Tool und die Ergebnisse davon

\todo{schreib bitte in die Bildunterschrift mit rein, was man hier sieht. Das gilt für alle Abbildungen, wo du das noch nicht tust}

Verlangsamung mit mehreren Threads

Diskussion:
Verlangsamung mittels memory blocking
Ideen: 
Ist die Eviction-Set-Search schnell genug für praktische Angriffe?\\

Eventuell: Vergleiche Bremsverhalten mit clflush in native c code,
Achtung aufwendig da Eviction-Set Zuordnung in c implementiert werden muss

Schätze ab wann ein Angriff auf mp_gcd möglich wäre (evaluerien mit delay parameter)\\


%kopfzeile tabelle absetzen
%einleitung ~4-6 (related work)
%schluss ~3-4

%4/7 sollte neuer content sein