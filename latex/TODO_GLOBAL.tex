Fragen:
introduction:
\todo{das hier finde ich albern. Du kannst dir gern den Absatz umdefinieren, sodass du bei jedem Absatz einen medskip hast. Aber hier plötzlich damit anfangen finde ich komisch.}
\todo{Sollte nur hier als Stilmittel verwendet werden, damit die Absätze den obigen Fragen besser zugeordnet werden können.}


basics
Hier sollte mehr stehen als ein Satz, ergibt sich vielleicht später. Hier gehört die Erklärung des Themas hin sowie eine Bestimmung der Rewlevanz des Themas für die Forschung.} 



results:
\todo{schreib bitte in die Bildunterschrift mit rein, was man hier sieht. Das gilt für alle Abbildungen, wo du das noch nicht tust}



(Wo kommen die Antworten auf die vier Eingangsfragen? Struktur der Arbeit inkl. Fazit ist mir nicht ganz klar; vielleicht rekurrierst du auf die Fragen im Fazit direkt, wenn es strukturell Sinn ergibt?)



Prüfen: Initialisierungsdauer linear mit der Anzahl der Angriffsinstanzen

%(fixen)
Aufgrund des Fehlens eines Beweises für die NP-Schwere oder eines effizienten Lösungsalgorithmus' ist dem Autor die Komplexität dieses Problems bewusst.


Algorithmus zu Pseudocode umbenennen

Masse erzeugen:
Verwandete Arbeiten mehr über RSA-Angriff schreiben siehe TODO
Analyse der Mod-Funktion (optional)
Analyse der Miller-Rabin-Funktion (optional)

Kapitel Implementation:
Benchmark zu Eviction-Set Suche vervollständigen
Probleme mit Timer evaluieren
Vervollständigung von Optimierung der Phasen (Contract init value)
Vervollständigung von Details der realen Implment
Vervollständigung von StoreForward

Kapitel Identifikation von Angriffszielen:
Funktionsnamen besser kenntlich machen

Beschreibung von Jan Tool und die Ergebnisse davon
Wenn dann einleitungssatz hinzufügen
%Zum Einsatz kam dabei ein Werkzeug (Welches ist das? Wie heißt es?), welches automatisch potenzielle Leakages aufdeckt.

\todo{schreib bitte in die Bildunterschrift mit rein, was man hier sieht. Das gilt für alle Abbildungen, wo du das noch nicht tust}


Diskussion:
Ideen: 
Ist die Eviction-Set-Search schnell genug für praktische Angriffe?\\

Eventuell: Vergleiche Bremsverhalten mit clflush in native c code,
Achtung aufwendig da Eviction-Set Zuordnung in c implementiert werden muss

Schätze ab wann ein Angriff auf mp_gcd möglich wäre (evaluerien mit delay parameter)\\

anführungszeichen fixen „Sub“

check: 01.10: Prime and Probe überall zu Prime-and-Probe geändert
01.10 : colliding address zu colliding-address


%kopfzeile tabelle absetzen
%einleitung ~4-6 (related work)
%schluss ~3-4

%4/7 sollte neuer content sein