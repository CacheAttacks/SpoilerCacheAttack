\chapter*{Abstract}
\todo{Der zweite Titel gefällt mir besser, was meinst du?}
%Einleitungssatz (zu thema hinführen): 
\todo{ja, du brauchst einen Einleitungssatz, aber ohne Meltdown/Spectre. Irgendwas in Richtung: Mikroarchitekturangriffe haben sich als sehr mächtig erwiesen, erfordern aber meist ausführung von code auf der Opfermaschine. Allerdings wurde durch PAPER gezeigt, dass auch aus Browser möglich. Solche Angriffe werden in Skriptsprachen usw. und sind besonders verherend, weil... . }
%Eventuell Meltdown und Spectre als Aufhänger nutzen
Browser-basierte Seitenkanalangriffe auf die Mikroarchitektur der CPU sind besonders verehrend, da bereits der Besuche einer Webseite ausreicht, um Opfer eines Angriffs zu werden.

%kurz näher beschreiben wie das in etwa funktionert:
So ein Angriff wird in Scriptsprachen wie Javascript implementiert und nutzt den geteilten L3-Caches der CPU aus, um Informationen über die Speicherzugriffe anderer Programme zu erhalten.
Der Angriff ist weder auf einen Exploit im Browser noch auf fahrlässiges Verhalten der Nutzer angewiesen.


%was passiert in dieser Arbeit/Was sind wesentlich Ergenisse?
%-Neue potenzielle gefährliche leakage in NSS und OpenPGP.js
%-versuchte Portierung eines nativen RSA-Schlüsselgenerierungsangriff auf OpenSSL zu browsbasierten Angriff auf NSS
%-Probleme bei Portierung wie Bremsen, verminderte Angriffsgeschwindigkeit im Browser werden analysiert
%-Eviction-Set-Suchalgorithmus wird verbessert, neue Variante beschrieben und mit altem Algorithmus verglichen

In dieser Arbeit \todo{wurde der Drive by angriff nachimplementiert und es wurde versucht, den nativen Angriff auf RSA-Primzahlgenerierung (QUELLE) in den Browser zu portieren. Zusätzlich wurden} werden neue Leakages in der RSA-Primzahlgenerierung von Mozilla NSS und OpenPGP.js analysiert und Nachteile die gegenüber einem nativen Angriff bestehen, erörtert.
Exemplarisch wird die Portierung eines nativ laufenden Angriffs auf die RSA-Primzahlgenerierung zu einem Browserangriff 
untersucht und auftretende Fallstricke\todo{besser: Schwierigkeiten} beschrieben.
Des Weiteren wird ein neuer \todo{Algorithmus zum finden von eviction sets} Eviction-Set-Algorithmus vorgestellt, der die in der Praxis wichtige Initalisierungsphase eines L3-Cache-Angriffs beschleunigt. 