\chapter{Diskussion}
\label{chapter:discussion}

%\section{Geschwindigkeit von Eviction-Set Suche}
%Wie in Abschnitt %\ref{} 
%beschrieben, unterliegen die Laufzeiten der Eviction sind

\section{Warum viele Kerne sinnvoll sind}
Da präzise Timer zurzeit mittels eines Shared-Array-Buffer erzeugt werden müssen, benötigt die Angreiferin mindestens drei virtuelle Kerne um ein Angriff auszuführen. Einen für die Iteration der Timer-Variable, einen für den Prime-and-Probe Angriffscode und einen auf dem das Opferprogramm läuft.

Theoretisch reichen deshalb bereits Prozessoren mit zwei physischen und vier virtuellen Kernen wie sie noch vielfach von Intel im Umlauf sind. Bei zwei physischen Kernen wird der Timer-Thread jedoch zeitweise exklusiv auf einen physischen Kern rechnen und sich zeitweise einen Kern mit einem der anderen Threads teilen.
Wenn ein anderer Thread auf dem gleichen Kern rechnet halbiert sich in etwa die Iterationsgeschwindigkeit der Timervariable, sodass die Werte des Timers starken Schwankungen unterliegen.
Problematisch ist dies beispielsweise in der Expand-Phase, da nicht erkannt wird, dass ein Candidate-Set bereits ein Eviction-Set für die Zeugenadresse ist, da die Werte des Timers das Gegenteil suggerieren.

Somit ist es essenziell zu erkennen wann der Timer-Thread ein Kern für sich alleine beansprucht.
Es könnte zur Laufzeit wiederholt nach einer festen Zeitspanne die Timergenauigkeit überprüft werden.
Der Wert der Funktion $performance.now()$ als Referenz heranzuziehen, wie im Kapitel 2 geschehen, ist aufgrund der geringen Auflösung (Chrome etwa 0,1ms) zu langsam.
TODO auf nuc testen.
Als Alternative kann etwa die Laufzeit einer Prime-and-Probe Operation auf einem wenig aktiven Cache-Set als Referenzwert genommen werden.
Hierbei ist zu beachten, dass die Prüfung im Thread des Angriffscodes läuft und dieser sich eventuell auch einen physischen Kern teilen muss.

Ein weiteres Problem ist die Wahl der Zeitspanne, da eine kleine viel Overhead erzeugen würde und eine große zu langsam reagiert, sodass in der Zwischenzeit bereits Timer-Werte falsch interpretiert wurden.

Wie erwähnt steht der Thread des Angriffscodes vor dem gleichen Problem, sodass sich die Dauer eine Prime-and-Probe Iteration im schlimmsten Fall verdoppeln kann.
Folglich sollte ein Angriff auch mit halbierter Geschwindigkeit der Prime-and-Probe Operation erfolgreich sein, da ansonsten Cache-Aktivitäten in bestimmten Zeitabschnitten verloren gehen.

Andersherum wird es im Laufe des Angriffs passieren, dass sich Opferprogramm und Timer einen physischen Kern teilen.
Sofern das Problem der halbierten Timerauflösung gelöst wird, wird die Auflösung des Prime-and-Probe-Angriffs effektiv verdoppelt, da auch das Opferprogramm mit halber Geschwindigkeit läuft.
Da die Zuordnung der Prozesse zu den virtuellen Kernen im Browser nicht beeinflussbar ist und somit aus Sicht der Angreiferin willkürlich erfolgt, werden nur zufällige Zeitabschnitte besser aufgelöst.
Allerdings sollten hier auch die Kosten gegengerechnet werden, die für die Erkennung der veränderten Timerauflösung entstehen.

Aufgrund der beschrieben Probleme und den Auswirkungen der Abmilderungen ebendieser ist es besser auf mindestens drei physische Kerne zurückgreifen zu können.
Sobald die Browserhersteller die Auflösung von $performance.now()$ wieder in den Nanosekundenbereich hieven, wären auch zwei physische Kerne ausreichend. 

Wenn das angegriffene Endgerät $n$ virtuelle Kerne besitzt können also $n-3$ für die Verlangsamung des Opferprogramms eingesetzt werden.

\section{Viel hilft viel?}

$n-1-k$ Kerne für Bremsung nutzbar mit $K$ Kerne für eigentliche Überwachung 

TODO multithread bench

\section{Warum OpenPGP.js nicht anfällig ist}

Montgomery Implementation von gcd beschreiben

OpenPGPjs gcd(p-1,e) im Gegensatz zu Mozilla NSS schon bei Primzahlgenerierung

https://github.com/openpgpjs/openpgpjs/blob/master/src/crypto/public_key/rsa.js

pow (Zeile 3332) scheinbar fixed window implementation:\\
https://github.com/indutny/bn.js/blob/e69c617b3297b99aca429f30842e27979ef9beb5/lib/bn.js

\section{Memory-locking: Mit Kanonen auf Spatzen schießen}

TODO benchmarks einfügen

Risenpart:
http://pages.cs.wisc.edu/~venkatv/pvstudy-usenixsecurity15.pdf
\cite{MemoryLockingRisenpart}

Wu:
https://www.usenix.org/system/files/conference/usenixsecurity12/sec12-final97.pdf
\cite{MemoryLockingWu}

\section{Wie viel schneller muss er sein?}

Beschreiben bei wie vielen Taktzyklen Pausen mp_gcd ausgelesen werden kann.

\section{Speicherallokation: Wie hilfsbereit ist das Betriebssystem?}

Wenn der Speicher wenig fragmentiert ist, etwa nach dem Booten und große zusammenhängende physischer Speicherbereiche frei ist, scheint Ubuntu 16.04.5 LTS dahin zu tendieren das $evictionBuffer$-Array in diese Bereiche zu mappen.
Angenommen der komplette evictionBuffer wird in einen zusammenhängeden physischen Speicherbereich gemappt. Dann ist zu erwarten, dass nach jeweils $2^8$ Speicherblöcken eine neue colliding address gefunden wird, da $2^8$ Speicherblöcke genau $2^20$ Bytes belegen und bei einer colliding address die letzten 20 Bit identisch sind.

Die ersten Versuche fanden statt, nachdem der Testrechner schon über 30 Minuten in Betrieb war und viele Tests und Programme ausgeführt wurden.
Wenn sich unter diesen Bedingungen der Abstand der ersten 10 Speicherblöcke zueinander angeschaut wird, welche eine colliding address zum Speicherblock 0 bilden, zeigen sich Abstandsfolgen wie 
\begin{align*} 
654, 938, 1438, 224, 372, 543, 464, 84, 38, 134\\
170, 236, 573, 66, 452, 124, 206, 228, 1026, 62
\end{align*}
Diese scheinen auch nach näherer Betrachtung rein willkürlich um den Erwartungswert 256 zu schwanken.

Interessant werden die Folgen wenn der Testrechner neu gestartet wird und der physische Adressbereich somit wenig fragmentiert ist.
Dann können Abstandsfolgen wie 
\begin{align*} 
256, 256, 256, 353, 256, 256, 258, 254, 256, 188\\
256, 256, 256, 256, 256, 256, 271, 256, 256, 256
\end{align*}
beobachtet werden.
Bei den gegebenen Beispielen sind vereinzelt von 256 abweichende Abstände zu beobachten, die scheinbar nicht aus einem Fehler bei der colliding address Erkennung resultieren.

Denn angenommen die Abweichungen stammen aus vereinzelten Fehlern bei der Erkennung, dann sollten Abstandsfolgen wie 256, 353, 159, 256 entstehen mit $353+159=2*256$.
Die beobachteten Folgen jedoch weisen nach der vereinzelten Abweichung häufig wieder den Abstand 256 auf, sodass die Abweichung kein Fehler in der Erkennung zu sein scheint.

Ein weiteres Indiz dafür liefert die erfolgreiche Verifikation, dass der Speicherblock mit dem abweichenden Abstand später Teil eines Eviction-Sets wird.

Die längste ununterbrochene Folge mit dem optimalen Abstand 256 war bei allen Tests immer auf den Wert 10 beschränkt, wobei etwa 60-80\% aller Abstände den optimalen Wert 256 aufweisen.
Somit scheint die Größe zusammenhängender physischer Speicherbereiche aus dem Autor nicht bekannten Gründen begrenzt zu sein.

Mit dieser Beobachtung lässt sich 

Bermerkungen Mem-Alloc von Firefox/Chrome: https://www.cs.vu.nl/~giuffrida/papers/anc-ndss-2017.pdf
\cite{ASLROnTheLine}

\section{Warum ist Google so mutig?}
\label{GooglePageIsolation}

Wie erwähnt hat Google bereits Anfang August 2018 mit der Chrome Version 68 \cite{ChromeSharedArrayBufferAgain} die SharedArrayBuffer standardmäßig wieder aktiviert.
Mozilla hingegen hat auch in der aktuellen Firefox Developer  Edition 63.0b8 (Stand Ende September 2018) die SharedArrayBuffer in der Standardeinstellung deaktiviert.

Firefox standardmäßig vier Content-Prozesse. D.h. Meltdown problematischer Angriff, da Inhalte anderer Webseiten im gleich Addressbereich liegen.

Google jeder Tab in einem Prozess. \cite{ChromeSiteIsolation}


\section{Wie wohl fühlen sich Meltdown und Spectre im Browser?}
\label{MeltdownSpectreBrowser}

https://alephsecurity.com/2018/06/26/spectre-browser-query-cache/
\cite{OvercomingSpectreBrowserMitigations}

Sofern Meltdown über Prozessgrenzen hinweg Speicher auslesen soll, muss der geschützte Speicher in den Adressebereich des Angriffsprozess gemappt werden.
Dies ist im Browser nicht ohne weiteres möglich, da Webassembly eine eigene Speicheradressierung mitbringt.
Auch Spectre

\section{Triviale Divsionen: Ist das sinnvoll oder kann das Weg?}

-Benchmarks von Mozilla NSS und OpenPGPjs ohne Divisionstests